\section{Conclusions and Future Work}
\label{sec:conclusion}

By translating only at the level of specifications, \RZ provides a
useful middle ground between ad-hoc implementations and
machine-generated implementations.  It allows much more flexible
implementation strategies, but relies on programmers to
verify properties of their code.

Further, \RZ can serve as a means of explaining constructive
mathematics to programmers.  Programmers who are not knowledgeable
about constructive mathematics can still understand the output of the
translation, which involves familiar concepts such as abstract types
and (classical) first-order logic.   Looking at such examples can
provide the necessary intuition behind the original logic, and better
explain why one might want to work with constructive rather than
classical logic to begin with.

Axioms parameterized by models (e.g., initiality) currently translate
into signatures of ML functors.  We have experimented with an
alternative translation of such axioms into polymorphic types.  In
this case the \Verb|initial| axiom of the natural numbers yields
the specification
\begin{Verbatim}
   val initial: 'a -> ('a -> 'a) -> N.s -> 'a
\end{Verbatim}
which is exactly the familiar iterator for natural numbers (i.e.,
given an initial value, a function, and a natural number, apply the
function that many times to the initial value).  Such types can be
much more natural and much simpler for programmers to understand.  The
theory behind the translation is well understood, being the
phase-splitting translation of Harper, Mitchell, and Moggi
\cite{harper+:popl90}.  Because of limitations of ML not every
parameterized axiom can be turned into polymorphism; ML allows only
prenex quantifiers, and the quantifiers can range over types but not
type operators.  However we would like to do so where it is possible
(the common case).  As an alternative, we could attempt to retarget
the output to a language like Haskell~\cite{haskell} which supports
the necessary polymorphic types, though Haskell's support of modules
is much weaker.

Another possible extension would be to allow dependent families in the
input language. Fortunately, this does not require finding a target
language that supports dependent types; we can use the underlying
(non-dependent) types, and then express the dependencies as additional
properties that must be verified for the implementation.


%%% Local Variables: 
%%% mode: latex
%%% TeX-master: "case"
%%% End: 
