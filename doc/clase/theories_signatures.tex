
\section{Theories and Signatures}
\label{sec:theories-signatures}

\begin{figure*}[t]
\[
\footnotesize
\renewcommand{\arraystretch}{1.0}
\begin{array}[t]{l}
\mbox{\textbf{Theory Elements}}\\
\SET\ s\ [\EQUALS\ \mathit{set}\ ]\\
\CONST\ c\ [\COLON \mathit{set}\ ]\ [\EQUALS\ \mathit{term}\ ]\\
{}[\STABLE{}]\ \RELATION\ r\ [\COLON \mathit{set}\ ]\ [\EQUALS \mathit{prop}\ ]\\
\EQUIVALENCE \COLON \mathit{set}\\
\MODEL\ M\COLON\mathit{theory}\\
\AXIOM\ a\ [\ M\COLON\mathit{theory}\ ]^{*}\ [x\COLON\mathit{set}\ ]^*\EQUALS\mathit{prop}\\
\\
\mbox{\textbf{Propositions}}\\
\TRUE\\
\FALSE\\
\NOT\ \Prop\\
\Prop \AAND \Prop\\
\Prop \OOR \Prop\\
\Prop \IIMPLY \Prop\\
\Prop \IIFF \Prop\\
r [\ \Term\ ]^*\\
\Term\EQUALS\Term\\ %[\IN\ \Set]\\
\ALL\ [x\COLON \Set] \PERIOD \Prop\\
\SOME\ [x\COLON \Set] \PERIOD \Prop\\
\UNIQUE\ [x\COLON \Set] \PERIOD \Prop\\
\end{array}
\qquad
\begin{array}[t]{l}
\mbox{\textbf{Sets}}\\
\ZERO\\
\ONE\\
\BOOL\\
s\\
\mathit{Model}\PERIOD \mathit{name}\\
\Set \TIMES \cdots \TIMES \Set\\
\Setexp \ARROW \Setexp\\
\Label\ [\COLON \Setexp\ ]\ \PLUS \cdots \PLUS \Label\ [\COLON \Setexp\ ]\\
\LBRACE \Ident [\ \COLON \Setexp\ ]\ \BAR \Proposition \RBRACE\\
\Setexp \PERCENT \metav{relation}\\
%\RZ\ \Setexp\\
\\
\mbox{\textbf{Terms}}\\
x\\
\LPAREN \Term\COMMA \cdots \COMMA \Term \RPAREN\\
\Term\PERIOD \metav{n}\\
\Label\ [\ \Term\ ]\\
\MATCH\ \Term\ \WITH\ \mbox{\textit{pattern-matches}}\\
\LAMBDA\ x\COLON\Set\ \PERIOD\ \Term\\
\Term\ \Term\\
\Term \PERCENT \metav{relation}\\
\LET\ x \PERCENT \metav{relation}\ \IN\ \Term\EQUALS\Term\\
\Term \SUBIN \Set\\
\Term \SUBOUT \Set\\
\THE\ x\ [\COLON\Set\ ] \PERIOD \Prop\\
\LET\ x\ [\COLON\Set\ ] \EQUALS\Term\ \IN\ \Term
\end{array}
\]  
\vspace{-0.6truecm}
\caption{Input Language Summary}
\label{fig:input}  
\end{figure*}

In this section we describe first-order theories and signatures.
Our system translates the former into the later.

\subsection{Theories}
\label{sec:theories}

A \emph{theory} is a description of a mathematical structure, such as
a group, a vector space, a directed graph, etc. A theory consists of
%
\begin{itemize}
\item a list of \emph{basic sets},
\item a list of \emph{basic constants} belonging to specified sets,
\item a list of \emph{basic relations} on specified sets,
\item a list of axioms.
\end{itemize}
%
To take a simple example, consider the theory of a semigroup in which
every element has a (possibly non-unique) square root; recall that a
semigroup is a set with an associative binary operation and a neutral
element.\footnote{An example of a semigroup with square roots is the
  complex numbers with multiplication as the binary operation.} In our
system it could be written as follows:
\vspace{-0.2truecm}
{
\small
\VerbatimInput{semigroup.thy}
}
\goodbreak\goodbreak
\noindent The theory is enclosed by \Verb|thy|\ldots\Verb|end|. This theory
defines one basic set \Verb|s|, and two basic constants: an element
\Verb|e| of \Verb|s| and a (curried) binary infix operator \Verb|*| on
the set \Verb|s|. The \Verb|implicit| operator is not part of the
theory proper, but signals to the type checker that bound
variables named \Verb|x| or \Verb|y| or \Verb|z| should be assumed to
range over \Verb|s| unless otherwise specified. Finally,
we have three axioms. Axiom arguments, e.g., \Verb|x|, \Verb|y|, and
\Verb|z| in the associativity axiom, name the free variables occuring
in the axiom. It is not too big a mistake to think of them as being
universally quantified.

It is important to note that theories do not include proofs, but
rather just the statements of the axioms (and theorems) specified to
hold. Thus although axioms can be defined, one cannot actually refer
to them within the theory.

There are several features of theories that our system supports other
than those shown in this example above; the input language is
summarized in Figure~\ref{fig:input}, where brackets imply optional
elements.


Theories may declare or define relations. They may be \Verb|stable|,
i.e., their computational interpretation is trivial (see
Section~\ref{sec:implementation} for further discussion of this
point). Axioms can universally quantify over all models of a theory.
This is useful for describing universality properties, such as
initiality of an algebra or finality of a coalgebra.
  
The propositions are the familiar ones from first-order logic;
$\UNIQUE$ is unique existence ($\exists!$). In addition to the basic
empty ($\ZERO$) and unit ($\ONE$) sets, one can form cartesian
products, function spaces, tagged disjoint unions, subsets, and
quotients by stable equivalence relations. The corresponding
introduction and elimination forms appear in the language of terms.
For example, $\Term \PERCENT \metav{relation}$ is the equivalence
class under $\metav{relation}$ containing $\Term$, while $\LET\ x
\PERCENT \metav{relation} \mbox{\Verb| = |} \Term_1\ \IN\ \Term_2$
binds $x$ to a representative of the equivalence class $\Term_1$ to be
used in $\Term_2$. The expression $\Term \SUBIN \Set$ injects $\Term$
into a given subset (recording a proof obligation of the term actually
being a member of the subset), while $\Term \SUBOUT \Set$ projects
$\Term$ from a subset out into its superset $\Set$. The value of the
description operator $\THE\ x \,.\, \Prop$ is the unique $x$
satisfying $\Prop$; using it incurs the obligation of proving that
there is exactly one such~$x$.


\subsection{Signatures}
\label{sec:signatures}

On the logical side, we have models described by theories.  Thus on
the programming side we should have implementations being described by
specifications.  Our tool thus translates theories into
\emph{signatures}, which are ML's module interfaces.

Signatures allow us to require the existence of certain types, as well
as values of given type.  This allows decidable typechecking, but we
need more expressiveness in order to faithfully translate the content
of a theory.  We therefore generate signatures augmented by assertion
comments, which specify constraints on the values and functions an
implementation beyond their type.  It is the responsibility of the
programmer to check that the implementation satisfies these
assertions, as \RZ does not attempt to do any theorem proving.

Assertions are written in ordinary classical first-order logic. Since
programmers typically are not trained in constructive logic, this may
make it easier to verify the assertions.
\goodbreak
The output for the theory \Verb|SQGROUP| above is then:
{\small \VerbatimInput{semigroup.mli}}

At the ML level we have required a type \Verb|s|, and three values
\Verb|e|, \Verb|*|, and \Verb|sqrt|, of types \Verb|s|,
\Verb|s->s->s|, and \Verb|s->s|, respectively. The third value was
generated from the square root axiom, which has a non-trivial
computational content, cf.\ Subsection~\ref{subsec:real-transl}.

Comments contain other requirements, not expressible in ML, that
further contstrain the allowed implementations. The assertion
\Verb|PER(=s=)| abbreviates the requirement that \Verb|=s=| be a
partial equivalence relation on~\Verb|s|; its domain \Verb+||s||+ is
the subset of terms of type \Verb|s| that realize semigroup
elements, and the relation \Verb|=s=| identifies (possibly different)
terms realizing the same abstract semigroup element. These data
together determine a modest set. The assertion following the
declarations of \Verb|e| asserts that \Verb|e| realizes a valid
semigroup element, and the one following \Verb|*| asserts that
\Verb|*| must not be affected by the choice of realizers. Both
\Verb|e| and \Verb|*| must of course still satisfy the \Verb|unit| and
\Verb|assoc| axioms. Finally, the new function \Verb|sqrt| derived
from the logic must compute square roots. Since the theory requires
existence but not uniqueness of square roots, there is no requirement
that \Verb|sqrt| be invariant with respect to the partial equivalence
relation on \Verb|s|; different realizers of the same semigroup
element are allowed to produce (realizers of) different square roots.


\subsection{Parameterized Theories}
\label{sec:param-theor-funct}

A theory may be parameterized by one or more models of other theories.
For example, a theory \Verb|Real| of the reals may be parameterized in
terms of a model \Verb|N| of the naturals.  A theory of free groups may be
parameterized in terms of the generating set.

Parameterized theories serve two purposes.  A model of a
parameterized theory is a generic implementation that, given any
implementation of the parameters, returns an implementation of the
resulting theory.  At the level of ML, this would be a function from
modules to modules, a so-called \emph{functor}, and so a
parameterized theory can be translated into the signature of a
functor.

Alternatively, once we have described a parameterized theory
\Verb|Real|, we may wish to use it to describe a single specific
implementation of real numbers based on a specific model \Verb|N1|
(implementation) of the natural numbers; this can be described as
an implementation satisfying the theory \Verb|Real(N1)|.

The dual nature of parameterized theories as being both a description
of a parameterized model (a $\Pi$ type) and something which can be
applied to a model to produce a specialized theory (a $\lambda$) is
very reminiscent the type inclusion of Automath~\cite{automath}.  ML
does not permit applications of functor signatures, however, so we
beta-reduce all theory applications before generating signatures;
\Verb|Real(N1)| would produce a signature for a real-number
implementation that refers directly to \Verb|N1| rather than to a
generic parameter \Verb|N|.


%%% Local Variables: 
%%% mode: latex
%%% TeX-master: "case"
%%% End: 
