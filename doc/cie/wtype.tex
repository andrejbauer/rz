\section{Inductive types}
\label{sec:outp-induct-types}

%\subsection{Inductive types}
\label{sec:inductive-types}

To demonstrate the use of dependent types we show how RZ handles
general inductive types, also known as W-types or general
trees~\cite{nordstroem90:_progr_martin_type_theor}. Recall that a
W-type is a set of well-founded trees, where the branching types of
trees are described by a family of sets $B = \set{T(x)}_{x \in S}$.
Each node in a tree has a \emph{branching type}~$x \in S$, which
determines that the successors of the node are labeled by the elements
of~$T(x)$. For example, to get non-empty binary trees whose leaves are
labeled by natural numbers, define
%
\begin{align*}
  S &= \set{\mathtt{cons}} \cup \set{\mathtt{leaf}(n) \such n \in \NN}
  \\
  T(\mathtt{cons}) &= \set{\mathtt{left}, \mathtt{right}}
  \\
  T(\mathtt{leaf}(n)) &= \emptyset.
\end{align*}
%
Then a node of type $\mathtt{cons}$ has two successors, indexed by
constants $\mathtt{left}$ and $\mathtt{right}$, while a node of type
$\mathtt{leaf}(n)$ does not have any successors.

Figure~\ref{fig:wtype} shows an RZ axiomatization of W-types.
%
\begin{figure}
  \centering
  \sourcefile{wtype.thy}
  \caption{General inductive types}
  \label{fig:wtype}
\end{figure}
%
The theory $\mathtt{Branching}$ describes that a branching type
consits of a set~$s$ and a set~$t$ depending on~$s$. The theory~$W$ is
parametrized by a branching type~$B$. It specifies a set~$w$ of
well-founded trees and a tree-forming operation $\mathtt{tree}$ with a
dependent type $\prod_{x \in B.s} (B.t(x) \to w) \to w$. Given a
branching type~$x$ and a map $f : B.t(x) \to w$, $\mathtt{tree}\,x\,f$
is the tree whose root has branching type~$x$ and whose successor
labeled by $\ell \in B.t(x)$ is the tree~$f(\ell)$. The inductive
nature of~$w$ is expressed with the axiom $\mathtt{induction}$, which
states that for every property $M.p$, if $M.p$ is an inductive
property then every tree satsifies it. A property is said to be
\emph{inductive} if a tree $\mathtt{tree}\,x\,f$ satisfies it whenever
all its successors satisfy it.

In the translation, see Appendix~\ref{sec:outp-induct-types},
dependencies at the level of types and terms disappear. A branching
type is determined by a pair of non-dependent types~$s$ and~$t$ but
the per $\per_t$ depends on~$\values{s}$. The theory~$W$ turns into a
signature for a functor receiving a branching type~$B$ and returning a
type~$w$, and an operation $\mathtt{tree}$ of type $B.s \to (B.t \to
w) \to w$. In this example we use phase-splitting (see
Section~\ref{sec:implementation}) to translate axiom
$\mathtt{induction}$ into a specification of a polymorphic function
(compare with output for axiom \texttt{free} in the previous example)
%
\begin{equation*}
  \mathtt{induction}:
  (B.s \to (B.t \to w) \to (B.t \to \poly{ty\_p}) \to \poly{ty\_p}) \to w \to \poly{ty\_p},
\end{equation*}
%
which is a form of recursion on well-founded trees. Instead of trying
to explain what $\mathtt{induction}$ is supposed to do, we show in
Figure~\ref{fig:wtype-implementation} a surprisingly simple, complete
hand-written implementation of W-types. The reader may be entertained
by figuring out how $\mathtt{induction}$ works.
%
\begin{figure}
  \centering
  \sourcefile{wtype.ml}
  \caption{An implementation of general inductive types.}
  \label{fig:wtype-implementation}
\end{figure}

The following is the unabridged output for the theory of inductive
types shown in Figure~\ref{fig:wtype}.

\sourcefile{wtype.mli}

%%% Local Variables: 
%%% mode: latex
%%% TeX-master: "cie.tex"
%%% End: 
