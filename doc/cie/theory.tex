\section{Typed realizability}
\label{sec:typed-realizability}

Outline:
%
\begin{enumerate}
\item explain how PER models are a natural semantic model for a
  programmer, because they capture ideas like
\item choice of representation and representation invariants,
\item definition of a PER model, which makes things more precise, in
  particular
\item what computational model we use (ML-like language, terminating
  values are used to represent things), and
\item what the category of PER's is (but sweep categories under the
  rug)
\item discuss relationship with ``ordinary'' realizability a la
  Kleene~\cite{KleeneSC:intint}, since logicians and theoreticians
  usually know that kind
\end{enumerate}

There ought to be some examples here.

\section{Singatures and specifications}
\label{sec:sing-spec}

Outline:
%
\begin{enumerate}
\item We introduce assertions by saying that signatures only describe a
  part of a specification
\item Our assertions talk about per's
\item They are written in the negative fragment,
\end{enumerate}


\section{The Input Language}
\label{sec:input-language}

Explain the input language and its semantics.

\section{Translation}
\label{sec:translation}

Explain the translation.




%%% Local Variables: 
%%% mode: latex
%%% TeX-master: "cie"
%%% End: 
