
\section{Translation}
\label{sec:translation}

Having seen the input and output languages for RZ, we now explain how
the translation from one to the other works. The mathematical basis
for the translation procedure is the \emph{realizability
  interpretation} of constructive type theory and logic in the
category of modest sets, which was described in
Section~\ref{sec:modest-sets-pers}. A theory is translated to a
specification, where the theory elements are translated as follows.

A set declaration $\iParameter{s}{\iSet}$ is translated to
%
\begin{align*}
  & \otyspec{s}. \\
  & \opropspec{({\approx_{s}})}{\oarrow{s}{\oarrow{s}{\oProp}}}.\\
  & \oassertion{\mathrm{per}_{s}}{
    \begin{aligned}[t]
      & \oforall{x\,y}{s}{(\oimply{\oper{s}{x}{y}}{\oper{s}{y}{x}})}
        \land {} \\
      & \oforall{x\,y\,z}{s}{(
        \oimply{
          \oper{s}{x}{y} \land \oper{s}{y}{z}
          }{\oper{s}{x}{z}}
        )}.
    \end{aligned}
  }
\end{align*}
%
This says that the programmer should define a type~$s$, and a binary
predicate~$\approx_s$ on $\values{s}$ which is symmetric and
transitive.\footnote{The predicate is \emph{not} an ocaml value of
  type $s \to s \to \mathtt{bool}$, but an abstract relation on the
  set $\values{s} \times \values{s}$. Only in special cases can we
  implement the per as a decidable test.} When a dependent set is
declared in the input, e.g., $\iParameter{t}{s \to \iSet}$, the
underlying output type is still non-dependent, but the per $\approx_t$
receives an extra argument so and has the type
$\oarrow{s}{\oarrow{t}{\oarrow{t}{\oProp}}}$. This is so because
dependent sets are interpreted by uniform families, as was explained
in Section~\ref{sec:uniform-families}.

An element declaration $\iParameter{x}{s}$ is translated to
%
\begin{align*}
  & \ovalspec{x}{s}. \\
  & \oassertion{\mathrm{support}_x}{\ototal{x}{s}}.
\end{align*}
%
which requires a definition of a value~$x$ of type~$s$ which is in the
support of~$s$. When $s$ is not a basic set, RZ computes the
interpretation of the underlying type and support.

Set and term definitions here.

In the realizability interpretation of constructive logic, a predicate
$\phi$ on a modest sets $A$ is not just a subset of $\setOf{A}$, but
rather a relationship $\rz$ between the values of an \emph{underlying
  type} of realizers $\typeOf{\phi}$ and $\setOf{A}$. For $v \in
\values{\typeOf{\phi}}$ and $t \in \setOf{A}$ we write $v \rz \phi(t)$
when $v$ and $x$ are related. This can be read as ``$v$ realizes the
fact that $\phi(t)$ holds.'' (Take further explanation from CLASE
paper here.)


%%% Local Variables: 
%%% mode: latex
%%% TeX-master: "cie"
%%% End: 
