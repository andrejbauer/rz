\section{Examples}
\label{sec:examples}

In this section we look at serveral examples which demonstrate various
points of RZ. Unfortunately, serious examples from computable
mathematics take too much space\footnote{Simply because even to
  describe real numbers, the most basic structure in analysis, it
  takes several operations and a dozen axioms or so.} and will have to
be presented separately.

\subsection{Decidable sets}
\label{sec:decidable-sets}

A set $S$ is said to be decidable when, for all $x, y \in S$, $x = y$
or $\lnot (x = y)$. Classically, all sets are decidable. In RZ a
decidable set is described as
%
\VerbatimInput{decidable1.thy}
%
which results in the following output;
%
\VerbatimInput{decidable1.mli}

\subsection{Finite sets}
\label{sec:finite-sets}


\subsection{Inductive types}
\label{sec:inductive-types}



\subsection{Axiom of choice}
\label{sec:axiom-choice}

Axiom of choice, intensional axiom of choice, presentation axiom.


\subsection{Modulus of Continuity}
\label{sec:we-show-modulus-of-continuity-example}

%%% Local Variables: 
%%% mode: latex
%%% TeX-master: "cie"
%%% End: 
