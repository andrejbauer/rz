\documentclass{article}

\usepackage{amsmath}
\usepackage{amssymb}
\usepackage{theorem}
\usepackage{graphicx}
\usepackage{mathpartir}

%%%% MACROS %%%%%

% Types and terms in ocaml
\newcommand{\ctype}{\mathtt{type}\;}
\newcommand{\cint}{\mathtt{int}}
\newcommand{\clist}[1]{#1\;\mathtt{list}}
\newcommand{\ctrue}{\mathtt{true}}
\newcommand{\cwhile}[2]{\mathtt{while}\;#1\;\mathtt{do}\;#2\;\mathtt{done}}
\newcommand{\cstring}[1]{\mathtt{"#1"}}
\newcommand{\cprint}[1]{\mathtt{print\_string}\;#1}
\newcommand{\cfun}[2]{\mathtt{fun}\;#1 \to #2}
\newcommand{\clet}[2]{\mathtt{let}\;#1 = #2}
\newcommand{\cletrec}[2]{\mathtt{let}\;\mathtt{rec}\;#1 = #2}

% Realizaiblity
\newcommand{\per}{\approx}
\newcommand{\rz}{\Vdash}
\newcommand{\Type}{\mathsf{Type}}
\newcommand{\terminating}[1]{[\![#1]\!]}
\newcommand{\typeOf}[1]{|#1|}
\newcommand{\support}[1]{\|#1\|}
\newcommand{\modestSet}[1]{(#1, \typeOf{#1}, {\rz_{#1}})}

% Sets
\newcommand{\set}[1]{\{#1\}}
\newcommand{\such}{\mid}

% Theorem-like environments
{
  \theorembodyfont{\itshape}
  \newtheorem{theorem}{Theorem}[section]
  \newtheorem{lemma}[theorem]{Lemma}
  \newtheorem{proposition}[theorem]{Proposition}
  \newtheorem{corollary}[theorem]{Corollary}
}
{
  \theorembodyfont{\rmfamily}
  \newtheorem{definition}[theorem]{Definition}
  \newtheorem{example}[theorem]{Example}
}

% Quantifiers. I am really used to these...
\newcommand{\all}[3]{\forall\, #1 \,{\in}\, #2\,.\left(#3\right)}
\newcommand{\some}[3]{\exists\, #1 \,{\in}\, #2\,.\left(#3\right)}
\newcommand{\exactlyone}[3]{\exists!\, #1 \,{\in}\, #2\,.\left(#3\right)}
\newcommand{\xall}[3]{\forall\, #1 \,{\in}\, #2\,.\,#3}
\newcommand{\xsome}[3]{\exists\, #1 \,{\in}\, #2\,.\,#3}
\newcommand{\xexactlyone}[3]{\exists!\, #1 \,{\in}\, #2\,.\,#3}


%%%%%%%%%%%%%%%%%%%%%%%%%%%%%%%%%%%%%%%%%%%%%%%%%%

\begin{document}
\title{RZ (Working title)}
\author{
  Andrej Bauer\\
  Faculty of Mathematics and Physics\\
  University of Ljubljana, Slovenia\\
  \texttt{Andrej.Bauer@fmf.uni-lj.si}
  \and
  Chrisopher A. Stone\\
  Computer Science Department\\
  Harvey Mudd College, USA\\
  \texttt{stone@cs.hmc.edu}
}
\maketitle

\begin{abstract}
  RZ is RZ.
\end{abstract}

\section{Introduction}
\label{sec:introduction}

Given the constants, relations, and axioms of a mathematical structure,
expressed in constructive logic, what must a programmer do in order to
implement that mathematical structure?

For some familiar cases, the answer is obvious. Code implementing a
group must have a type to represent the group elements, a constant, a
unary operator, and a binary operator, each satisfying appropriate
axioms that guarantee the constant is a unit of the binary operator,
the unary operator computes inverses, and that the binary operator is
associative.

But for more interesting structures, especially those arising in
constructive analysis, the answer is much less obvious. Significant
research goes into finding suitable representations~\cite{???}. What
do you need to implement the real numbers (a complete, ordered field)?
What about a complete metric space? Or a [cs: I'm hoping Andrej will
fix this part up and add more interesting examples examples.] �`
\bigskip

The theory of realizability provides guidance in this task.
Realizability is a fundamental theoretical tool for the study of
logic, but it also has direct application to the design and
implementation of programs. It can be used to find a description of
the data structure directly corresponding to a mathematical
specification.

Extracting code interfaces by hand from mathematical axioms quickly
grows tedious, especially because different but logically equivalent
sets of axioms correspond to different, although interdefinable,
interfaces for code. One might want to see several of these, since
some interfaces can be more useful than others in practice.

And few programmers --- even those with solid backgrounds in
mathematics and logic --- are familiar with constructive logic or
realizability. Programmers are more familiar with language constructs
describing interfaces: C++ header files, ML signatures, Java
interfaces, and the like.

\bigskip

We have therefore implemented a system, called RZ, which serves as a
bridge between the logical world and the programming world. It
automatically translates specifications in constructive logic into
actual interface code in a programming language (currently Objective
Caml~\cite{ocaml}, but we are considering other languages as well). The
constructive part of the original specification turns into interface
code listing types and values that must exist in an implementation.

The remainder of the specification is maintained as assertions about
these types and values. Because these assertions have no computational
content, they can be interpreted constructively or classically equally
well, and in particular make equal sense to programmers (and
mathematicians) who are used to working in classical logic.

\bigskip

RZ was specifically designed as a lightweight system. Although the
realizability translation can be extended to transforming complete
proofs into complete code~\cite{komagata+:tr95}, we have not
implemented this. Other systems, including Coq~\cite{coq}, already
perform this task well.

But Coq works best when the entire task (from specification to code
generation) is performed within the same system. In contrast, the
interfaces generated by RZ can be implemented in any fashion, as long
as the assertions are satisfied. Code can be written by hand ---
allowing imperative, concurrent, and other arbitrary language
features, not just a ``functional'' subset. Or, the system could
generate a Coq interface as an \emph{input}, where the distinction
between computational (Set) and non-computational (Prop) is
automatically determined, and a corresponding implementation can be
provided through logical techniques.

The RZ system translates this to the language understood by many
programmers (typed interfaces with assertions in classical logic)

\bigskip

In this paper we show how RZ makes practical use of realizability and
other theoretical techniques, and provide some interesting motivational
examples.


%%% Local Variables: 
%%% mode: latex
%%% TeX-master: "cie"
%%% End: 

\section{Typed realizability}
\label{sec:typed-realizability}

Outline:
%
\begin{enumerate}
\item explain how PER models are a natural semantic model for a
  programmer, because they capture ideas like
\item choice of representation and representation invariants,
\item definition of a PER model, which makes things more precise, in
  particular
\item what computational model we use (ML-like language, terminating
  values are used to represent things), and
\item what the category of PER's is (but sweep categories under the
  rug)
\item discuss relationship with ``ordinary'' realizability a la
  Kleene~\cite{KleeneSC:intint}, since logicians and theoreticians
  usually know that kind
\end{enumerate}

There ought to be some examples here.

\section{Singatures and specifications}
\label{sec:sing-spec}

Outline:
%
\begin{enumerate}
\item We introduce assertions by saying that signatures only describe a
  part of a specification
\item Our assertions talk about per's
\item They are written in the negative fragment,
\end{enumerate}


\section{The Input Language}
\label{sec:input-language}

Explain the input language and its semantics.

\section{Translation}
\label{sec:translation}

Explain the translation.




%%% Local Variables: 
%%% mode: latex
%%% TeX-master: "cie"
%%% End: 

\section{Examples}
\label{sec:examples}

In this section we look at several examples which demonstrate various
points of RZ. Unfortunately, serious examples from computable
mathematics take too much space\footnote{The most basic structure
  in analysis (the real numbers) alone 
  requires several operations and a dozen or more axioms.} and will have to
be presented separately. The main theme is that constructively
reasonable axioms yield computationally reasonable operations.

\subsection{Decidable sets}
\label{sec:decidable-sets}

A set $S$ is said to be decidable when, for all $x, y \in S$, $x = y$
or $\lnot (x = y)$. In classical mathematics all sets are decidable, 
\iflong
because decidability of equality is just an instance of the law of
excluded middle.  But
\else
but
\fi % \iflong
RZ requires an axiom
%
\begin{source}
Parameter s : Set.
Axiom eq: \iForall x y : s, x = y \iOr \iNot (x = y).
\end{source}
%
to produce a realizer for equality
%
\begin{source}
val eq : s \iTo s \iTo [`or0 | `or1]
Assertion eq = \iForall (x:\iT{s}, y:\iT{s}),
                   (match eq x y with
                      `or0 \iImply x \iPer{s} y
                    | `or1 \iImply \iNot (x \iPer{s} y) )
\end{source}
%
We read this as follows: $\f{eq}$ is a function which takes
arguments~$\f{x}$ and~$\f{y}$ of type~$\f{s}$ and returns
$\mathtt{`or0}$ or $\mathtt{`or1}$. If it returns $\mathtt{`or0}$,
then $\oper{\f{s}}{\f{x}}{\f{y}}$, and if it returns
$\mathtt{`or1}$, then $\onot{(\oper{\f{s}}{\f{x}}{\f{y}})}$. In
other words $\f{eq}$ is a decision procedure%
\iflong
which tells when
values~$\f{x}$ and~$\f{y}$ represent the same element of the modest
set.
\else % \iflong
.
\fi % \iflong

\iflong
\subsection{Examples with obligations}
\label{sec:exampl-with-oblig}

In this section we show how RZ produces obligations, is sometimes able
to optimize them away, and show the effect of hoisting.

Consider how we might define division of real numbers. Assuming the
set of real numbers~$\f{real}$, a constant $\f{zero}$, and
multiplication operation~$\f{*}$ have already been declared and
axiomatized, we might write:
%
\begin{source}
Definition nonZeroReal := \{x : real | \iNot (x = zero)\}.
Parameter inv : nonZeroReal \iTo real.
Axiom inverse : \iForall x : real, \iNot (x = zero) -> x * (inv x) = one.
Definition (/) (x : real) (y : nonZeroReal) := x * (inv y).
\end{source}
%
We have defined the set of non-zero reals $\f{nonZeroReal}$ and
the inverse operation~$\f{inv}$ on it. Division $\f{x/y}$ is defined
as $\f{x * inv\; y}$. This does \emph{not} mean that the
programmer must necessarily implement division this way, only that the
implementation of $\f{x/y}$ must be equivalent to $\f{x * inv\;y}$.

In the axiom $\f{inverse}$, RZ encounters the subexpression
$\f{inv \;x}$. Because $\f{x}$ is quantified as an element of
$\f{real}$ rather than $\f{nonZeroReal}$, the typechecking
phase inserts a coercion that makes the expression well-typed.
Translation sees $\f{inv}(\f{x} \mathbin{{:}}
\f{nonZeroReal})$ instead of $\f{inv\ x}$ and translates this to
%
\begin{source}
inv (assure u:unit . \iNot (x \iPer{real} zero) in (x, u))
\end{source}
%
If this were the final output, the programmer would have to verify
that~$x$ is not zero, and provide a trivial realizer for it. However,
in this case the thinning phase first removes the trivial realizer,
%
\iflong
\begin{source}
inv (assure \iNot (x \iPer{real} zero) in x)
\end{source}%
\fi % \iflong
%
and then the optimizer determines that the obligation is not needed
because the whole expressions appears under the hypothesis that~$x$ is
not zero. So in the end the programmer sees
%
\begin{source}
(**  Assertion inverse =
  \iForall (x:\iT{real}),  \iNot (x \iPer{real} zero) \iTo (x * inv x) \iPer{real} one
*)
\end{source}
%
Assuming further that a strict linear order $<$ on~$\f{real}$ has
been axiomatized, we might proceed by relating it to $\f{inv}$:
%
\begin{showInput}
\iAxiom{\f{inv\_positive}}{
\iforall{\f{x}}{\f{real}}{\oimply{\f{zero} < \f{x}}{\f{zero} < \f{inv\ x}}}
}.
%Axiom inv_positive: forall x : real, zero < x -> zero < inv x.
\end{showInput}%
%
Once again $\f{inv\;x}$ appears in the input, bt this time the
optimizer is unable to remove the obligation, so the output is
%
\begin{showOutput}
\oassertion{\f{inv\_positive}}{}\\
\qquad
\oforallt{\f{x}}{\f{real}}{
  \oimply{
    \f{zero} < \f{x}}
  {\f{zero} < \f{inv}
    (\oobligx{\onot{(\oper{\f{real}}{\f{x}}{\f{zero}})}}{\f{x}})}
}
%(**  Assertion inv_positive =
%       forall (x:||real||),  zero < x ->
%         zero < inv (assure (not (x =real= zero)) in x)
%*)
\end{showOutput}
%
Local obligations can sometimes be hard to read, but if we activate the hoisting phase
(see Section~\ref{sec:implementation}), the obligation can be moved
to the top level. As this is done, the hypotheses under which the
obligation appears are collected, and we get
%
\begin{showOutput}
\oassertion{\f{inv\_positive}}{}\\
\qquad
\oobligx{
\oforallt{\f{x}}{\f{real}}{
\oimply{\f{zero}<\f{x}}{\onot{(\oper{\f{real}}{\f{x}}{\f{zero}})}}
}
}{}\\
\qquad\qquad{
\oforallt{\f{x}}{\f{real}}{
  \oimply{
    \f{zero} < \f{x}}
  {\f{zero} < \f{inv\ x}}
}}
%(**  Assertion inv_positive =
%  assure (forall (x:||real||),  zero < x -> not (x =real= zero))
%    in forall (x:||real||),  zero < x -> zero < inv x
%*)
\end{showOutput}%
%
Now it is easier to understand what must be checked, namely that
positive reals are not zero---an easy consequence of irreflexivity
of~$<$, but not something that RZ optimizer is aware of.

Lastly, we could define the golden ratio as the positive solution of
$x^2 = x + 1$,
%
\begin{source}
the x : real, (zero < x /\ x*x = x + one)
\end{source}
%
Not surprisingly, RZ cannot determine that there is a unique such~$x$,
so it outputs an obligation:
%
\begin{source}
assure x:real.
  (x : ||real|| /\ zero < x /\ x * x =real= x + one /\
     (forall (x':||real||),  zero < x' /\ x' * x' =real= x' + one ->
        x =real= x'))
  in x
\end{source}
\fi % \iflong

\iflong
\subsection{Finite sets}
\label{sec:finite-sets}

\begin{figure}[t]
\begin{showInputSmall}
\iDefinition{\f{Semilattice}}{\mathsf{thy}}\\
\qquad \iParameter{\f{s}}{\iSet}.\\
\qquad \iParameter{\f{zero}}{\f{s}}.\\
\qquad \iParameter{\f{join}}{\f{s}\to\f{s}\to\f{s}}.\\
\\
\qquad \iAxiom{\f{commutative}}{\forall \f{x},\f{y}:\f{s}.\ \f{join\ x\ y} = \f{join\ y\ x}}.\\
\qquad \iAxiom{\f{associative}}{\forall \f{x},\f{y},\f{z}:\f{s}.\ \f{join}\,(\f{join\ x\ y})\,\f{z} = \f{join}\,\f{x}\, (\f{join\ y\ z})}.\\
\qquad \iAxiom{\f{idempotent}}{\forall \f{x}:\f{s}.\ \f{join\ x\ x} = \f{x}}.\\
\qquad \iAxiom{\f{neutral}}{\forall \f{x}:\f{s}.\ \f{join\ x\ zero} = \f{x}}.\\
\mathsf{end}.
\end{showInputSmall}
  \caption{The theory of a semilattice}
  \label{fig:semilattice}
\end{figure}

\iflong
There are many characterizations of finite sets, but the one that
works best constructively is due to Kuratowski, who identified the
finite subsets of~$A$ as the least family~$K(A)$ of subsets of~$A$
that contains the empty set and is closed under unions with
singletons. This characterization relies on powersets, which are not
available in RZ. But the gist of it, namely that $K(A)$ is an
\emph{inital} structure a suitable sort, can be expressed as follows.

\else
%
The family $K(A)$ of finite subsets of a set~$A$ may be characterized
as the free $\vee$-semilattice generated by~$A$.
%
\fi
%
Recall that a \emph{$\vee$-semilattice} is a set~$S$ with a
constant~$0 \in S$ and an associative, commutative, and idempotent
operation ``join'' $\vee$ on~$S$ such that $0$ is the neutral element
for~$\vee$, see Figure~\ref{fig:semilattice} for RZ axiomatization of
semilattices.
%
The Kuratowski finite sets~$K(A)$ are the \emph{free} semilattice
generated by a set~$A$, where $\vee$ is union and $0$ is the empty
set. This is formalized in RZ as shown in Figure~\ref{fig:kuratowski}.
%
\begin{figure}
\begin{showInput}
\iDefinition{\texttt{K}\ (\texttt{A} : \f{thy}\ \  \iParameter{\f{a}}{\iSet}.\ \  \texttt{end})}{\f{thy}}\\
\qquad \iInclude{\f{Semilattice}}.\\
\qquad \iParameter{\f{singleton}}{\f{A.a} \to \f{s}}.\\
\qquad \iDefinition{\f{fin}}{\f{s}}.\\
\qquad \iDefinition{\f{emptyset}}{\f{zero}}.\\
\qquad \iDefinition{\f{union}}{\f{join}}.\\
\\
\qquad \iAxiom{\f{free}}{} \forall \f{S}:\f{Semilattice}.\ \forall\f{f}:{\f{A.a}\to\f{S.s}}.\ \exists!g:\f{fin}{\to}{\f{S.s}}.\\
\qquad \qquad \qquad \qquad \qquad \f{g\ emptyset} = \f{S.zero}\ \land\\
\qquad \qquad \qquad \qquad \qquad \forall\f{x}:\f{A.a}.\ \f{f\ x} = \f{g}\ (\f{singleton\ x})\ \land\\
\qquad \qquad \qquad \qquad \qquad \forall\f{u},\f{v}:\f{fin}.\ \f{g}(\f{union\ u\ v}) = \f{S.join}\,(\f{g\ u})\,(\f{g\ v}).\\
\f{end}.
\end{showInput}
  \caption{Kuratowski finite sets}
  \label{fig:kuratowski}
\end{figure}
%
The theory $\f{K}$ is parametrized by a model~$\f{A}$ which contains a
set~$\f{a}$. In the first line we include the theory of semilattices.
Then we postulate an operation $\f{singleton}$ which injects the
generators into the semilattice. The three definitions are just a
convenience, so that we can refer to the parts of $\f{K(A)}$ by their
natural names, e.g., $\f{emptyset}$ instead of $\f{zero}$. The axiom
$\f{free}$ expresses the fact that $\f{K}(\f{A})$ is the free
semilattice on~$\f{A.a}$: for every semilattice $\f{S}$ and a map
$\f{f} : \f{A.a} \to \f{S.s}$ from the generators to the underlying
set of~$\f{S}$, there exists a unique semilattice homomorphism $\f{g}
: \f{fin} \to \f{S.s}$ such that $\f{f}(\f{x}) = \f{g}(\f{singleton\;
  x})$.

The output for $\f{Semilattice}$ and~$\f{K}$ specifies
values of suitable types for each declared constant and operation. All
axioms but the last one are equations and have straightforward
translations in terms of underlying pers. The output for the axiom
$\f{free}$ is shown in Figure~\ref{fig:free}.
%
\begin{figure}
  \centering
\begin{showOutputSmall}
\f{module}\ \f{Free} : \f{functor} (\f{S} : \f{Semilattice}) \to \f{sig}\\
\quad \ovalspec{\f{free}}{\oarrow{(\oarrow{\f{A.a}}{\f{S.s}})}{\oarrow{\f{fin}}{\f{S.s}}}}
\\
\quad
\oassertion{\f{free}}{}
\\
\quad\quad
\oforallt{\f{f}}{\oarrow{\f{A.a}}{\f{S.s}}}{
\olet{\f{g}}{\f{free\ f}}{}}\\
\qquad\qquad \ototal{\f{g}}{\oarrow{\f{fin}}{\f{S.s}}} \land \oper{\f{S.s}}{\f{g\ emptyset}}{\f{S.zero}} \land {}\\
\qquad\qquad (\oforallt{\f{x}}{\f{A.a}}{\oper{\f{S.s}}{\f{f\ x}}{\f{g(singleton\ x)}}}) \land {} \\
\qquad\qquad (\oforallt{\f{u},\f{v}}{\f{fin}}{\oper{\f{S.s}}{\f{g(union\ u\ v)}}{\f{S.join\ (g\ u)\ (g\ v)}}}) \land {} \\
\qquad\qquad
((\oforall{\f{h}}{\oarrow{\f{fin}}{\f{S.s}}}{}\\
\qquad\qquad\qquad
\ototal{\f{h}}{\oarrow{\f{fin}}{\f{S.s}}} \land
\oper{\f{S.s}}{\f{h\ emptyset}}{\f{S.zero}} \land {} \\
\qquad\qquad\qquad (\oforallt{\f{x}}{\f{A.a}}{\oper{\f{S.s}}{\f{f\ x}}{\f{h(singleton\ x)}}}) \land {} \\
\qquad\qquad\qquad
(\oforallt{\f{u},\f{v}}{\f{fin}}{\oper{\f{S.s}}{\f{h(union\ u\ v)}}{\f{S.join\ (h\ u)\ (h\ v)}}})) \Rightarrow {} \\
\qquad\qquad
\oforall{\f{x},\f{y}}{\f{fin}}{\oimply{\oper{\f{fin}}{\f{x}}{\f{y}}}{\oper{\f{S.s}}{\f{g\ x}}{\f{h\ y}}}})
%module Free : functor (S : Semilattice) ->
%sig
%val free : (A.a -> S.s) -> fin -> S.s
%(**  Assertion free = 
%forall (f:||A.a -> S.s||), 
%  let g = free f in g : ||fin -> S.s|| /\ 
%  g emptyset =S.s= S.zero /\ 
%  (forall (x:||A.a||),  f x =S.s= g (singleton x)) /\ 
%  (forall (u:||fin||, v:||fin||), g (union u v) =S.s= S.join (g u) (g v)) /\ 
%  (forall h:fin -> S.s,  h : ||fin -> S.s|| /\ 
%     h emptyset =S.s= S.zero /\ 
%     (forall (x:||A.a||), f x =S.s= h (singleton x)) /\ 
%     (forall (u:||fin||, v:||fin||), 
%        h (union u v) =S.s= S.join (h u) (h v)) ->
%     forall x:fin, y:fin,  x =fin= y -> g x =S.s= h y)
%*)
%end
\end{showOutputSmall}%
  \caption{Output of axiom $\texttt{free}$.}
  \label{fig:free}
\end{figure}
%
Because the axiom quantifies over all models~$\f{S}$ of the theory
$\f{Semilattice}$ its translation is a functor~$\f{Free}$ which
accepts an implementation of a semilattice~$S$ and yields a realizer
$\f{free}$ validating the axiom. The computational meaning of
$\f{free}$ is a combination map and fold operation, taking a map
$\f{f} : \f{A.a} \to \f{S.s}$ and a finite set~$\f{u} = \set{x_1,
  \ldots, x_n}$, and return $\f{f}(x_1) \vee \cdots \vee \f{f}(x_n)$,
where $\vee$ is the join operation on the semilattice~$S$.

Applying phase-splitting to this axiom yields the even simpler
specification
%
\begin{equation*}
\mathtt{val}\ \f{free} : \alpha \to (\alpha \to \alpha \to \alpha) \to (\f{A.a}\to\alpha) \to \f{fin} \to \alpha	
\end{equation*}
%
(with an appropriate assertion)
which replaces the module parameter \texttt{S} by two extra term arguments term (corresponding to the module components \texttt{S.zero} and \texttt{S.join}) 
and a type argument $\alpha$ for the type of lattice elements (corresponding to the module input \texttt{S.s}).  This is even
more recognizable as a folding operation over the set.


It is important to note that, in contrast to \texttt{fold} operators found in typical functional
languages, \texttt{free} is only expected to work for suitable \texttt{join} arguments (e.g., idempotent and order independent).  These
sets are not the typical finite-set data structure: there is no membership predicate, nor
is there a way to compute the size of a set.  There is no
assumption that equality is decidable for set elements; this permits
finite sets of  exact real numbers, for example.  Decidable equality
is required both for membership and for detecting
whether the same element has been added twice to the same set\footnote{The natural implementation would thus
be an unordered collection of elements, possibly with duplicates.}.

Some operations are nevertheless computable.  Using \texttt{free} one
can determine whether a finite set is empty.  In the case of a set of exact
real numbers, we cannot compute their sum, but we could compute maximum or minimum.

More common set implementations (e.g., the \texttt{Set} module in the OCaml standard library)
implement sets over values with either decidable total order; these could also be
formalized in RZ.
\fi % \iflong

\subsection{Inductive types}
\label{sec:inductive-types}

To demonstrate the use of dependent types we show how RZ handles
general inductive types, also known as W-types or general
trees~\cite{nordstroem90:_progr_martin_type_theor}. Recall that a
W-type is a set of well-founded trees, where the branching types of
trees are described by a family of sets $B = \set{T(x)}_{x \in S}$.
Each node in a tree has a \emph{branching type}~$x \in S$, which
determines that the successors of the node are labeled by the elements
of~$T(x)$.
%
\iflong
%
For example, to get non-empty binary trees whose leaves are
labeled by natural numbers, define
%
\begin{align*}
  S &= \set{\f{cons}} \cup \set{\f{leaf}(n) \such n \in \NN}
  \\
  T(\f{cons}) &= \set{\f{left}, \f{right}}
  \\
  T(\f{leaf}(n)) &= \emptyset.
\end{align*}
%
Then a node of type $\f{cons}$ has two successors, indexed by
constants $\f{left}$ and $\f{right}$, while a node of type
$\f{leaf}(n)$ does not have any successors.
\par
%
\fi % iflong
%
Figure~\ref{fig:wtype} shows an RZ axiomatization of W-types.
%
\begin{figure}
\begin{source}
Parameter W : [B : Branching] \iTo
thy
  Parameter w : Set.
  Parameter tree : [x : B.s] \iTo (B.t x \iTo w) \iTo w.
  Axiom induction:
    \iForall M : thy Parameter p : w \iTo Prop. end,
    (\iForall x : B.s, \iForall f : B.t x \iTo w,
       ((\iForall y : B.t x, M.p (f y)) \iTo M.p (tree x f))) \iTo
    \iForall t : w, M.p t.
end.
\end{source}
  \caption{General inductive types}
  \label{fig:wtype}
\end{figure}
%
The theory $\f{Branching}$ describes that a branching type
consists of a set~$\f{s}$ and a set~$\f{t}$ depending on~$\f{s}$. The theory~$\f{W}$ is
parameterized by a branching type~$\f{B}$. It specifies a set~$\f{w}$ of
well-founded trees and a tree-forming operation $\f{tree}$ with a
dependent type $\Pi_{\f{x} \in \f{B.s}} (\f{B.t(x)} \to \f{w}) \to \f{w}$.
%
\iflong
%
Given a
branching type~$\f{x}$ and a map $\f{f} : \f{B.t(x)} \to \f{w}$, $\f{tree}\;\f{x}\;\f{f}$
is the tree whose root has branching type~$\f{x}$ and whose successor
labeled by $\ell \in \f{B.t}(\f{x})$ is the tree~$\f{f}(\ell)$.
%
\fi
%
The inductive nature of~$\f{w}$ is expressed with the axiom
$\f{induction}$, which states that for every property $\f{M.p}$, if $\f{M.p}$
is an inductive property then every tree satisfies it. A property is
said to be \emph{inductive} if a tree $\f{tree}\;\f{x}\;\f{f}$ satisfies it
whenever all its successors satisfy it.

\iflong
In the translation, see Appendix~\ref{sec:outp-induct-types} for a
complete output, dependencies at the level of types and terms disappear.
\else
In the translation dependencies at the level of types and terms disappear.
\fi
%
A branching type is determined by a pair of non-dependent types~$\f{s}$
and~$\f{t}$ but the per $\per_{\f{t}}$ depends on~$\values{\f{s}}$. The theory~$\f{W}$
turns into a signature for a functor receiving a branching type~$\f{B}$
and returning a type~$\f{w}$, and an operation $\f{tree}$ of type
$\f{B.s} \to (\f{B.t} \to \f{w}) \to \f{w}$.  One can use phase-splitting
to translate axiom
$\f{induction}$ into a specification of a polymorphic function
%
\begin{equation*}
  \f{induction:
  (B.s \to (B.t \to w) \to (B.t \to \poly{\alpha}) \to \poly{\alpha}) \to w \to \poly{\alpha}},
\end{equation*}
%
which is a form of recursion on well-founded trees. Instead of trying
to explain what $\f{induction}$ is supposed to do, we show a surprisingly simple,
hand-written implementation of W-types. The reader may enjoy figuring out how it works:
%
\sourcefile{wtype.ml}


\subsection{Axiom of choice}
\label{sec:axiom-choice}

RZ can help explain why a generally
accepted axiom is not constructively valid. Consider the Axiom of
Choice:
%
\begin{source}
Parameter a b : Set.
Parameter r : a \iTo b \iTo Prop.
Axiom ac: (\iForall x : a, \iExists y : b, r x y) \iTo
          (\iExists c : a \iTo b, \iForall x : a, r x (c x)).
\end{source}
%
The relevant part of the output is
%
\begin{source}
val ac : (a \iTo b * ty_r) \iTo (a \iTo b) * (a \iTo ty_r)
Assertion ac =
  \iForall f:a \iTo b * ty_r,
    (\iForall (x:\iT{a}),  let (p,q) = f x in p : \iT{b} \iAnd r x p q) \iTo
    let (g,h) = ac f in
      g : \iT{a \iTo b} \iAnd (\iForall (x:\iT{a}),  r x (g x) (h x))
\end{source}
%
This requires a function $\f{ac}$ which accepts a function $\f{f}$
and computes a pair of functions $\f{(g,h)}$. The input function~$\f{f}$ takes
an $\ototal{\f{x}}{\f{a}}$ and returns a pair $\f{(p,q)}$ such that $\f{q}$ realizes
the fact that $\f{r\;x\;p}$ holds. The output functions $\f{g}$ and $\f{h}$ taking
$\ototal{\f{x}}{\f{a}}$ as input must be such that $\f{h\;x}$ realizes
$\f{r\;x\;(g\;x)}$. Crucially, the requirement $\ototal{\f{g}}{\oarrow{\f{a}}{\f{b}}}$
says tht $\f{g}$ must be extensional, i.e., map equivalent realizers to
equivalent realizers. We could define~$\f{h}$ as the first component
of~$\f{f}$, but we cannot hope to implement~$\f{g}$ in general because the
second component of~$f$ is not assumed to be extensional.

The \emph{Intensional} Axiom of Choice allows the choice function to
depend on the realizers:
%
\begin{source}
Parameter a b : Set.
Parameter r : a \iTo b \iTo Prop.
Axiom iac: (\iForall x : a, \iExists y : b, r x y) \iTo
           (\iExists c : rz a \iTo b, \iForall x : rz a, r (rz x) (c x)).
\end{source}
%
Now the output is
%
\begin{source}
val iac : (a \iTo b * ty_r) \iTo (a \iTo b) * (a \iTo ty_r)
Assertion iac =
  \iForall f:a \iTo b * ty_r,
    (\iForall (x:\iT{a}),  let (p,q) = f x in p : \iT{b} \iAnd r x p q) \iTo
    let (g,h) = iac f in
      (\iForall x:a, x : \iT{a} \iTo g x : \iT{b}) \iAnd (\iForall (x:\iT{a}),  r x (g x) (h x))
\end{source}
%
which is exactly the same as before, \emph{except} that the
troublesome requirement $\ototal{\f{g}}{\oarrow{\f{a}}{\f{b}}}$ turned into
$\oforall{\f{x}}{\f{a}}{(\oimply{\ototal{\f{x}}{\f{a}}}{\ototal{\f{g\;x}}{\f{b}}})}$, which
is weaker. We can impement $\f{iac}$ as
%
\begin{source}
let iac f = (fun x -> fst (f x)), (fun x -> snd (f x))
\end{source}
%
The Intensional Axiom of Choice is in fact just an instance of the
usual Axiom of Choice applied to~$\irz{A}$ and~$B$. Combined with the
fact that~$\irz{A}$ covers~$A$, this establishes the validity of
\emph{Presentation Axiom}~\cite{barwise75:_admis_sets_struc}, which
states that every set is an image of one satisfying the axiom of
choice.

\subsection{Modulus of Continuity}
\label{sec:we-show-modulus-of-continuity-example}

As a last example we show how certain constructive principles require
the use of computational effects. To keep the example short, we
presume that we are already given the set of natural
numbers~$\f{nat}$ with the usual structure.

A \emph{type 2 functional} is a map $f : (\f{nat} \to \f{nat})
\to \f{nat}$. It is said to be continuous if the output of $f(a)$
depends only on an initial segment of the sequence~$a$. We can express
this axiom in RZ as follows:
%
\begin{source}
Axiom continuity: \iForall f : (nat \iTo nat) \iTo nat, \iForall a : nat \iTo nat,
  \iExists k, \iForall b : nat \iTo nat, (\iForall m, m \iLeq k \iTo a m = b m) \iTo f a = f b.
\end{source}
%
The axiom says that for any $\f{f}$ and $\f{a}$ there exists $\f{k} \in
\f{nat}$ such that $\f{f(b) = f(a)}$ as soon as the sequences~$\f{a}$
and~$\f{b}$ agree on the first $\f{k}$ terms. The axiom is translated to the
specification
%
\begin{source}
val continuity : ((nat \iTo nat) \iTo nat) \iTo (nat \iTo nat) \iTo nat
Assertion continuity =
  \iForall (f:\iT{(nat \iTo nat) \iTo nat}, a:\iT{nat \iTo nat}),
    let p = continuity f a in p : \iT{nat} \iAnd
    (\iForall (b:\iT{nat \iTo nat}),
       (\iForall (m:\iT{nat}),  m \iLeq p \iTo a m \iPer{nat} b m) \iTo f a \iPer{nat} f b)
\end{source}
%
which says that $\f{continuity\;f\;a}$ is a number~$\f{p}$ such that
$\f{f(a) = f(b)}$ whenever $\f{a}$ and $\f{b}$ agree on the first~$\f{p}$ terms. In
other words, $\f{continuity}$ is a \emph{modulus of continuity}
functional. It cannot be implemented in a purely functional
language,\footnote{There are models of $\lambda$-calculus which validate
  the choice principle~$AC_{2,0}$, but this principle contradicts the
  existence of a modulus of continuity functional,
  see~\cite[9.6.10]{Troelstra:van-Dalen:88:2}.} but with the use of
store we can implement it as
%
\begin{source}
let continuity f a =
  let p = ref 0 in
  let a' n = (p := max !p n; a n) in
    f a' ; !p
\end{source}
%
To compute a modulus for~$\f{f}$ at~$\f{a}$, the program creates a
function~$\f{a'}$ which is just like~$\f{a}$ except that it stores in~$\f{p}$ the
largest argument at which it has been called. Then $\f{f\;a'}$ is
computed, its value it discarded, and the value of~$\f{p}$ is returned.
The program works because~$\f{f}$ is assumed to be extensional and must
therefore not distinguish between extensionally equal sequences~$\f{a}$
and~$\f{a'}$.



%%% Local Variables: 
%%% mode: latex
%%% TeX-master: "cie"
%%% End: 

\section{Implementation}
\label{sec:implementation}

The RZ implementation consists of several sequential passes.

After the initial parsing, a \emph{type reconstruction} phase checks
that the input is well-typed (and checks for well-formedness to the
extent that it is easily decidable), and if successful produces an
annotated result with all variables explicitly tagged with types. The
type checking phase uses a system of dependent types, with limited
subtyping (implicit coercions) for sum types and subset types. The
details are fairly standard, so are omitted here. One non-obvious
consequence of the realizability translation, however, is that the
subset types $\isubset{\ix}{\iS}{\iand{\ipp_1(\ix)}{\ipp_2(\ix)}}$ and
$\isubset{\ix}{\iS}{\iand{\ipp_2(\ix)}{\ipp_1(\ix)}}$ are not
equal, but only isomorphic in general. An
explicit coercion is required to go from one type to the other,
because subset values are pairs containing realizers for
$\iand{\ipp_1(\ix)}{\ipp_2(\ix)}$ and
$\iand{\ipp_2(\ix)}{\ipp_1(\ix)}$, and these realizers have
potentially different types $|\ipp_1(\ix)|\mathtt{*}|\ipp_2(\ix)|$ and
$|\ipp_2(\ix)|\mathtt{*}|\ipp_1(\ix)|$ respectively.

Next the realizability translation is performed as described in
Section~\ref{sec:translation}, producing interface code. The
flexibility of the full input language (e.g., $n$-ary sum types and
dependent product types) makes the translation code fairly involved,
and so it is performed in a ``naive'' fashion whenever possible. The
immediate result of the translation is not easily readable.
 
\internal{Chris}{Uh oh...Kuratowski has been moved to a later section.
Should we move this section after
the examples?}

Thus, three more passes simplify the output before it is displayed to
the user. A \emph{thinning} pass removes all references to trivial
realizers produced by stable formulas. For example, direct translation
of the $\mathtt{free}$ axiom in the output for Kuratowski-finite sets
yields a value specification for $\mathtt{free}$ of type
%
\begin{equation*}
  (\mathtt{A.a} \to \mathtt{S.s}) \to 
  (\mathtt{fin} \to \mathtt{S.s}) * (\ounit * (\mathtt{A.a} \to
  \ounit) *
  (\mathtt{fin} \to \mathtt{fin} \to \ounit))
\end{equation*}
%
where $\ounit$ is the unit (terminal) type classifying the trivial
realizer. Thinning replaces this by the isomorphic type
%
\begin{equation*}
  (\mathtt{A.a} \to \mathtt{S.s}) \to \mathtt{fin} \to \mathtt{S.s}
\end{equation*}
%
and appropriately modifies references to $\mathtt{free}$ in the assertions to account for this change in type.

Next, an \emph{optimization} pass applies an ad-hoc collection of
basic logical and term simplifications in order to make the output more readable. 
Logical simplifications include applications of truth table rules
($\iand{\itrue}{\ip}$ becomes $\ip$), detection of syntactically
identical premises and conclusions
($\iimply{\ip_1}{\iand{\ip_1}{\ip_2}}$ becomes
$\iimply{\ip_1}{\ip_2}$), and optimization of other common patterns we have
seen arise
($\iforall{\ix}{\is}{\iimply{(\iequal{\ix}{\ie})}{\ipp(\ix)}}$ becomes
$\ipp(\ie)$). We do not attempt real theorem proving 
so some redundancy may remain, but in practice the optimization pass
can help significantly.

Finally, the user can specify whether two optional steps occur.
RZ can optionally performs a \emph{phase-splitting} pass~\cite{harper+:popl90}. 
This is an ML-specific transformation that replace certain
uses of parameterized modules (a heavyweight language construct) by
parameterized types and polymorphic values. The idea is that although
functors map modules containing types and terms to other modules containing types
and terms, constraints on the programming language ensure that output types
depend only on input types (and not input terms).  Thus, we can split each
functor into a mapping from input types to output types, and then a separate
(polymorphic) term mapping input types and terms to an output term.

For example (ignoring
assertions for simplicity) the entire module
\begin{source}
module Free : functor (S : Semilattice) ->
                    sig
                      val free : (A.a -> S.s) -> fin -> S.s
                    end	
\end{source}   
appearing in the output of the Kuratowski example can be replaced by the single polymorphic function
\begin{source}
val free : 's -> ('s -> 's -> 's) -> (A.a -> 's) -> fin -> 's	
\end{source}
which replaces the module parameter \texttt{S} by two extra term arguments term (corresponding to the module components \texttt{S.zero} and \texttt{S.join}) 
and a type argument \texttt{'s} for the type of lattice elements (corresponding to the module input \texttt{S.s}).

The other optional transformation is a \emph{hoisting} pass which lifts obligations in the output out to top-level positions.  This can make it easier to see exactly what one is obliged to provide.  When identical obligations appear in separate subterms of a term, hoisting can lift and merge these obligations, reducing redundancy.  However, moving obligations far from where they are used can make it harder to see why the obligation is required at all (and hence how one might satisfy the obligation), and so hoisting is turned off by default.

For example, in the following input (extracted from a larger description of an ordered field)
\begin{source}
Parameter s : Set.
Parameter zero : s.
Parameter inverse : {x : s | not (x = zero)} -> s.

Parameter lt : s -> s -> Stable.
Definition positive (x:s) := lt zero x.
Axiom lt_irr: forall x:s, not (lt x x).

Axiom order_inv: forall x:s, positive x -> positive (inverse x).
\end{source}
the axiom translates to the assertion:
\begin{source}
	(**  Assertion order_inv = 
          forall (x:||s||),  positive x ->
            positive (inv (assure (not (x =real= zero)) in x))
   *).
\end{source}
Here the system has noticed that for \Verb|inverse x| to make sense, we must know that
\Verb|x| is non-zero.   This requires non-trivial theorem proving and hence remains as 
an obligation for the user.  

We must prove \Verb|not (x =s= zero)| not for all \Verb|x|, but only under
the premises in force where the obligation occurs.  This is slightly clearer when hoisting moves the
obligation to the top level, after which it could be verified in the same way as all other assertions:
\begin{source}
   (**  Assertion order_inv = 
          assure (forall (x:||real||),  positive x -> not (x =real= zero))
            in forall (x:||real||),  positive x -> positive (inv x)
   *)	
\end{source}


%%% Local Variables: 
%%% mode: latex
%%% TeX-master: "cie"
%%% End: 

\section{Related Work}
\label{sec:related-work}

\subsection{Coq}
\label{sec:comparison-with-coq}

Coq is a very flexible system, with complete support for
theorem-proving and creating trusted code. One common pattern of use
for Coq is to write code in Coq's functional language (values whose
types are \texttt{Set} in Coq), to state and prove theorems stating
that the code behaves correctly (where the theorems are Coq values
whose types are \texttt{Prop} in Coq), and then have Coq produce
guaranteed correct code in ML. In such cases, RZ is complementary to
Coq. RZ can clarify the constructive content of mathematical
structures and hence suggest an appropriate division between Coq's
\texttt{Set} and \texttt{Prop}, i.e., which values and which theorems
should appear in the input to Coq. (It should be easy to have RZ
produce output in Coq syntax, and we hope to do this eventually.)

In general, RZ is a smaller and more lightweight system and thus more
flexible where it applies. It is not always practical or necessary to
do theorem proving in order to provide an implementation; interfaces
generated by RZ can be implemented in any manner from theorem proving
to directly writing code. And, RZ provides a way to talk with
programmers about implications for constructive without bringing in
full theorem proving.

\internal{Andrej}{I don't parse the previous sentence.}

\subsection{Other tools}

Komagata and Schmidt~\cite{komagata+:tr95} describe a system that uses
a similar realizability translation to ours. Like Coq, the system
translates formal proofs to programs, rather than

An interesting technical difference is that the algorithm they use,
attributed to John Hatcliff, does thinning as it goes along, rather
than making this a separate pass. For example, the translation of the
conjunction-introduction rule has four cases, depending on whether the
left and/or right propositions being proved are [almost?] negative, in
which case the trivial contribution can be immediately discarded.

\subsection{Other Models of Computability}
\label{sec:models-of-computability}

Most formulations of computable mathematics are based on realizability
models~\cite{Bauer:00}, even though they were not initially developed,
nor are they usually presented within the framework of realizability:
Recursive Mathematics~\cite{ershov98:_handb_recur_mathem} is based on
the original realizability by Turing machines~\cite{KleeneSC:intint};
Type Two Effectivity~\cite{Wei00} on function
realizability~\cite{KleeneSC:fouim} and relative function
realizability~\cite{BirkedalL:devttc}, while topological and domain
representations~\cite{Bla97a,Bauer:Birkedal:Scott:98} are based on
realizability over the graph model
$\mathcal{P}\omega$~\cite{ScottD:dattl}. A common feature of these is
that they use models of computation which are suitable for the
theoretical studies of computability, rather than for practical
programming. Since we want to emphasize the practical aspects of
realizability, we have chosen instead an actual real-world programming
language.


%%% Local Variables: 
%%% mode: latex
%%% TeX-master: "cie"
%%% End: 

\section{Conclusions and Future Work}
\label{sec:conclusion}

By translating only at the level of specifications, \emph{RZ} provides a
useful middle ground between ad-hoc implementations and
machine-generated implementations --- allowing much more flexible
implementation strategies, but relying on the programmer to
verify properties of their code.

Further, \emph{RZ} can serve as a means of explaining constructive
mathematics to programmers.  Programmers who are not knowledgeable
about constructive mathematics can still understand the output of the
translation, which involves familiar concepts such as abstract types
and (classical) first-order logic.   Looking at such examples can
provide the necessary intuition behind the original logic, and better
explain why one might want to work with constructive rather than
classical logic to begin with.

\bigskip
 
Axioms parameterized by models (e.g., initiality) currently translate
into signatures of ML functors.  We have experimented with an
alternative translation of such axioms into polymorphic types.  In
this case the \Verb|inital| axiom of the natural numbers yields
the specification
\begin{Verbatim}
   val initial: 
      'a -> ('a -> 'a) -> N.s -> 'a
\end{Verbatim}
which is exactly the familiar iterator for natural numbers (i.e.,
given an initial value, a function, and a natural number, apply the
function that many times to the initial value).  Such types can be
much more natural and much simpler for programmers to understand.  The
theory behind the translation is well understood, being the
phase-splitting translation of Harper, Mitchell, and Moggi
\cite{harper+:popl90}.  Because of limitations of ML not every
parameterized axiom can be turned into polymorphism; ML allows only
prenex quantifiers, and the quantifiers can range over types but not
type operators.  However we would like to do so where it is possible
(the common case).  As an alternative, we could attempt to retarget
the output to a language like Haskell~\cite{haskell} which supports
the necessary polymorphic types, though Haskell's support of modules
is much weaker.

Another possible extension would be to allow dependent families in the
input language. Fortunately, this does not require finding a target
language that supports dependent types; we can use the underlying
(non-dependent) types, and then express the dependencies as additional
properties that must be verified for the implementation.


%%% Local Variables: 
%%% mode: latex
%%% TeX-master: "case"
%%% End: 


%% Bibliography
% Bibliography
\bibliographystyle{alpha}

\bibliography{rzbib}


\end{document}

%%% Local Variables: 
%%% mode: latex
%%% TeX-master: t
%%% End: 
