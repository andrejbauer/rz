\section{The Input Language}
\label{sec:input-language}

\begin{figure}[htbp]
	\internal{CS}{code here}
	\caption{caption}
	\label{fig:typicalinput}
\end{figure}

Figure~\ref{fig:typicalinput} shows a sample input for the RZ system that
generates the output shown in \internal{CS}{somewhere}. The input is the
definition of one or more \emph{theories}. A \emph{theory} is a generalized
logical signature with associated axioms, similar to a Coq module signature.
The components of a theory may specify that a certain set must exist, that an
element of a set must exist, that a proposition or first-order predicate must
exist, a model of some other theory must exist, or specify an axiom that must
hold.

\internal{CS}{Actually, I should probably just follow the CASE approach of introducing the syntax in-line as each construct is mentioned.  This could save significant space for the short paper.}

As shown in in Figure~\ref{fig:typicalinput}, propositions and predicates
appearing in theories may use full first-order constructive logic, not just the
negative fragment. The grammar for logical inputs is shown in
Figure~\ref{fig:input}. Most of this should be familiar, including the use of
lambda abstraction are to define predicates.

The language of sets is rich, going well beyond the type systems of typical
programming languages. In addition to any base sets postulated in a theory, one
can construct dependent cartesian products and dependent function spaces. We
also supports disjoint unions (with labeled tags), quotient spaces (a set
modulo an equivalence relation), subsets (elements of a set satisfying a
predicate). RZ even permits explicit references to sets of realizers.

The term language then primarily follows the set structure by providing
introduction and elimination constructs. For product sets we have tuples and
projection, and for function spaces we have lambda abstractions and
applications. One can inject a term into a dependent union, or do case analyses
on the members of a union. We can refer to the equivalence class of a given
term, or pick a representative of a given equivalence class (as long as what we
do with it does not depend on the particular representative). Very similarly,
we can produce the set of realizers for a term, or choose a representative from
a given set of realizers (as long as what we do with it does not depend on the
particular representative). We can inject a term into a subset (if it satisfies
the appropriate predicate), or project an element of a subset into the
enclosing set. Finally, the term language allows local definitions of term
variables, and definite descriptions (as long as there is a unique element
satisfying the predicate in question).

From the parenthesized qualifiers, it is clear that checking the
well-formedness of terms is not decidable. RZ checks what it can, but does not
attempt any serious theorem proving. Uncheckable constraints remain as
explicitly-marked \emph{obligations} in the final output, and must be verified
by other means.






\newcommand{\ip}{p}
\newcommand{\ix}{x}
\newcommand{\is}{s}
\newcommand{\ie}{e}
\newcommand{\iP}{P}
\newcommand{\iS}{S}
\newcommand{\iprop}{\mathtt{prop}}
\newcommand{\itrue}{\mathtt{true}}
\newcommand{\ifalse}{\mathtt{false}}
\newcommand{\inot}[1]{\mathtt{not}\,#1}
\newcommand{\iimply}[2]{#1\,\mbox{\texttt{->}}\,#2}
\newcommand{\iiff}[2]{#1\,\mbox{\texttt{<->}}\,#2}
\newcommand{\iequal}[2]{#1\,\mathtt{=}\,#2}
\newcommand{\iands}[2]{#1\,\mathtt{/\backslash}\, #2}
\newcommand{\iors}[2]{#1\,\mathtt{\backslash/}\,#2}
\newcommand{\iforall}[3]{\texttt{forall}\,#1\,\texttt{:}\,#2\texttt{,}\ #3}
\newcommand{\iexists}[3]{\texttt{exists}\,#1\,\texttt{:}\,#2\texttt{,}\ #3}
\newcommand{\iunique}[3]{\texttt{unique}\,#1\,\texttt{:}\,#2\texttt{,}\ #3}
\newcommand{\ilambda}[3]{\texttt{lambda}\,#1\,\texttt{:}\,#2\texttt{,}\ #3}
\newcommand{\iapps}[3]{#1\ #2\,\cdots\,#3}

\newcommand{\ituples}[2]{\texttt{(}#1\texttt{,}\ldots\texttt{,}#2\texttt{)}}
\newcommand{\iproj}[2]{#1\texttt{.}#2}
\newcommand{\iconstrain}[2]{#1\,\texttt{:}\,#2}
\newcommand{\ithe}[3]{\texttt{the}\,#1\,\texttt{:}\,#2\texttt{,}\, #3}
\newcommand{\ilet}[3]{\texttt{let}\,#1\,\texttt{=}\,#2\,\texttt{in}\, #3}
\newcommand{\iquot}[2]{#1\,\texttt{\%}\,#2}
\newcommand{\iletquot}[4]{\texttt{let}\ #1\texttt{\%}\,#2\,\texttt{=}\,#3\ \texttt{in}\ #4}
\newcommand{\irz}[1]{\texttt{rz}\,#1}
\newcommand{\iletrz}[3]{\texttt{let}\ \texttt{rz}\,#1\,\texttt{=}\,#2\ \texttt{in}\ #3}

\newcommand{\iprods}[2]{#1\,\texttt{*}\cdots\texttt{*}\,#2}
\newcommand{\isums}[2]{#1\,\texttt{+}\cdots\texttt{+}\,#2}

\begin{figure}
	\[
	\begin{array}{rll}
		\ip ::= 
		    & \itrue \ | \ \ifalse\ |\ \inot{\ip}\ |\ \iands{\ip}{\ip}\ |\ 
		       \iors{\ip}{\ip}\ | \ \iimply{\ip}{\ip}\ |\ \iiff{\ip}{\ip}\\
		  | & \iP\\
		  | & \ilambda{\ix}{\is}{\ip} \ |\ \iapps{\ip}{\ie}{\ie}\\
		  | & \iequal{\ie}{\ie}\\
		  | & \iforall{\ix}{\is}{\ip} \ |\ 
		      \iexists{\ix}{\is}{\ip} \ |\
		      \iunique{\ix}{\is}{\ip}\\[10pt]
		
		\is ::= 
		    & ()\ |\ \iprods{\is}{\is}\\
		  | & \ilambda{\ix}{\is}{\is}\ |\ 
		      \iapps{\is}{\ie_1}{\ie_n}\\
		  | & \isums{\is}{\is}\\
		
		\\[10pt]
		\ie ::=
		    & \ix\\
		  | & \iapps{\ie}{\ie}{\ie}\\
		  | & \ilambda{\ix}{\is}{\ie}\\
		  | & \ituples{\ie}{\ie}\\
		  | & \iproj{\ie}{n}\\
		  | & \iconstrain{\ie}{\is}\\
		  | & \ithe{\ix}{\is}{\ip}\\
		  | & \ilet{\ix}{\ie}{\ie}\\
		  | & \iquot{\ie}{\ip}\\
		  | & \iletquot{\ix}{\ip}{\ie}{\ie}\\
		  | & \irz{\ie}\\
		  | & \iletrz{\ix}{\ie}{\ie}\\
	\end{array}
	\]
	\label{fig:input}
	\caption{Input Syntax}
\end{figure}



Explain the input language and its semantics.



\internal{AB}{The plan here seems to be as follows. Before this
  section we will say that assertions are written in the negative
  fragment, and that in the input language we use constructive logic,
  which turns out to be more convenient. So just go ahead and
  introduce constructive logic without worrying what its
  interpretation might be. Then in the translation secton we'll
  explain the realizability interpretation of constructive logic.
  Actually, the logic is first-order intuitionistic logic. The fact
  the the logic is intuitionistic is not a choice but reality: the
  interpretation does not validate the Law of Excluded Middle. This
  should be commented on.}


%%% Local Variables: 
%%% mode: latex
%%% TeX-master: "cie"
%%% End: 
%!TEX root = cie

