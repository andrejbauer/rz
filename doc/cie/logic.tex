\section{The Input Language}
\label{sec:input-language}

Explain the input language and its semantics.

\internal{AB}{The plan here seems to be as follows. Before this
  section we will say that assertions are written in the negative
  fragment, and that in the input language we use constructive logic,
  which turns out to be more convenient. So just go ahead and
  introduce constructive logic without worrying what its
  interpretation might be. Then in the translation secton we'll
  explain the realizability interpretation of constructive logic.
  Actually, the logic is first-order intuitionistic logic. The fact
  the the logic is intuitionistic is not a choice but reality: the
  interpretation does not validate the Law of Excluded Middle. This
  should be commented on.}


%%% Local Variables: 
%%% mode: latex
%%% TeX-master: "cie"
%%% End: 
