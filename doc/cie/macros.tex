%%%% MACROS %%%%%

% Private comments, should be left in the source when the paper is finished.
% Use this version for visible comments:
\newcommand{\comment}[1]{\paragraph{Comment:} #1}
% For invisible comments:
%\newcommand{\comment}[1]{}

% Internal notes, used during writing the paper. The first argument is
% the author of the note. When the paper is finished, these should
% all be gone. Delete an internal note as soon as you resolve it.
\newcommand{\internal}[2]{\paragraph{Note by #1:} #2}

% Types and terms in ocaml
\newcommand{\ctype}{\mathtt{type}\;}
\newcommand{\cint}{\mathtt{int}}
\newcommand{\clist}[1]{#1\;\mathtt{list}}
\newcommand{\ctrue}{\mathtt{true}}
\newcommand{\cwhile}[2]{\mathtt{while}\;#1\;\mathtt{do}\;#2\;\mathtt{done}}
\newcommand{\cstring}[1]{\mathtt{"#1"}}
\newcommand{\cprint}[1]{\mathtt{print\_string}\;#1}
\newcommand{\cfun}[2]{\mathtt{fun}\;#1 \to #2}
\newcommand{\clet}[2]{\mathtt{let}\;#1 = #2}
\newcommand{\cletrec}[2]{\mathtt{let}\;\mathtt{rec}\;#1 = #2}
\newcommand{\cfst}[1]{\mathtt{fst}\;#1}
\newcommand{\csnd}[1]{\mathtt{snd}\;#1}
\newcommand{\cmod}{\mathbin{\mathtt{mod}}}

% Realizaiblity
\newcommand{\per}{\approx}
\newcommand{\perty}[1]{\per_{\mathtt{#1}}}
\newcommand{\rz}{\Vdash}
\newcommand{\Type}{\mathsf{Type}}
\newcommand{\values}[1]{[\![#1]\!]}
\newcommand{\typeOf}[1]{|#1|}
\newcommand{\setOf}[1]{\langle #1 \rangle}
\newcommand{\support}[1]{\|#1\|}

% Sets
\newcommand{\set}[1]{\{#1\}}
\newcommand{\such}{\mid}
\newcommand{\pair}[1]{\langle #1 \rangle}

\newcommand{\family}[2]{\set{#1}_{#2}}
\newcommand{\depsum}[2]{{\Sigma_{#1} #2}}
\newcommand{\depprod}[2]{{\Pi_{#1} #2}}

% Common sets
\newcommand{\NN}{\mathbb{N}}
\newcommand{\ZZ}{\mathbb{Z}}
\newcommand{\RR}{\mathbb{R}}

% Theorem-like environments
\iffalse
{
  \theorembodyfont{\itshape}
  \newtheorem{theorem}{Theorem}[section]
  \newtheorem{lemma}[theorem]{Lemma}
  \newtheorem{proposition}[theorem]{Proposition}
  \newtheorem{corollary}[theorem]{Corollary}
}
{
  \theorembodyfont{\rmfamily}
  \newtheorem{definition}[theorem]{Definition}
  \newtheorem{example}[theorem]{Example}
}
\fi

% Quantifiers. I am really used to these...
\newcommand{\all}[3]{\forall\, #1 \,{\in}\, #2\,.\left(#3\right)}
\newcommand{\some}[3]{\exists\, #1 \,{\in}\, #2\,.\left(#3\right)}
\newcommand{\exactlyone}[3]{\exists!\, #1 \,{\in}\, #2\,.\left(#3\right)}
\newcommand{\xall}[3]{\forall\, #1 \,{\in}\, #2\,.\,#3}
\newcommand{\xsome}[3]{\exists\, #1 \,{\in}\, #2\,.\,#3}
\newcommand{\xexactlyone}[3]{\exists!\, #1 \,{\in}\, #2\,.\,#3}


%%%%%%%%%%%%%%%%%%%%%%%%%%%%%%%%%%%%
%% Macros for input & output language

% ASCII syntax
\iffalse
\newcommand{\ip}{p}
\newcommand{\ix}{x}
\newcommand{\is}{s}
\newcommand{\ie}{e}
\newcommand{\iP}{P}
\newcommand{\iS}{S}
\newcommand{\iprop}{\mathtt{prop}}
\newcommand{\itrue}{\mathtt{true}}
\newcommand{\ifalse}{\mathtt{false}}
\newcommand{\inot}[1]{\mathtt{not}\,#1}
\newcommand{\iimply}[2]{#1\,\mbox{\texttt{->}}\,#2}
\newcommand{\iiff}[2]{#1\,\mbox{\texttt{<->}}\,#2}
\newcommand{\iequal}[2]{#1\,\mathtt{=}\,#2}
\newcommand{\iands}[2]{#1\,\mathtt{/\backslash}\, #2}
\newcommand{\iors}[2]{#1\,\mathtt{\backslash/}\,#2}
\newcommand{\iforall}[3]{\texttt{forall}\,#1\,\texttt{:}\,#2\texttt{,}\ #3}
\newcommand{\iexists}[3]{\texttt{exists}\,#1\,\texttt{:}\,#2\texttt{,}\ #3}
\newcommand{\iunique}[3]{\texttt{unique}\,#1\,\texttt{:}\,#2\texttt{,}\ #3}
\newcommand{\ilambda}[3]{\texttt{lambda}\,#1\,\texttt{:}\,#2\texttt{,}\ #3}
\newcommand{\iapps}[3]{#1\ #2\,\cdots\,#3}

\newcommand{\ituples}[2]{\texttt{(}#1\texttt{,}\ldots\texttt{,}#2\texttt{)}}
\newcommand{\iproj}[2]{#1\texttt{.}#2}
\newcommand{\iconstrain}[2]{#1\,\texttt{:}\,#2}
\newcommand{\ithe}[3]{\texttt{the}\,#1\,\texttt{:}\,#2\texttt{,}\, #3}
\newcommand{\ilet}[3]{\texttt{let}\,#1\,\texttt{=}\,#2\,\texttt{in}\, #3}
\newcommand{\iquot}[2]{#1\,\texttt{\%}\,#2}
\newcommand{\iletquot}[4]{\texttt{let}\ #1\texttt{\%}\,#2\,\texttt{=}\,#3\ \texttt{in}\ #4}
\newcommand{\irz}[1]{\texttt{rz}\,#1}
\newcommand{\iletrz}[3]{\texttt{let}\ \texttt{rz}\,#1\,\texttt{=}\,#2\ \texttt{in}\ #3}

\newcommand{\iprods}[2]{#1\,\texttt{*}\cdots\texttt{*}\,#2}
\newcommand{\isums}[2]{#1\,\texttt{+}\cdots\texttt{+}\,#2}

% Output language
\newcommand{\ity}{t}
\newcommand{\iTy}{T}
\newcommand{\iper}[1]{\mathbin{\mathtt{=}{#1}\mathtt{=}}}
\newcommand{\iunit}{\mathtt{unit}}
\fi

% Spiffed up syntax using mathematical symbols
\newcommand{\ip}{\varphi}
\newcommand{\ipp}{\rho}
\newcommand{\ix}{x}
\newcommand{\is}{s}
\newcommand{\ie}{e}
\newcommand{\iP}{p}
\newcommand{\iS}{\alpha}
\newcommand{\itrue}{\top}
\newcommand{\ifalse}{\bot}
\newcommand{\inot}[1]{\lnot #1}
\newcommand{\iimply}[2]{#1\Rightarrow #2}
\newcommand{\iiff}[2]{#1 \Leftrightarrow #2}
\newcommand{\iequal}[2]{#1 = #2}
\newcommand{\iand}[2]{#1\land #2}
\newcommand{\ior}[2]{#1\lor #2}
\newcommand{\iforall}[3]{\forall #1{:}#2.\ #3}
\newcommand{\iexists}[3]{\exists #1{:}#2.\ #3}
\newcommand{\iunique}[3]{\exists! #1{:}#2.\ #3}
\newcommand{\ilambda}[3]{\lambda #1{:}#2.\ #3}
\newcommand{\iapp}[2]{#1\ #2}

\newcommand{\ituple}[2]{(#1,\ldots,#2)}
\newcommand{\iproj}[2]{\mathop{\pi_{#2}} #1}
\newcommand{\iconstrain}[2]{#1:#2}
\newcommand{\ithe}[3]{\iota\,#1{:}#2{.}\, #3}
\newcommand{\ilet}[3]{\textsf{let}\,#1{=}#2\,\textsf{in}\, #3}
\newcommand{\ieclass}[2]{[#1]_{#2}}
\newcommand{\ileteclass}[4]{\textsf{let}\ [#1]_{#2}\,{=}#3\ \textsf{in}\ #4}
\newcommand{\irz}[1]{\textsf{rz}\,#1}
\newcommand{\iletrz}[3]{\textsf{let}\ \textsf{rz}\,#1\,{=}\,#2\ \textsf{in}\ #3}
\newcommand{\iinj}[2]{#1\,#2}
\newcommand{\imatches}[7]{\textsf{match}\,#1\,\textsf{with}\,\iinj{#2}{#3}\Rightarrow{#4}\,|\cdots|\iinj{#5}{#6}\Rightarrow{#7}}
\newcommand{\imatch}[7]{\textsf{match}\,#1\,\textsf{with}\,\iinj{#2}{#3}\Rightarrow{#4}\,|\,\iinj{#5}{#6}\Rightarrow{#7}}
\newcommand{\iselect}[2]{#1{.}#2}

\newcommand{\itag}{\mathit{tag}}
\newcommand{\iprod}[3]{#1{:}#2\times#3}
\newcommand{\isum}[4]{#1{:}#2+#3{:}#4}
\newcommand{\iquot}[2]{#1{\bigm/} #2}
\newcommand{\isubset}[3]{\{\,#1{:}#2\ |\ #3\,\}}

\newcommand{\iDefinition}[2]{\mathsf{Definition}\ #1\ :=\ #2.}
\newcommand{\iParameter}[2]{\mathsf{Parameter}\ #1\ : #2.}
\newcommand{\iAxiom}[2]{\mathsf{Axiom}\ #1\ : #2.}
\newcommand{\ipt}{\tau}
\newcommand{\ik}{\kappa}
\newcommand{\idarrow}[3]{(#1{:}#2){\to}#3}
\newcommand{\ite}{\theta}
\newcommand{\ith}{\Theta}
\newcommand{\iTH}{T}
\newcommand{\ithy}[1]{\mathsf{thy}\ #1\ \mathsf{end}}
\newcommand{\im}{M}
\newcommand{\iM}{m}
\newcommand{\il}{l}
\newcommand{\iSet}{\mathsf{Set}}
\newcommand{\iStable}{\mathsf{Stable}}
\newcommand{\iProp}{\mathsf{Prop}}
\newcommand{\iEquiv}[1]{\mathsf{Equiv}(#1)}
\newcommand{\ibar}{\ |\ }

\newcommand{\ity}{\tau}
\newcommand{\iTy}{T}
\newcommand{\iper}[1]{\approx_{#1}}
\newcommand{\iunit}{\mathsf{unit}}
\newcommand{\isumty}[4]{[`#1\,\mathsf{of}\,#2 \mid `#3\,\mathsf{of}\,#4]}
\newcommand{\iinjty}[2]{`#1\,#2}
\newcommand{\imatchesty}[7]{\textsf{match}\,#1\,\textsf{with}\,\iinjty{#2}{#3}\to{#4}\,|\cdots|\iinjty{#5}{#6}\to{#7}}
\newcommand{\ioblig}[3]{\textsf{assure}\,#1{:}#2\,\mathsf{in}\,#3}
\newcommand{\isig}[1]{\textsf{sig}\,#1\,\mathsf{end}}

%%% Local Variables: 
%%% mode: latex
%%% TeX-master: "cie"
%%% End: 
