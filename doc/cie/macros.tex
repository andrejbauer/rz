%%%% MACROS %%%%%

% Private comments, should be left in the source when the paper is finished.
% Use this version for visible comments:
\newcommand{\comment}[1]{\paragraph{Comment:} #1}
% For invisible comments:
%\newcommand{\comment}[1]{}

% Internal notes, used during writing the paper. The first argument is
% the author of the note. When the paper is finished, these should
% all be gone. Delete an internal note as soon as you resolve it.
\newcommand{\internal}[2]{\par\fbox{\parbox{0.9\textwidth}{{Note by #1:} #2}}\par}

% Types and terms in ocaml
\newcommand{\ctype}{\mathtt{type}\;}
\newcommand{\cint}{\mathtt{int}}
\newcommand{\clist}[1]{#1\;\mathtt{list}}
\newcommand{\ctrue}{\mathtt{true}}
\newcommand{\cbool}{\mathtt{bool}}
\newcommand{\cwhile}[2]{\mathtt{while}\;#1\;\mathtt{do}\;#2\;\mathtt{done}}
\newcommand{\cstring}[1]{\mathtt{"#1"}}
\newcommand{\cprint}[1]{\mathtt{print\_string}\;#1}
\newcommand{\cfun}[2]{\mathtt{fun}\;#1 \to #2}
\newcommand{\clet}[2]{\mathtt{let}\;#1 = #2}
\newcommand{\cletrec}[2]{\mathtt{let}\;\mathtt{rec}\;#1 = #2}
\newcommand{\cfst}[1]{\mathtt{fst}\;#1}
\newcommand{\csnd}[1]{\mathtt{snd}\;#1}
\newcommand{\cmod}{\mathbin{\mathtt{mod}}}
\newcommand{\poly}[1]{\text{\texttt{\char39}}#1}

% Realizaiblity
\newcommand{\per}{\approx}
\newcommand{\perty}[1]{\per_{\mathtt{#1}}}
\newcommand{\rz}{\Vdash}
\newcommand{\Type}{\mathsf{Type}}
\newcommand{\values}[1]{[\![#1]\!]}
\newcommand{\typeOf}[1]{|#1|}
\newcommand{\setOf}[1]{\langle #1 \rangle}
\newcommand{\support}[1]{\|#1\|}

% Sets
\newcommand{\set}[1]{\{#1\}}
\newcommand{\such}{\mid}
\newcommand{\pair}[1]{\langle #1 \rangle}
\newcommand{\epito}{\twoheadrightarrow}
\newcommand{\family}[2]{\set{#1}_{#2}}
\newcommand{\depsum}[2]{{\Sigma_{#1} #2}}
\newcommand{\depprod}[2]{{\Pi_{#1} #2}}

% Common sets
\newcommand{\NN}{\mathbb{N}}
\newcommand{\ZZ}{\mathbb{Z}}
\newcommand{\RR}{\mathbb{R}}

% Theorem-like environments
\iffalse
{
  \theorembodyfont{\itshape}
  \newtheorem{theorem}{Theorem}[section]
  \newtheorem{lemma}[theorem]{Lemma}
  \newtheorem{proposition}[theorem]{Proposition}
  \newtheorem{corollary}[theorem]{Corollary}
}
{
  \theorembodyfont{\rmfamily}
  \newtheorem{definition}[theorem]{Definition}
  \newtheorem{example}[theorem]{Example}
}
\fi

% Quantifiers. I am really used to these...
\newcommand{\all}[3]{\forall\, #1 \,{\in}\, #2\,.\left(#3\right)}
\newcommand{\some}[3]{\exists\, #1 \,{\in}\, #2\,.\left(#3\right)}
\newcommand{\exactlyone}[3]{\exists!\, #1 \,{\in}\, #2\,.\left(#3\right)}
\newcommand{\xall}[3]{\forall\, #1 \,{\in}\, #2\,.\,#3}
\newcommand{\xsome}[3]{\exists\, #1 \,{\in}\, #2\,.\,#3}
\newcommand{\xexactlyone}[3]{\exists!\, #1 \,{\in}\, #2\,.\,#3}


%%%%%%%%%%%%%%%%%%%%%%%%%%%%%%%%%%%%
%% Macros for input & output language

\newcommand{\f}[1]{\mathtt{#1}}

% ASCII syntax
\iffalse
\newcommand{\ip}{p}
\newcommand{\ix}{x}
\newcommand{\is}{s}
\newcommand{\ie}{e}
\newcommand{\iP}{P}
\newcommand{\iS}{S}
\newcommand{\iprop}{\mathtt{prop}}
\newcommand{\itrue}{\mathtt{true}}
\newcommand{\ifalse}{\mathtt{false}}
\newcommand{\inot}[1]{\mathtt{not}\,#1}
\newcommand{\iimply}[2]{#1\,\mbox{\texttt{->}}\,#2}
\newcommand{\iiff}[2]{#1\,\mbox{\texttt{<->}}\,#2}
\newcommand{\iequal}[2]{#1\,\mathtt{=}\,#2}
\newcommand{\iands}[2]{#1\,\mathtt{/\backslash}\, #2}
\newcommand{\iors}[2]{#1\,\mathtt{\backslash/}\,#2}
\newcommand{\iforall}[3]{\texttt{forall}\,#1\,\texttt{:}\,#2\texttt{,}\ #3}
\newcommand{\iexists}[3]{\texttt{exists}\,#1\,\texttt{:}\,#2\texttt{,}\ #3}
\newcommand{\iunique}[3]{\texttt{unique}\,#1\,\texttt{:}\,#2\texttt{,}\ #3}
\newcommand{\ilambda}[3]{\texttt{lambda}\,#1\,\texttt{:}\,#2\texttt{,}\ #3}
\newcommand{\iapps}[3]{#1\ #2\,\cdots\,#3}

\newcommand{\ituples}[2]{\texttt{(}#1\texttt{,}\ldots\texttt{,}#2\texttt{)}}
\newcommand{\iproj}[2]{#1\texttt{.}#2}
\newcommand{\isubin}[2]{#1\,\texttt{:>}\,#2}
\newcommand{\isubout}[2]{#1\,\texttt{:<}\,#2}
\newcommand{\ithe}[3]{\texttt{the}\,#1\,\texttt{:}\,#2\texttt{,}\, #3}
\newcommand{\ilet}[3]{\texttt{let}\,#1\,\texttt{=}\,#2\,\texttt{in}\, #3}
\newcommand{\iquot}[2]{#1\,\texttt{\%}\,#2}
\newcommand{\iletquot}[4]{\texttt{let}\ #1\texttt{\%}\,#2\,\texttt{=}\,#3\ \texttt{in}\ #4}
\newcommand{\irz}[1]{\texttt{rz}\,#1}
\newcommand{\iletrz}[3]{\texttt{let}\ \texttt{rz}\,#1\,\texttt{=}\,#2\ \texttt{in}\ #3}

\newcommand{\iprods}[2]{#1\,\texttt{*}\cdots\texttt{*}\,#2}
\newcommand{\isums}[2]{#1\,\texttt{+}\cdots\texttt{+}\,#2}

\fi

% Spiffed up syntax using mathematical symbols
\newcommand{\ip}{\varphi}
\newcommand{\ipp}{\rho}
\newcommand{\ix}{x}
\newcommand{\is}{s}
\newcommand{\ie}{e}
\newcommand{\iP}{p}
\newcommand{\iS}{\alpha}
\newcommand{\itrue}{\top}
\newcommand{\ifalse}{\bot}
\newcommand{\inot}[1]{\lnot #1}
\newcommand{\iimply}[2]{#1\Rightarrow #2}
\newcommand{\iiff}[2]{#1 \Leftrightarrow #2}
\newcommand{\iequal}[2]{#1 = #2}
\newcommand{\iand}[2]{#1\land #2}
\newcommand{\ior}[2]{#1\lor #2}
\newcommand{\iforall}[3]{\forall #1{:}#2.\ #3}
\newcommand{\iexists}[3]{\exists #1{:}#2.\ #3}
\newcommand{\iunique}[3]{\exists! #1{:}#2.\ #3}
\newcommand{\ilambda}[3]{\lambda #1{:}#2.\ #3}
\newcommand{\iapp}[2]{#1\ #2}
\newcommand{\ipcases}[7]{\texttt{match}\,#1\,\texttt{with}\,\iinj{#2}{#3}{\Rightarrow}{#4}\,|\cdots|\iinj{#5}{#6}{\Rightarrow}{#7}}
\newcommand{\ipcase}[7]{\texttt{match}\,#1\,\texttt{with}\,\iinj{#2}{#3}{\Rightarrow}{#4}\,|\,\iinj{#5}{#6}{\Rightarrow}{#7}}

\newcommand{\ituple}[2]{(#1,\ldots,#2)}
\newcommand{\iproj}[2]{\mathop{\pi_{#2}} #1}
\newcommand{\isubin}[2]{#1\,{{:}{>}}\,#2}
\newcommand{\isubout}[2]{#1\,{{:}{<}}\,#2}
\newcommand{\icoerce}[2]{#1\,{:}\,#2}
\newcommand{\ithe}[3]{\iota\,#1{:}#2{.}\, #3}
\newcommand{\ilet}[3]{\texttt{let}\,#1{=}#2\,\texttt{in}\, #3}
\newcommand{\ieclass}[2]{[#1]_{#2}}
\newcommand{\ileteclass}[4]{\texttt{let}\ [#1]_{#2}\,{=}#3\ \texttt{in}\ #4}
\newcommand{\irz}[1]{\texttt{rz}\,#1}
\newcommand{\iletrz}[3]{\texttt{let}\ \texttt{rz}\,#1\,{=}\,#2\ \texttt{in}\ #3}
\newcommand{\iinj}[2]{{#1}\,#2}
\newcommand{\imatches}[7]{\texttt{match}\,#1\,\texttt{with}\,\iinj{#2}{#3}{\Rightarrow}{#4}\,|\cdots|\iinj{#5}{#6}{\Rightarrow}{#7}}
\newcommand{\imatch}[7]{\texttt{match}\,#1\,\texttt{with}\,\iinj{#2}{#3}{\Rightarrow}{#4}\,|\,\iinj{#5}{#6}{\Rightarrow}{#7}}
\newcommand{\iselect}[2]{#1{.}#2}

\newcommand{\itag}{\mathit{tag}}
\newcommand{\iprod}[3]{[#1{:}#2]\times#3}
\newcommand{\isum}[4]{#1{:}#2+#3{:}#4}
\newcommand{\iquot}[2]{#1{\bigm/} #2}
\newcommand{\isubset}[3]{\{#1{:}#2\ |\ #3\}}
\newcommand{\isunit}{1}
\newcommand{\isvoid}{0}

% I removed periods because I use these in the middle of text.
\newcommand{\iDefinition}[2]{\ensuremath{\mathtt{Definition}\ #1 \mathbin{{:}{=}} #2}}
\newcommand{\iParameter}[2]{\ensuremath{\mathtt{Parameter}\ #1 : #2}}
\newcommand{\iAxiom}[2]{\ensuremath{\mathtt{Axiom}\ #1 : #2}}
\newcommand{\iInclude}[1]{\ensuremath{\mathtt{Include}\ #1}}
\newcommand{\ipt}{\Pi}
\newcommand{\ik}{\kappa}
\newcommand{\iarrow}[2]{#1{\to}#2}
\newcommand{\idarrow}[3]{[#1{:}#2]{\to}#3}
\newcommand{\ite}{\theta}
\newcommand{\ith}{\Theta}
\newcommand{\iTH}{T}
\newcommand{\ithy}[1]{\mathtt{thy}\ #1\ \mathtt{end}}
\newcommand{\im}{M}
\newcommand{\iM}{m}
\newcommand{\il}{l}
\newcommand{\iSet}{\mathtt{Set}}
\newcommand{\iStable}{\mathtt{Stable}}
\newcommand{\iProp}{\mathtt{Prop}}
\newcommand{\iEquiv}[1]{\mathtt{Equiv}(#1)}
\newcommand{\ibar}{\ |\ }

%%%%%%%%%%%%%%%%%%%%%%%%%%%%%%%%%%%%%%%%%%%%%%%%%%%%%%%%%%%%%%%%%%%%%%
%  Output-specific syntax (did not bother to produce ASCII version yet)
\newcommand{\obar}{\ |\ }
%%%% types
\newcommand{\oty}{\tau} % type
\newcommand{\opty}{\alpha} % polymorphic type
\newcommand{\oTY}{T} % type name
\newcommand{\oprod}[2]{#1 \times #2} % product
\newcommand{\oarrow}[2]{#1 \to #2} % arrow
\newcommand{\ounit}{\mathtt{unit}} % unit type
\newcommand{\oselect}[2]{#1{.}#2}  % projection from a module
\newcommand{\osumty}[4]{`#1\,\mathtt{of}\,#2 + \cdots + `#3\,\mathtt{of}\,#4}
\newcommand{\osumtyx}[4]{`#1\,\mathtt{of}\,#2 + `#3\,\mathtt{of}\,#4}
\newcommand{\ol}{l} % label
%%%% terms
\renewcommand{\oe}{e} % expression
\newcommand{\ox}{x} % variable
\newcommand{\olambda}[3]{\mathtt{fun}\,#1{:}#2 \to #3} % abstraction
\newcommand{\oapp}[2]{#1\,#2} % application
\newcommand{\otuple}[2]{(#1,\ldots,#2)} % tuple
\newcommand{\oproj}[2]{\mathtt{p}_{#2}\,#1} % projection
\newcommand{\oinj}[2]{`#1\,#2} % sum injection
\newcommand{\oinjx}[1]{`#1} % sum injection
\newcommand{\omatch}[7]{\mathtt{match}\;#1\;\mathtt{with}\,\oinj{#2}{#3}\to{#4}\obar\oinj{#5}{#6}\to{#7}}
\newcommand{\omatchx}[5]{\mathtt{match}\;#1\;\mathtt{with}\,\oinjx{#2}\to{#3}\obar\oinjx{#4}\to{#5}}
\newcommand{\omatches}[7]{\mathtt{match}\;#1\;\mathtt{with}\,\oinj{#2}{#3}\to{#4}\,|\cdots\,|\oinj{#5}{#6}\to{#7}}
\newcommand{\ooblig}[4]{\mathtt{assure}\,#1{:}#2.\,#3\,\mathtt{in}\,#4}
\newcommand{\oobligx}[2]{\mathtt{assure}\,#1\,\mathtt{in}\,#2}
\newcommand{\olet}[3]{\mathtt{let}\,#1{=}#2\,\mathtt{in}\,#3}
%%%% propositions
\newcommand{\op}{p} % proposition
\newcommand{\oP}{P} % atomic proposition
\newcommand{\otrue}{\top} % true
\newcommand{\ofalse}{\bot} % false
\newcommand{\onot}[1]{\lnot #1} % not
\newcommand{\oand}[2]{#1 \land #2} % and
\newcommand{\oimply}[2]{#1 \Rightarrow #2} % imply
\newcommand{\oiff}[2]{#1 \Leftrightarrow #2} % iff
\newcommand{\opcase}[7]{\mathtt{match}\;#1\;\mathtt{with}\,\oinj{#2}{#3}{\Rightarrow}{#4}\,|\,\oinj{#5}{#6}\Rightarrow{#7}}
\newcommand{\opcasex}[5]{\mathtt{match}\;#1\;\mathtt{with}\,\oinjx{#2}{\Rightarrow}{#3}\,|\,\oinjx{#4}\Rightarrow{#5}}
\newcommand{\opcases}[7]{\mathtt{match}\;#1\;\mathtt{with}\,\oinj{#2}{#3}\Rightarrow{#4}\,|\cdots\,|\oinj{#5}{#6}\Rightarrow{#7}}
\newcommand{\ototal}[2]{#1{:}\support{#2}} % support membership
\newcommand{\oper}[3]{#2\,{\approx_{#1}}\,#3} % per
\newcommand{\oequal}[2]{#1{=}#2} % observational equality
\newcommand{\oforall}[3]{\forall #1{:}#2.\,#3}
\newcommand{\oforallt}[3]{\forall #1{:}\support{#2}.\,#3}
%%% Modest sets
\newcommand{\oS}{S}
\newcommand{\os}{s}
%%% Modules
\newcommand{\om}{m} % module
\newcommand{\oM}{M} % module name
%%% Proposition kinds
\newcommand{\opt}{\Pi}
\newcommand{\oProp}{\mathtt{bool}}
%%% Module type elements
\newcommand{\ote}{\theta}
\newcommand{\ovalspec}[2]{\mathtt{val}\;#1:#2}
\newcommand{\otyspec}[1]{\mathtt{type}\;#1}
\newcommand{\otydef}[2]{\mathtt{type}\;#1=#2}
\newcommand{\omodulespec}[2]{\mathtt{module}\;#1:#2}
\newcommand{\osignatdef}[2]{\mathtt{module}\;\mathtt{type}\,#1=#2}
\newcommand{\opropspec}[2]{\mathtt{predicate}\; #1:#2}
\newcommand{\oassertion}[2]{\mathtt{assertion}\; #1 : #2}
\newcommand{\oA}{A} % Assertion name
%%% Module type
\newcommand{\omt}{\Sigma}
\newcommand{\oMT}{S}
\newcommand{\osig}[1]{\mathtt{sig}\,#1\,\mathtt{end}}
\newcommand{\ofunctor}[3]{\mathtt{functor}\, (#1{:}#2) \to #3}

%%% Local Variables: 
%%% mode: latex
%%% TeX-master: "cie"
%%% End: 
