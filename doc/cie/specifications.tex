\section{Specifications as signatures with assertions}
\label{sec:spec-sign-assert}


In programming we distinguish between \emph{implementation} and
\emph{specification} of a structure. In OCaml these two notions are
expressed with modules and module types, respectively.\footnote{In
  object-oriented languages implementations and specifications are
  expressed with classes and interfaces, while in Haskell they
  correspond to modules and declarations.} A module defines types and
values, while a module type simply lists the types, type definitions,
and values provided by a module. For a complete specification, a
module type must also be annotated with \emph{assertions} which
specify the required properties of declared types and values.
%
\begin{figure}
  \centering
  \sourcefile{group.ml}
  \caption{The module type $\mathtt{Ab}$ and module
    $\mathtt{Z7}$.}
  \label{fig:module-example}
\end{figure}
%
For example, if we look at the definition of module type $\mathtt{Ab}$
and module~$\mathtt{Z7}$ in Figure~\ref{fig:module-example}, we can
probably guess that $\mathtt{Z7}$ implements the group~$\ZZ_7$ and
that $\mathtt{Ab}$ is a signature for an Abelian group. However, for a
complete description of~$\mathtt{Z7}$ we still need to define a
per~$\perty{t}$ on $\values{\mathtt{t}}$ that tells us when two
integers represent the same element of~$\ZZ_7$, e.g.,
%
\begin{equation*}
  u \perty{t} v \iff
  u \mathbin{\mathrm{mod}} 7 = v \mathbin{\mathrm{mod}} 7 \;.
\end{equation*}
%
Similiarly, $\mathtt{Ab}$ by itself does not describe an Abelian
group, only its signature. A complete description would contain the
following further \emph{assertions}:
%
\begin{enumerate}
\item
  \label{enum:t-per}%
  there is a per $\perty{t}$ on $\values{\mathtt{t}}$,
\item
  \label{enum:zero-total}%
  $\mathtt{zero} \in \support{\mathtt{t}}$.
\item
  \label{enum:neg-total}%
  for all $u, v \in \values{\mathtt{t}}$, if $u \perty{t} v$ then
  $\mathtt{neg} \; u \perty{t} \mathtt{neg} \; v$,
\item
  \label{enum:add-total}%
  for all $u_1, u_2, v_1, v_2 \in \values{\mathtt{t}}$, if $u_1
  \perty{t} v_1$ and $u_2 \perty{t} v_2$ then $\mathtt{add} \; (u_1,
  u_2) \perty{t} \mathtt{add} \; (v_1, v_2)$,
\item 
  \label{enum:ab-group-axiom-1}%
  for all $u \in \support{\mathtt{t}}$, $\mathtt{add} \;
  (\mathtt{zero}, u) \perty{t} u$,
\item
  \label{enum:ab-group-axiom-2}%
  for all $u \in \support{\mathtt{t}}$, $\mathtt{add} \; (u,
  \mathtt{neg} \; u) \perty{t} \mathtt{zero}$,
\item
  \label{enum:ab-group-axiom-3}%
  for all $u, v, w \in \support{\mathtt{t}}$, $\mathtt{add} \;
  (\mathtt{add} \; (u, v), w) \perty{t} \mathtt{add} \; (u,
  \mathtt{add} \; (v, w))$,
\item
  \label{enum:ab-group-axiom-4}%
  for all $u, v \in \support{\mathtt{t}}$, $\mathtt{add} \; (u, v)
  \perty{t} \mathtt{add} (v, u)$.
\end{enumerate}
%
Assertions \ref{enum:zero-total}--\ref{enum:add-total} state that
$\mathtt{zero}$, $\mathtt{neg}$, and $\mathtt{add}$ realize a
constant, a unary, and a binary operation, respectively, while
assertions \ref{enum:ab-group-axiom-1}--\ref{enum:ab-group-axiom-4}
correspond to axioms for Abelian groups. RZ generates complete
specifications (module types with assertions), such as the above
module type~$\mathtt{Ab}$ with assertions
\ref{enum:t-per}--~\ref{enum:ab-group-axiom-4}. The output is written
in the language of specifications summarized in
Figure~\ref{fig:output}.

\begin{figure}
  \[
  \begin{array}{rl@{\qquad}l}
    \noalign{\textbf{Types}}
    \oty ::= 
    & \oTY  \obar \oselect{\oM}{\oTY} &\mbox{Type names}\\
    | & \ounit \obar \oprod{\oty_1}{\oty_2} &\mbox{Unit and cartesian product}\\
    | & \oarrow{\oty_1}{\oty_2} & \mbox{Function type}\\
    | & \osumty{\ol_1}{\oty_1}{\ol_n}{\oty_n} &\mbox{Disjoint sum}\\
    | & \opty & \mbox{Polymorphic types}\\[5pt]
    
    \noalign{\textbf{Terms}}	
    \oe ::=
    & \ox \obar \oselect{\oM}{\ox} &\mbox{Term names}\\
    | & \olambda{\ox}{\oty_1}{\oe} \obar 
    \oapp{\oe_1}{\oe_2} &\mbox{Functions and application}\\
    | & ()
    \obar \otuple{\oe_1}{\oe_n} 
    \obar \oproj{\oe}{n}&\mbox{Tuples and projection}\\
    | & \oinj{\ol}{\oe} 
    \obar (\omatches{\oe_0}{\ol_1}{\ox_1}{\oe_1}{\ol_n}{\ox_n}{\oe_n})&\mbox{Injection and projection from a sum}\\
    | & \ooblig{\ox}{\oty}{\op}{\oe} \obar \oobligx{\op}{\oe} &\mbox{Obligations}\\
    | & \olet{\ox}{\oe_1}{\oe_2}&\mbox{Local definitions}\\[5pt]
    

    \noalign{\textbf{Propositions (negative fragment)}}
    \op ::= 
    & \oP \obar \oselect{\oM}{\oP}&\mbox{Atomic
      proposition}\\
    | & \otrue  \obar \ofalse \obar \onot{\op_1} \obar \oand{\op_1}{\op_2} \obar 
    \oimply{\op_1}{\op_2} \obar \oiff{\op_1}{\op_2} & \mbox{Predicate logic}\\
    | & \olambda{\ox}{\oty}{\op}  \obar \oapp{\op}{\oe} &
    \mbox{Propositional functions and application}\\
    | & \oper{\os}{\oe_1}{\oe_2} \obar \ototal{\oe}{\os} & \mbox{Pers and support} \\
    | & \oequal{\oe_1}{\oe_2} & \mbox{(Observational) term equality}\\
    | & \oforall{\ox}{\oty}{\op}  \obar 
    \oforallt{\ox}{\os}{\op} & \mbox{Term quantifiers}\\[5pt]
    
    \noalign{\textbf{Basic modest sets}}
    \os ::=
    & \oS \obar \oapp{\os}{\oe} & \\[5pt]

    \noalign{\textbf{Modules}}		
    \om ::= 
    & \oM  \obar \oselect{\om}{\oM}&\mbox{Model names}\\
    | & \oapp{\om_1}{\om_2}&\mbox{Application of parameterized model}\\[5pt]
    
    \noalign{\textbf{Proposition Kinds}}
    \opt ::=
    & \oProp & \mbox{Classifier for propositions}\\
    |& \oarrow{\oty}{\opt} & \mbox{Classifier for a predicate/relation}\\[5pt] 
    
    \noalign{\textbf{Specification elements}}
    \ote ::=
    & \ovalspec{\ox}{\oty}&\mbox{Value declaration}\\
    | & \otyspec{\oTY}&\mbox{Type declaration}\\
    | & \otydef{\oTY}{\oty}&\mbox{Type definition}\\
    | & \omodulespec{\om}{\omt}&\mbox{Module declaration}\\
    | & \osignatdef{\oMT}{\omt}&\mbox{Specification definition}\\
    | & \opropspec{\oP}{\opt}&\mbox{Proposition declaration}\\
    | & \oassertion{\oA}{\op}&\mbox{Assertion}\\[5pt]

    \noalign{\textbf{Specifications (module types with assertions)}}
    \omt ::= 
    & \oMT \obar \oselect{\om}{\oMT} & \mbox{Specification names}\\
    | & \osig{\ote_1 \ldots \ote_n} & \mbox{Specification elements}\\
    | & \ofunctor{\om}{\omt_1}{\omt_2} & \mbox{Parametrized specification}\\
    | & \oapp{\omt}{\om} & \mbox{Specification application}
  \end{array}
  \]
  \label{fig:output}
  \caption{The syntax of specifications (Simplified)}
\end{figure}

RZ ever produces only a small subset of OCaml types (the unit type,
products, function types, polymorphic variant types, and parameter
types). Correspondingly, the language of terms produced is fairly simple
(tuples, functions, polymorphic variants, and local definitions).
However, the programmer is free to implement a specification using any
types and terms that exist in OCaml. 

A special kind of term is an
\emph{obligation} $\ooblig{\ox}{\oty}{\op}{\oe}$ which means ``in term
$e$, let $x$ be any element of $\values{\oty}$ that satisfies~$\op$''.
An obligation is equivalent to a combination of the indefinite
description operator~\cite{epsilon-reference} and a local definition,
$\olet{\ox}{(\varepsilon \ox {:} \oty.\,\op)}{\oe}$, where
$\varepsilon \ox {:} \oty.\, \op$ means ``any $x \in \values{\oty}$
such that $\op$''. The alternative form $\oobligx{\op}{\oe}$ stands
for $\ooblig{\_}{\ounit}{\op}{\oe}$. 

Obligations arise from the fact that well-formedness 
of the input language is undecidable; see Section~\ref{sec:input-language}.  
In such cases the system computes a realizability translation, but also produces obligations 
to be checked.  The programmer must
replace each obligation with a value satisfying the
obligation (i.e., demonstrate that the obligation can be satisfied). 
If such values do not exist, the specification is
unimplementable.

\iffalse
Obligations serve a double role.
On one hand they are \emph{local assertions} that refer to local
values inside a term, e.g.,
%
\begin{equation*}
  \olambda{x}{\mathtt{real}}{(\oobligx{x^2-x+1 \neq 0}{x/(x^2-x+1)})}.
\end{equation*}
%
On the other hand they are \emph{local declarations} that require the
implementation of a local value inside a term, e.g.,
%
\begin{equation*}
  \olambda{x}{\mathtt{real}}{(
    \ooblig{y}{\mathtt{real}}{\phi(x,y)}{x+y}
    )},
\end{equation*}
%
requires that the programer define a local value~$y$ such
that$\phi(x,y)$ holds. In principle, local specifications could always
be expressed as global ones, but at a significant cost to clarity and
conciseness.

RZ outputs obligations when it encounters an uncheckable typing
constraint, or a term in the input whose well-formedness requires the
existence of a suitable term in the output, see
Section~\ref{sec:exampl-with-oblig}.
\fi
Assertions are expressed in the \emph{negative fragment} of
first-order logic, which contains constants for truth and falsehood,
negation, conjunction, implication, equivalence, and universal
quantification (but no disjunction or existential). This is the part
of first-order logic that has no computational content in the
realizability interpretation. Consequently, the classical and
constructive interpretations of assertions agree. This is quite
desirable, since RZ acts as a bridge between constructive mathematics
and real-world programmers, which typically are not familiar with
constructive logic. 
\iffalse
The basic predicates from which assertions are
built are observatinal equality $\oequal{\oe_1}{\oe_2}$, pers
$\oper{s}{\oe_1}{\oe_2}$ on abstract types, and basic predicates
declared as parameters. The formula $\ototal{\oe}{\os}$ stands for
$\oper{\os}{\oe}{\oe}$, while $\oforallt{\ox}{\os}{\op}$ is a
shorthand for
$\oforall{\ox}{\os}{(\oimply{\oper{\os}{\ox}{\ox}}{p})}$.
\fi

A specification is an OCaml module type with assertions. It may
contain value declarations, type declarations and definitions, module
declarations, specification definitions, proposition declarations, and
assertions. A specification $\ofunctor{\om}{\omt_1}{\omt_2}$ is a
module type of an OCaml functor. Such a parametrized specification may
be applied to a module. Because OCaml does not support such
applications, RZ reduces them away.

\internal{Andrej}{Maybe you can clarify the point about parameterized
  specifications applied to models (modules)? I seem to remember you
  once compared this to something that Automath has.}

%%% Local Variables: 
%%% mode: latex
%%% TeX-master: "cie"
%%% End: 
