
\section{Implementation}
\label{sec:implementation}

\subsection{Pre-translation}

After parsing, our implementation does type checking.  The type
checker does simple type reconstruction.  Instead of doing full
unification, we require that the types of all bound variables must
either be given at the binding site, given through an \Verb|implicit|
declaration, or be obvious from their definition.  

Unlike ML datatypes, we do not require disjoint union type
to be declared before they are used, or to have different unions
involve different tags.  Therefore, a very small amount of implicit 
subtyping is done between sums.  Otherwise, if we had a
specifiation of queues of integers that included
\begin{Verbatim}
  set iqueue
  const emptyQueue
  const enqueue : int*iqueue -> iqueue
  const dequeue : iqueue -> `None + `Some:int*iqueue
\end{Verbatim}
\goodbreak\noindent
then the axiom
\begin{Verbatim}
  dequeue emptyQueue = `None
\end{Verbatim}
would fail to typecheck (the most-precise set for the left-hand-side
is a two-element disjoint unition, while the most-precise set
for the right is a one-element disjoint union.)
\goodbreak

The type checker will also try to convert between $\Set$ and a
subset type $\LBRACE
x \ \COLON \Setexp\ \BAR \Proposition \RBRACE$ as necessary in order
to type check.  Thus, 
\begin{Verbatim}
  set real
  set nz_real = {x:real | not (x=zero)}
  const one   : real
  const inv   : nz_real -> nz_real
  const ( * ) : real * real -> real

  axiom field (x : real) =
    not (x=zero)  =>  x * (inv x) = one
\end{Verbatim}
is allowed, instead of requiring
\begin{Verbatim}
  axiom field (x : real) =
    not (x=zero)  =>  x * ((inv (x:>nz_real)) :< real) = one
\end{Verbatim}
In this case, since \Verb|:>| and \Verb|:<| has computational content
(going into a subset involves pairing the item with the realizer of
the proposition; going out of a subset is then a first projection),
the typechecker rewrites the former version of \Verb|field| into the
latter before passing it on to the translation phase. If injections
into subsets cannot be justified in the theory (e.g., if the
\Verb|field| axiom lacked the premise \Verb|not (x=zero)|) then the
theory will still translate, but the generated assertions will not be
satisfied by any implementation.

\subsection{Realizability Translation}
\label{subsec:real-transl}

We first discuss the realizability translation of sets and terms, and
then focus on the translation of logic, which is the interesting part.

A set~$S$ is translated into a modest set $(\ut{S}, \tot{S},
{\per{S}})$ according to its structure: a basic set is translated to a
modest set whose underlying type is abstract, a product is translated
to a product of modest sets, a function space is translated to the
exponential of modest sets, etc. Thus we simply use the rich
categorical structure of~$\Mod{\PL}$ to interpret all the set
constructors.

Similarly, terms are translated to suitable ML terms according to
their structure. Note however, that there are terms whose validity
cannot be checked by RZ because that would require it to prove
arbitrary theorems. Such an example is the definite description
operator $\THE\ x \,.\, \phi(x)$, whose validity can be confirmed only
by proving that there exists a unique~$x$ satisfying~$\phi(x)$. In
such cases RZ emits proof obligations for the programmer to verify.
Note however, that the translated terms always have valid ML types,
even if the accompanying proof obligations are not satisfied.

The driving force behind the realizability translation of logic is a
theorem, see e.g.\ \cite[Thm.~4.4.10]{Troelstra:van-Dalen:88:1}, which
says that under the realizability interpretation every formula~$\phi$
is equivalent to one that says, informally speaking, ``there
exists~$r$, such that $r$ realizes $\phi$''. Furthermore, the formula
``$r$ realizes $\phi$'' is computationally trivial. We explain what
precisely this means next.

In classical logic a doubly negated formula $\lnot\lnot\phi$ is
equivalent to~$\phi$. Constructively, this is not true in general. To
see this, recall that in constructive logic $\lnot\phi$ is defined as
$\phi \implies \bot$ and observe that the underlying type of realizers
of $\lnot\lnot\phi$ is $(\ut{\phi} \to \mathtt{unit}) \to
\mathtt{unit}$. Terms of this type cannot be converted to terms of
type $\ut{\phi}$ in a general way (although conversion in the reverse
direction can be done quite easily, which shows that $\phi$ implies
$\lnot\lnot\phi$). Furthermore, terms of type $(\ut{\phi} \to
\mathtt{unit}) \to \mathtt{unit}$ do not compute much of anything, so
we might as well replace them by a special \emph{trivial realizer}
devoid of any computational meaning. We can think of the trivial
realizer as a term which witnesses validity of a formula but does not
compute anything.

In some cases it may happen even in constructive logic that
$\lnot\lnot\phi$ is equivalent to~$\phi$. When this is so we
call~$\phi$ a \emph{$\lnot\lnot$-stable formula}, or just \emph{stable
  formula} for short. Stable formulas have trivial realizers, as they
are equivalent to doubly negated formulas. Among the stable formulas
the \emph{almost negative} formulas are important because they can be
easily recognized syntactically: they are built from any combination
of $\land$, $\implies$, $\forall$, $=$, and those basic predicates
that are known to be stable, but $\exists$ and $\lor$ are only allowed
to appear on the left side of an~$\implies$.\footnote{A
  \emph{negative} formula is one that does not contain $\exists$ and
  $\lor$ at all.}

The following theorem is a precise formulation of the claims we made
in a paragraph above.

\begin{theorem}
  For every formula $\phi$ there exists a set~$R_\phi$ and an almost
  negative formula $\phi'$ such that in the realizability
  interpretation $\phi$ is equivalent to $\exists r \in R_\phi . \,
  \phi'(r)$.
\end{theorem}

We omit the proof, as it is fairly standard and involves a
straightforward induction on the structure of~$\phi$. The set~$R_\phi$
in the theorem is simply the set of terms of the underlying
type~$\ut{\phi}$ of realizers, while the intuitive meaning of
$\phi'(r)$ is ``$r$ realizes $\phi$''.

RZ translates an axiom~$\phi$, or any other proposition it encounters,
by computing its underlying type $\ut{\phi}$ and the almost negative
formula $\phi'$ from the above theorem. In the output signature it
then emits
%
\begin{align*}
  & \mathtt{val}\ r : \ut{\phi} \\
  & \text{(* Assertion $\phi'(r)$ *)}
\end{align*}
%
This way the axiom~$\phi$ has been separated into its computational
content~$r$ and a statement $\phi'$ which describes when~$r$ is a
valid realizer of~$\phi$. Because $\phi'$ is almost negative it has no
computational content, which means that its classical and constructive
readings agree. Therefore a constructive mathematician and a classical
programmer will agree on the meaning of $\phi'(r)$.

RZ recognizes almost negative formulas and optimizes away their
realizers, as described below. In addition, the user may declare a
basic predicate or relation to be stable, which will be taken into
account by RZ during translation and optimization.

It seems worth noting that the computational irrelevance of
stable propositions is akin to \emph{proof irrelevance} studied by
Pfenning~\cite{pfenning01:_inten_exten_proof_irrel_modal_type_theor}.
This is not surprising in view of the well known fact that double
negation enjoys many formal properties of a modal possibility
operator.

\subsection{Optimization}

Propositions without constructive content have trivial realizers, and
so a final ``unit elimination'' pass both removes these and does 
peephole simplification of the resulting signature.   Without an optimizer,
the axioms of the theory \Verb|SQGROUP| would produce
\begin{Verbatim}
val unit : s -> top * top
(** Assertion unit (x:||s||) =  
     x * e =s= x and e * x =s= x
*)

val sqrt : s -> s * top
(** Assertion sqrt (x:||s||) =  
     pi0(sqrt x) : ||s|| and 
     pi0(sqrt x) * pi0(sqrt x) =s= x
*)
\end{Verbatim}
where \Verb|top| is the type of trivial realizers; we use 
\Verb|top| instead of \Verb|unit| to emphasize that these trivial realizers are terminating and hence safe to eliminate; this would not necessarily
be true for terms of type \Verb|unit|.  The optimizer can
easily tell from the types that the realizers for the \Verb|unit| and
\Verb|assoc| axioms are trivial and can be discarded, and that
although \Verb|sqrt| cannot be discarded entirely, part of its return
value is unnecessary.  Assertions that reference discarded or optimized
constants are automatically rewritten to preserve well-typedness, and we obtain the translation of \Verb|SQGROUP| shown previously, which contains no
occurrences of \Verb|top|.


%%% Local Variables: 
%%% mode: latex
%%% TeX-master: "case"
%%% End: 
