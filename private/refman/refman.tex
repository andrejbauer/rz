\documentclass[11pt]{article}
\usepackage{fancyvrb,xspace,fullpage,eqnarray,array,ifpdf}
%\usepackage{ccfonts,eulervm}

\ifpdf
\usepackage{pdfsync}
\fi

\DefineShortVerb{\|}

\newcommand{\metav}[1]{\mbox{\textit{#1}}}

\newcommand{\Case}{\metav{case}}
\newcommand{\Ident}{\metav{name}}
\newcommand{\Label}{\metav{\texttt{\`}label}}
\newcommand{\MId}{\metav{Modelid}}
\newcommand{\TId}{\metav{Theoryid}}
\newcommand{\SId}{\metav{setid}}
\newcommand{\PId}{\metav{propid}}
\newcommand{\EId}{\metav{termid}}
\newcommand{\Id}{\metav{id}}
\newcommand{\Phrase}{\metav{phrase}}
\newcommand{\Class}{\metav{class}}
\newcommand{\Setexp}{\metav{set}}
\newcommand{\Specification}{\metav{spec}}
\newcommand{\Propexp}{\metav{prop}}
\newcommand{\Termexp}{\metav{term}}
\newcommand{\Theoryexp}{\metav{theory}}
\newcommand{\Modelexp}{\metav{model}}
\newcommand{\Setkind}{\metav{kind}}
\newcommand{\Propkind}{\metav{propkind}}
\newcommand{\Theorykind}{\metav{thrykind}}
\newcommand{\Params}{\metav{params}}
\newcommand{\Pat}{\metav{pat}}

\SaveVerb{VWildcard}|_|
\SaveVerb{VColon}|:|
\SaveVerb{VLParen}|(|
\SaveVerb{VRParen}|)|
\SaveVerb{VLBrack}|[|
\SaveVerb{VRBrack}|]|
\SaveVerb{VTimes}|*|
\SaveVerb{VPlus}|+|
\SaveVerb{VDArrow}|=>|
\SaveVerb{VArrow}|->|
\SaveVerb{VComma}|,|

\newcounter{grammaritem}


\newcommand{\includeexample}[1]{
   \subsection{#1}
   \VerbatimInput[%frame=single,
                  label=\fbox{#1},numbers=left,
                  %framesep=5mm,
                 obeytabs=true,showtabs=true]{../../examples/misc/#1}}

\newcommand{\NB}{\textbf{NB: }}

\newcommand{\Maybe}[1]{$[${}#1{}$]$}
\newcommand{\Repeat}[1]{$[${}#1{}$]^*$}
\newcommand{\RRepeat}[1]{$[${}#1{}$]^+$}


\newcolumntype{E}{>{\quad\refstepcounter{grammaritem}}l<{\ (\arabic{grammaritem})}}
\input badline
 
\title{RZ Reference Manual}
\date{Version of: \today}
\author{Andrej Bauer and Chris Stone}

\begin{document}
\maketitle

\VerbatimFootnotes

\section{Running RZ}

\begin{Verbatim}
	Usage:  ./rz <options> [filenames]
	  --hoist Hoist all assurances
	  --nohoist Don't hoist assurances [default]
	  --opt Optimize translation [default]
	  --noopt Don't optimize translation
	  --poly Convert functors to polymorphism [default]
	  --nopoly Don't convert functors to polymorphism
	  --save Send output to .mli file [default]
	  --nosave No output to .mli file
	  --show Show output on stdout [default]
	  --noshow No output on stdout
	  --sigapp Retain signature applications
	  --nosigapp Expand away signature applications [default]
	  --thin Remove trivial realizers [default]
	  --nothin Leave trivial realizers
	  --dump_infer Dump result of type inference
	  --columns <int>  Number of columns in output [78]
	  --longerr <int>  Minimum lines in a 'long' error message [3]
	  -help  Display this list of options
	  --help  Display this list of options
\end{Verbatim}

Input files have the extension |.thy| (by convention, not a requirement), and the 
output file has the same name but with its extension replaced by |.mli|.

\section{Input Syntax}

\subsection{Comments}

Two sorts of comments are allowed:
\begin{itemize}
\item Line comments, which go from |#| to the end-of-line.
\item Multi-line comments, which are marked |(*|$\cdots$|*)| and which may be nested.  |#| is not treated as a
  special line-comment character within a multi-line comment, e.g.,
  the line |(* # *)| is a complete comment.
\end{itemize}

\NB{Multi-line comments are retained in the final ML output, while line
  comments are discarded.}

\subsection{Names}
\label{sec:names}

\begin{center}
\begin{tabular}{rRlE}
	\Id &::= & \EId & Term identifier\\
	      &\mid & \SId & Set identifier\\
	      &\mid & \PId & Proposition/predicate identifier\\
	      &\mid & \MId & Model identifier\\
	      &\mid & \TId & Theory identifier\\
\end{tabular}
\end{center}

\subsubsection*{Notes}

Model and theory names must begin with an uppercase letter.

Term, set, and predicate identifiers can be:
\begin{enumerate}
\item An ordinary lowercase alphanumeric identifier, beginning with a
  letter or underscore, containing letters, digits, underscores, and
  single quotes. (Integers such as |0| are therefore not valid names for constants.)
\item Or, a unary operator (any purely symbolic characters beginning with |?|, |~|, or |!|) enclosed in parentheses
\item Or a binary operator (any other identifier composed \emph{only} of
  symbolic characters) enclosed in parentheses\footnote{The parenthesized version of a binary operator \Verb|*| must be
written \Verb|(* )| rather than as \Verb|(*)| in order to avoid
creating the end-of-comment token \Verb|*)|.}.  The precedence of a
  binary operator is determined by its first character (or two).  From
  lowest precedence to highest precedence, the initial symbols are:
\begin{Verbatim}
  =   <   >   &   $
  @   ^
  +   -
  *   /   %
  **
\end{Verbatim}
  When a binary operator such as |&| is actually applied, the prefix form |(&) X Y| and infix form |X & Y| are interchangeable.
\end{enumerate}

\NB{Why do we require parentheses for unary (prefix) operators?}

\subsection{Syntax for Function Parameters}

\begin{center}
	\begin{tabular}{rRlE}
		\Params & ::= & \RRepeat{\Id\ \textbar\ \UseVerb{VWildcard}}\\
		& \mid & \RRepeat{\Id\ \textbar\ \UseVerb{VWildcard}}\ |:|\ \Class\\
		& \mid & \RRepeat{\UseVerb{VLParen}\ \Repeat{\Id\ \textbar\ \UseVerb{VWildcard}}\ \UseVerb{VColon}\ \Class\ \UseVerb{VRParen}}\\ 
	\end{tabular}
\end{center}

\subsection{Theories}

\begin{center}
\begin{tabular}{rRlE}
\Theoryexp & ::= & \TId & Previously-defined theory\\
    & \mid & \Modelexp|.|\TId & Previously-defined theory in a model\\
    & \mid & |thy|\ \Repeat{\Specification}\ |end| & Theory of a model\\
    & \mid & |[|\ \MId\ |:|\ \Theoryexp\ |]|\ |->|\ \Theoryexp & Theory of a family of models\\
    & \mid & |fun|\ \Params\ |=>| \Theoryexp & Mapping from models to theories\\
    & \mid & \Theoryexp\ \Modelexp & Application of a mapping\\
    & \mid & |(|\ \Theoryexp\ |)| & Grouping parentheses\\
\end{tabular}
\end{center}



\iffalse
\noindent Theories can be defined as follows:
\begin{itemize}
\item $\THEORY\ \Theoryname\ [\MParam]^*\ \EQUALS\ \Theoryexp$

   A named theory with the given definitions.  It may be parameterized
   by one or more models; each model parameter, if present, must be written
   with the syntax $\LPAREN \MId \COLON \Theoryexp \RPAREN$.
\end{itemize}
\fi


\subsection{Theory Elements}
\begin{center}
	\begin{tabular}{rRlE}
		\Specification &::=& |Parameter|\ \Repeat{\Id}\ |:| \Class\ |.|&Abstract constant declaration\label{gr:te:parameter}\\
	 	& \mid & |Parameter|\ \RRepeat{\Verb+(+ \Repeat{\Id}\ \Verb+:+\ \Class\ \Verb+)+}\ |.|&Alternate form\label{gr:te:parameter2}\\
	    & \mid & |Axiom|\ \Id\ |:| \Propexp\ |.|&Logical assertion\label{gr:te:axiom}\\
		& \mid & |Definition|\ \Id\ \Repeat{\Id\ \textbar\ \Verb+(+\ \Id\ \Verb+:+ \Class\ \Verb+)+}\ |:=| \Phrase\ |.|&Definition\label{gr:te:definition}\\
		& \mid & |implicit| \Repeat{\EId}\ |:|\ \Setexp\ |.|&Implicit range for variable\label{gr:te:implicit}\\
	\end{tabular}
\end{center}
\begin{itemize}
\item[(\ref{gr:te:axiom})] The keyword |Axiom| can be replaced by |Hypothesis| without change in meaning.\footnote{Actually, as far as RZ is concerned, the keywords |Axiom|, |Hypothesis|, and |Parameter| are indistinguishable.  We recommend that |Axiom| or |Hypothesis| be used for logical assertions, and |Parameter| be used for all other constants.}

\item[(\ref{gr:te:implicit})] 
  The given identifiers will be assumed to range over values in the
  given set, unless otherwise specified.  Unlike all other theory elements, this places no actual demand
  on a model satisfying this theory (it only simplifies writing
  the rest of the theory), and hence generates to
  no output.  
  
  Like all other specifications, |implicit| has the remainder of the
  enclosing |thy|$\cdots$|end| as its scope.
\end{itemize}


\subsection{Set Kinds}

\begin{center}
	\begin{tabular}{rRlE}
		\Setkind &::= & |Set|\\
	    &\mid & |[|\ \EId\ |:|\ \Setexp\ |]|\ |->| \Setkind\\
	\end{tabular}
\end{center}


If there is no dependency, the kind of a family of sets can be 
\Setexp\ |->|\ \Setkind.

\subsection{Proposition Kinds}

\begin{center}
\begin{tabular}{rRlE}
   \Propkind &::= & |Prop| & All propositions\label{gr:pk:prop}\\
         &   \mid & |Stable| & Stable propositions\label{gr:pk:stable}\\
         &   \mid & |Equiv|\ \Setexp & Equivalence relation\label{gr:pk:equiv}\\
         &   \mid & |[|\ \EId\ |:|\ \Setexp\ |]|\ |->| \Propkind
& Predicate with the given domain\label{gr:pk:arrow}\\
\end{tabular}
\end{center}

\subsubsection*{Notes}

\begin{itemize}
\item[(\ref{gr:pk:prop})] |Prop| classifies any well-formed formula that is true or false.
\item[(\ref{gr:pk:stable})] Stable propositions can be treated as arbitrary propositions.
\[
\mbox{\texttt{Stable}} \preceq\ \mbox{\texttt{Prop}}
\]
\item[(\ref{gr:pk:equiv})] Binary equivalence relations must be stable and take their arguments in a curried fashion.
\[
\mbox{\texttt{Equiv}}\ t\ \preceq\ t\texttt{\ ->\ }t\texttt{\ ->\ Stable}
\]
\item[(\ref{gr:pk:arrow})]
	If there is no dependency, the kind of a predicate can be written
	\Setexp\ |->|\ \Propkind.
\end{itemize}

\subsection{Theory Kinds}

\begin{center}
	\begin{tabular}{rRlE}
	 \Theorykind & ::= & |TheoryModel|\\
	  & \mid & |[|\ \MId\ |:|\ \Theoryexp\ |]|\ |->|\ \Theorykind
	\end{tabular}
\end{center}

The typechecker internally makes use of theory kinds to distinguish those
theories which classify models from those which map models to theories.
There is \emph{no} external syntax for theory kinds (because there's no
such thing as an ``abstract'' theory, without definition). They are
listed here merely for completeness.


\subsection{Propositions}

\begin{center}
	\begin{tabular}{rRlE}
	 \Propexp & ::= & |True|\\
	   & \mid & |False| \\
	   & \mid & \PId\\
	   & \mid & \Modelexp|.|\PId\\
	   & \mid & |(|\ \Propexp\ |)|\\
	   & \mid & \Propexp\ \Termexp\\
	   & \mid & \Propexp\ |:|\ \Propkind\\  
	   & \mid & |match|\ \Termexp\ |with|\ \Pat\ |=>| \Propexp \Verb+|+ $\cdots$\\
	   & \mid & |fun| \Params\ |=>|\ \Propexp\\
	   & \mid & \Termexp\ |=|\ \Termexp\\
	   & \mid & |not|\ \Propexp\\
	   & \mid & \Propexp\ |/\|\ $\cdots$\ |/\|\ \Propexp\\
	   & \mid & \Propexp\ |\/|\ $\cdots$\ |\/|\ \Propexp\\
	   & \mid & \Propexp\ |->|\ \Propexp\\
	   & \mid & \Propexp\ |<->|\ \Propexp\\
	   & \mid & |forall|\ \Params\ |,|\ \Propexp\\
	   & \mid & |exists|\ \Params\ |,|\ \Propexp\\
	   & \mid & |exists1|\ \Params\ |,|\ \Propexp\\
	\end{tabular}
\end{center}

The propositions are listed roughly in decreasing order of precedence.
(This suggests |:|, |match|, and |fun| need to be moved.)

\iffalse
& \mid & $\metav{longrelation}\ [\ \Term\ ]^*$\\
   $\metav{longunaryrelation}\ \Term$\\
   $\Term\ \metav{longbinaryrelation}\ \Term$\\	 
\fi

\subsection{Sets}

\begin{center}
	\begin{tabular}{rRlE} 
	\Setexp & ::= & |empty| & Empty (void) set\label{gr:s:empty}\\
	   &\mid& |unit| & Unit set, whose only member is |()|\label{gr:s:unit} \\
	   &\mid& \SId & Set identifier \label{gr:s:id}\\
	   &\mid& \Modelexp|.|\SId & Set from a model\label{gr:s:mproj}\\
	   &\mid& |(|\ \Setexp\ |)|&Grouping parentheses\label{gr:s:parens}\\
	   &\mid& \Setexp\ \Termexp&Application\label{gr:s:app}\\
	   &\mid& |fun|\ \Params\ |=>|\ \Setexp&Dependent set\label{gr:s:fun}\\
	   &\mid& \RRepeat{\UseVerb{VLBrack}\ \EId\ \UseVerb{VColon}\ \Setexp\ \UseVerb{VRBrack}\ \UseVerb{VTimes}}\ \Setexp& Dependent sum\label{gr:s:times}\\
	   &\mid& |[|\ \EId\ |:|\ \Setexp\ |]|\ |->| \Setexp&
	Dependent product\label{gr:s:arrow}\\
	   &\mid& \Setexp\ \verb+%+ \Propexp&Quotient set\label{gr:s:quotient}\\
	   &\mid& \Setexp\ |:|\ \Setkind&Kind coercion\label{gr:s:coerce}\\
	   &\mid& |rz|\ \Setexp & Set of realizers\label{gr:s:rz}\\
	   &\mid& |{|\ \EId\ |:|\ \Setexp\ \Verb+|+\ \Propexp\ |}| &
	Subset\label{gr:s:subset}\\
	   &\mid& \Label\ \Maybe{\UseVerb{VColon}\ \Setexp}\ \RRepeat{\UseVerb{VPlus}\ \Label\ \Maybe{\UseVerb{VColon}\ \Setexp}}& Tagged disjoint union\\
	\end{tabular}
\end{center}

\subsubsection*{Notes}

\begin{itemize}
\item[(\ref{gr:s:empty})] The syntax |{}| can be used as a synonym for |empty|.
\item[(\ref{gr:s:id})] Set identifiers must be symbolic or lowercase, as described in \S\ref{sec:names}.
\item[(\ref{gr:s:fun})] Sets may only have dependencies on terms (i.e., all identifiers in $\Params$ must be term identifiers, and all classifiers in $\Params$ must be sets).  All parameters must either have a set explicitly declared, or else they must have previously appeared in an |implicit|.
\item[(\ref{gr:s:times})] The full syntax for cartesian product supports dependencies, e.g. |[x:s] * [y:t(x)] * u x y| (where |s|:|Set|, |t|:|s->Set|, and |u|:|[x:s]->t(x)->Set|) stands
for 
\[
\Sigma_{x:s}\ \Sigma_{y:t(x)}\ u(x)(y).
\]
If there is no dependency on a component then the brackets, the identifier, and the colon can call be omitted.  In the simplest form, we have just the cartesian product |a*b*c|.
\item[(\ref{gr:s:arrow})] The full syntax for sets of functions supports dependencies, e.g. |[x:s] -> t(x)| (where |s|:|Set| and |t|:|s->Set|) stands
for 
\[
\Pi_{x:s}\ t(x)
\]
If there is no dependency on the argument then the brackets, the identifier, and the colon can call be omitted, and we have just |s->t|.
\item[(\ref{gr:s:coerce})] Sets have unique kinds up to equivalence, so all RZ does in this case is verify that the given set does have the given kind.
\end{itemize}

\subsection{Terms}

\begin{center}
	\begin{tabular}{rRlE}
	 \Termexp & ::= & |()| & Unit value\label{gr:t:unit}\\
	   &\mid& \EId &Term identifier \label{gr:t:id}\\
	   &\mid& \Modelexp|.|\EId\\
	   &\mid& \Termexp|.|$n$\\
	   &\mid& |(|\ \Termexp\ |)|\\
	   &\mid& |(|\ \Termexp\ \RRepeat{\UseVerb{VComma}\ \Termexp}\ |)|\\
	   &\mid& \Label\ \Maybe{\Termexp}\\
	   &\mid& \Termexp\ \Termexp\\
	   &\mid& |fun|\ \Params\ |=>|\ \Termexp\\
	   &\mid& \Termexp\ |:|\ \Setexp\\
	   & \mid & |match|\ \Termexp\ |with|\ \Pat\ |=>| \Termexp \Verb+|+ $\cdots$\\
	   &\mid& |the|\ \EId\ \Maybe{\UseVerb{VColon}\ \Setexp}\ |,|\ \Termexp\\
	   &\mid& |let|\ \EId\ \Maybe{\UseVerb{VColon}\ \Setexp}\ |=|\ \Termexp\ |in|\ \Termexp\\
	   &\mid& |let|\ |[|\ \EId\ |]|\ |=|\ \Termexp\ |in|\ \Termexp\\
       &\mid& \Termexp\ \verb+%+\ \Propexp\\
	   &\mid& |let|\ |rz|\ \EId\ |=|\ \Termexp\ |in|\ \Termexp\\
	   &\mid& |rz|\ \Termexp\\
	\end{tabular}
\end{center}

\ref{gr:t:id}\ \ref{gr:t:unit}

\end{document}

\subsection{Sets}

Sets can be of the form:

\begin{itemize}
\item |empty|\\
      |{}|

  The empty set.

\item |unit|\\

  The unit set, whose only member is \Verb|()|.

\item $\metav{longset}$
  
  The set is specified by 0 or more projections from models, followed
  by the alphanumeric set name.

\item \LPAREN \Setexp \RPAREN

  Just the given \Setexp (i.e., grouping parentheses).


\item $\Label\ [\ \COLON \Setexp\ ]\ (\ \PLUS\ \Label\ [\ \COLON \Setexp\ ]\ )^{*}$

  A $k$-ary sum type for $k\ge 1$.  The parentheses and brackets
  are EBNF.

\item \LBRACE \Ident [\ \COLON \Setexp\ ] \WITH \Proposition \RBRACE
      
  A subset type.  The bound variable must either be given a set
  explicitly, or must have previously appeared in an \IMPLICIT
  declaration.  In contrast to most other binding sites, parentheses
  are neither required nor allowed if the set is specified.  (These
  braces are concrete syntax.)


\item \LBRACE \Ident [\COLON \Setexp ]\BAR \Proposition \RBRACE
      
      A subset type for \emph{stable} propositions.  The bound variable must
      either be given a set explicitly, or must have previously
      appeared in an \IMPLICIT declaration.  Again, paraentheses
      are not required or allowed if the set is specified.

\item \Setexp \ARROW \Setexp

  Set of functions (exponential).

  \textbf{All functions?}

\item % \EQUIVPROP\\
      \PROP\\
      \STABLEPROP

  Sets of equivalence propositions, all propositions, and stable
  propositions respectively.  Relations can be viewed as functions
  to these sets.  These sets may not appear in specifying the set
  for a bound variable or in an \IMPLICIT declaration (though
  the error may not be caught by the typechecker until a later use).

\iffalse
\EQUIVPROP is used internally by the typechecker,
  but there's currently no way to specify this set in the source.
  Is this a bug?

No --- it's not a set in the topos, so we don't 
\fi

\item \Setexp \PERCENT \metav{longrelationname}\\
      \Setexp \PERCENT \LPAREN \metav{longbinaryrelation} \RPAREN

  The set of equivalence class, under the given relation on the
  given set. Currently a quotient by a symbolic binary relation requires
  parentheses.

\item \RZ \Setexp

  Underlying set of realizers of the given set.
\end{itemize}


The subset relation is generated by width subtyping on sums, and
(currently) an extremely limited subtyping relation on
stable subsets and the corresponding entire sets.


\subsection{Terms}

\end{document}

\item \CONST \metav{constname} = \Term\\
      \CONST \LPAREN \metav{constname} \COLON \Setexp \RPAREN = \Term\\
      
      A local abbreviation of a term.  The set containing the constant
      may be inferred from the \Term.  The possible names are the same
      as in the previous case for \CONST.

\item{} [\STABLE{}] \RELATION \metav{relationname} \COLON \Setexp\\
     {} [\STABLE{}] \PREDICATE \metav{relationname} \COLON \Setexp
  
     A relation of the given name exists, with the given set as its
     domain (but see below).  The relation is not assumed to be stable
     unless explicitly stated with the \STABLE keyword.  The name of
     the relation can be:
\begin{enumerate}
\item  An ordinary alphanumeric identifier;
\item  A unary operator (symbolic characters beginning with \Verb|?|, \Verb|~|, or \Verb|!|) enclosed in parentheses;
\item  Or a binary operator (any other identifier composed of symbolic characters) enclosed in parentheses, in which case the \Setexp should be of the (uncurried) form $\Setexp \TIMES \Setexp$

See the above discussion of \CONST for the precedences of binary
operators.

\NB Despite the syntax of the specification, the system \emph{uncurries}
binary operators ranging over a product of sets.  Thus uses of
\Verb|<| can be written either in the usual infix form \Verb|x < y| or
in the curried form \Verb|( < ) x y|.
\end{enumerate}

\NB If the set specified is of the (curried) form $\ldots \TO \PROP$,
this declares not a predicate on functions, but rather a proposition
that takes its arguments ($\cdots$) in curried form.

\item{} [\STABLE{}] \RELATION \metav{relationname}\ \EQUALS\ \Proposition\\
     {} [\STABLE{}] \PREDICATE \metav{relationname}\ \EQUALS\ \Proposition
  
     A local definition of a relation (or stable relation) of the
     given name exists.  The arguments implicitly give the domain of
     the relation; each argument must either be of the form $\LPAREN
     \Id \COLON \Setexp \RPAREN$ or must be a single
     $\Id$ that has previously appeared in an \IMPLICIT
     declaration.


\item $\AXIOM\ \metav{axiomname}\ [\MParam]^*\ [\Param]^*\ =\ \Proposition$\\
      $\COROLLARY\ \metav{axiomname}\ [\MParam]^*\ [\Param]^*\ =\ \Proposition$\\
      $\LEMMA\ \metav{axiomname}\ [\MParam]^*\ [\Param]^*\ =\ \Proposition$\\
      $\PROPOSITION\ \metav{axiomname}\ [\MParam]^*\ [\Param]^*\ =\ \Proposition$\\
      $\THEOREM\ \metav{axiomname}\ [\MParam]^*\ [\Param]^*\ =\ \Proposition$
      
      Each model parameter, if present, must be written
      with the syntax $\LPAREN \MId \COLON \Theoryexp \RPAREN$.
      Each term parameter, if present, must either be
      explicitly typed with the syntax $\LPAREN \Id \COLON 
      \Setexp \RPAREN$ or must be a single $\Id$ that has
      previously appeared in an \IMPLICIT declaration.

      \textbf{Need to change the parser so that model parameters are
      surrounded by parentheses as stated here, rather than 
      square brackets.}

\item \MODEL \MId \COLON \Theoryexp\\
      \STRUCTURE \MId \COLON \Theoryexp\\

      A model of the given theory must exist.  Names of models must be
      capitalized.

\item \EQUIV \Id \COLON \Setexp\\
      \EQUIVALENCE \Id \COLON \Setexp
  
  An stable equivalence relation on the given set (i.e., a predicate on the
  cartesian product of the set with itself) must exist.  The
  name can be alphanumeric or a binary operator in parentheses.

\item $\EQUIV\ \Id\ \Param\ \Param\ =\ \Proposition$\\
      $\EQUIVALENCE\ \Id\ \Param\ \Param\ =\ \Proposition$
 
  A local definition of an stable equivalence relation.  Both term parameters
  must be present and must either be explicitly typed with the syntax
  $\LPAREN \Id \COLON \Setexp \RPAREN$ or must be a single
  $\Id$ that has previously appeared in an \IMPLICIT
  declaration.  

  The proposition must be almost negative, so that the system can 
  verify that the equivalence relation is indeed stable.
  
\end{itemize}

Propositions can be:

\begin{itemize}
\item |True|

\item |False|

\item |not|\ \Propexp

item 
     
      
      Application of the given relation to the specified terms.  The
      relation is specified by 0 or more projections from models,
      followed by the relation name.  If a symbolic binary relation
      is applied in prefix form, the entire sequence of projections
      must be parenthesized. 
      Examples: \Verb|x < y| or \Verb|lt x y| or \Verb|Numeric.Nat.lt x y| or
      \Verb|( Numeric.Nat.< ) x y| or \Verb|x Numeric.Nat.< y| .

\item \Exp\ |=|\ \Exp

   Compares the terms for equivalence.

\item $\ALL\ \Param\ \PERIOD\ \Proposition$\\
      $\FORALL\ \Param\ \PERIOD\ \Proposition$

   Parameter must be implicitly typed or parenthesized, as usual.

\item $\EXISTS\ \Param\ \PERIOD\ \Proposition$\\
      $\SOME\ \Param\ \PERIOD\ \Proposition$

   Parameter must be implicitly typed or parenthesized, as usual.

\item $\EXISTSONE\ \Param\ \PERIOD\ \Proposition$\\
      $\SOMEONE\ \Param\ \PERIOD\ \Proposition$

   Parameter must be implicitly typed or parenthesized, as usual.

\item $\MATCH\ \Term\ \WITH\ \Case\ [\BAR\ \Case\ ]^*\ \END$
  
  Generalized if-then-else proposition, where
  $\Term$ must be a member of a sum.  For non-value-carrying
  sum labels the \Case must be of the form  $\Label\ \ARROW\
  \Proposition$ .  Otherwise, it should be of the pattern-matching
  form $\Label\ \Param\ \ARROW\ \Proposition$, where \Param must
  either be a \Ident or of the form $\LPAREN\ \Id\ \COLON\ 
  \Setexp\ \RPAREN$, and this parameter can be used in the
  corresponding \Proposition.  The cases must be non-redundant
  and exhaustive.
  
  \NB A single identifier in a pattern-matching case need not
  be described in an \IMPLICIT; there is enough information from
  looking at the \Term and the \Label.
\end{itemize}

\textbf{It would be good to explain the order of operations}


\subsection{Terms}

Terms can be of the following forms:
\begin{itemize}
\item $\metav{longfunctionname}\ [\ \Term\ ]^*$\\
      $\metav{longunaryfunction}\ \Term$\\
      $\Term\ \metav{longbinaryfunction}\ \Term$\\
  
  Constants or applications of constants to 0 or more arguments.  The
  function is specified by 0 or more projections from models, followed
  by the relation name.  If a unary or binary operator is applied in
  prefix form, the entire sequence of projections must be
  parenthesized.  Examples: \Verb|zero| or \Verb|x + y| or 
 \Verb|plus x y| or \Verb|Numeric.Nat.plus x y| or \Verb|( Numeric.Nat.+ ) x y|
  or \Verb|x Numeric.Nat.+ y| .

\item $\LPAREN\ \Term\ \RPAREN$

  Just the given \Term.

\item $\LPAREN\ \Term\ \COLON\ \Setexp\ \RPAREN$

  Just the given \Term, but treated as a member of the specified
  \Setexp (which must be a superset of the inferred set for the term).

\item $\LPAREN\ \Term\ [\ \COMMA\ \Term\ ]^{+}\ \RPAREN$

  A $k$-tuple for $k\ge 2$.

\item \Verb|()|

  The unique term of type \UNIT (i.e., the 0-tuple).

\item $\Term \PERIOD \metav{n}$
 
  Projection from a tuple, where \metav{n} is a natural number.  Projection
  is 0-based, i.e., \Verb|(x,y,z).1| is \Verb|y|.

\item $\Label\ [\ \Term\ ]$

  An injection into a sum type; if no term is given then it is
  assumed that the $\Label$ is non-value-carrying.

\item $\MATCH\ \Term\ \WITH\ \Case\ [\BAR\ \Case\ ]^*\ \END$
  
  Generalized if-then-else proposition, where
  $\Term$ must be a member of a sum.  For non-value-carrying
  sum labels the \Case must be of the form  $\Label\ \ARROW\
  \Term$ .  Otherwise, it should be of the pattern-matching
  form $\Label\ \Param\ \ARROW\ \Term$, where \Param must
  either be a \Ident or of the form $\LPAREN\ \Id\ \COLON\ 
  \Setexp\ \RPAREN$, and this parameter can be used in the
  corresponding \Term.  The cases must be non-redundant
  and exhaustive; all their terms must either be inferred
  as members of the same set, or all compatible members
  of a sum.
  
  \NB A single identifier in a pattern-matching case need not
  be described in an \IMPLICIT; there is enough information from
  looking at the \Term and the \Label.

\item $0$\\
      $1$

  \textbf{Bug: these are both (distinct) valid term variables, but we can't actually
  define constants with these names.  Also, other integers are not permissible
  as either terms or named constants.}

\item \LET \Param \EQUALS \Term \IN \Term
  
  Local definition of a term.  As usual, either parentheses and a set
  are required for the defined variable, or \Param must be a lowercase
  variable that has appeared in a previous \IMPLICIT.

\item \LAMBDA \Param \PERIOD \Term
  
  An anonymous function.  Either parentheses and a set are required
  for the argument, or \Param must be a lowercase variable that has
  appeared in a previous \IMPLICIT.

\item \THE \Param \PERIOD \Proposition

  The description operator.  Either parentheses and a set are required
  for the argument, or \Param must be a lowercase variable that has
  appeared in a previous \IMPLICIT.  

  \THE should be used only when there exists exactly one value making
  the proposition true; the system does not verify this, but produces
  an obligation to be checked by the user.

\item \Term \PERCENT \metav{longrelationname}\\
      \Term \PERCENT \LPAREN \metav{longbinaryrelation} \RPAREN

  The equivalence class, under the given relation, containing the
  given term.  The result belongs in a quotient set.  As shown,
  currently a quotient by a symbolic binary relation requires
  parentheses.

\item \LET \Param \PERCENT \metav{longrelationname} \EQUALS \Term \IN \Term\\
      \LET \Param \PERCENT \LPAREN \metav{longbinaryrelation} \RPAREN \EQUALS \Term \IN \Term\\

  Picks an arbitrary member of the equivalence class given by the first
  term to use in computing the second term.  No type annotation or
  \IMPLICIT for the variable containing this memory is required.

  \NB This should not be used unless the choice of representative has 
  no effect on the value of this whole term.
  
\item \LBRACK \Term \RBRACK

  The element whose realizer is the given term.

  \NB These brackets are concrete syntax rather than EBNF.

\item \LET \LBRACK \Param \RBRACK \EQUALS \Term \IN \Term

  Picks an arbitrary realizer for the first term, to be used
  in the last term.

  \NB This should not be used unless the choice of representative has 
  no effect on the value of this last term.

\item \Term \SUBIN \Setexp
  
  Coerces the given term of some set S to the given sub-set of $S$.
  Should only be used if we know the term is in the subset.

\item \Term \SUBOUT \Setexp

  Coerces the given term from a sub-set of some set S to just
  the set $S$.  Always safe if the term is known to belong to the
  given subset.
  
\end{itemize}

\clearpage
\section{Sample Inputs}
\includeexample{nat.thy}

\clearpage
\includeexample{kuratowski.thy}
Need to change brackets to parentheses

\clearpage
\includeexample{sums.thy}
\includeexample{subset.thy}

\clearpage
\includeexample{field.thy}

\clearpage
\includeexample{choice.thy}

\section{Output}

\subsection{Output Terms and Types}

\subsection{The Translation}

\subsection{Optimization}

\end{document}
