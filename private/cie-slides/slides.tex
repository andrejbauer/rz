\documentclass[t,handout]{beamer}

\usepackage{palatino}
\usepackage{amsfonts}
\usepackage[all]{xypic}

\usepackage{fancyvrb}
% To show source
\DefineVerbatimEnvironment{source}{Verbatim}%
{fontsize=\small,commandchars=\\\{\}}

\DefineVerbatimEnvironment{sourcex}{Verbatim}%
{fontsize=\footnotesize,commandchars=\\\{\}}

% macros within source
\newcommand{\iTo}{{\ensuremath{\to}}}
\newcommand{\iForall}{$\forall$}
\newcommand{\iOr}{$\lor$}
\newcommand{\iAnd}{$\land$}
\newcommand{\iImply}{$\Rightarrow$}
\newcommand{\iNot}{$\lnot$}
\newcommand{\iExists}{$\exists$}
\newcommand{\iExistsOne}{$\exists!$}
\newcommand{\iT}[1]{$\|$#1$\|$}
\newcommand{\iPer}[1]{$\approx_{\mathtt{#1}}$}
\newcommand{\iIff}{$\leftrightarrow$}
\newcommand{\iLeq}{$\leq$}
\newcommand{\iAlpha}{$\alpha$}

\newcommand{\per}{\approx}
\newcommand{\typeOf}[1]{|#1|}
\newcommand{\sem}[1]{[\![#1]\!]}
\newcommand{\dom}[1]{\mathrm{dom}(#1)}

\newcommand{\rz}{\Vdash}


\title{RZ: a Tool for Bringing\\Constructive and Computable\\
  Mathematics Closer to Practice}
\author{Andrej Bauer \and Chris Stone}
\institute{Department of Mathematics and Physics\\
  University of Ljubljana, Slovenia
  \and
  Computer Science Department\\
  Harvey Mudd College, USA}
\date{CiE 2007, Siena, June 2007}

\beamerdefaultoverlayspecification{<+->}
\mode<presentation>
% \usetheme{Goettingen}
\usecolortheme{rose}
\usefonttheme{serif}
\setbeamertemplate{navigation symbols}{}

\begin{document}

\begin{frame}
  \titlepage
\end{frame}

\begin{frame}
  \frametitle{Theory and practice}

  \begin{itemize}
  \item The theory of constructive \& computable mathematics:
    % 
    \begin{itemize}[<.->]
    \item Structures from analysis and topology are studied.
    \item \emph{Informal descriptions} of algorithms via Turing machines.
    \item Deals mostly with: ``What \emph{can} be computed?''
    \item Efficiency of computation is desired.
    \end{itemize}
  \item 
    Practice of computing:
    % 
    \begin{itemize}[<.->]
    \item Emphasis on discrete mathematics.
    \item \emph{Implementations} of practical data structures and
      algorithms.
    \item Deals with: ``How \emph{fast} can we compute?''
    \item Speed is essential.
    \end{itemize}
  \end{itemize}
\end{frame}

\begin{frame}
  \frametitle{Can we bring constructive math closer to practice?}

  \begin{itemize}
  \item Sacrificing performance for correctness is unacceptable.
    \begin{itemize}[<.->]
    \item Currently programs extracted from formal proofs are inefficient.
    \item Programmers should be free to implement efficient code.
    \item Provide support for proving correctness of implementation.
    \end{itemize}
  \item It is tricky to correctly implement structures from analysis
    and topology.
    \begin{itemize}[<.->]
    \item We should link mathematical models with practical programming.
    \item Give programmers tools that automate tasks.
    \end{itemize}
  \end{itemize}
\end{frame}

\begin{frame}
  \frametitle{Our contribution}

  \begin{itemize}
  \item A theory of representations based on Objective Caml.
    \begin{itemize}[<.->]
    \item We replaced Turing machines (type I and II) with a
      real-world programming language.
    \item Representations can \emph{actually} be implemented.
    \item Other programming languages could be used.
    \end{itemize}
  \item But we do not work with representations directly.
    % 
    \begin{itemize}[<.->]
    \item Cumbersome and generally too low a level of abstraction to
      do mathematics.
    \item How do we know which representation of a given set is the
      right one?
    \end{itemize}
  \item Instead, we use representations as a model in which to
    interpret constructive mathematics.
    % 
    \begin{itemize}[<.->]
    \item Use Kleene's realizability interpretation adapted to OCaml.
    \item The translation of a constructive theory is a
      \emph{specification} describing how to implement it in OCaml.
    \end{itemize}
  \item Most importantly, we built a tool RZ which
    \emph{automatically} translates constructive logic to representations.
  \end{itemize}
\end{frame}

\begin{frame}
  \frametitle{Representations}

  \begin{itemize}
  \item Representations are a successful idea in computable
    mathematics:
    % 
    \begin{itemize}[<.->]
    \item numbered sets,
    \item Type Two Effectivity representations,
    \item domain-theoretic representations,
    \item equilogical spaces.
    \end{itemize}
  \item Phrased in various forms:
    %
    \begin{itemize}[<.->]
    \item partial surjections,
    \item partial equivalence relations,
    \item modest sets,
    \item assemblies,
    \item multi-valued partial surjections,
    \item realizability relations.
    \end{itemize}
  \item Can be described to programmers without much trouble.
  \end{itemize}
\end{frame}

\begin{frame}
  \frametitle{Representations in Objective Caml}

  \begin{itemize}
  \item A representation $\delta: \mathtt{t} \to X$ consists of:
    % 
    \begin{itemize}[<.->]
    \item represented set $X$
    \item representing datatype $\mathtt{t}$
    \item partial surjection $\delta : \mathtt{t} \to X$
    \end{itemize}
  \item 
    Define the partial equivalence relation (per) $\per$ on
    $\mathtt{t}$ by
    % 
    \begin{equation*}
      u \per v \iff u, v \in \dom{\delta} \land \delta(u) = \delta(v) \;.
    \end{equation*}
    % 
  \item 
    We may recover $\delta : \mathtt{t} \to X $ from $(\mathtt{t},
    \per)$ up to isomorphism:
    % 
    \begin{gather*}
      \|\mathtt{t}\| = \{ u \in \mathtt{t} \mid u \per u \}
      \\
      X \cong \|\mathtt{t}\|/{\per}, \quad \dom{\delta} =
      \|\mathtt{t}\|, \quad \delta(u) = [u]_{\per}
    \end{gather*}
  \item Note: $\delta$ and $\per$ are \emph{not} required to be
    computable, they live ``outside'' the programming language.
  \end{itemize}
\end{frame}

\begin{frame}
  \frametitle{Constructions of representations}

  Representations, together with a suitable notion of morphisms, form a
  rich category with many constructions:
  %
  \begin{itemize}[<.->]
  \item products $A \times B$ and disjoint sums $A + B$,
  \item function spaces $A \to B$,
  \item dependent sums $\Sigma_{i \in A} B(i)$ and products $\Pi_{i
      \in A} B(i)$,
  \item subsets $\{x : A \mid \phi(x)\}$,
  \item quotients $A/{\rho}$,
  \item but \emph{no} powersets.
  \end{itemize}
  %
  This is a convenient ``toolbox'' for constructive mathematics.

\end{frame}


\begin{frame}
  \frametitle{Realizability interpretation of logic}

  \begin{itemize}
  \item A formalization of Brouwer-Heyting-Kolmogorov interpretation
    of intuitionistic logic.
  \item Validity of a proposition $\phi$ is witnessed by a
    \emph{realizer}:
    %
    \begin{equation*}
      r \rz \phi \qquad \text{``$r$ is computational witness of $\phi$''}
    \end{equation*}
  \item Note: $r$ could be any OCaml value, need not correspond to a
    proof under the Curry-Howard correspondence.
  \item The type of $r$ and $\Vdash$ are defined inductively on the
    structure of~$\phi$, e.g.:
    %
    \begin{xalignat*}{2}
      &\langle r_1, r_2 \rangle \rz \phi_1 \land \phi_2
      &\text{iff}\quad
      &\text{$r_1 \rz \phi_1$ and $r_2 \rz \phi_2$}
      \\
      &r \rz \phi \implies \psi
      &\text{iff}\quad
      &\text{whenever $s \rz \phi$ then $r(s) \rz \psi$}
      \\
      &\cdots & &\cdots
    \end{xalignat*}
  \end{itemize}
\end{frame}

\begin{frame}
  \frametitle{RZ}

  \begin{itemize}
  \item Input: one or more theories
  \item Output: OCaml module type specifications
  \item Translation has several phases:
    \begin{enumerate}[<.->]
    \item Type-checking: does the input make sense?
    \item Translation via realizability interpretation
    \item Thinning: remove computationally irrelevant realizers
    \item Optimization: perform further simplifications to output
    \item Phase splitting (will not explain here, read the paper)
    \end{enumerate}
  \end{itemize}
\end{frame}

\begin{frame}[fragile]
  \frametitle{Input}

  A theory consists of declarations, definitions, and axioms.

\begin{source}
Definition Ab :=
thy
  Parameter t : Set.
  Parameter zero : t.
  Parameter neg : t \iTo t.
  Parameter add : t * t \iTo t.
  Definition sub (u : t) (v : t) := add(u, neg v).
  Axiom zero_neutral: \iForall u : t, add(zero,u) = zero.
  Axiom neg_inverse: \iForall u : t, add(u,neg u) = zero.
  Axiom add_assoc:
   \iForall u v w : t, add(add(u,v),w) = add(u,add(v,w)).
  Axiom abelian: \iForall u v : t, add(u,v) = add(v,u).
end.    
\end{source}

Theories can be \emph{parametrized}, e.g., the theory of a vector
space parametrized by a field, \texttt{VectorSpace(F:Field)}.

\end{frame}


\begin{frame}[fragile]
  \frametitle{Translation and output}

\begin{itemize}[<+->]
\item
  Consider the input:
  %
\begin{source}
Axiom lpo : \iForall f : nat \iTo nat,
  [`zero: \iForall n : nat, f n = zero] \iOr
  [`nonzero: \iNot (\iForall n : nat, f n = zero)].    
\end{source}

\item
  In the output we get a \emph{value declaration} and an \emph{assertion}:
%
\begin{source}
val lpo : (nat \iTo nat) \iTo [`zero | `nonzero]
(**  assertion lpo : 
  \iForall (f:\iT{nat \iTo nat}), 
    (match lpo f with
       `zero \iImply \iForall (n:\iT{nat}), f n \iPer{nat} zero 
     | `nonzero \iImply \iNot (\iForall (n:\iT{nat}), f n \iPer{nat} zero))
*)
\end{source}
\item The value \texttt{lpo} is the computational content of the axiom.
\item An implementation of \texttt{lpo} must satisfy the assertion.
\item Assertion is free of computational content, thus its
  constructive and classical readings agree.
\end{itemize}

\end{frame}

\begin{frame}[fragile]
  \frametitle{Example: ``All functions are continuous''}

  \begin{itemize}[<+->]
  \item 
    Input:
%
\begin{sourcex}
Axiom modulus:
\iForall f : (nat \iTo nat) \iTo nat, \iForall a : nat \iTo nat,
  \iExists k : nat, \iForall b : nat \iTo nat,
    (\iForall m : nat, m \iLeq k \iTo a m = b m) \iTo f a = f b.
\end{sourcex}
%
\item RZ output:
%
\begin{sourcex}
val modulus : ((nat \iTo nat) \iTo nat) \iTo (nat \iTo nat) \iTo nat
(**  Assertion modulus = 
  \iForall (f:\iT{(nat \iTo nat) \iTo nat}, a:\iT{nat \iTo nat}), 
    let p = modulus f a in p : \iT{nat} \iAnd 
    (\iForall (b:\iT{nat \iTo nat}), 
       (\iForall (m:\iT{nat}), m \iLeq p \iTo a m \iPer{nat} b m) \iTo
       f a \iPer{nat} f b) *)
\end{sourcex}
\item Implementation:
%
\begin{source}
let modulus f a =
  let p = ref 0 in
  let a' n = (p := max !p n; a n) in
    ignore (f a') ; !p
\end{source}
\end{itemize}


\end{frame}

\begin{frame}
  \frametitle{Remarks}

  \begin{itemize}
  \item We have implemented real numbers using RZ:
    \begin{itemize}[<.->]
    \item see Bauer \& Kavkler at CCA 2007.
    \end{itemize}
  \item We would like to implement more advanced structures:
    %
    \begin{itemize}[<.->]
    \item manifolds, Hilbert spaces, analytic functions, \ldots
    \item we expect these to be painfully slow at first.
    \end{itemize}
  \item Even if you do not want to implement anything, you can use RZ
    to \emph{automatically} compute representations from constructive
    definitions.
  \item It would be interesting to connect RZ with a tool that allows
    formal verification of correctness, such as Coq.
  \end{itemize}

\end{frame}

\end{document}
