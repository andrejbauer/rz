\documentclass{entcs}
\usepackage{prentcsmacro}

%\usepackage{times}
\usepackage{amsmath}
\usepackage{amssymb}
%\usepackage{theorem}
\usepackage{fancyvrb,xspace}

%% MACROS

\newcommand{\ut}[1]{|#1|}
\newcommand{\tot}[1]{\|#1\|}
\newcommand{\per}[1]{\approx_{#1}}
\newcommand{\PL}{\mathcal{P}}
\newcommand{\RZ}{RZ\xspace}
\newcommand{\Mod}[1]{\mathsf{Mod}(#1)}
\newcommand{\rz}{\Vdash}
%\newcommand{\note}[1]{\texttt{[#1]}}
\newcommand{\fst}[1]{\texttt{fst}\,#1}
\newcommand{\snd}[1]{\texttt{snd}\,#1}
\newcommand{\inl}[1]{\texttt{inl}\,#1}
\newcommand{\inr}[1]{\texttt{inr}\,#1}

\newcommand{\comment}[1]{\textbf{[#1]}}

%% THEOREM-LIKE ENVIRONMENTS


\newcommand{\keywd}[1]{\mbox{\texttt{#1}}\xspace}
\newcommand{\ALL}{\keywd{all}}
\newcommand{\AND}{\keywd{and}}
\newcommand{\AXIOM}{\keywd{axiom}}
\newcommand{\BOOL}{\keywd{bool}}
\newcommand{\CONST}{\keywd{const}}
\newcommand{\COROLLARY}{\keywd{corollary}}
\newcommand{\EQUIV}{\keywd{equiv}}
\newcommand{\EQUIVALENCE}{\keywd{equivalence}}
\newcommand{\EQUIVPROP}{\keywd{Equiv}}
\newcommand{\END}{\keywd{end}}
\newcommand{\EXISTS}{\keywd{exists}}
\newcommand{\EXISTSONE}{\keywd{exists1}}
\newcommand{\FALSE}{\keywd{false}}
\newcommand{\FORALL}{\keywd{forall}}
\newcommand{\IMPLICIT}{\keywd{implicit}}
\newcommand{\IFF}{\keywd{iff}}
\newcommand{\IMPLY}{\keywd{implies}}
\newcommand{\IN}{\keywd{in}}
\newcommand{\LAMBDA}{\keywd{lam}}
\newcommand{\LEMMA}{\keywd{lemma}}
\newcommand{\LET}{\keywd{let}}
\newcommand{\MATCH}{\keywd{match}}
\newcommand{\MODEL}{\keywd{model}}
\newcommand{\NOT}{\keywd{not}}
\newcommand{\OR}{\keywd{or}}
\newcommand{\PROP}{\keywd{Prop}}
\newcommand{\PROPOSITION}{\keywd{proposition}}
\newcommand{\PREDICATE}{\keywd{predicate}}
\newcommand{\RELATION}{\keywd{relation}}
%\newcommand{\RZ}{\keywd{rz}}
\newcommand{\SET}{\keywd{set}}
\newcommand{\SOME}{\keywd{some}}
\newcommand{\STABLE}{\keywd{stable}}
\newcommand{\STABLEPROP}{\keywd{Stable}}
\newcommand{\STRUCTURE}{\keywd{structure}}
\newcommand{\THE}{\keywd{the}}
\newcommand{\THEOREM}{\keywd{theorem}}
\newcommand{\THEORY}{\keywd{theory}}
\newcommand{\THY}{\keywd{thy}}
\newcommand{\TRUE}{\keywd{true}}
\newcommand{\UNIQUE}{\keywd{unique}}
\newcommand{\UNIT}{\mbox{\Verb| unit |}}
\newcommand{\WITH}{\keywd{with}}

\newcommand{\metav}[1]{\mbox{\textit{#1}}\xspace}

\newcommand{\Case}{\metav{case}}
\newcommand{\Ident}{x}
\newcommand{\Identifier}{\Ident}
\newcommand{\Label}{\metav{\Verb|`|label}}
\newcommand{\MIdentifier}{\metav{Modelname}}
\newcommand{\TIdentifier}{\metav{Theoryname}}
\newcommand{\Setexp}{\metav{set}}
\newcommand{\Set}{\metav{set}}
\newcommand{\Specification}{\metav{specification}}
\newcommand{\Proposition}{\metav{proposition}}
\newcommand{\Prop}{\metav{prop}}
\newcommand{\Param}{\metav{param}}
\newcommand{\MParam}{\metav{modelparam}}
\newcommand{\Term}{\metav{term}}
\newcommand{\Theoryexp}{\metav{theory}}

\newcommand{\AAND}{\mbox{\Verb| \&\& |}}
\newcommand{\ARROW}{\mbox{\Verb| -> |}}
\newcommand{\BAR}{\mbox{\Verb+ | +}}
\newcommand{\COLON}{\mbox{\Verb| : |}}
\newcommand{\COMMA}{\mbox{\Verb| , |}}
\newcommand{\EQUALS}{\mbox{\Verb| = |}}
\newcommand{\HASH}{\mbox{\Verb| \# |}}
\newcommand{\IIFF}{\mbox{\Verb| <=> |}}
\newcommand{\IIMPLY}{\mbox{\Verb| => |}}
\newcommand{\LBRACE}{\mbox{\Verb|\{ |}}
\newcommand{\LBRACK}{\mbox{\Verb|[ |}}
\newcommand{\LCOMMENT}{\mbox{\Verb|(* |}}
\newcommand{\LPAREN}{\mbox{\Verb|( |}}
\newcommand{\OOR}{\mbox{\Verb+ || +}}
\newcommand{\ONE}{\mbox{\Verb|1|}}
\newcommand{\PERCENT}{\mbox{\Verb+ \% +}}
\newcommand{\PLUS}{\mbox{\Verb| + |}}
\newcommand{\RBRACE}{\mbox{\Verb| \}|}}
\newcommand{\RBRACK}{\mbox{\Verb| ]|}}
\newcommand{\RCOMMENT}{\mbox{\Verb| *)|}}
\newcommand{\RPAREN}{\mbox{\Verb| )|}}
\newcommand{\PERIOD}{\mbox{\Verb| . |}}
\newcommand{\SUBIN}{\mbox{\Verb| :> |}}
\newcommand{\SUBOUT}{\mbox{\Verb| :< |}}
\newcommand{\TIMES}{\mbox{\Verb| * |}}
\newcommand{\TO}{\mbox{\Verb| -> |}}
\newcommand{\ZERO}{\mbox{\Verb|0|}}

%%%%%%%%%%%%%%%%%%%%%%%%%%%%%%%%%%%%%%%%%%%%%%%%%%
%%%%%%%%%%%%%%%%%%%%%%%%%%%%%%%%%%%%%%%%%%%%%%%%%%
\begin{document}

\def\lastname{Bauer and Stone}
\begin{frontmatter}
  \title{Specifications via Realizability} 
  \author{Andrej Bauer\thanksref{myemail}}
  \address{Department of Mathematics and Physics\\ 
    University of Ljubljana\\
    Ljubljana, Slovenia} \author{Christopher A. Stone\thanksref{coemail}}
  \address{Computer Science Department\\Harvey Mudd College\\
    Claremont, CA, USA} \thanks[myemail]{Email:
    \href{mailto:Andrej.Bauer@andrej.com} {\texttt{\normalshape
        Andrej.Bauer@andrej.com}}} \thanks[coemail]{Email:
    \href{mailto:stone@cs.hmc.edu} {\texttt{\normalshape
        stone@cs.hmc.edu}}}
\begin{abstract} 
  We present a system, called~\RZ, for automatic generation of program
  specifications from mathematical theories. We translate mathematical
  theories to specifications by computing their realizability
  interpretations in the ML language augmented with assertions (as
  comments). While the system is best suited for descriptions of those
  data structures that can be easily described in mathematical
  language (e.g., finitely presented groups, real arithmetic, graphs,
  etc.), it also elucidates the relationship between data structures
  and constructive mathematics.
\end{abstract}
\begin{keyword}
  Realizability, Constructive Logic, ML.
\end{keyword}
\end{frontmatter}

\section{Introduction}
\label{sec:introduction}

Kleene~\cite{KleeneSC:intint} introduced realizability as a model of
intuitionistic arithmetic based on partial computable functions. The
idea has been studied and generalized by various
authors~\cite{TroelstraA:rea,HylandJ:efft,HylandJ:trit,OostenJ:exer}.
Building on the idea of \emph{typed realizability} by
Longley~\cite{Longley00}, we have constructed a tool \emph{RZ} which
can translate mathematical theories into specifications for code,
explaining what is necessary in order to believe that we have a correct
implementation of the mathematical theory.

As the realizability interpretation validates the laws of
\emph{intuitionistic} logic, our input theories are intuitionistic or
constructive. Thus, \emph{RZ} extracts the computational
meaning of a constructive theory and expresses it as a programming
specification.

We emphasize that RZ does \emph{not} extract programs from proofs---in
fact, there is no way to write a proof in our system. We just
determine what the programs are supposed to do, i.e., we provide
specifications for them. We leave it to the programmer, or to another
tool, to construct the programs as he or she best knows. This leaves
the programmer completely free to write the most efficient programs
that do not necessarily correspond to any proofs.

The original aim of RZ was to help with development of data structures
for computable mathematics. If one sets out to actually compute
realizability interpretations of theories of constructive mathematics,
one quickly wishes for an automated way of doing it. With a tool like
RZ it is much easier to experiment and try out variations of a theory
until a suitable specification is obtained. We have also discovered
that RZ can be used to explain and teach constructive mathematics to
programmers, who are typically trained in classical mathematics,
because it translates constructive statements to easily understood
requirements about programs (expressed in classical logic).

We assume throughout that we have chosen a fixed programming
language~$\PL$. Any ML-like language will
do~\cite{milner+:definition}. Our implementation of RZ uses Objective
Caml~\cite{ocaml} but could easily be adopted to other variants of ML.
The essential features we require of~$\PL$ are product, function, and
sum types, as well as support for module interfaces.

The paper is organized as follows. Section~\ref{sec:realizability}
contains a brief overview of realizability. In
Section~\ref{sec:theories-signatures} we describe theories and
signatures, which are the inputs and the outputs of RZ, respectively. In
Section~\ref{sec:implementation} we discuss various point of
implementation. In Section~\ref{sec:examples} we show typical examples
and conclude with Section~\ref{sec:conclusion}.


%%% Local Variables: 
%%% mode: latex
%%% TeX-master: "case"
%%% End: 



\section{Realizability}
\label{sec:realizability}

We briefly motivate the main idea of (typed) realizability. When we
represent a set of mathematical objects~$S$ in a programming
language~$\PL$ there are two natural steps to take: first choose an
\emph{underlying type~$\ut{S}$} of representing values, and second
specify how the values of type~$\ut{S}$ represent, or \emph{realize},
elements of the set~$S$. For example, consider how we might represent
the set~$D$ of simple finite directed graphs (whose vertices are
labeled by integers). As the underlying datatype we might choose
$\ut{D} = \mathtt{int} \; \mathtt{list} * (\mathtt{int} *
\mathtt{int}) \; \mathtt{list}$, and represent a graph~$G \in D$ as a
pair of lists $(v,e)$ where $v = [x_1; \ldots; x_n]$ is the list of
vertices and $e = [e_1; \ldots; e_m]$ is the list of edges. Formally,
we write
%
\begin{equation*}
  (v, e) \rz_D G
\end{equation*}
%
and read it as ``$(v,e)$ realizes~$G \in D$''. Observe that each graph
is realized by at least one pair of lists, and that no pair of lists
represents more than one graph. (As commonly occurs, most graphs are
represented by many different pairs of lists.) This leads us to the definition given
below. We shall abuse notation slightly and write $t \in \ut{S}$ to
mean that $t$ is a closed expression of type $\ut{S}$.


\begin{figure*}
  \centering
  \parbox[t]{0.35\textwidth}{
    \begin{align*}
      \ut{\top} &= \mathtt{unit} \\
      \ut{\bot} &= \mathtt{unit} \\
      \ut{x =_S y} &= \mathtt{unit} \\
      \ut{\phi \land \psi} &= \ut{\phi} * \ut{\psi} \\
      \ut{\phi \implies \psi} &= \ut{\phi} \to \ut{\psi} \\
      \ut{\phi \lor \psi} &= \ut{\phi} + \ut{\psi} \\
      \ut{\forall x \in A .\, \phi(x)} &= \ut{A} \to \ut{\phi} \\
      \ut{\exists x \in A .\, \phi(x)} &= \ut{A} \times \ut{\phi}
    \end{align*}
  }
  \vline
  \quad
  \parbox[t]{0.5\textwidth}{
    \begin{align*}
      () \rz \top &
      \\
      () \rz x =_S y
        &\quad\text{iff}\quad 
      x \per{S} y
      \\
      (t_1,t_2) \rz \phi \land \psi
        &\quad\text{iff}\quad
        \text{$t_1 \rz \phi$ and $t_2 \rz \psi$}
      \\
      t \rz \phi \implies \psi
        &\quad\text{iff}\quad
        \text{for all $u \in \ut{\phi}$, if $u \rz \phi$ then $t\,u
          \rz \psi$}
      \\
      \inl{t} \rz \phi \lor \psi
        &\quad\text{iff}\quad
        \text{$t \rz \phi$}
      \\
      \inr{t} \rz \phi \lor \psi
        &\quad\text{iff}\quad
        \text{$t \rz \psi$}
      \\
      t \rz \forall x \in A . \phi(x)
        &\quad\text{iff}\quad
        \text{for all $u \in \ut{A}$, if $u \rz_A x$ then $t\,u \rz \phi(x)$}
      \\
      (t_1, t_2) \rz \exists x \in A . \phi(x)
        &\quad\text{iff}\quad
        \text{$t_1 \rz_A x$ and $t_2 \rz \phi(x)$}
    \end{align*}
  }
  \caption{Realizability interpretation of logic (outline)}
  \label{fig:rz-logic}
\end{figure*}

\begin{definition}
  A \emph{modest set}\footnote{Modest sets were so named by Dana
    Scott. They are ``modest'' because their size cannot exceed the
    number of expressions of the underlying datatype.} is a triple
  $(S, \ut{S}, {\rz_S})$ where $S$ is a set, $\ut{S}$ is a type and
  $\rz_S$ is a relation between expressions of type~$\ut{S}$ and
  elements of~$S$, satisfying:
  % 
  \begin{enumerate}
  \item For every $x \in S$ there is $t \in \ut{S}$ such that $t \rz_S
    x$.
  \item If $t \rz_S x$ and $t \rz_S y$ then $x = y$.
  \end{enumerate}
  %
  A \emph{realized function} $f : (S, \ut{S}, {\rz_S}) \to (T, \ut{T},
  {\rz_T})$ between modest sets is a function $f : S \to T$ for which
  there exists $u \in \ut{S} \to \ut{T}$ such that
  %
  \begin{equation*}
    t \rz_S x \implies u\,t \rz_T f(x) \;.
  \end{equation*}
  %
  We say that $u$ \emph{realizes}~$f$.
\end{definition}

The realizer~$u$ of a realized function~$f$ is more commonly known as
an ``implementation of~$f$'' or an ``algorithm for computing~$f$''.

Modest sets and realized functions form a category of \emph{modest
  sets~$\Mod{\PL}$}. In realizability theory this is a well known
category with good properties. It is regular and locally bi-cartesian
closed, which allows us to interpret first-order logic and a rich type
theory. Here we only outline the main ideas behind the realizability
interpretation of logic. See e.g.~\cite{Bauer:00} for details.

In the realizability interpretation of logic, each formula~$\phi$ is
assigned a set of \emph{realizers} which can be thought of as
computations that witness the validity of~$\phi$. The situation is
somewhat similar (but not equivalent) to the propositions-as-types
translation of logic into type theory, where the proofs of a
proposition correspond to terms of the corresponding type. More
precisely, to each formula~$\phi$ we assign an underlying type
$\ut{\phi}$ of realizers. However, unlike in the propositions-as-types
translation, not all terms of type $\ut{\phi}$ are necessarily valid
realizers for~$\phi$. We write $t \rz \phi$ when $t \in \ut{\phi}$ is
a realizer for~$\phi$. The underlying types and the
realizability relation~$\rz$ are defined inductively on the structure
of~$\phi$; an outline is shown in Figure~\ref{fig:rz-logic}. We say that a
formula~$\phi$ is \emph{valid} in~$\Mod{\PL}$ if it has at least one
realizer.

We shall not dwell any further on the technicalities involving the
category of modest sets, but rather proceed to a concrete description
of our realizability translation. There is one technical point,
though, which we first take care of. A modest set is a triple $(S,
\ut{S}, {\rz_S})$ in which~$S$ is an arbitrary set. For an automated
system it would be convenient if it did not have to refer to arbitrary
sets but rather just to ingredients that are already present in the
programming language, such as types and sets of expressions. Up to
equivalence of categories, modest sets can be construed as triples
$(\ut{S}, \tot{S}, {\per{S}})$ where $\ut{S}$ is a type, $\tot{S}$ is
a subset of expressions of type~$\ut{S}$, called the \emph{total
  values},\footnote{We do \emph{not} require that a total value must
  be a terminating expression.} and $\per{S}$ is an equivalence
relation on~$\tot{S}$. The relationship between this representation of
a modest set and the original one is as follows:
%
\begin{itemize}
\item $\tot{S}$ is the set of those $t \in \ut{S}$ that
  realize something, i.e., there is $x \in S$ such that $t \rz_S x$.
  These correspond to implementations that satisfy
  the representation invariant, e.g., graphs where the list of edges
  mentions only integers in the list of nodes, a subset of
  all values of type $\mathtt{int} \; \mathtt{list} * (\mathtt{int} *
\mathtt{int}) \; \mathtt{list}$.
\item $t \per{S} u$ if $t$ and $u$ realize the same element, i.e.,
  there is $x \in S$ such that $t \rz_S x$ and $u \rz_S x$.
  This relation equates alternate concrete representations of the same
  abstract value, e.g., equating two concrete graph representations differing
  only in the order of the nodes or the order of the edges.
\end{itemize}
%
The alternative view of a modest set $(\ut{S}, \tot{S}, {\per{S}})$
only refers to objects and concepts from the programming language. It
is better suited for our purposes.

Note that the equivalence relation on~$\tot{S}$ is also a
\emph{partial} equivalence relation on~$\ut{S}$, which shows that
modest sets are in fact equivalent to PER models.


%%% Local Variables: 
%%% mode: latex
%%% TeX-master: "case"
%%% End: 



\section{Theories and Signatures}
\label{sec:theories-signatures}

\begin{figure*}[t]
\[
\begin{array}[t]{l}
\mbox{\textbf{Theory Elements}}\\
\SET\ s\ [\EQUALS\ \mathit{set}\ ]\\
\CONST\ c\ [\COLON \mathit{set}\ ]\ [\EQUALS\ \mathit{term}\ ]\\
{}[\STABLE{}]\ \RELATION\ r\ [\COLON \mathit{set}\ ]\ [\EQUALS \mathit{prop}\ ]\\
\EQUIVALENCE \COLON \mathit{set}\\
\MODEL\ M\COLON\mathit{theory}\\
\AXIOM\ a\ [\ M\COLON\mathit{theory}\ ]^{*}\ [x\COLON\mathit{set}\ ]^*\EQUALS\mathit{prop}\\
\\
\mbox{\textbf{Propositions}}\\
\TRUE\\
\FALSE\\
\NOT\ \Prop\\
\Prop \AAND \Prop\\
\Prop \OOR \Prop\\
\Prop \IIMPLY \Prop\\
\Prop \IIFF \Prop\\
r [\ \Term\ ]^*\\
\Term\EQUALS\Term\\ %[\IN\ \Set]\\
\ALL\ [x\COLON \Set] \PERIOD \Prop\\
\SOME\ [x\COLON \Set] \PERIOD \Prop\\
\UNIQUE\ [x\COLON \Set] \PERIOD \Prop\\
\end{array}
\qquad
\begin{array}[t]{l}
\mbox{\textbf{Sets}}\\
\ZERO\\
\ONE\\
\BOOL\\
s\\
\mathit{Model}\PERIOD \mathit{name}\\
\Set \TIMES \cdots \TIMES \Set\\
\Setexp \ARROW \Setexp\\
\Label\ [\COLON \Setexp\ ]\ \PLUS \cdots \PLUS \Label\ [\COLON \Setexp\ ]\\
\LBRACE \Ident [\ \COLON \Setexp\ ]\ \BAR \Proposition \RBRACE\\
\Setexp \PERCENT \metav{relation}\\
%\RZ\ \Setexp\\
\\
\mbox{\textbf{Terms}}\\
x\\
\LPAREN \Term\COMMA \cdots \COMMA \Term \RPAREN\\
\Term\PERIOD \metav{n}\\
\Label\ [\ \Term\ ]\\
\MATCH\ \Term\ \WITH\ \mbox{\textit{pattern-matches}}\\
\LAMBDA\ x\COLON\Set\ \PERIOD\ \Term\\
\Term\ \Term\\
\Term \PERCENT \metav{relation}\\
\LET\ x \PERCENT \metav{relation}\ \IN\ \Term\EQUALS\Term\\
\Term \SUBIN \Set\\
\Term \SUBOUT \Set\\
\THE\ x\ [\COLON\Set\ ] \PERIOD \Prop\\
\LET\ x\ [\COLON\Set\ ] \EQUALS\Term\ \IN\ \Term\\
\end{array}
\]  
\caption{Input Language Summary}
\label{fig:input}  
\end{figure*}

In this section we describe first-order theories and signatures.
Our system translates the former into the later.

\subsection{Theories}
\label{sec:theories}

A \emph{theory} is a description of a mathematical structure, such as
a group, a vector space, a directed graph, etc. A theory consists of
%
\begin{itemize}
\item a list of \emph{basic sets},
\item a list of \emph{basic constants} belonging to specified sets,
\item a list of \emph{basic relations} on specified sets,
\item a list of axioms.
\end{itemize}
%
To take a simple example, consider the theory of a semigroup in which
every element has a (possibly non-unique) square root; recall that a
semigroup is a set with an associative binary operation and a neutral
element.\footnote{An example of a semigroup with square roots is the
  complex numbers with multiplication as the binary operation.} In our
system it could be written as follows:
%
\VerbatimInput{semigroup.thy}
%
The theory is enclosed by \Verb|thy|\ldots\Verb|end|. This theory
defines one basic set \Verb|s|, and two basic constants: an element
\Verb|e| of \Verb|s| and a (curried) binary infix operator \Verb|*| on
the set \Verb|s|. The \Verb|implicit| operator is not part of the
theory proper, but signals to the type checker that bound
variables named \Verb|x| or \Verb|y| or \Verb|z| should be assumed to
range over \Verb|s| unless otherwise specified. Finally,
we have three axioms. Axiom arguments, e.g., \Verb|x|, \Verb|y|, and
\Verb|z| in the associativity axiom, name the free variables occuring
in the axiom. It is not too big a mistake to think of them as being
universally quantified.

It is important to note that theories do not include proofs, but
rather just the statements of the axioms (and theorems) specified to
hold. Thus although axioms can be defined, one cannot actually refer
to them within the theory.

There are several features of theories that our system supports other
than those shown in this example above; the input language is
summarized in Figure~\ref{fig:input}, where brackets imply optional
elements.


Theories may declare or define relations. They may be \Verb|stable|,
i.e., their computational interpretation is trivial (see
Section~\ref{sec:implementation} for further discussion of this
point). Axioms can universally quantify over all models of a theory.
This is useful for describing universaility properties, such as
initiality of an algebra or finality of a coalgebra.
  
The propositions are the familiar ones from first-order logic;
$\UNIQUE$ is unique existence ($\exists!$). In addition to the basic
empty ($\ZERO$) and unit ($\ONE$) sets, one can form cartesian
products, function spaces, tagged disjoint unions, subsets, and
quotients by stable equivalence relations. The corresponding
introduction and elimination forms appear in the language of terms.
For example, $\Term \PERCENT \metav{relation}$ is the equivalence
class under $\metav{relation}$ containing $\Term$, while $\LET\ x
\PERCENT \metav{relation} \mbox{\Verb| = |} \Term_1\ \IN\ \Term_2$
binds $x$ to a representative of the equivalence class $\Term_1$ to be
used in $\Term_2$. The expression $\Term \SUBIN \Set$ injects $\Term$
into a given subset (recording a proof obligation of the term actually
being a member of the subset), while $\Term \SUBOUT \Set$ projects
$\Term$ from a subset out into its superset $\Set$. The value of the
description operator $\THE\ x \,.\, \Prop$ is the unique $x$
satisfying $\Prop$; using it incurs the obligation of proving that
there is exactly one such~$x$.


\subsection{Signatures}
\label{sec:signatures}

On the logical side, we have models described by theories.  Thus on
the programming side we should have implementations being described by
specifications.  Our tool thus translates theories into
\emph{signatures}, which are ML's module interfaces.

Signatures allow us to require the existence of certain types, as well
as values of given type.  This allows decidable typechecking, but we
need more expressiveness in order to faithfully translate the content
of a theory.  We therefore generate signatures augmented by assertion
comments, which specify constraints on the values and functions an
implementation beyond their type.  It is the responsibility of the
programmer to check that the implementation satisfies these
assertions, as RZ does not attempt to do any theorem proving.

Assertions are written in ordinary classical first-order logic. Since
programmers typically are not trained in constructive logic, this may
make it easier to verify the assertions.

The output for the theory \Verb|SQGROUP| above is then:
\VerbatimInput{semigroup.mli} 

At the ML level we have required a type \Verb|s|, a value \Verb|e| of
type \Verb|s| and a binary function on \Verb|s|; \comment{I thought
  that names like \Verb|( * )| here were supposed to be translated
  away in the ML code?} these were mentioned explicitly in the theory.
For interestingly, the constructive content of the existential in the
\Verb|sqrt| axiom requires that a witness function be provided as
well; thus, the ML signature further requires a function \Verb|sqrt|
mapping \Verb|s| to \Verb|s|.

Comments contain other properties (not expressible in ML) that further
constrain the allowed implementations.  The proposition \Verb|PER(=s=)| 
abbreviates a requirement that \Verb|=s=| be a partial equivalence 
on \Verb|s|; its domain \Verb+||s||+ is the
subset of terms of type $s$ that represent group elements, and the
relation equates different implementations corresponding to the same
abstract group element.  Then, \Verb|e| must represent a
group element, and the \Verb|*| operation must not be affected by the
choice of representatives.  These must of course still satisfy the
\Verb|unit| and \Verb|assoc| axioms.  Finally, the new function
\Verb|sqrt| derived from the logic must compute square roots.  Since
the theory requires existence but not uniqueness of square roots, its
computational content does not require that \Verb|sqrt| be invariant
with respect to the partial equivalence relation on \Verb|s|;
different implementations of the same group element are allowed to produce
(implementations of) different square roots.  


\subsection{Parameterized Theories}
\label{sec:param-theor-funct}

A theory may be parameterized by one or more models of other theories.
For example, a theory \Verb|Real| of the reals may be parameterized in
terms of a model \Verb|N| of the naturals.  A theory of free groups may be
parameterized in terms of the generating set.

Parameterized theories serve two purposes.  First a model of a
parameterized theory would be a generic implementation that, given any
implementation of the parameters, returns an implementation of the
resulting theory.  At the level of ML, this would be a function from
modules to modules, which is called a \emph{functor}, and so a
parameterized theory can be translated into the signature of a
functor.

Alternatively, once we have described a parameterized theory
\Verb|Real|, we may wish to use it to describe a single specific
implementation of real numbers based on a specific model \Verb|N1|
(implementation) of the natural numbers; this can be described as
an implementation satisfying the theory \Verb|Real(N1)|.

The dual nature of parameterized theories as being both a description
of a parameterized model (a $\Pi$ type) and something which can be
applied to a model to produce a specialized theory (a $\lambda$) is
very reminiscent the type inclusion of Automath~\cite{automath}.  ML
does not permit applications of functor signatures, however, so we
beta-reduce all theory applications before generating signatures;
\Verb|Real(N1)| would produce a signature for a real-number
implementation that refers directly to \Verb|N1| rather than to a
generic parameter \Verb|N|.


%%% Local Variables: 
%%% mode: latex
%%% TeX-master: "case"
%%% End: 


\section{Implementation}
\label{sec:implementation}

The RZ implementation consists of several sequential passes.

After the initial parsing, a \emph{type reconstruction} phase checks
that the input is well-typed (and checks for well-formedness to the
extent that it is easily decidable), and if successful produces an
annotated result with all variables explicitly tagged with types. The
type checking phase uses a system of dependent types, with limited
subtyping (implicit coercions) for sum types and subset types. The
details are fairly standard, so are omitted here. One non-obvious
consequence of the realizability translation, however, is that the
subset types $\isubset{\ix}{\iS}{\iand{\ipp_1(\ix)}{\ipp_2(\ix)}}$ and
$\isubset{\ix}{\iS}{\iand{\ipp_2(\ix)}{\ipp_1(\ix)}}$ are not
equal, but only isomorphic in general. An
explicit coercion is required to go from one type to the other,
because subset values are pairs containing realizers for
$\iand{\ipp_1(\ix)}{\ipp_2(\ix)}$ and
$\iand{\ipp_2(\ix)}{\ipp_1(\ix)}$, and these realizers have
potentially different types $|\ipp_1(\ix)|\mathtt{*}|\ipp_2(\ix)|$ and
$|\ipp_2(\ix)|\mathtt{*}|\ipp_1(\ix)|$ respectively.

Next the realizability translation is performed as described in
Section~\ref{sec:translation}, producing interface code. The
flexibility of the full input language (e.g., $n$-ary sum types and
dependent product types) makes the translation code fairly involved,
and so it is performed in a ``naive'' fashion whenever possible. The
immediate result of the translation is not easily readable.
 
\internal{Chris}{Uh oh...Kuratowski has been moved to a later section.
Should we move this section after
the examples?}

Thus, three more passes simplify the output before it is displayed to
the user. A \emph{thinning} pass removes all references to trivial
realizers produced by stable formulas. For example, direct translation
of the $\mathtt{free}$ axiom in the output for Kuratowski-finite sets
yields a value specification for $\mathtt{free}$ of type
%
\begin{equation*}
  (\mathtt{A.a} \to \mathtt{S.s}) \to 
  (\mathtt{fin} \to \mathtt{S.s}) * (\ounit * (\mathtt{A.a} \to
  \ounit) *
  (\mathtt{fin} \to \mathtt{fin} \to \ounit))
\end{equation*}
%
where $\ounit$ is the unit (terminal) type classifying the trivial
realizer. Thinning replaces this by the isomorphic type
%
\begin{equation*}
  (\mathtt{A.a} \to \mathtt{S.s}) \to \mathtt{fin} \to \mathtt{S.s}
\end{equation*}
%
and appropriately modifies references to $\mathtt{free}$ in the assertions to account for this change in type.

Next, an \emph{optimization} pass applies an ad-hoc collection of
basic logical and term simplifications in order to make the output more readable. 
Logical simplifications include applications of truth table rules
($\iand{\itrue}{\ip}$ becomes $\ip$), detection of syntactically
identical premises and conclusions
($\iimply{\ip_1}{\iand{\ip_1}{\ip_2}}$ becomes
$\iimply{\ip_1}{\ip_2}$), and optimization of other common patterns we have
seen arise
($\iforall{\ix}{\is}{\iimply{(\iequal{\ix}{\ie})}{\ipp(\ix)}}$ becomes
$\ipp(\ie)$). We do not attempt real theorem proving 
so some redundancy may remain, but in practice the optimization pass
can help significantly.

Finally, the user can specify whether two optional steps occur.
RZ can optionally performs a \emph{phase-splitting} pass~\cite{harper+:popl90}. 
This is an ML-specific transformation that replace certain
uses of parameterized modules (a heavyweight language construct) by
parameterized types and polymorphic values. The idea is that although
functors map modules containing types and terms to other modules containing types
and terms, constraints on the programming language ensure that output types
depend only on input types (and not input terms).  Thus, we can split each
functor into a mapping from input types to output types, and then a separate
(polymorphic) term mapping input types and terms to an output term.

For example (ignoring
assertions for simplicity) the entire module
\begin{source}
module Free : functor (S : Semilattice) ->
                    sig
                      val free : (A.a -> S.s) -> fin -> S.s
                    end	
\end{source}   
appearing in the output of the Kuratowski example can be replaced by the single polymorphic function
\begin{source}
val free : 's -> ('s -> 's -> 's) -> (A.a -> 's) -> fin -> 's	
\end{source}
which replaces the module parameter \texttt{S} by two extra term arguments term (corresponding to the module components \texttt{S.zero} and \texttt{S.join}) 
and a type argument \texttt{'s} for the type of lattice elements (corresponding to the module input \texttt{S.s}).

The other optional transformation is a \emph{hoisting} pass which lifts obligations in the output out to top-level positions.  This can make it easier to see exactly what one is obliged to provide.  When identical obligations appear in separate subterms of a term, hoisting can lift and merge these obligations, reducing redundancy.  However, moving obligations far from where they are used can make it harder to see why the obligation is required at all (and hence how one might satisfy the obligation), and so hoisting is turned off by default.

For example, in the following input (extracted from a larger description of an ordered field)
\begin{source}
Parameter s : Set.
Parameter zero : s.
Parameter inverse : {x : s | not (x = zero)} -> s.

Parameter lt : s -> s -> Stable.
Definition positive (x:s) := lt zero x.
Axiom lt_irr: forall x:s, not (lt x x).

Axiom order_inv: forall x:s, positive x -> positive (inverse x).
\end{source}
the axiom translates to the assertion:
\begin{source}
	(**  Assertion order_inv = 
          forall (x:||s||),  positive x ->
            positive (inv (assure (not (x =real= zero)) in x))
   *).
\end{source}
Here the system has noticed that for \Verb|inverse x| to make sense, we must know that
\Verb|x| is non-zero.   This requires non-trivial theorem proving and hence remains as 
an obligation for the user.  

We must prove \Verb|not (x =s= zero)| not for all \Verb|x|, but only under
the premises in force where the obligation occurs.  This is slightly clearer when hoisting moves the
obligation to the top level, after which it could be verified in the same way as all other assertions:
\begin{source}
   (**  Assertion order_inv = 
          assure (forall (x:||real||),  positive x -> not (x =real= zero))
            in forall (x:||real||),  positive x -> positive (inv x)
   *)	
\end{source}


%%% Local Variables: 
%%% mode: latex
%%% TeX-master: "cie"
%%% End: 


\section{Examples}
\label{sec:examples}

In this section we look at several examples which demonstrate various
points of RZ. Unfortunately, serious examples from computable
mathematics take too much space\footnote{The most basic structure
  in analysis (the real numbers) alone 
  requires several operations and a dozen or more axioms.} and will have to
be presented separately. The main theme is that constructively
reasonable axioms yield computationally reasonable operations.

\subsection{Decidable sets}
\label{sec:decidable-sets}

A set $S$ is said to be decidable when, for all $x, y \in S$, $x = y$
or $\lnot (x = y)$. In classical mathematics all sets are decidable, 
\iflong
because decidability of equality is just an instance of the law of
excluded middle.  But
\else
but
\fi % \iflong
RZ requires an axiom
%
\begin{source}
Parameter s : Set.
Axiom eq: \iForall x y : s, x = y \iOr \iNot (x = y).
\end{source}
%
to produce a realizer for equality
%
\begin{source}
val eq : s \iTo s \iTo [`or0 | `or1]
Assertion eq = \iForall (x:\iT{s}, y:\iT{s}),
                   (match eq x y with
                      `or0 \iImply x \iPer{s} y
                    | `or1 \iImply \iNot (x \iPer{s} y) )
\end{source}
%
We read this as follows: $\f{eq}$ is a function which takes
arguments~$\f{x}$ and~$\f{y}$ of type~$\f{s}$ and returns
$\mathtt{`or0}$ or $\mathtt{`or1}$. If it returns $\mathtt{`or0}$,
then $\oper{\f{s}}{\f{x}}{\f{y}}$, and if it returns
$\mathtt{`or1}$, then $\onot{(\oper{\f{s}}{\f{x}}{\f{y}})}$. In
other words $\f{eq}$ is a decision procedure%
\iflong
which tells when
values~$\f{x}$ and~$\f{y}$ represent the same element of the modest
set.
\else % \iflong
.
\fi % \iflong

\iflong
\subsection{Examples with obligations}
\label{sec:exampl-with-oblig}

In this section we show how RZ produces obligations, is sometimes able
to optimize them away, and show the effect of hoisting.

Consider how we might define division of real numbers. Assuming the
set of real numbers~$\f{real}$, a constant $\f{zero}$, and
multiplication operation~$\f{*}$ have already been declared and
axiomatized, we might write:
%
\begin{source}
Definition nonZeroReal := \{x : real | \iNot (x = zero)\}.
Parameter inv : nonZeroReal \iTo real.
Axiom inverse : \iForall x : real, \iNot (x = zero) -> x * (inv x) = one.
Definition (/) (x : real) (y : nonZeroReal) := x * (inv y).
\end{source}
%
We have defined the set of non-zero reals $\f{nonZeroReal}$ and
the inverse operation~$\f{inv}$ on it. Division $\f{x/y}$ is defined
as $\f{x * inv\; y}$. This does \emph{not} mean that the
programmer must necessarily implement division this way, only that the
implementation of $\f{x/y}$ must be equivalent to $\f{x * inv\;y}$.

In the axiom $\f{inverse}$, RZ encounters the subexpression
$\f{inv \;x}$. Because $\f{x}$ is quantified as an element of
$\f{real}$ rather than $\f{nonZeroReal}$, the typechecking
phase inserts a coercion that makes the expression well-typed.
Translation sees $\f{inv}(\f{x} \mathbin{{:}}
\f{nonZeroReal})$ instead of $\f{inv\ x}$ and translates this to
%
\begin{source}
inv (assure u:unit . \iNot (x \iPer{real} zero) in (x, u))
\end{source}
%
If this were the final output, the programmer would have to verify
that~$x$ is not zero, and provide a trivial realizer for it. However,
in this case the thinning phase first removes the trivial realizer,
%
\iflong
\begin{source}
inv (assure \iNot (x \iPer{real} zero) in x)
\end{source}%
\fi % \iflong
%
and then the optimizer determines that the obligation is not needed
because the whole expressions appears under the hypothesis that~$x$ is
not zero. So in the end the programmer sees
%
\begin{source}
(**  Assertion inverse =
  \iForall (x:\iT{real}),  \iNot (x \iPer{real} zero) \iTo (x * inv x) \iPer{real} one
*)
\end{source}
%
Assuming further that a strict linear order $<$ on~$\f{real}$ has
been axiomatized, we might proceed by relating it to $\f{inv}$:
%
\begin{showInput}
\iAxiom{\f{inv\_positive}}{
\iforall{\f{x}}{\f{real}}{\oimply{\f{zero} < \f{x}}{\f{zero} < \f{inv\ x}}}
}.
%Axiom inv_positive: forall x : real, zero < x -> zero < inv x.
\end{showInput}%
%
Once again $\f{inv\;x}$ appears in the input, bt this time the
optimizer is unable to remove the obligation, so the output is
%
\begin{showOutput}
\oassertion{\f{inv\_positive}}{}\\
\qquad
\oforallt{\f{x}}{\f{real}}{
  \oimply{
    \f{zero} < \f{x}}
  {\f{zero} < \f{inv}
    (\oobligx{\onot{(\oper{\f{real}}{\f{x}}{\f{zero}})}}{\f{x}})}
}
%(**  Assertion inv_positive =
%       forall (x:||real||),  zero < x ->
%         zero < inv (assure (not (x =real= zero)) in x)
%*)
\end{showOutput}
%
Local obligations can sometimes be hard to read, but if we activate the hoisting phase
(see Section~\ref{sec:implementation}), the obligation can be moved
to the top level. As this is done, the hypotheses under which the
obligation appears are collected, and we get
%
\begin{showOutput}
\oassertion{\f{inv\_positive}}{}\\
\qquad
\oobligx{
\oforallt{\f{x}}{\f{real}}{
\oimply{\f{zero}<\f{x}}{\onot{(\oper{\f{real}}{\f{x}}{\f{zero}})}}
}
}{}\\
\qquad\qquad{
\oforallt{\f{x}}{\f{real}}{
  \oimply{
    \f{zero} < \f{x}}
  {\f{zero} < \f{inv\ x}}
}}
%(**  Assertion inv_positive =
%  assure (forall (x:||real||),  zero < x -> not (x =real= zero))
%    in forall (x:||real||),  zero < x -> zero < inv x
%*)
\end{showOutput}%
%
Now it is easier to understand what must be checked, namely that
positive reals are not zero---an easy consequence of irreflexivity
of~$<$, but not something that RZ optimizer is aware of.

Lastly, we could define the golden ratio as the positive solution of
$x^2 = x + 1$,
%
\begin{source}
the x : real, (zero < x /\ x*x = x + one)
\end{source}
%
Not surprisingly, RZ cannot determine that there is a unique such~$x$,
so it outputs an obligation:
%
\begin{source}
assure x:real.
  (x : ||real|| /\ zero < x /\ x * x =real= x + one /\
     (forall (x':||real||),  zero < x' /\ x' * x' =real= x' + one ->
        x =real= x'))
  in x
\end{source}
\fi % \iflong

\iflong
\subsection{Finite sets}
\label{sec:finite-sets}

\begin{figure}[t]
\begin{showInputSmall}
\iDefinition{\f{Semilattice}}{\mathsf{thy}}\\
\qquad \iParameter{\f{s}}{\iSet}.\\
\qquad \iParameter{\f{zero}}{\f{s}}.\\
\qquad \iParameter{\f{join}}{\f{s}\to\f{s}\to\f{s}}.\\
\\
\qquad \iAxiom{\f{commutative}}{\forall \f{x},\f{y}:\f{s}.\ \f{join\ x\ y} = \f{join\ y\ x}}.\\
\qquad \iAxiom{\f{associative}}{\forall \f{x},\f{y},\f{z}:\f{s}.\ \f{join}\,(\f{join\ x\ y})\,\f{z} = \f{join}\,\f{x}\, (\f{join\ y\ z})}.\\
\qquad \iAxiom{\f{idempotent}}{\forall \f{x}:\f{s}.\ \f{join\ x\ x} = \f{x}}.\\
\qquad \iAxiom{\f{neutral}}{\forall \f{x}:\f{s}.\ \f{join\ x\ zero} = \f{x}}.\\
\mathsf{end}.
\end{showInputSmall}
  \caption{The theory of a semilattice}
  \label{fig:semilattice}
\end{figure}

\iflong
There are many characterizations of finite sets, but the one that
works best constructively is due to Kuratowski, who identified the
finite subsets of~$A$ as the least family~$K(A)$ of subsets of~$A$
that contains the empty set and is closed under unions with
singletons. This characterization relies on powersets, which are not
available in RZ. But the gist of it, namely that $K(A)$ is an
\emph{inital} structure a suitable sort, can be expressed as follows.

\else
%
The family $K(A)$ of finite subsets of a set~$A$ may be characterized
as the free $\vee$-semilattice generated by~$A$.
%
\fi
%
Recall that a \emph{$\vee$-semilattice} is a set~$S$ with a
constant~$0 \in S$ and an associative, commutative, and idempotent
operation ``join'' $\vee$ on~$S$ such that $0$ is the neutral element
for~$\vee$, see Figure~\ref{fig:semilattice} for RZ axiomatization of
semilattices.
%
The Kuratowski finite sets~$K(A)$ are the \emph{free} semilattice
generated by a set~$A$, where $\vee$ is union and $0$ is the empty
set. This is formalized in RZ as shown in Figure~\ref{fig:kuratowski}.
%
\begin{figure}
\begin{showInput}
\iDefinition{\texttt{K}\ (\texttt{A} : \f{thy}\ \  \iParameter{\f{a}}{\iSet}.\ \  \texttt{end})}{\f{thy}}\\
\qquad \iInclude{\f{Semilattice}}.\\
\qquad \iParameter{\f{singleton}}{\f{A.a} \to \f{s}}.\\
\qquad \iDefinition{\f{fin}}{\f{s}}.\\
\qquad \iDefinition{\f{emptyset}}{\f{zero}}.\\
\qquad \iDefinition{\f{union}}{\f{join}}.\\
\\
\qquad \iAxiom{\f{free}}{} \forall \f{S}:\f{Semilattice}.\ \forall\f{f}:{\f{A.a}\to\f{S.s}}.\ \exists!g:\f{fin}{\to}{\f{S.s}}.\\
\qquad \qquad \qquad \qquad \qquad \f{g\ emptyset} = \f{S.zero}\ \land\\
\qquad \qquad \qquad \qquad \qquad \forall\f{x}:\f{A.a}.\ \f{f\ x} = \f{g}\ (\f{singleton\ x})\ \land\\
\qquad \qquad \qquad \qquad \qquad \forall\f{u},\f{v}:\f{fin}.\ \f{g}(\f{union\ u\ v}) = \f{S.join}\,(\f{g\ u})\,(\f{g\ v}).\\
\f{end}.
\end{showInput}
  \caption{Kuratowski finite sets}
  \label{fig:kuratowski}
\end{figure}
%
The theory $\f{K}$ is parametrized by a model~$\f{A}$ which contains a
set~$\f{a}$. In the first line we include the theory of semilattices.
Then we postulate an operation $\f{singleton}$ which injects the
generators into the semilattice. The three definitions are just a
convenience, so that we can refer to the parts of $\f{K(A)}$ by their
natural names, e.g., $\f{emptyset}$ instead of $\f{zero}$. The axiom
$\f{free}$ expresses the fact that $\f{K}(\f{A})$ is the free
semilattice on~$\f{A.a}$: for every semilattice $\f{S}$ and a map
$\f{f} : \f{A.a} \to \f{S.s}$ from the generators to the underlying
set of~$\f{S}$, there exists a unique semilattice homomorphism $\f{g}
: \f{fin} \to \f{S.s}$ such that $\f{f}(\f{x}) = \f{g}(\f{singleton\;
  x})$.

The output for $\f{Semilattice}$ and~$\f{K}$ specifies
values of suitable types for each declared constant and operation. All
axioms but the last one are equations and have straightforward
translations in terms of underlying pers. The output for the axiom
$\f{free}$ is shown in Figure~\ref{fig:free}.
%
\begin{figure}
  \centering
\begin{showOutputSmall}
\f{module}\ \f{Free} : \f{functor} (\f{S} : \f{Semilattice}) \to \f{sig}\\
\quad \ovalspec{\f{free}}{\oarrow{(\oarrow{\f{A.a}}{\f{S.s}})}{\oarrow{\f{fin}}{\f{S.s}}}}
\\
\quad
\oassertion{\f{free}}{}
\\
\quad\quad
\oforallt{\f{f}}{\oarrow{\f{A.a}}{\f{S.s}}}{
\olet{\f{g}}{\f{free\ f}}{}}\\
\qquad\qquad \ototal{\f{g}}{\oarrow{\f{fin}}{\f{S.s}}} \land \oper{\f{S.s}}{\f{g\ emptyset}}{\f{S.zero}} \land {}\\
\qquad\qquad (\oforallt{\f{x}}{\f{A.a}}{\oper{\f{S.s}}{\f{f\ x}}{\f{g(singleton\ x)}}}) \land {} \\
\qquad\qquad (\oforallt{\f{u},\f{v}}{\f{fin}}{\oper{\f{S.s}}{\f{g(union\ u\ v)}}{\f{S.join\ (g\ u)\ (g\ v)}}}) \land {} \\
\qquad\qquad
((\oforall{\f{h}}{\oarrow{\f{fin}}{\f{S.s}}}{}\\
\qquad\qquad\qquad
\ototal{\f{h}}{\oarrow{\f{fin}}{\f{S.s}}} \land
\oper{\f{S.s}}{\f{h\ emptyset}}{\f{S.zero}} \land {} \\
\qquad\qquad\qquad (\oforallt{\f{x}}{\f{A.a}}{\oper{\f{S.s}}{\f{f\ x}}{\f{h(singleton\ x)}}}) \land {} \\
\qquad\qquad\qquad
(\oforallt{\f{u},\f{v}}{\f{fin}}{\oper{\f{S.s}}{\f{h(union\ u\ v)}}{\f{S.join\ (h\ u)\ (h\ v)}}})) \Rightarrow {} \\
\qquad\qquad
\oforall{\f{x},\f{y}}{\f{fin}}{\oimply{\oper{\f{fin}}{\f{x}}{\f{y}}}{\oper{\f{S.s}}{\f{g\ x}}{\f{h\ y}}}})
%module Free : functor (S : Semilattice) ->
%sig
%val free : (A.a -> S.s) -> fin -> S.s
%(**  Assertion free = 
%forall (f:||A.a -> S.s||), 
%  let g = free f in g : ||fin -> S.s|| /\ 
%  g emptyset =S.s= S.zero /\ 
%  (forall (x:||A.a||),  f x =S.s= g (singleton x)) /\ 
%  (forall (u:||fin||, v:||fin||), g (union u v) =S.s= S.join (g u) (g v)) /\ 
%  (forall h:fin -> S.s,  h : ||fin -> S.s|| /\ 
%     h emptyset =S.s= S.zero /\ 
%     (forall (x:||A.a||), f x =S.s= h (singleton x)) /\ 
%     (forall (u:||fin||, v:||fin||), 
%        h (union u v) =S.s= S.join (h u) (h v)) ->
%     forall x:fin, y:fin,  x =fin= y -> g x =S.s= h y)
%*)
%end
\end{showOutputSmall}%
  \caption{Output of axiom $\texttt{free}$.}
  \label{fig:free}
\end{figure}
%
Because the axiom quantifies over all models~$\f{S}$ of the theory
$\f{Semilattice}$ its translation is a functor~$\f{Free}$ which
accepts an implementation of a semilattice~$S$ and yields a realizer
$\f{free}$ validating the axiom. The computational meaning of
$\f{free}$ is a combination map and fold operation, taking a map
$\f{f} : \f{A.a} \to \f{S.s}$ and a finite set~$\f{u} = \set{x_1,
  \ldots, x_n}$, and return $\f{f}(x_1) \vee \cdots \vee \f{f}(x_n)$,
where $\vee$ is the join operation on the semilattice~$S$.

Applying phase-splitting to this axiom yields the even simpler
specification
%
\begin{equation*}
\mathtt{val}\ \f{free} : \alpha \to (\alpha \to \alpha \to \alpha) \to (\f{A.a}\to\alpha) \to \f{fin} \to \alpha	
\end{equation*}
%
(with an appropriate assertion)
which replaces the module parameter \texttt{S} by two extra term arguments term (corresponding to the module components \texttt{S.zero} and \texttt{S.join}) 
and a type argument $\alpha$ for the type of lattice elements (corresponding to the module input \texttt{S.s}).  This is even
more recognizable as a folding operation over the set.


It is important to note that, in contrast to \texttt{fold} operators found in typical functional
languages, \texttt{free} is only expected to work for suitable \texttt{join} arguments (e.g., idempotent and order independent).  These
sets are not the typical finite-set data structure: there is no membership predicate, nor
is there a way to compute the size of a set.  There is no
assumption that equality is decidable for set elements; this permits
finite sets of  exact real numbers, for example.  Decidable equality
is required both for membership and for detecting
whether the same element has been added twice to the same set\footnote{The natural implementation would thus
be an unordered collection of elements, possibly with duplicates.}.

Some operations are nevertheless computable.  Using \texttt{free} one
can determine whether a finite set is empty.  In the case of a set of exact
real numbers, we cannot compute their sum, but we could compute maximum or minimum.

More common set implementations (e.g., the \texttt{Set} module in the OCaml standard library)
implement sets over values with either decidable total order; these could also be
formalized in RZ.
\fi % \iflong

\subsection{Inductive types}
\label{sec:inductive-types}

To demonstrate the use of dependent types we show how RZ handles
general inductive types, also known as W-types or general
trees~\cite{nordstroem90:_progr_martin_type_theor}. Recall that a
W-type is a set of well-founded trees, where the branching types of
trees are described by a family of sets $B = \set{T(x)}_{x \in S}$.
Each node in a tree has a \emph{branching type}~$x \in S$, which
determines that the successors of the node are labeled by the elements
of~$T(x)$.
%
\iflong
%
For example, to get non-empty binary trees whose leaves are
labeled by natural numbers, define
%
\begin{align*}
  S &= \set{\f{cons}} \cup \set{\f{leaf}(n) \such n \in \NN}
  \\
  T(\f{cons}) &= \set{\f{left}, \f{right}}
  \\
  T(\f{leaf}(n)) &= \emptyset.
\end{align*}
%
Then a node of type $\f{cons}$ has two successors, indexed by
constants $\f{left}$ and $\f{right}$, while a node of type
$\f{leaf}(n)$ does not have any successors.
\par
%
\fi % iflong
%
Figure~\ref{fig:wtype} shows an RZ axiomatization of W-types.
%
\begin{figure}
\begin{source}
Parameter W : [B : Branching] \iTo
thy
  Parameter w : Set.
  Parameter tree : [x : B.s] \iTo (B.t x \iTo w) \iTo w.
  Axiom induction:
    \iForall M : thy Parameter p : w \iTo Prop. end,
    (\iForall x : B.s, \iForall f : B.t x \iTo w,
       ((\iForall y : B.t x, M.p (f y)) \iTo M.p (tree x f))) \iTo
    \iForall t : w, M.p t.
end.
\end{source}
  \caption{General inductive types}
  \label{fig:wtype}
\end{figure}
%
The theory $\f{Branching}$ describes that a branching type
consists of a set~$\f{s}$ and a set~$\f{t}$ depending on~$\f{s}$. The theory~$\f{W}$ is
parameterized by a branching type~$\f{B}$. It specifies a set~$\f{w}$ of
well-founded trees and a tree-forming operation $\f{tree}$ with a
dependent type $\Pi_{\f{x} \in \f{B.s}} (\f{B.t(x)} \to \f{w}) \to \f{w}$.
%
\iflong
%
Given a
branching type~$\f{x}$ and a map $\f{f} : \f{B.t(x)} \to \f{w}$, $\f{tree}\;\f{x}\;\f{f}$
is the tree whose root has branching type~$\f{x}$ and whose successor
labeled by $\ell \in \f{B.t}(\f{x})$ is the tree~$\f{f}(\ell)$.
%
\fi
%
The inductive nature of~$\f{w}$ is expressed with the axiom
$\f{induction}$, which states that for every property $\f{M.p}$, if $\f{M.p}$
is an inductive property then every tree satisfies it. A property is
said to be \emph{inductive} if a tree $\f{tree}\;\f{x}\;\f{f}$ satisfies it
whenever all its successors satisfy it.

\iflong
In the translation, see Appendix~\ref{sec:outp-induct-types} for a
complete output, dependencies at the level of types and terms disappear.
\else
In the translation dependencies at the level of types and terms disappear.
\fi
%
A branching type is determined by a pair of non-dependent types~$\f{s}$
and~$\f{t}$ but the per $\per_{\f{t}}$ depends on~$\values{\f{s}}$. The theory~$\f{W}$
turns into a signature for a functor receiving a branching type~$\f{B}$
and returning a type~$\f{w}$, and an operation $\f{tree}$ of type
$\f{B.s} \to (\f{B.t} \to \f{w}) \to \f{w}$.  One can use phase-splitting
to translate axiom
$\f{induction}$ into a specification of a polymorphic function
%
\begin{equation*}
  \f{induction:
  (B.s \to (B.t \to w) \to (B.t \to \poly{\alpha}) \to \poly{\alpha}) \to w \to \poly{\alpha}},
\end{equation*}
%
which is a form of recursion on well-founded trees. Instead of trying
to explain what $\f{induction}$ is supposed to do, we show a surprisingly simple,
hand-written implementation of W-types. The reader may enjoy figuring out how it works:
%
\sourcefile{wtype.ml}


\subsection{Axiom of choice}
\label{sec:axiom-choice}

RZ can help explain why a generally
accepted axiom is not constructively valid. Consider the Axiom of
Choice:
%
\begin{source}
Parameter a b : Set.
Parameter r : a \iTo b \iTo Prop.
Axiom ac: (\iForall x : a, \iExists y : b, r x y) \iTo
          (\iExists c : a \iTo b, \iForall x : a, r x (c x)).
\end{source}
%
The relevant part of the output is
%
\begin{source}
val ac : (a \iTo b * ty_r) \iTo (a \iTo b) * (a \iTo ty_r)
Assertion ac =
  \iForall f:a \iTo b * ty_r,
    (\iForall (x:\iT{a}),  let (p,q) = f x in p : \iT{b} \iAnd r x p q) \iTo
    let (g,h) = ac f in
      g : \iT{a \iTo b} \iAnd (\iForall (x:\iT{a}),  r x (g x) (h x))
\end{source}
%
This requires a function $\f{ac}$ which accepts a function $\f{f}$
and computes a pair of functions $\f{(g,h)}$. The input function~$\f{f}$ takes
an $\ototal{\f{x}}{\f{a}}$ and returns a pair $\f{(p,q)}$ such that $\f{q}$ realizes
the fact that $\f{r\;x\;p}$ holds. The output functions $\f{g}$ and $\f{h}$ taking
$\ototal{\f{x}}{\f{a}}$ as input must be such that $\f{h\;x}$ realizes
$\f{r\;x\;(g\;x)}$. Crucially, the requirement $\ototal{\f{g}}{\oarrow{\f{a}}{\f{b}}}$
says tht $\f{g}$ must be extensional, i.e., map equivalent realizers to
equivalent realizers. We could define~$\f{h}$ as the first component
of~$\f{f}$, but we cannot hope to implement~$\f{g}$ in general because the
second component of~$f$ is not assumed to be extensional.

The \emph{Intensional} Axiom of Choice allows the choice function to
depend on the realizers:
%
\begin{source}
Parameter a b : Set.
Parameter r : a \iTo b \iTo Prop.
Axiom iac: (\iForall x : a, \iExists y : b, r x y) \iTo
           (\iExists c : rz a \iTo b, \iForall x : rz a, r (rz x) (c x)).
\end{source}
%
Now the output is
%
\begin{source}
val iac : (a \iTo b * ty_r) \iTo (a \iTo b) * (a \iTo ty_r)
Assertion iac =
  \iForall f:a \iTo b * ty_r,
    (\iForall (x:\iT{a}),  let (p,q) = f x in p : \iT{b} \iAnd r x p q) \iTo
    let (g,h) = iac f in
      (\iForall x:a, x : \iT{a} \iTo g x : \iT{b}) \iAnd (\iForall (x:\iT{a}),  r x (g x) (h x))
\end{source}
%
which is exactly the same as before, \emph{except} that the
troublesome requirement $\ototal{\f{g}}{\oarrow{\f{a}}{\f{b}}}$ turned into
$\oforall{\f{x}}{\f{a}}{(\oimply{\ototal{\f{x}}{\f{a}}}{\ototal{\f{g\;x}}{\f{b}}})}$, which
is weaker. We can impement $\f{iac}$ as
%
\begin{source}
let iac f = (fun x -> fst (f x)), (fun x -> snd (f x))
\end{source}
%
The Intensional Axiom of Choice is in fact just an instance of the
usual Axiom of Choice applied to~$\irz{A}$ and~$B$. Combined with the
fact that~$\irz{A}$ covers~$A$, this establishes the validity of
\emph{Presentation Axiom}~\cite{barwise75:_admis_sets_struc}, which
states that every set is an image of one satisfying the axiom of
choice.

\subsection{Modulus of Continuity}
\label{sec:we-show-modulus-of-continuity-example}

As a last example we show how certain constructive principles require
the use of computational effects. To keep the example short, we
presume that we are already given the set of natural
numbers~$\f{nat}$ with the usual structure.

A \emph{type 2 functional} is a map $f : (\f{nat} \to \f{nat})
\to \f{nat}$. It is said to be continuous if the output of $f(a)$
depends only on an initial segment of the sequence~$a$. We can express
this axiom in RZ as follows:
%
\begin{source}
Axiom continuity: \iForall f : (nat \iTo nat) \iTo nat, \iForall a : nat \iTo nat,
  \iExists k, \iForall b : nat \iTo nat, (\iForall m, m \iLeq k \iTo a m = b m) \iTo f a = f b.
\end{source}
%
The axiom says that for any $\f{f}$ and $\f{a}$ there exists $\f{k} \in
\f{nat}$ such that $\f{f(b) = f(a)}$ as soon as the sequences~$\f{a}$
and~$\f{b}$ agree on the first $\f{k}$ terms. The axiom is translated to the
specification
%
\begin{source}
val continuity : ((nat \iTo nat) \iTo nat) \iTo (nat \iTo nat) \iTo nat
Assertion continuity =
  \iForall (f:\iT{(nat \iTo nat) \iTo nat}, a:\iT{nat \iTo nat}),
    let p = continuity f a in p : \iT{nat} \iAnd
    (\iForall (b:\iT{nat \iTo nat}),
       (\iForall (m:\iT{nat}),  m \iLeq p \iTo a m \iPer{nat} b m) \iTo f a \iPer{nat} f b)
\end{source}
%
which says that $\f{continuity\;f\;a}$ is a number~$\f{p}$ such that
$\f{f(a) = f(b)}$ whenever $\f{a}$ and $\f{b}$ agree on the first~$\f{p}$ terms. In
other words, $\f{continuity}$ is a \emph{modulus of continuity}
functional. It cannot be implemented in a purely functional
language,\footnote{There are models of $\lambda$-calculus which validate
  the choice principle~$AC_{2,0}$, but this principle contradicts the
  existence of a modulus of continuity functional,
  see~\cite[9.6.10]{Troelstra:van-Dalen:88:2}.} but with the use of
store we can implement it as
%
\begin{source}
let continuity f a =
  let p = ref 0 in
  let a' n = (p := max !p n; a n) in
    f a' ; !p
\end{source}
%
To compute a modulus for~$\f{f}$ at~$\f{a}$, the program creates a
function~$\f{a'}$ which is just like~$\f{a}$ except that it stores in~$\f{p}$ the
largest argument at which it has been called. Then $\f{f\;a'}$ is
computed, its value it discarded, and the value of~$\f{p}$ is returned.
The program works because~$\f{f}$ is assumed to be extensional and must
therefore not distinguish between extensionally equal sequences~$\f{a}$
and~$\f{a'}$.



%%% Local Variables: 
%%% mode: latex
%%% TeX-master: "cie"
%%% End: 


\section{Conclusions and Future Work}
\label{sec:conclusion}

By translating only at the level of specifications, \emph{RZ} provides a
useful middle ground between ad-hoc implementations and
machine-generated implementations --- allowing much more flexible
implementation strategies, but relying on the programmer to
verify properties of their code.

Further, \emph{RZ} can serve as a means of explaining constructive
mathematics to programmers.  Programmers who are not knowledgeable
about constructive mathematics can still understand the output of the
translation, which involves familiar concepts such as abstract types
and (classical) first-order logic.   Looking at such examples can
provide the necessary intuition behind the original logic, and better
explain why one might want to work with constructive rather than
classical logic to begin with.

\bigskip
 
Axioms parameterized by models (e.g., initiality) currently translate
into signatures of ML functors.  We have experimented with an
alternative translation of such axioms into polymorphic types.  In
this case the \Verb|inital| axiom of the natural numbers yields
the specification
\begin{Verbatim}
   val initial: 
      'a -> ('a -> 'a) -> N.s -> 'a
\end{Verbatim}
which is exactly the familiar iterator for natural numbers (i.e.,
given an initial value, a function, and a natural number, apply the
function that many times to the initial value).  Such types can be
much more natural and much simpler for programmers to understand.  The
theory behind the translation is well understood, being the
phase-splitting translation of Harper, Mitchell, and Moggi
\cite{harper+:popl90}.  Because of limitations of ML not every
parameterized axiom can be turned into polymorphism; ML allows only
prenex quantifiers, and the quantifiers can range over types but not
type operators.  However we would like to do so where it is possible
(the common case).  As an alternative, we could attempt to retarget
the output to a language like Haskell~\cite{haskell} which supports
the necessary polymorphic types, though Haskell's support of modules
is much weaker.

Another possible extension would be to allow dependent families in the
input language. Fortunately, this does not require finding a target
language that supports dependent types; we can use the underlying
(non-dependent) types, and then express the dependencies as additional
properties that must be verified for the implementation.


%%% Local Variables: 
%%% mode: latex
%%% TeX-master: "case"
%%% End: 


%%%%%%%%%%%%%%%%%%%%%%%%%%%%%%%%%%%%%%%%%%%%%%%%%%
% Bibliography

\bibliographystyle{plain}

\bibliography{case}


\end{document}

%%% Local Variables: 
%%% mode: latex
%%% TeX-master: t
%%% End: 
