\documentclass{article}

\begin{document}

\title{Automatic derivation of data structures \\
  from computable mathematics}
\author{Andrej Bauer\\
Faculty of Mathematics and Physics\\
University of Ljubljana, Slovenia\\
\emph{and}\\
Institute of Mathematics, Physics, and Mechanics\\
Ljubljana, Slovenia
}

\maketitle

We report on how to use the realizability interpretation of
constructive logic to automatically derive specifications from
axiomatizations of mathematical theories. For example, the
interpretation of the axioms of real numbers, when suitably
interpreted, gives a specification for exact real arithmetic.

There are tools which use this idea (or the related idea of
propositions-as-types) to automatically extract programs from formal
proofs, such as Coq~\cite{Coq} and Minlog~\cite{Minlog}. However, more
often than not the extracted programs are orders of magnitude slower
than hand-written versions, especially when complex mathematical
structures are involved.
%
It therefore makes sense to separate extraction of programs into two levels:
%
\begin{enumerate}
\item Extract \emph{specifications} for data structures and programs from
  definitions of structures and statements of theorems.
\item Extract \emph{implementations} of data structures and programs from
  constructions of structures and formal proofs of theorems.
\end{enumerate}
%
In joint work with Christopher Stone we developed a tool RZ~\cite{RZ,RZWeb}
which performs the first level of extraction automatically. It outputs
specifications as signatures in Objective Caml language~\cite{Ocaml}.
The extraction works on ``small scale'' (specification of data types
and values) as well as ``large scale'' (specification of whole program
modules). It uses Objective Caml module system to express a hierachy
of mathematical structures and connections between them. RZ performs a
number of optimizations and simplifications in order to output
readable and useful specifications.

In joint work with Iztok Kavkler~\cite{Era} we showed that the
extracted specifications are actually useful in practice. We
implemented exact real numbers Era following a specification produced
by RZ. Our implementation approaches the performance of the
state-of-the-art implementations of exact real numbers such as
RealLib~\cite{RealLib} and iRRAM~\cite{iRRAM}.

\begin{thebibliography}{10}\label{bibliography}

\bibitem{Era}
Bauer, A. and Kavkler, I.
\newblock Implementing real numbers with {RZ}.
\newblock In Weihrauch, K. and Zhong, N., editors, {\em Fourth International
  Conference on Computability and Complexity in Analysis}, Electronic Notes in
  Theoretical Computer Science, 2007.

\bibitem{RZ}
Bauer, A. and Stone, C.
\newblock {RZ}: a tool for bringing constructive and computable mathematics
  closer to programming practice.
\newblock In {\em Computability in Europe 2007}, June 2007.
\newblock To apear in a special issue of Journal of Logic and Computation.

\bibitem{RZWeb}
Bauer, A. and Stone, C., \emph{{RZ}}, \texttt{http://math.andrej.com/rz/}.

\bibitem{Minlog}
Benl, H., Berger, U., Schwichtenberg, H., Seisenberger, M., Zuber, W.:
\newblock Proof theory at work: Program development in the {M}inlog system.
\newblock In Bibel, W., Schmidt, P.H., eds.: Automated Deduction: A Basis for
  Applications. Volume {II}, Systems and Implementation Techniques.
\newblock Kluwer Academic Publishers, Dordrecht (1998)

\bibitem{Coq}
Bertot, Y., Cast\'eran, P.:
\newblock Interactive Theorem Proving and Program Development.
\newblock Springer (2004)

\bibitem{RealLib}
Lambov B.
\newblock {RealLib}: An efficient implementation of exact real arithmetic.
\newblock {\em Mathematical Structures in Computer Science}, 17:81--98, 2007.

\bibitem{Ocaml}
Leroy, X., Doligez, D., Garrigue, J., R\'emy, D., Vouillon, J.:
\newblock The {Objective Caml} system, documentation and user's manual -
  release 3.08.
\newblock Technical report, INRIA (July 2004)

\bibitem{iRRAM}
M\"{u}ller N.
\newblock The {iRRAM}: Exact arithmetic in {C}++.
\newblock In Jens Blanck, Vasco Brattka, and Peter Hertling, editors, {\em
  Computability and Complexity in Analysis: 4th International Workshop, CCA
  2000 Swansea, UK, September 17, 2000, Selected Papers}, number 2064 in
  Lecture Notes in Computer Science, pages 222--252. Springer, 2001.

\end{thebibliography}




\end{document}
