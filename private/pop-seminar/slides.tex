\documentclass[compress,t]{beamer}

\usepackage{palatino}
\usepackage{amsfonts}
\usepackage[all]{xypic}

\usepackage{fancyvrb}
% To show source
\DefineVerbatimEnvironment{source}{Verbatim}%
{fontsize=\small,commandchars=\\\{\}}

\DefineVerbatimEnvironment{sourcex}{Verbatim}%
{fontsize=\footnotesize,commandchars=\\\{\}}

% macros within source
\newcommand{\iTo}{{\ensuremath{\to}}}
\newcommand{\iForall}{$\forall$}
\newcommand{\iOr}{$\lor$}
\newcommand{\iAnd}{$\land$}
\newcommand{\iImply}{$\Rightarrow$}
\newcommand{\iNot}{$\lnot$}
\newcommand{\iExists}{$\exists$}
\newcommand{\iExistsOne}{$\exists!$}
\newcommand{\iT}[1]{$\|$#1$\|$}
\newcommand{\iPer}[1]{$\approx_{\mathtt{#1}}$}
\newcommand{\iIff}{$\leftrightarrow$}
\newcommand{\iLeq}{$\leq$}
\newcommand{\iAlpha}{$\alpha$}

\newcommand{\per}{\approx}
\newcommand{\typeOf}[1]{|#1|}

\title{How to Connect\\the Theory and Practice\\of Computable Mathematics}
\author{Andrej Bauer}
\institute{Department of Mathematics and Physics\\
University of Ljubljana, Slovenia
\and
{\normalsize Joint work with Iztok Kavkler and Chris Stone}
}
\date{CMU SCS, POP Seminar, April 2007}

%\beamerdefaultoverlayspecification{<+->}
\mode<presentation>
%\usetheme{Goettingen}
\usecolortheme{rose}
\usefonttheme{serif}
\setbeamertemplate{navigation symbols}{}

\begin{document}

\begin{frame}
    \titlepage
\end{frame}


\begin{frame}[fragile]
\frametitle{How would you implement a group?}
\begin{source}
Definition Group :=
thy
  Parameter s : Set.
  Parameter e : s.
  Parameter mul : s \iTo s \iTo s.
  Parameter inv : s \iTo s.
  Axiom associative :
    \iForall x y z : s, mul (mul x y) z = mul x (mul y z).
  Axiom neutral :
     \iForall x : s, mul e x = x \iAnd mul x e = x.
  Axiom inverse :
    \iForall x : s, mul (inv x) x = e \iAnd mul x (inv x) = e.
end.
\end{source}
\end{frame}

\begin{frame}[fragile]
\frametitle{A signature for a group implementation}
\begin{source}
module type Group =
 sig
   type s
   val e : s
   val mul : s \iTo s \iTo s
   val inv : s \iTo s
 end
\end{source}

\pause

The axioms should become \emph{assertions}, e.g.,
%
\begin{source}
assertion neutral :
  \iForall x : s, mul x one = x \iAnd mul one x = x.
\end{source}
%

But what do ``$=$'' and ``$\forall \mathtt{x} : \mathtt{s}$'' mean?
\end{frame}

\begin{frame}
  \frametitle{An implementation of graphs}
  
  \begin{tabular}{cl}
   \includegraphics[width=0.4\textheight]{digraph}
   &
   \parbox{0.5\textwidth}{
     \footnotesize
      $V = [1; 4; 3; 2]$,\\
      $E = [(1,2); (2,2); (3,1); (2,3); (3,2)]$
    }
  \end{tabular}

  \pause

  \begin{gather*}
    \mathtt{type\ graph\ =\ int\ list\ *\ (int * int)\ list}
  \end{gather*}

  \pause

  We also need a relation on values of type $\mathtt{graph}$
  % 
  \begin{equation*}
    (v_1,e_1) \per (v_2,e_2)
  \end{equation*}
  %
  meaning ``$(v_1, e_1)$ and $(v_2, e_2)$ represent the same graph''.
  Relation $\per$ is symmetric and transitive, a \emph{partial
    equivalence relation (per)}.
\end{frame}

\begin{frame}
  \frametitle{The category of pers}

  \begin{itemize}

  \item Object $(\tau, {\per_\tau})$:
    % 
    \begin{itemize}
    \item $\tau$ is an ML (monomorphic) type
    \item $\per_\tau$ is a per on values of type~$\tau$.
    \item define the \emph{support} $\|\tau\| = \{v : \tau \mid v
      \per_\tau v \}$.
    \end{itemize}
  \item Morphism $f : (\tau, {\per_\tau}) \to (\sigma, {\per_\sigma})$
    is represented by an \emph{extensional} value $f : \tau \to \sigma$:
    %
    \begin{equation*}
      u \per_\tau v \implies f\; u \per_\sigma f\; v \;.
    \end{equation*}
    %
  \end{itemize}  

  The category supports the interpretation of first-order logic,
  dependent types, subsets and quotients (but no powersets).

\end{frame}

\begin{frame}
  \frametitle{The realizability interpretation of logic}

  \begin{itemize}
  \item<1-> Classical predicate $p : A \to \mathtt{bool}$
  \item<2-> Realizability predicate on a per $(\sigma, {\per_\sigma})$:
    %
    \begin{equation*}
      p : \sigma \to \tau_p \to \mathtt{bool}
    \end{equation*}
    %
    \begin{itemize}
    \item strict: $p\;u\;v \Rightarrow u \per_{\sigma} u$
    \item extensional: $p\;u\;v \land u \per_\sigma u' \Rightarrow p\;u'\;v$
    \end{itemize}
  \item<3-> Read $p\;u\;v$ as ``$v$ realizes/witnesses the fact that $u$ has
    property $p$''.
  \end{itemize}
\end{frame}

\begin{frame}
  \frametitle{Realizability predicates}

  Underlying type $\typeOf{p} = \tau_p$ of realizers:
  %
  \begin{align*}
    \typeOf{e_1 = e_2} &= \mathtt{unit} \\
    \typeOf{p \land q} &= \typeOf{p} * \typeOf{q} \\
    \typeOf{p \Rightarrow q} &= \typeOf{p} \to \typeOf{q} \\
    \text{etc.}
  \end{align*}

  Interpretation of logical connectives:
  %
  \begin{align*}
    (e_1 = e_2)\; ()     &\iff \text{$e_1 \per_\tau e_2$} \\
    (p \land q)\; (u, v) &\iff \text{$p\; u$ and $q\; v$} \\
    (p \Rightarrow q)\; f  &\iff \text{if $p\; u$ then $q\; (f\; u)$}\\
    \text{etc.}
  \end{align*}
\end{frame}

\begin{frame}
  \frametitle{Observations about realizability}

  \begin{itemize}
  \item<1-> Unlike in propositions-as-types, realizers may be partial
    functions, and may use computational effects.
  \item<2-> Dependent sets are translated to \emph{simple} types and
    \emph{dependent} pers.
  \item<3-> A part of a realizer may be ``trivial'', e.g., $r : \tau \to
    \sigma \times (\tau \to \mathtt{unit})$ may be \emph{thinned} to
    $\tau \to \sigma$.
  \item<4-> \emph{Negative} formulas (built from $\top$, $\bot$, $=$,
    $\land$, $\Rightarrow$, $\forall$) have trivial realizers.
  \item<5-> The formula ``$v$ realizes $p$'' is negative.
  \end{itemize}

\end{frame}

\begin{frame}
  \frametitle{RZ}

  \begin{itemize}
  \item \textbf{Input:} Theories
  \item \textbf{Output:} Specifications (ML signatures with assertions)    
  \end{itemize}

  \pause

  The translation procedure has several passes:
  %
  \begin{itemize}
  \item Translation:
    \begin{itemize}
    \item ML signature: computational content of a theory
    \item assertions: the non-computational part
    \end{itemize}
  \item Thinning: removes trivial (parts of) realizers
  \item Optimization: simplifications such as $\top \land \phi \mapsto
    \phi$.
  \item Phase-splitting: replace functors by polymorphic values
  \end{itemize}
\end{frame}


\begin{frame}[fragile]
  \frametitle{Obligations}

Input:
%
\begin{source}
Axiom inv_positive:
  \iForall x : real, zero < x \iTo zero < inv x.
\end{source}

Output:
%
\begin{source}
assertion inv_positive:
   \iForall (x:\iT{real}), zero < x \iTo
   zero < inv (assure (not (x \iPer{real} zero)) in x)
\end{source}

\pause

General form:
%
\begin{equation*}
  \mathtt{assure}\ x : \tau, \phi\ \mathtt{in}\ e
\end{equation*}
%
Equivalently expressed with Hilbert's \emph{indefinite description}
operator $\varepsilon$:
%
\begin{equation*}
  \mathtt{let}\ x = \varepsilon\, x : \tau. \phi\ \mathtt{in}\ e
\end{equation*}

\end{frame}


\begin{frame}[fragile]
  \frametitle{Parameterized theories}

\begin{sourcex}
Definition ChainCompletion
  (I : Integer)
  (P : Poset I) :=
thy
  include CompletePoset I.
  Parameter incl : P.s \iTo s.
  Definition incl_chain (a : P.chain) :=
    (fun n : I.nat \iImply incl (a n)) : chain.
  Axiom incl_injective :
    \iForall x y : P.s, incl x = incl y \iTo x = y.
  Axiom incl_monotone :
    \iForall x y : P.s, P.below x y \iTo below (incl x) (incl y).
  Axiom incl_continuous:
    \iForall x : P.s, \iForall a : P.chain,
      P.supremum a x \iTo supremum (incl_chain a) (incl x).
  Axiom stage :
    \iForall x : s, \iExists a : P.chain, supremum (incl_chain a) x.
  Definition extend (f : P.continuous) :=
    the g : continuous, \iForall x : P.s, incl (f x) = g (incl x).
end. 
\end{sourcex}
\end{frame}

\begin{frame}[fragile]
  \frametitle{Parameterized axioms}

  Input:
  %
\begin{source}
Axiom induction :
  \iForall M : thy Parameter p : nat \iTo Prop. end,
  M.p zero \iTo (\iForall k, M.p k \iTo M.p (succ k)) \iTo
  \iForall k, M.p k.    
\end{source}

Output:
%
\begin{sourcex}
val induction : \iAlpha \iTo (nat \iTo \iAlpha \iTo \iAlpha) \iTo nat \iTo \iAlpha
(**  assertion \iAlpha [p:nat \iTo \iAlpha \iTo bool] induction : 
  (\iForall x:nat, a:\iAlpha,  p x a \iTo x : \iT{nat}) \iTo
  (\iForall x:nat, y:nat, a:\iAlpha,  x \iPer{nat} y \iTo p x a \iTo p y a) \iTo
  \iForall x:\iAlpha,  p zero x \iTo
  \iForall f:nat \iTo \iAlpha \iTo \iAlpha,
  (\iForall (k:\iT{nat}), \iForall y:\iAlpha,  p k y \iTo p (succ k) (f k y)) \iTo
  \iForall (k:\iT{nat}),  p k (induction x f k)
*)
\end{sourcex}

\end{frame}

\begin{frame}[fragile]
  \frametitle{Computational effects}

Input:
%
\begin{sourcex}
Axiom continuity:
  \iForall f : (nat \iTo nat) \iTo nat, \iForall a : nat \iTo nat,
  \iExists k, \iForall b : nat \iTo nat,
    (\iForall m, m \iLeq k \iTo a m = b m) \iTo f a = f b.
\end{sourcex}

Output:
%
\begin{sourcex}
val continuity : ((nat \iTo nat) \iTo nat) \iTo (nat \iTo nat) \iTo nat
assertion continuity :
  \iForall (f:\iT{(nat \iTo nat) \iTo nat}, a:\iT{nat \iTo nat}),
    let p = continuity f a in p : \iT{nat} \iAnd
    (\iForall (b:\iT{nat \iTo nat}),
       (\iForall (m:\iT{nat}),  m \iLeq p \iTo a m \iPer{nat} b m) \iTo f a \iPer{nat} f b)
\end{sourcex}

\pause

Realized by:
%
\begin{sourcex}
let continuity f a =
  let p = ref 0 in
  let a' n = (p := max !p n; a n) in
    f a' ; !p
\end{sourcex}
\end{frame}

\begin{frame}
  \frametitle{Real numbers}

  Two ways of approximating a real number by a dyadic\footnote{A
    \emph{dyadic} rational is of the form $m \cdot 2^{-e}$.} rational:
  %
  \begin{itemize}
  \item<1-> Old:
    %
    \begin{gather*}
      \mathtt{val\ approx : real \to nat \to dyadic}\\
      |x - \mathtt{approx}\;x\;k| \leq 2^{-k}
    \end{gather*}
    %
  \item<2-> New:
    \begin{gather*}
      \mathtt{val\ stage : real \to nat \to dyadic}\\
      x = \lim_{k \to \infty} \mathtt{stage}\;x\;k
    \end{gather*}
  \end{itemize}
\end{frame}

\begin{frame}[fragile]
  \frametitle{Two axiomatizations of reals}

  \begin{itemize}
  \item<1-> A complete Archimedean ordered field.
    %
\begin{source}
Axiom dense:
  \iForall x : real, \iForall k : nat, \iExists d : dyadic,
    abs (x - incl d) \iLeq incl (pow2 (neg k)).
\end{source}
% 
\item<2-> Via Scott's \emph{interval domain}.
\begin{source}
Axiom stage :
  \iForall x : real, \iExists a : ID.chain,
    supremum (incl_chain a) x.
\end{source}
\end{itemize}
\end{frame}

\begin{frame}
  \frametitle{Wish list}

  Realizability:
  %
  \begin{itemize}
  \item allow realizers which expose computational effects
  \item adopt the internal logic of pers to such realizers
  \end{itemize}

  ML:
  \begin{itemize}
  \item support for logical assertions
  \item mappings from structures to signatures
  \end{itemize}
\end{frame}
\end{document}
