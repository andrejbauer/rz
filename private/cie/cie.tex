\pagestyle{plain}

\usepackage{amsmath}
\usepackage{amssymb}
%\usepackage{theorem}
\usepackage{graphicx}
\usepackage{mathpartir}
\usepackage{fancyvrb}
\usepackage{url}
\usepackage{ifpdf}

\ifpdf
\usepackage{pdfsync}
\fi


%%%% MACROS %%%%%

% Private comments, should be left in the source when the paper is finished.
% Use this version for visible comments:
\newcommand{\comment}[1]{\paragraph{Comment:} #1}
% For invisible comments:
%\newcommand{\comment}[1]{}

% Internal notes, used during writing the paper. The first argument is
% the author of the note. When the paper is finished, these should
% all be gone. Delete an internal note as soon as you resolve it.
\newcommand{\internal}[2]{\par\fbox{\parbox{0.9\textwidth}{{Note by #1:} #2}}\par}

% Types and terms in ocaml
\newcommand{\ctype}{\mathtt{type}\;}
\newcommand{\cint}{\mathtt{int}}
\newcommand{\clist}[1]{#1\;\mathtt{list}}
\newcommand{\ctrue}{\mathtt{true}}
\newcommand{\cbool}{\mathtt{bool}}
\newcommand{\cwhile}[2]{\mathtt{while}\;#1\;\mathtt{do}\;#2\;\mathtt{done}}
\newcommand{\cstring}[1]{\mathtt{"#1"}}
\newcommand{\cprint}[1]{\mathtt{print\_string}\;#1}
\newcommand{\cfun}[2]{\mathtt{fun}\;#1 \to #2}
\newcommand{\clet}[2]{\mathtt{let}\;#1 = #2}
\newcommand{\cletrec}[2]{\mathtt{let}\;\mathtt{rec}\;#1 = #2}
\newcommand{\cfst}[1]{\mathtt{fst}\;#1}
\newcommand{\csnd}[1]{\mathtt{snd}\;#1}
\newcommand{\cmod}{\mathbin{\mathtt{mod}}}
\newcommand{\poly}[1]{\text{\texttt{\char39}}#1}

% Realizaiblity
\newcommand{\per}{\approx}
\newcommand{\perty}[1]{\per_{\mathtt{#1}}}
\newcommand{\rz}{\Vdash}
\newcommand{\Type}{\mathsf{Type}}
\newcommand{\values}[1]{[\![#1]\!]}
\newcommand{\typeOf}[1]{|#1|}
\newcommand{\setOf}[1]{\langle #1 \rangle}
\newcommand{\support}[1]{\|#1\|}

% Sets
\newcommand{\set}[1]{\{#1\}}
\newcommand{\such}{\mid}
\newcommand{\pair}[1]{\langle #1 \rangle}
\newcommand{\epito}{\twoheadrightarrow}
\newcommand{\family}[2]{\set{#1}_{#2}}
\newcommand{\depsum}[2]{{\Sigma_{#1} #2}}
\newcommand{\depprod}[2]{{\Pi_{#1} #2}}

% Common sets
\newcommand{\NN}{\mathbb{N}}
\newcommand{\ZZ}{\mathbb{Z}}
\newcommand{\RR}{\mathbb{R}}

% Theorem-like environments
\iffalse
{
  \theorembodyfont{\itshape}
  \newtheorem{theorem}{Theorem}[section]
  \newtheorem{lemma}[theorem]{Lemma}
  \newtheorem{proposition}[theorem]{Proposition}
  \newtheorem{corollary}[theorem]{Corollary}
}
{
  \theorembodyfont{\rmfamily}
  \newtheorem{definition}[theorem]{Definition}
  \newtheorem{example}[theorem]{Example}
}
\fi

% Quantifiers. I am really used to these...
\newcommand{\all}[3]{\forall\, #1 \,{\in}\, #2\,.\left(#3\right)}
\newcommand{\some}[3]{\exists\, #1 \,{\in}\, #2\,.\left(#3\right)}
\newcommand{\exactlyone}[3]{\exists!\, #1 \,{\in}\, #2\,.\left(#3\right)}
\newcommand{\xall}[3]{\forall\, #1 \,{\in}\, #2\,.\,#3}
\newcommand{\xsome}[3]{\exists\, #1 \,{\in}\, #2\,.\,#3}
\newcommand{\xexactlyone}[3]{\exists!\, #1 \,{\in}\, #2\,.\,#3}


%%%%%%%%%%%%%%%%%%%%%%%%%%%%%%%%%%%%
%% Macros for input & output language

% ASCII syntax
\iffalse
\newcommand{\ip}{p}
\newcommand{\ix}{x}
\newcommand{\is}{s}
\newcommand{\ie}{e}
\newcommand{\iP}{P}
\newcommand{\iS}{S}
\newcommand{\iprop}{\mathtt{prop}}
\newcommand{\itrue}{\mathtt{true}}
\newcommand{\ifalse}{\mathtt{false}}
\newcommand{\inot}[1]{\mathtt{not}\,#1}
\newcommand{\iimply}[2]{#1\,\mbox{\texttt{->}}\,#2}
\newcommand{\iiff}[2]{#1\,\mbox{\texttt{<->}}\,#2}
\newcommand{\iequal}[2]{#1\,\mathtt{=}\,#2}
\newcommand{\iands}[2]{#1\,\mathtt{/\backslash}\, #2}
\newcommand{\iors}[2]{#1\,\mathtt{\backslash/}\,#2}
\newcommand{\iforall}[3]{\texttt{forall}\,#1\,\texttt{:}\,#2\texttt{,}\ #3}
\newcommand{\iexists}[3]{\texttt{exists}\,#1\,\texttt{:}\,#2\texttt{,}\ #3}
\newcommand{\iunique}[3]{\texttt{unique}\,#1\,\texttt{:}\,#2\texttt{,}\ #3}
\newcommand{\ilambda}[3]{\texttt{lambda}\,#1\,\texttt{:}\,#2\texttt{,}\ #3}
\newcommand{\iapps}[3]{#1\ #2\,\cdots\,#3}

\newcommand{\ituples}[2]{\texttt{(}#1\texttt{,}\ldots\texttt{,}#2\texttt{)}}
\newcommand{\iproj}[2]{#1\texttt{.}#2}
\newcommand{\isubin}[2]{#1\,\texttt{:>}\,#2}
\newcommand{\isubout}[2]{#1\,\texttt{:<}\,#2}
\newcommand{\ithe}[3]{\texttt{the}\,#1\,\texttt{:}\,#2\texttt{,}\, #3}
\newcommand{\ilet}[3]{\texttt{let}\,#1\,\texttt{=}\,#2\,\texttt{in}\, #3}
\newcommand{\iquot}[2]{#1\,\texttt{\%}\,#2}
\newcommand{\iletquot}[4]{\texttt{let}\ #1\texttt{\%}\,#2\,\texttt{=}\,#3\ \texttt{in}\ #4}
\newcommand{\irz}[1]{\texttt{rz}\,#1}
\newcommand{\iletrz}[3]{\texttt{let}\ \texttt{rz}\,#1\,\texttt{=}\,#2\ \texttt{in}\ #3}

\newcommand{\iprods}[2]{#1\,\texttt{*}\cdots\texttt{*}\,#2}
\newcommand{\isums}[2]{#1\,\texttt{+}\cdots\texttt{+}\,#2}

\fi

% Spiffed up syntax using mathematical symbols
\newcommand{\ip}{\varphi}
\newcommand{\ipp}{\rho}
\newcommand{\ix}{x}
\newcommand{\is}{s}
\newcommand{\ie}{e}
\newcommand{\iP}{p}
\newcommand{\iS}{\alpha}
\newcommand{\itrue}{\top}
\newcommand{\ifalse}{\bot}
\newcommand{\inot}[1]{\lnot #1}
\newcommand{\iimply}[2]{#1\Rightarrow #2}
\newcommand{\iiff}[2]{#1 \Leftrightarrow #2}
\newcommand{\iequal}[2]{#1 = #2}
\newcommand{\iand}[2]{#1\land #2}
\newcommand{\ior}[2]{#1\lor #2}
\newcommand{\iforall}[3]{\forall #1{:}#2.\ #3}
\newcommand{\iexists}[3]{\exists #1{:}#2.\ #3}
\newcommand{\iunique}[3]{\exists! #1{:}#2.\ #3}
\newcommand{\ilambda}[3]{\lambda #1{:}#2.\ #3}
\newcommand{\iapp}[2]{#1\ #2}

\newcommand{\ituple}[2]{(#1,\ldots,#2)}
\newcommand{\iproj}[2]{\mathop{\pi_{#2}} #1}
\newcommand{\isubin}[2]{#1\,{{:}{>}}\,#2}
\newcommand{\isubout}[2]{#1\,{{:}{<}}\,#2}
\newcommand{\ithe}[3]{\iota\,#1{:}#2{.}\, #3}
\newcommand{\ilet}[3]{\textsf{let}\,#1{=}#2\,\textsf{in}\, #3}
\newcommand{\ieclass}[2]{[#1]_{#2}}
\newcommand{\ileteclass}[4]{\textsf{let}\ [#1]_{#2}\,{=}#3\ \textsf{in}\ #4}
\newcommand{\irz}[1]{\textsf{rz}\,#1}
\newcommand{\iletrz}[3]{\textsf{let}\ \textsf{rz}\,#1\,{=}\,#2\ \textsf{in}\ #3}
\newcommand{\iinj}[2]{`{#1}\,#2}
\newcommand{\imatches}[7]{\textsf{match}\,#1\,\textsf{with}\,\iinj{#2}{#3}\Rightarrow{#4}\,|\cdots|\iinj{#5}{#6}\Rightarrow{#7}}
\newcommand{\imatch}[7]{\textsf{match}\,#1\,\textsf{with}\,\iinj{#2}{#3}\Rightarrow{#4}\,|\,\iinj{#5}{#6}\Rightarrow{#7}}
\newcommand{\iselect}[2]{#1{.}#2}

\newcommand{\itag}{\mathit{tag}}
\newcommand{\iprod}[3]{#1{:}#2\times#3}
\newcommand{\isum}[4]{#1{:}#2+#3{:}#4}
\newcommand{\iquot}[2]{#1{\bigm/} #2}
\newcommand{\isubset}[3]{\{#1{:}#2\ |\ #3\}}
\newcommand{\isunit}{1}
\newcommand{\isvoid}{0}

% I removed periods because I use these in the middle of text.
\newcommand{\iDefinition}[2]{\mathsf{Definition}\ #1 \mathbin{{:}{=}} #2}
\newcommand{\iParameter}[2]{\mathsf{Parameter}\ #1 : #2}
\newcommand{\iAxiom}[2]{\mathsf{Axiom}\ #1 : #2}
\newcommand{\ipt}{\tau}
\newcommand{\ik}{\kappa}
\newcommand{\iarrow}[2]{#1{\to}#2}
\newcommand{\idarrow}[3]{(#1{:}#2){\to}#3}
\newcommand{\ite}{\theta}
\newcommand{\ith}{\Theta}
\newcommand{\iTH}{T}
\newcommand{\ithy}[1]{\mathsf{thy}\ #1\ \mathsf{end}}
\newcommand{\im}{M}
\newcommand{\iM}{m}
\newcommand{\il}{l}
\newcommand{\iSet}{\mathsf{Set}}
\newcommand{\iStable}{\mathsf{Stable}}
\newcommand{\iProp}{\mathsf{Prop}}
\newcommand{\iEquiv}[1]{\mathsf{Equiv}(#1)}
\newcommand{\ibar}{\ |\ }

%%%%%%%%%%%%%%%%%%%%%%%%%%%%%%%%%%%%%%%%%%%%%%%%%%%%%%%%%%%%%%%%%%%%%%
%  Output-specific syntax (did not bother to produce ASCII version yet)
\newcommand{\obar}{\ |\ }
%%%% types
\newcommand{\oty}{\tau} % type
\newcommand{\opty}{\alpha} % polymorphic type
\newcommand{\oTY}{T} % type name
\newcommand{\oprod}[2]{#1 \times #2} % product
\newcommand{\oarrow}[2]{#1 \to #2} % arrow
\newcommand{\ounit}{\mathsf{unit}} % unit type
\newcommand{\oselect}[2]{#1{.}#2}  % projection from a module
\newcommand{\osumty}[4]{[`#1\,\mathsf{of}\,#2 \mid \cdots \mid `#3\,\mathsf{of}\,#4]}
\newcommand{\osumtyx}[4]{[`#1\,\mathsf{of}\,#2 \mid `#3\,\mathsf{of}\,#4]}
\newcommand{\ol}{l} % label
%%%% terms
\renewcommand{\oe}{e} % expression
\newcommand{\ox}{x} % variable
\newcommand{\olambda}[3]{\mathsf{fun}\,#1{:}#2 \to #3} % abstraction
\newcommand{\oapp}[2]{#1\,#2} % application
\newcommand{\otuple}[2]{(#1,\ldots,#2)} % tuple
\newcommand{\oproj}[2]{\mathsf{p}_{#2}\,#1} % projection
\newcommand{\oinj}[2]{`#1\,#2} % sum injection
\newcommand{\omatch}[7]{\mathsf{match}\,#1\,\mathsf{with}\,\oinj{#2}{#3}\to{#4}\obar\oinj{#5}{#6}\to{#7}}
\newcommand{\omatches}[7]{\mathsf{match}\,#1\,\mathsf{with}\,\oinj{#2}{#3}\to{#4}\,|\cdots|\oinj{#5}{#6}\to{#7}}
\newcommand{\ooblig}[4]{\mathsf{assure}\,#1{:}#2.\,#3\,\mathsf{in}\,#4}
\newcommand{\oobligx}[2]{\mathsf{assure}\,#1\,\mathsf{in}\,#2}
\newcommand{\olet}[3]{\mathsf{let}\,#1{=}#2\,\mathsf{in}\,#3}
%%%% propositions
\newcommand{\op}{p} % proposition
\newcommand{\oP}{P} % atomic proposition
\newcommand{\otrue}{\top} % true
\newcommand{\ofalse}{\bot} % false
\newcommand{\onot}[1]{\lnot #1} % not
\newcommand{\oand}[2]{#1 \land #2} % and
\newcommand{\oimply}[2]{#1 \Rightarrow #2} % imply
\newcommand{\oiff}[2]{#1 \Leftrightarrow #2} % iff
\newcommand{\ototal}[2]{#1{:}\support{#2}} % support membership
\newcommand{\oper}[3]{#2\,{\approx_{#1}}\,#3} % per
\newcommand{\oequal}[2]{#1{=}#2} % observational equality
\newcommand{\oforall}[3]{\forall #1{:}#2.\,#3}
\newcommand{\oforallt}[3]{\forall #1{:}\support{#2}.\,#3}
%%% Modest sets?
\newcommand{\oS}{S}
\newcommand{\os}{s}
%%% Modules
\newcommand{\om}{m} % module
\newcommand{\oM}{M} % module name
%%% Proposition kinds
\newcommand{\opt}{\Pi}
\newcommand{\oProp}{\mathsf{bool}}
%%% Module type elements
\newcommand{\ote}{\theta}
\newcommand{\ovalspec}[2]{\mathsf{val}\;#1:#2}
\newcommand{\otyspec}[1]{\mathsf{type}\;#1}
\newcommand{\otydef}[2]{\mathsf{type}\;#1=#2}
\newcommand{\omodulespec}[2]{\mathsf{module}\;#1:#2}
\newcommand{\osignatdef}[2]{\mathsf{module}\;\mathsf{type}\,#1=#2}
\newcommand{\opropspec}[2]{\mathsf{predicate}\; #1:#2}
\newcommand{\oassertion}[2]{\mathsf{assertion}\; #1 : #2}
\newcommand{\oA}{A} % Assertion name
%%% Module type
\newcommand{\omt}{\Sigma}
\newcommand{\oMT}{S}
\newcommand{\osig}[1]{\mathsf{sig}\,#1\,\mathsf{end}}
\newcommand{\ofunctor}[3]{\mathsf{functor}\, (#1{:}#2) \to #3}

%%% Local Variables: 
%%% mode: latex
%%% TeX-master: "cie"
%%% End: 


%%%%%%%%%%%%%%%%%%%%%%%%%%%%%%%%%%%%%%%%%%%%%%%%%%
%
% Slightly better macros for displaying source code.
% We can fiddle with the style later, I just stole this.
% (Use number=left to get source line numbers.)

% To include a source file
\newcommand{\sourcefile}[1]{%
\VerbatimInput[xleftmargin=1.5em,fontsize=\footnotesize]{#1}}

% To include lots of source
\newcommand{\sourcefilex}[1]{%
\VerbatimInput[xleftmargin=1.5em,fontsize=\scriptsize]{#1}}

% To show source
\DefineVerbatimEnvironment{source}{Verbatim}%
{xleftmargin=1.5em,fontsize=\footnotesize,commandchars=\\\{\}}

% macros within source
\newcommand{\iTo}{$\to$}
\newcommand{\iForall}{$\forall$}
\newcommand{\iOr}{$\lor$}
\newcommand{\iAnd}{$\land$}
\newcommand{\iImply}{$\Rightarrow$}
\newcommand{\iNot}{$\lnot$}
\newcommand{\iExists}{$\exists$}
\newcommand{\iExistsOne}{$\exists!$}
\newcommand{\iT}[1]{$\|$#1$\|$}
\newcommand{\iPer}[1]{$\approx_{\mathtt{#1}}$}
\newcommand{\iIff}{$\Leftrightarrow$}
\newcommand{\iLeq}{$\leq$}
\newcommand{\iRz}{$\rz$}

%%%%%%%%%%%%%%%%%%%%%%%%%%%%%%%%%%%%%%%%%%%%%%%%%%

\begin{document}
\title{RZ: a Tool for Bringing\\
  Constructive and Computable Mathematics\\
  Closer to Programming Practice}
\author{
  Andrej Bauer\inst{1} \and
  Christopher A. Stone\inst{2}}

\institute{
  Faculty of Mathematics and Physics, University of Ljubljana, Slovenia\\
  \email{Andrej.Bauer@fmf.uni-lj.si}
  \and
  Computer Science Department, 
  Harvey Mudd College, USA\\
  \email{stone@cs.hmc.edu}
}
\maketitle

\begin{abstract}
  Realizability theory 
\iflong 
  is not only a fundamental tool in logic
  and computability, but also 
  has direct application to the design and
  implementation of programs: it
\fi % \iflong
  can produce interfaces for the data
  structure corresponding to a
  mathematical theory.
%
  Our tool, called RZ,
  serves as a bridge between
\iflong the worlds of \fi
  constructive
  mathematics and programming% 
\iflong 
.
By using the realizability
  interpretation of constructive mathematics, RZ 
  translates
\else
\ by translating
\fi % \iflong 
  specifications in constructive logic into annotated
  interface code in Objective Caml.
%
  The system supports
  a rich input language allowing descriptions of
  complex mathematical structures. RZ does not extract code from
  proofs, but allows any implementation method, from handwritten code to code extracted from
  proofs by other tools.
\end{abstract}

\begin{center}
  Version of \today.
\end{center}

\section{Introduction}
\label{sec:introduction}

Given the constants, functions, relations, and axioms of a mathematical
structure, what qualifies as a correct and complete computer implementation?

For some familiar cases, the answer is obvious. Code implementing a group must
have a type to represent the group elements, as well as a constant, a unary
operator, and a binary operator, each satisfying appropriate axioms (the
constant is the neutral element, the unary operator computes inverses, and that
the binary operator is associative).

But for more interesting structures, especially those arising in
mathematical analysis, the answer is much less obvious. How do we
implement the real numbers (a Cauchy-complete Archimedean ordered
field), or a compact metric space, or a space of smooth functions?
Significant research goes into finding representations that provide a
satisfactory theory of computability on such
structures~\cite{Wei00,TZ98,Bla97,EL00}, and implementations of exact
real arithmetic~\cite{Mue00,Lam05a} show that the theory can be put
into practice quite successfully.

The theory of realizability provides guidance in development of
computable mathematics.   We show that, in addition to being a
fundamental tool in logic and theory of computability, realizability
has direct application to the design and implementation of programs.
It can be used to find a description of the data structure directly
corresponding to a mathematical specification.

Unfortunately, extracting code interfaces by hand from mathematical axioms
quickly grows tedious. Worse, different but logically equivalent sets of axioms
correspond to different, although interdefinable, interfaces for code. One
might want to compare several variations, since some interfaces can be more
useful than others in practice.

And few programmers --- even those with strong backgrounds in
mathematics and classical logic --- are familiar with constructive logic or
realizability. Programmers are more familiar with language constructs
describing interfaces (e.g., C++ header files, ML signatures, or Java
interfaces) and logical assertions (e.g., preconditions
and postconditions).

\bigskip

We have therefore implemented a system, called RZ, to serve as a bridge between
the logical world and the programming world. RZ automatically translates
specifications in constructive logic into actual interface code in a
programming language (currently Objective Caml~\cite{ocaml}, but we are
considering other languages as well).

The constructive part of the original specification turns into interface code
listing types and values that must exist in an implementation. The remainder of
the specification is maintained as assertions about these types and values.
Because these assertions have no computational content, they can be interpreted
constructively or classically equally well, and in particular make equal sense
to programmers (and mathematicians) who are more comfortable thinking in
classical logic.

\bigskip

RZ was specifically designed as a lightweight system. Although the
realizability translation can be extended to transforming complete
proofs into complete code~\cite{komagata+:tr95}, we have not
implemented this. Other systems, including Coq~\cite{coqart}, already
perform this task well.

But Coq works best when the entire task (from specification to code
generation) is performed within the same system. In contrast, the
interfaces generated by RZ can be implemented in any fashion, as long
as the assertions are satisfied. Code can be written by hand ---
allowing imperative, concurrent, and other arbitrary language
features, not just a ``functional'' subset. Or, the system could
generate a Coq interface as an \emph{input}, where the distinction
between computational (Set) and non-computational (Prop) is
automatically determined, and a corresponding implementation can be
provided through logical techniques.


\internal{AB}{from here on the introduction seems ``undone''}

The RZ system translates this to the language understood by many
programmers (typed interfaces with assertions in classical logic)

\bigskip

In this paper we show how RZ makes practical use of realizability and
other theoretical techniques, and provide some interesting motivational
examples.


%%% Local Variables: 
%%% mode: latex
%%% TeX-master: "cie"
%%% End: 

\section{Typed realizability}
\label{sec:typed-realizability}

RZ is based on \emph{typed realizability} by John
Longley~\cite{Longley99}.   This variant of realizability corresponds most
directly to programmers' intuition about implementations.

We approach typed realizability and its relationship to
real-world programming by way of example. Suppose we are asked to
design a data structure for the set $\mathcal{G}$ of all finite
simple%
\iflong
\footnote{At most one arrow between any two vertices.}
\fi % \iflong
\ directed graphs with vertices labeled by distinct integers. 
%
\iflong
An exemplar
directed graph~$G$ is shown in Figure~\ref{fig:digraph}.
%
\begin{figure}
  \centering
  \includegraphics[width=0.3\textwidth]{digraph}
  \caption{A finite directed graph $G$}
  \label{fig:digraph}
\end{figure}
\fi % \iflong
%
A common representation is a pair of lists $(\ell_V, \ell_A)$, where
$\ell_V$ is the list of vertex labels and $\ell_A$ is the \emph{adjacency list} 
representing the arrows by pairing the labels of each source and target.
\iflong
In our example,
$\ell_V = [1; 2;
3; 4]$ and $\ell_A = [(1,2); (2,2); (2,3); (3,2); (3;1)]$.
\fi % \iflong
%
Thus we define the datatype of graphs as\footnote{We use OCaml
  notation in which $\clist{t}$ classifies finite lists of elements of
  type~$t$, and $t_1 * t_2$ classifies pairs containing a value of
  type $t_1$ and a value of type $t_2$.}
%
\begin{equation*}
  \ctype \mathtt{graph} = \clist{\cint} \ \ * \ \ \clist{(\cint * \cint)}
\end{equation*}
%
However, this is not a complete description of the representation, as
there would be representation invariants and conditions not expressed by the type, e.g.,
%
\iflong
\begin{enumerate}
\item The order in which vertices and arrows are listed is not
  important%
; for example, $[1;2;3;4]$ and $[4;1;2;3]$ represent the same vertices.
\item Each vertex and arrow must be listed exactly once.
\item The source and target of each arrow must appear in the list of vertices.
\end{enumerate}
\else % \iflong
the order in which vertices and arrows are listed is not
important, each vertex and arrow must be listed exactly once, and
the source and target of each arrow must appear in the list of vertices.
\fi %\iflong

%
Thus, to implement the mathematical set~$\mathcal{G}$, we must not
only decide on the underlying datatype $\mathtt{graph}$, but also
determine what values of that type represent which elements
of~$\mathcal{G}$. As we shall see next, this can be expressed either
using a \emph{realizability relation} or a \emph{partial equivalence
  relation (per)}.


\subsection{Modest sets and pers}
\label{sec:modest-sets-pers}

\iflong
We now define typed realizability as it
applies to OCaml. Other general-purpose programming languages could be
used instead, as long as they provide the usual ground types, product
and function types.\footnote{It is also convenient to work with a
language that supports sum types, as this allows a more natural
representation of disjoint unions.}
\else
We now define typed realizability as it
applies to OCaml. Other general-purpose programming languages could be
used instead.
\fi % \iflong

Let $\Type$ be the collection of all (non-parametric) OCaml types. To
each type $t \in \Type$ we assign the set $\values{t}$ of values of
type~$t$ that behave \emph{functionally} in the sense of
Longley~\cite{longley99when}. Such values are represented by
terminating expressions that do not throw exceptions or return
different results on different invocations. They may \emph{use}
exceptions, store, and other computational effects, provided they
appear functional from the outside; a useful example using
computational effects is presented in
Section~\ref{sec:we-show-modulus-of-continuity-example}. A functional
value of function type may diverge as soon as it is applied. The
collection $\Type$ with the assignment of functional values
$\values{t}$ to each $t \in \Type$ forms a \emph{typed partial
  combinatory algebra (TPCA)}%
\iflong
, which provides a theoretical basis for
the definition of a realizability model that suits our needs%
\fi%\iflong
.

Going back to our example, we see that an implementation of directed
graphs $\mathcal{G}$ specifies a datatype $\typeOf{\mathcal{G}} =
\mathtt{graph}$ together with a \emph{realizability relation}
$\rz_{\mathcal{G}}$ between $\mathcal{G}$ and
$\values{\mathtt{graph}}$. The meaning of $(\ell_V, \ell_A)
\rz_\mathcal{G} G$ is ``OCaml value $(\ell_V, \ell_A)$
represents/realizes/implements graph $G$''.
%
\iflong
%
There are two natural
conditions that $\rz_\mathcal{G}$ ought to satisfy: (1) for every $G
\in \mathcal{G}$ there should be at least one realizer $(\ell_V,
\ell_A)$ representing it, and (2) if $(\ell_V, \ell_A)$ represents
both $G$ and $G'$ then $G = G'$.\footnote{The latter condition is
  called \emph{modesty} and is not strictly necessary for the
  development of the theory, though programmers would naturally expect
  it to hold.} If $(\ell_V, \ell_A)$ and $(\ell'_V, \ell'_A)$
represent the same graph (e.g., because $\ell_V$ is a permutation of
$\ell'_V$, and similarly for $\ell_A$ and $\ell'_A$) we say that they
are \emph{equivalent} and write $(\ell_V, \ell_A) \per_\mathcal{G}
(\ell'_V, \ell'_A)$. The relation $\per_\mathcal{G}$ is a
\emph{partial} equivalence relation (symmetric and transitive, but not
reflexive) because not every $(\ell_V, \ell_A) \in
\values{\mathtt{graph}}$ represents a graph.

\smallskip

A general definition is in order. A \emph{modest set} is 
%
\else % iflong
%
Generalizing from this, we define a \emph{modest set} to be
%
\fi
%
a triple $A = (\setOf{A}, \typeOf{A}, {\rz_A})$ where $\setOf{A}$ is
the \emph{underlying set}, $\typeOf{A} \in \Type$ is the
\emph{underlying type}, and $\rz_A$ is a \emph{realizability relation}
between $\values{\typeOf{A}}$ and $\setOf{A}$, satisfying
%
\iflong
% 
\begin{enumerate}
\item \emph{totality:} for every $x \in \setOf{A}$ there is $v \in
  \values{\typeOf{A}}$ such that $v \rz_A x$, and
\item \emph{modesty:} if $u \rz_A x$ and $u \rz_A y$ then $x = y$.
\end{enumerate}
%
\else % iflong
%
(1) \emph{totality:} for every $x \in \setOf{A}$ there is $v \in
\values{\typeOf{A}}$ such that $v \rz_A x$, and (2) \emph{modesty:} if
$u \rz_A x$ and $u \rz_A y$ then $x = y$.
%
\fi % iflong
%
The \emph{support} of $A$ is the set $\support{A} = \set{v \in
  \values{\typeOf{A}} \such \xsome{x}{\setOf{A}}{v \rz_A x}}$ of those
values that realize something. We define the relation $\per_A$ on
$\values{\typeOf{A}}$ by
%
\begin{equation*}
  u \per_A v
  \iff
  \some{x}{\setOf{A}}{u \rz_A x \land v \rz_A x} \;.
\end{equation*}
%
From totality and modesty of $\rz_A$ it follows that $\per_A$ is a per,
i.e., symmetric and transitive. Observe that $\support{A} = \set{v \in
  \values{\typeOf{A}} \such v \per_A v}$, whence $\per_A$
restricted to $\support{A}$ is an equivalence relation. In fact, we
may recover a modest set up to isomorphism from $\typeOf{A}$ and
$\per_A$ by taking $\setOf{A}$ to be the set of equivalence classes of
$\per_A$, and $v \rz_A x$ to mean $v \in x$.

The two views of implementations, as modest sets $(\setOf{A},
\typeOf{A}, {\rz_A})$, and as pers $(\typeOf{A}, {\per_A})$, are
equivalent.\footnote{And there is a third view, as a partial surjection
  $\delta_A : {\subseteq}\values{\typeOf{A}} \epito \setOf{A}$, with
  $\delta_A(v) = x$ when $v \rz_A x$. This is how realizability is
  presented in Type Two Effectivity~\cite{Wei00}.} 
%
We concentrate on the view of modest sets as pers. They are more
convenient to use in RZ because they refer only to types and values,
as opposed to arbitrary sets.
%
Nevertheless, it is useful to understand
how modest sets and pers arise from natural programming practice.

\iflong
%
Modest sets form a category whose objects are modest sets and
morphisms are the realized functions. A \emph{realized function} $f :
A \to B$ is a function $f : \setOf{A} \to \setOf{B}$ for which there
exists $v \in \values{\typeOf{A} \to \typeOf{B}}$ such that, for all
$x \in \setOf{A}$ and $u \in \typeOf{A}$,
%
\begin{equation}
  \label{eq:rz-function-space}
  u \rz_A x \implies v\;u \rz_B f(x) \;.
\end{equation}
%
This condition is just a mathematical expression of the usual idea
that~$v$ is an implementation of~$f$ if it does to realizers
what~$f$ does to the elements they represent.
\fi

\iflong
The equivalent category of pers has as objects
\else
Pers form a category whose objects are
\fi
%
pairs $A = (\typeOf{A}, {\per_A})$ where $\typeOf{A} \in \Type$ and
$\per_A$ is a per on $\values{\typeOf{A}}$. A morphism $A \to B$ is
represented by a function $v\in \values{\typeOf{A} \to \typeOf{B}}$
such that, for all $u, u' \in \support{A}$,
%
\iflong
\begin{equation}
  \label{eq:per-exponential}
  u \per_A u' \implies v\;u \per_B v\;u' \;.
\end{equation}
%
Values $v$ and $v'$ that both satisfy~\eqref{eq:per-exponential}
\else % \iflong
$u \per_A u' \implies v\;u \per_B v\;u'$.  Two such functions $v$ and $v'$
\fi
represent the same morphism if, for all $u, u' \in \support{A}$,
$u \per_A u'$ implies $v\;u \per_B v'\;u'$.

\iflong
%
The category of pers has a very rich structure. For example, we may
form a cartesian product $A \times B$ of pers $A$ and $B$ by
%
\begin{align*}
  \typeOf{A \times B} &= \typeOf{A} * \typeOf{B},\\
  (u_1, v_1) \per_{A \times B} (u_2, v_2) &\iff
  u_1 \per_A u_2 \land v_1 \per_B v_2.
\end{align*}
%
The projections $\pi_1 : A \times B \to A$ and $\pi_2 : A \times B \to
B$ are realized by $\mathtt{fst}$ and $\mathtt{snd}$, respectively.

The morphisms between pers~$A$ and~$B$ again form a per
$B^A$, also written as $A \to B$, called the \emph{exponential} of~$A$
and~$B$, with
%
\begin{align*}
  \typeOf{B^A} &= \typeOf{A} \to \typeOf{B},\\
  \support{B^A} &=
  \set{v \in \values{\typeOf{A} \to \typeOf{B}} \such 
    \all{u,u'}{\values{\typeOf{A}}}{u \per_A u' \implies v\,u \per_B v\,u'}}
  \\
  u \per_{B^A} v &\iff u, v \in \support{A} \land
  \xall{w}{\support{A}}{u\;w \per_B v\;w}.
\end{align*}
%
The evaluation map $e : B^A \times A \to B$ is realized by OCaml
application, $\cfun{(u,v)}{u\;v}$. If a function $f : C \times A \to
B$ is realized by $v$, then its transpose $\tilde{f} : C \to B^A$,
$\tilde{f}(z)(x) = f(z,x)$, is realized by $\cfun{z\;x}{v \; (z,x)}$.
This shows that the category of pers is cartesian closed. In
Section~\ref{sec:transl-sets-terms} we review other canonical
constructions on modest sets.

\bigskip
\else % \iflong
%
The category of pers has a very rich structure, namely that of a
regular locally cartesian closed category~\cite{Bauer:00}. This
suffices for the interpretation of first-order logic and (extensional)
dependent types~\cite{JacobsB:cltt}.
%
\fi % \iflong

\iflong
%
As an example we consider the cyclic group on seven elements $(\ZZ_7,
0, {-}, {+})$. To implement the group, we must give a representation
of $\ZZ_7$ as a per~$Z = (\typeOf{Z}, {\per_Z})$, and
provide realizers for the neutral element~$0$, negation~$-$, and
addition~$+$. 

One possibility is to choose $\cint$ as the underlying type
$\typeOf{Z}$, and to let $\support{Z}$ be only the integers \texttt{0}
through \texttt{6}. Then negation and addition must work modulo
\texttt{7} (i.e., must return an integer in the range
\texttt{0}--\texttt{6} when given integers in this range). The neutral
element would be the integer constant \texttt{0}, and the equivalence
$\per_Z$ would be integer equality.

Alternatively, we could take $\cint$ as the underlying type
$\typeOf{Z}$, but let $\support{Z}$ include all integers. In this
case, negation and addition could be simply integer addition and
negation\footnote{Taking care to prevent integer
  overflow.}. Here the neutral element could be implemented as any
integer multiple of \texttt{7}, and the equivalence $\per_Z$ would be
equivalence-modulo-7.

Both of these pers happen to be \emph{decidable}, i.e., it can be
algorithmically decided whether two values represent the same element
of~$\ZZ_7$, by code for integer equality and code for integer
equivalence-modulo-7 respectively. \fi % \iflong

\iflong
%
Not all pers are decidable.
%
\else
%
Not all pers are \emph{decidable}, i.e., there may be no algorithm for deciding
when two values are equivalent.
%
\fi
%
Examples include implementations of semigroups with an undecidable
word problem~\cite{post47:_recur_unsol_probl_thue}%
\iflong, implementations of computable sets of integers (which might
be realized by membership functions of type $\cint\to\cbool$),\fi
\ and implementations of computable real numbers (which might be
realized by infinite Cauchy sequences).
%
\iflong
%
There is no presupposition that pers are computable
(implementable). We can require decidable equivalence by adding a
suitable axiom; see Section~\ref{sec:decidable-sets}. \fi

\subsection{Interpretation of logic}
\label{sec:interpretation-logic}

In the realizability interpretation of logic, each formula~$\phi$ is
assigned a set of \emph{realizers}, which can be thought of as
computations that witness the validity of~$\phi$. The situation is
somewhat similar, but not equivalent, to the propositions-as-types
translation of logic into type theory, where proofs of a
proposition correspond to terms of the corresponding type. More
precisely, to each formula~$\phi$ we assign an underlying type
$\typeOf{\phi}$ of realizers, but unlike the propositions-as-types
translation, not all terms of type $\typeOf{\phi}$ are necessarily
valid realizers for~$\phi$, and some terms that are realizers may not
correspond to any proofs, for example, if they denote partial
functions or use computational effects.

It is customary to write $t \rz \phi$ when $t \in
\values{\typeOf{\phi}}$ is a realizer for~$\phi$. The underlying types
and the realizability relation~$\rz$ are defined inductively on the
structure of~$\phi$; an outline is shown in Figure~\ref{fig:rz-logic}.
We say that a formula~$\phi$ is \emph{valid} if it has at least one
realizer.
%
\begin{figure*}[t]
  \textbf{Underlying types of realizers:}
\[
  \begin{array}{ll@{\qquad}ll}
    \typeOf{\itrue} &= \mathtt{unit} &
    \typeOf{\ifalse} &= \mathtt{unit} \\
    \typeOf{\iequal{x}{y}} &= \mathtt{unit} &
    \typeOf{\iand{\phi}{\psi}} &= \oprod{\typeOf{\phi}}{\typeOf{\psi}} \\
    \typeOf{\iimply{\phi}{\psi}} &= \oarrow{\typeOf{\phi}}{\typeOf{\psi}} &
    \typeOf{\ior{\phi}{\psi}} &=
    \osumtyx{\mathtt{or}_0}{\typeOf{\phi_0}}{\mathtt{or}_1}{\typeOf{\phi_1}} \\
    \typeOf{\iforall{x}{A}{\phi}} &= \oarrow{\typeOf{A}}{\typeOf{\phi}} &
    \typeOf{\iexists{x}{A}{\phi}} &= \oprod{\typeOf{A}}{\typeOf{\phi}}
  \end{array}
\]
  \textbf{Realizers:}
\[
  \begin{array}{ll}
    () \rz \itrue & \\
    () \rz \iequal{x}{y}
    &\quad\text{iff}\quad 
    x = y
    \\
    (t_1,t_2) \rz \iand{\phi}{\psi}
    &\quad\text{iff}\quad
    \text{$t_1 \rz \phi$ and $t_2 \rz \psi$}
    \\
    t \rz \iimply{\phi}{\psi}
    &\quad\text{iff}\quad
    \text{for all $u \in \typeOf{\phi}$, if $u \rz \phi$ then $t\,u
      \rz \psi$}
    \\
    \oinj{\mathtt{or}_0}{t} \rz \ior{\phi}{\psi}
    &\quad\text{iff}\quad
    \text{$t \rz \phi$}
    \\
    \oinj{\mathtt{or}_1}{t} \rz \ior{\phi}{\psi}
    &\quad\text{iff}\quad
    \text{$t \rz \psi$}
    \\
    t \rz \iforall{x}{A}{\phi(x)}
    &\quad\text{iff}\quad
    \text{for all $u \in \typeOf{A}$, if $u \rz_A x$ then $t\,u \rz \phi(x)$}
    \\
    (t_1, t_2) \rz \iexists{x}{A}{\phi(x)}
    &\quad\text{iff}\quad
    \text{$t_1 \rz_A x$ and $t_2 \rz \phi(x)$}
  \end{array}
\]
\vspace{-0.5cm}
  \caption{Realizability interpretation of logic (outline)}
  \label{fig:rz-logic}
\end{figure*}

In classical mathematics, a predicate on a set~$X$ may be viewed as a
subset of~$X$ or a (possibly non-computable) function $X \to \oProp$,
where $\oProp = \set{\ofalse, \otrue}$ is the set of truth values.
Accordingly, since in realizability propositions are witnessed by
realizers,
%
\iflong
a predicate~$\phi$ on a modest set~$A$ may be viewed as a
subset of $\setOf{A} \times \values{\typeOf{\phi}}$, or a (possibly
non-computable) function $\setOf{A} \times \values{\typeOf{\phi}} \to
\set{\ofalse, \otrue}$. In terms of pers, which is what RZ uses,
\fi
%
a predicate~$\phi$ on a per~$A = (\typeOf{A}, {\per_A})$ is a
(possibly non-computable) function $\phi : \values{\typeOf{A}} \times
\values{\typeOf{\phi}} \to \oProp$ that is
%
\iflong
\begin{itemize}
\item \emph{strict:} if $\phi(u,v)$ then $u \in \support{A}$, and
\item \emph{extensional:} if $\phi(u_1,v)$ and $u_1 \per_A u_2$ then
  $\phi(u_2,v)$.
\end{itemize}
\else % \iflong
\emph{strict} (if $\phi(u,v)$ then $u \in \support{A}$) 
and \emph{extensional} (if $\phi(u_1,v)$ and $u_1 \per_A u_2$ then
  $\phi(u_2,v)$).
\fi

\iflong
We illustrate how the realizability interpretation extracts the
computational content of a proposition. To make an interesting
example, suppose
\else % \iflong
Suppose
\fi % \iflong
we have implemented the real
numbers~$\RR$ as a per~$R = (\mathtt{real}, {\per_R})$, and
consider  
\iflong
the statement
that every cubic $x^3 + a x + b$ has a root,
%
\begin{equation}
  \label{eq:square-root}%
  \iforall{a}{R}{\iforall{b}{R}{\iexists{x}{R}{\iequal{x^3 + a x + b}{0}}}}.
\end{equation}
\else % \iflong
$\iforall{a}{R}{\iforall{b}{R}{\iexists{x}{R}{\iequal{x^3 + a x + b}{0}}}}$.
\fi % \iflong
%
By computing according to Figure~\ref{fig:rz-logic}, we see that
a realizer for this proposition is a value~$r$ of type
$\oarrow{\mathtt{real}}{\oarrow{\mathtt{real}}{\oprod{\mathtt{real}}{\mathtt{unit}}}}$
such that, if $t$ realizes $a \in \RR$ and $u$ realizes $b \in
\RR$, then $\oapp{\oapp{r}{t}}{u} = (v, w)$ with $v$ realizing a real
number~$x$ such that $x^3 + a x + b = 0$, and $w$ is trivial. (This
can be ``thinned'' to a realizer of type
$\oarrow{\mathtt{real}}{\oarrow{\mathtt{real}}{\mathtt{real}}}$ that
does not bother to compute~$w$.) In essence, the realizer~$r$
computes a root of the cubic equation. Note
that $r$ is \emph{not} extensional, i.e., different realizers~$t$
and~$u$ for the same~$a$ and~$b$ may result in different roots. 
To put this in another way, $r$ realizes a \emph{multi-valued}
function\footnote{The multi-valued nature of the realizer comes from
  the fact that it computes \emph{any one} of many values, not that it
  computes \emph{all} of the many values.} rather than a per
morphism. It is well known in computable mathematics that certain
operations, such as equation solving, are only computable if we allow
them to be multi-valued. They arise naturally in RZ as translations of
$\forall\exists$~statements.

\iflong
There are propositions whose realizers are ``irrelevant'' or free of
computational content. For example, realizers for $\itrue$ and
equality have type $\ounit$. Another example is a negation
$\inot{\phi}$, which is defined to be the same as
$\iimply{\phi}{\ifalse}$, whose realizers have type
$\oarrow{\typeOf{\phi}}{\ounit}$. Such realizers do not compute
anything useful, and we may as well throw them away. Sometimes only a
part of a realizer is computationally irrelevant, as we saw in the
last example. Propositions that are free of computational content
are characterized as the \emph{$\lnot\lnot$-stable propositions}. A
proposition~$\phi$ is said to be $\lnot\lnot$-stable, or just
\emph{stable} for short, when $\iimply{\inot{\inot{\phi}}}{\phi}$ is
valid. Any \emph{negative} proposition, i.e., one built from $\itrue$,
$\ifalse$, $=$, $\land$, $\Rightarrow$ and $\forall$ is stable, but
there may be other propositions that are stable and are not written
in the negative form.

It would be unproductive to bother the programmer with requirements
for useless code.
%
\else
%
Some propositions, such as equality and negation, have ``irrelevant'' realizers
free of computational content. Sometimes only a
part of a realizer is computationally irrelevant. 
Propositions that are free of computational content are
characterized as the \emph{$\lnot\lnot$-stable propositions}. A
proposition~$\phi$ is said to be $\lnot\lnot$-stable, or just
\emph{stable} for short, when $\iimply{\inot{\inot{\phi}}}{\phi}$ is
valid.
%
\fi
%
On input, one can specify whether abstract predicates
have computational content. On output, extracted realizers
go through a \emph{thinning} phase, which removes
irrelevant realizers.


\iflong
\subsection{Uniform families of modest sets}
\fi % \iflong
\label{sec:uniform-families}

Many structures are naturally viewed as families of sets, or sets
depending on parameters, or \emph{dependent types} as they are called
in type theory. For example, the $n$-dimensional Euclidean space
$\RR^n$ depends on the dimension $n \in \NN$, the Banach space
$\mathcal{C}([a,b])$ of uniformly continuous real functions on the
closed interval $[a,b]$ depends on $a, b \in \RR$ such that $a < b$,
etc. In general, a family of sets $\family{A_i}{i \in I}$ is an
assignment of a set $A_i$ to each $i \in I$ from an \emph{index
  set}~$I$.

\iflong
In the category of modest sets the appropriate notion is that of a
\emph{uniform} family~$\family{A_i}{i \in I}$, which is an assignment
of a modest set $A_i = (\setOf{A_i}, \typeOf{A}, {\rz_{A_i}})$ to each
$i \in \setOf{I}$, where $I$ is an index modest
set~\cite[6.3]{JacobsB:cltt}. The uniformity comes from the
requirement that all the~$A_i$'s share the same underlying
type~$\typeOf{A_i} = \typeOf{A}$. It is a desirable restriction from
the implementation point of view, because it removes dependencies at
the level of types. Note also that there is no dependency on the
realizers, only on the elements of the underlying set.

We may express uniform families in terms of pers, too.
\else
In the category of pers the appropriate notion is that of a
\emph{uniform} family.
%
\fi
A uniform family of pers $\family{A_i}{i \in I}$ indexed
by a per~$I$ is given by an underlying type $\typeOf{A}$ and a family
of pers $(\per_{A_i})_{i \in \values{\typeOf{I}}}$ that is
% 
\iflong
\begin{itemize}
\item \emph{strict:} if $u \per_{A_i} v$ then $i \in \support{I}$, and
\item \emph{extensional:} if $u \per_{A_i} v$ and $i \per_I j$ then $u
  \per_{A_j} v$.
\end{itemize}
\else % \iflong
strict (if $u \per_{A_i} v$ then $i \in \support{I}$) and
extensional (if $u \per_{A_i} v$ and $i \per_I j$ then $u
  \per_{A_j} v$).
\fi % \iflong

\iflong
We may form the \emph{sum} $\depsum{i \in I}{A_i}$ of a uniform family
$\family{A_i}{i \in I}$ as
%
\begin{align*}
  \typeOf{\depsum{i \in I}{A_i}} &=
  \typeOf{I} \times \typeOf{A}
  \\
  (i_1, u_1) \per_{\depsum{i \in I}{A_i}} (i_2, u_2)
  &\iff
  i_1 \per_I i_2 \land u_1 \per_{A_{i_1}} u_2
\end{align*}
%
and the \emph{product} $\depprod{i \in I}{A_i}$ as
%
\begin{align*}
  \typeOf{\depprod{i \in I}{A_i}} &=
  \typeOf{I} \to \typeOf{A}
  \\
  \support{\depprod{i \in I}{A_i}} &=
  \set{v \in \values{\typeOf{I} \to \typeOf{A}} \such
    \all{i, j}{\values{\typeOf{I}}}{i \per_{I} j \implies v\,i
      \per_{A_i} v\,j}
    } \\
  u \per_{\depprod{i \in I}{A_i}} v
  &\iff
  u, v \in \support{\depprod{i \in I}{A_i}} \land
  \all{i,j}{\values{\typeOf{I}}}{
    i \per_I j \implies u\;i \per_{A_i} v\;j
  }.
\end{align*}
%
These constructions allow us to interpret (extensional) dependent type
theory in the category of modest sets.
\else % \iflong
We can also form the \emph{sum} $\depsum{i \in I}{A_i}$ or
\emph{product} $\depprod{i \in I}{A_i}$ of
a uniform family, allowing an interpretation of (extensional) dependent type
theory.
\fi % \iflong

\iflong
As an example of a uniform family we consider the cyclic group
$(\ZZ_n, 0, {-}, {+})$ of order~$n$. To keep things simple, we assume
that~$n$ ranges over natural numbers that can be represented by
type~$\cint$ (i.e., $\typeOf{N}=\cint$), and that $\per_N$ is equality.
%
The uniform family $\family{Z_n}{n \in N}$ is then like the cyclic
group of order~$7$, with $7$ replaced by~$n$. Ignoring overflow, the
underlying type would be $\typeOf{Z_n} = \cint$. Any of the
implementations suggested for $\ZZ_7$ would work here, with~$7$
replaced by the parameter~$n$; in one case we would have $u \per_{Z_n}
v \iff u = v$ and in the other $u \per_{Z_n} v \iff u
\mathbin{\mathrm{mod}} n = v \mathbin{\mathrm{mod}} n$.
%
Negation would be specified as a constant of dependent type
$\Pi_{n \in N} Z_n \to Z_n$. Its realizer \texttt{neg} would then have
type $\typeOf{N}\to\typeOf{Z_n}\to{\typeOf{Z_n}}$, i.e.,
$\cint\to\cint\to\cint$, so that $\mathtt{neg}(n)$ would be a realizer
for negation on $\ZZ_n$. The realizer for addition would similarly
take an extra argument $n$.

\fi % \iflong


%%% Local Variables: 
%%% mode: latex
%%% TeX-master: "cie-long"
%%% End: 

\section{Specifications as signatures with assertions}
\label{sec:spec-sign-assert}


In programming we distinguish between \emph{impementation} and
\emph{specification} of a structure. In OCaml these two notions are
expressed with modules and module types, respectively.\footnote{In
  object-oriented languages implementations and specifications are
  expressed with classes and interfaces, while in Haskell they
  correspond to modules and declarations.} A module defines types and
values, while a module type simply lists the types, type definitions,
and values provided by a module. For a complete specification, a
module type must also be annotated with \emph{assertions} which
specify the required properties of declared types and values.
%
\begin{figure}
  \centering
  \sourcefile{group.ml}
  \caption{The module type $\mathtt{Ab}$ and module
    $\mathtt{Z7}$.}
  \label{fig:module-example}
\end{figure}
%
For example, if we look at the definition of module type $\mathtt{Ab}$
and module~$\mathtt{Z7}$ in Figure~\ref{fig:module-example}, we can
probably guess that $\mathtt{Z7}$ implements the group~$\ZZ_7$ and
that $\mathtt{Ab}$ is a signature for an Abelian group. However, for a
complete description of~$\mathtt{Z7}$ we still need to define a
per~$\perty{t}$ on $\values{\mathtt{t}}$ that tells us when two
integers represent the same element of~$\ZZ_7$, e.g.,
%
\begin{equation*}
  u \perty{t} v \iff
  u \mathbin{\mathrm{mod}} 7 = v \mathbin{\mathrm{mod}} 7 \;.
\end{equation*}
%
Similiarly, $\mathtt{Ab}$ by itself does not describe an Abelian
group, only its signature. A complete description would contain the
following further \emph{assertions}:
%
\begin{enumerate}
\item
  \label{enum:t-per}%
  there is a per $\perty{t}$ on $\values{\mathtt{t}}$,
\item
  \label{enum:zero-total}%
  $\mathtt{zero} \in \support{\mathtt{t}}$.
\item
  \label{enum:neg-total}%
  for all $u, v \in \values{\mathtt{t}}$, if $u \perty{t} v$ then
  $\mathtt{neg} \; u \perty{t} \mathtt{neg} \; v$,
\item
  \label{enum:add-total}%
  for all $u_1, u_2, v_1, v_2 \in \values{\mathtt{t}}$, if $u_1
  \perty{t} v_1$ and $u_2 \perty{t} v_2$ then $\mathtt{add} \; (u_1,
  u_2) \perty{t} \mathtt{add} \; (v_1, v_2)$,
\item 
  \label{enum:ab-group-axiom-1}%
  for all $u \in \support{\mathtt{t}}$, $\mathtt{add} \;
  (\mathtt{zero}, u) \perty{t} u$,
\item
  \label{enum:ab-group-axiom-2}%
  for all $u \in \support{\mathtt{t}}$, $\mathtt{add} \; (u,
  \mathtt{neg} \; u) \perty{t} \mathtt{zero}$,
\item
  \label{enum:ab-group-axiom-3}%
  for all $u, v, w \in \support{\mathtt{t}}$, $\mathtt{add} \;
  (\mathtt{add} \; (u, v), w) \perty{t} \mathtt{add} \; (u,
  \mathtt{add} \; (v, w))$,
\item
  \label{enum:ab-group-axiom-4}%
  for all $u, v \in \support{\mathtt{t}}$, $\mathtt{add} \; (u, v)
  \perty{t} \mathtt{add} (v, u)$.
\end{enumerate}
%
Assertions \ref{enum:zero-total}--\ref{enum:add-total} state that
$\mathtt{zero}$, $\mathtt{neg}$, and $\mathtt{add}$ realize a
constant, a unary, and a binary operation, respectively, while
assertions \ref{enum:ab-group-axiom-1}--\ref{enum:ab-group-axiom-4}
correspond to axioms for Abelian groups. RZ generates complete
specifications (module types with assertions), such as the above
module type~$\mathtt{Ab}$ with assertions
\ref{enum:t-per}--~\ref{enum:ab-group-axiom-4}. The output is written
in the language of specifications summarized in
Figure~\ref{fig:input}.

\begin{figure}
  \[
  \begin{array}{rl@{\qquad}l}
    \noalign{\textbf{Types}}
    \oty ::= 
    & \oTY  \obar \oselect{\oM}{\oTY} &\mbox{Type names}\\
    | & \ounit \obar \oprod{\oty_1}{\oty_2} &\mbox{Unit and cartesian product}\\
    | & \oarrow{\oty_1}{\oty_2} & \mbox{Function type}\\
    | & \osumty{\ol_1}{\oty_1}{\ol_n}{\oty_n} &\mbox{Disjoint sum}\\
    | & \opty & \mbox{Polymorphic types}\\[5pt]
    
    \noalign{\textbf{Terms}}	
    \oe ::=
    & \ox \obar \oselect{\oM}{\ox} &\mbox{Term names}\\
    | & \olambda{\ox}{\oty_1}{\oe} \obar 
    \oapp{\oe_1}{\oe_2} &\mbox{Functions and application}\\
    | & ()
    \obar \otuple{\oe_1}{\oe_n} 
    \obar \oproj{\oe}{n}&\mbox{Tuples and projection}\\
    | & \oinj{\ol}{\oe} 
    \obar (\omatches{\oe_0}{\ol_1}{\ox_1}{\oe_1}{\ol_n}{\ox_n}{\oe_n})&\mbox{Injection and projection from a sum}\\
    | & \ooblig{\ox}{\oty}{\op}{\oe} \obar \oobligx{\op}{\oe} &\mbox{Obligations}\\
    | & \olet{\ox}{\oe_1}{\oe_2}&\mbox{Local definitions}\\[5pt]
    

    \noalign{\textbf{Propositions (negative fragment)}}
    \op ::= 
    & \oP \obar \oselect{\oM}{\oP}&\mbox{Atomic
      proposition}\\
    | & \otrue  \obar \ofalse \obar \onot{\op_1} \obar \oand{\op_1}{\op_2} \obar 
    \oimply{\op_1}{\op_2} \obar \oiff{\op_1}{\op_2} & \mbox{Predicate logic}\\
    | & \olambda{\ox}{\oty}{\op}  \obar \oapp{\op}{\oe} &
    \mbox{Propositional functions and application}\\
    | & \oper{\os}{\oe_1}{\oe_2} \obar \ototal{\oe}{\os} & \mbox{Pers and support} \\
    | & \oequal{\oe_1}{\oe_2} & \mbox{(Observational) term equality}\\
    | & \oforall{\ox}{\oty}{\op}  \obar 
    \oforallt{\ox}{\os}{\op} & \mbox{Term quantifiers}\\[5pt]
    
    \noalign{\textbf{Basic modest sets}}
    \os ::=
    & \oS \obar \oapp{\os}{\oe} & \\[5pt]

    \noalign{\textbf{Modules}}		
    \om ::= 
    & \oM  \obar \oselect{\om}{\oM}&\mbox{Model names}\\
    | & \oapp{\om_1}{\om_2}&\mbox{Application of parameterized model}\\[5pt]
    
    \noalign{\textbf{Proposition Kinds}}
    \opt ::=
    & \oProp & \mbox{Classifier for propositions}\\
    |& \oarrow{\oty}{\opt} & \mbox{Classifier for a predicate/relation}\\[5pt] 
    
    \noalign{\textbf{Specification elements}}
    \ote ::=
    & \ovalspec{\ox}{\oty}&\mbox{Value declaration}\\
    | & \otyspec{\oTY}&\mbox{Type declaration}\\
    | & \otydef{\oTY}{\oty}&\mbox{Type definition}\\
    | & \omodulespec{\om}{\omt}&\mbox{Module declaration}\\
    | & \osignatdef{\oMT}{\omt}&\mbox{Specification definition}\\
    | & \opropspec{\oP}{\opt}&\mbox{Proposition declaration}\\
    | & \oassertion{\oA}{\op}&\mbox{Assertion}\\[5pt]

    \noalign{\textbf{Specifications (module types with assertions)}}
    \omt ::= 
    & \oMT \obar \oselect{\om}{\oMT} & \mbox{Specification names}\\
    | & \osig{\ote_1 \ldots \ote_n} & \mbox{Specification elements}\\
    | & \ofunctor{\om}{\omt_1}{\omt_2} & \mbox{Parametrized specification}\\
    | & \oapp{\omt}{\om} & \mbox{Specification application}
  \end{array}
  \]
  \label{fig:input}
  \caption{The syntax of specifications (Simplified)}
\end{figure}

RZ ever produces only a small subset of OCaml types (the unit type,
products, function types, polymorphic variant types, and parameter
types). Correspondingly, the language of terms is fairly simple
(tuples, functions, polymorphic variants, and local definitions).
However, the programmer is free to implement a specification using any
types and terms that exist in OCaml. A special kind of term is an
\emph{obligation} $\ooblig{\ox}{\oty}{\op}{\oe}$ which means ``in term
$e$, let $x$ be any element of $\values{\oty}$ that satisfies~$\op$''.
An obligation is equivalent to a combination of the indefinite
description operator~\cite{epsilon-reference} and a local definition,
$\olet{\ox}{(\varepsilon \ox {:} \oty.\,\op)}{\oe}$, where
$\varepsilon \ox {:} \oty.\, \op$ means ``any $x \in \values{\oty}$
such that $\op$''. The alternative form $\oobligx{\op}{\oe}$ stands
for $\ooblig{\_}{\ounit}{\op}{\oe}$. Obligations serve a double role.
On one hand they are \emph{local assertions} that refer to local
values inside a term, e.g.,
%
\begin{equation*}
  \olambda{x}{\mathtt{real}}{(\oobligx{x^2-x+1 \neq 0}{x/(x^2-x+1)})}.
\end{equation*}
%
On the other hand they are \emph{local declarations} that require the
implementation of a local value inside a term, e.g.,
%
\begin{equation*}
  \olambda{x}{\mathtt{real}}{(
    \ooblig{y}{\mathtt{real}}{\phi(x,y)}{x+y}
    )},
\end{equation*}
%
requires that the programer define a local value~$y$ such
that$\phi(x,y)$ holds. In principle, local specifications could always
be expressed as global ones, but at a significant cost to clarity and
conciseness.

RZ outputs obligations when it encounters an uncheckable typing
constraint, or a term in the input whose well-formedness requires the
existence of a suitable term in the output, see
Section~\ref{sec:exampl-with-oblig}. The programmer is supposed to
replace an obligation with an instance of a value satisfying the
obligation. If such a value does not exist, the specification is
unimplementable.

Assertions are expressed in the \emph{negative fragment} of
first-order logic, which contains constants for truth and falsehood,
negation, conjunction, implication, equivalence, and universal
quantification (but no disjunction or existential). This is the part
of first-order logic that has no computational content in the
realizability interpretation. Consequently, the classical and
constructive interpretations of assertions agree. This is quite
desirable, since RZ acts as a bridge between constructive mathematics
and real-world programmers, which typically are not familiar with
constructive logic. The basic predicates from which assertions are
built are observatinal equality $\oequal{\oe_1}{\oe_2}$, pers
$\oper{s}{\oe_1}{\oe_2}$ on abstract types, and basic predicates
declared as parameters. The formula $\ototal{\oe}{\os}$ stands for
$\oper{\os}{\oe}{\oe}$, while $\oforallt{\ox}{\os}{\op}$ is a
shorthand for
$\oforall{\ox}{\os}{(\oimply{\oper{\os}{\ox}{\ox}}{p})}$.

A specification is an OCaml module type with assertions. It may
contain value declarations, type declarations and definitions, module
declarations, specification definitions, proposition declarations, and
assertions. A specification $\ofunctor{\om}{\omt_1}{\omt_2}$ is a
module type of an OCaml functor. Such a parametrized specification may
be applied to a module. Because OCaml does not support such
applications, RZ reduces them away.

\internal{Andrej}{Maybe you can clarify the point about parameterized
  specifications applied to models (modules)? I seem to remember you
  once compared this to something that Automath has.}

%%% Local Variables: 
%%% mode: latex
%%% TeX-master: "cie"
%%% End: 

\iffalse
\newcommand{\ip}{p}
\newcommand{\ix}{x}
\newcommand{\is}{s}
\newcommand{\ie}{e}
\newcommand{\iP}{P}
\newcommand{\iS}{S}
\newcommand{\iprop}{\mathtt{prop}}
\newcommand{\itrue}{\mathtt{true}}
\newcommand{\ifalse}{\mathtt{false}}
\newcommand{\inot}[1]{\mathtt{not}\,#1}
\newcommand{\iimply}[2]{#1\,\mbox{\texttt{->}}\,#2}
\newcommand{\iiff}[2]{#1\,\mbox{\texttt{<->}}\,#2}
\newcommand{\iequal}[2]{#1\,\mathtt{=}\,#2}
\newcommand{\iands}[2]{#1\,\mathtt{/\backslash}\, #2}
\newcommand{\iors}[2]{#1\,\mathtt{\backslash/}\,#2}
\newcommand{\iforall}[3]{\texttt{forall}\,#1\,\texttt{:}\,#2\texttt{,}\ #3}
\newcommand{\iexists}[3]{\texttt{exists}\,#1\,\texttt{:}\,#2\texttt{,}\ #3}
\newcommand{\iunique}[3]{\texttt{unique}\,#1\,\texttt{:}\,#2\texttt{,}\ #3}
\newcommand{\ilambda}[3]{\texttt{lambda}\,#1\,\texttt{:}\,#2\texttt{,}\ #3}
\newcommand{\iapps}[3]{#1\ #2\,\cdots\,#3}

\newcommand{\ituples}[2]{\texttt{(}#1\texttt{,}\ldots\texttt{,}#2\texttt{)}}
\newcommand{\iproj}[2]{#1\texttt{.}#2}
\newcommand{\iconstrain}[2]{#1\,\texttt{:}\,#2}
\newcommand{\ithe}[3]{\texttt{the}\,#1\,\texttt{:}\,#2\texttt{,}\, #3}
\newcommand{\ilet}[3]{\texttt{let}\,#1\,\texttt{=}\,#2\,\texttt{in}\, #3}
\newcommand{\iquot}[2]{#1\,\texttt{\%}\,#2}
\newcommand{\iletquot}[4]{\texttt{let}\ #1\texttt{\%}\,#2\,\texttt{=}\,#3\ \texttt{in}\ #4}
\newcommand{\irz}[1]{\texttt{rz}\,#1}
\newcommand{\iletrz}[3]{\texttt{let}\ \texttt{rz}\,#1\,\texttt{=}\,#2\ \texttt{in}\ #3}

\newcommand{\iprods}[2]{#1\,\texttt{*}\cdots\texttt{*}\,#2}
\newcommand{\isums}[2]{#1\,\texttt{+}\cdots\texttt{+}\,#2}
\fi

\newcommand{\ip}{\varphi}
\newcommand{\ipp}{\rho}
\newcommand{\ix}{x}
\newcommand{\is}{s}
\newcommand{\ie}{e}
\newcommand{\iP}{p}
\newcommand{\iS}{\alpha}
\newcommand{\iprop}{\mathtt{prop}}
\newcommand{\itrue}{\top}
\newcommand{\ifalse}{\bot}
\newcommand{\inot}[1]{\lnot #1}
\newcommand{\iimply}[2]{#1\Rightarrow #2}
\newcommand{\iiff}[2]{#1 \Leftrightarrow #2}
\newcommand{\iequal}[2]{#1 = #2}
\newcommand{\iands}[2]{#1\land #2}
\newcommand{\iors}[2]{#1\lor #2}
\newcommand{\iforall}[3]{\forall #1{:}#2.\ #3}
\newcommand{\iexists}[3]{\exists #1{:}#2.\ #3}
\newcommand{\iunique}[3]{\exists! #1{:}#2.\ #3}
\newcommand{\ilambda}[3]{\lambda #1{:}#2.\ #3}
\newcommand{\iapps}[3]{#1\ #2\,\cdots\,#3}

\newcommand{\ituples}[2]{\texttt{(}#1\texttt{,}\ldots\texttt{,}#2\texttt{)}}
\newcommand{\iproj}[2]{#1\texttt{.}#2}
\newcommand{\iconstrain}[2]{#1\,\texttt{:}\,#2}
\newcommand{\ithe}[3]{\texttt{the}\,#1\,\texttt{:}\,#2\texttt{,}\, #3}
\newcommand{\ilet}[3]{\textsf{let}\,#1{=}#2\,\textsf{in}\, #3}
\newcommand{\ieclass}[2]{[#1]_{#2}}
\newcommand{\ileteclass}[4]{\texttt{let}\ [#1]_{#2}\,\texttt{=}\,#3\ \texttt{in}\ #4}
\newcommand{\irz}[1]{\texttt{rz}\,#1}
\newcommand{\iletrz}[3]{\texttt{let}\ \texttt{rz}\,#1\,\texttt{=}\,#2\ \texttt{in}\ #3}

\newcommand{\iprod}[3]{#1{:}#2\times#3}
\newcommand{\isums}[2]{#1+\cdots+\,#2}
\newcommand{\iquot}[2]{#1{\bigm/} #2}
\newcommand{\isubset}[3]{\{\,#1{:}#2\ |\ #3\,\}}


\section{The Input Language}
\label{sec:input-language}

\begin{figure}[htbp]
\VerbatimInput{kuratowski.thy}
	\caption{caption}
	\label{fig:typicalinput}
\end{figure}

Figure~\ref{fig:typicalinput} shows a sample input for the RZ system that
generates the output shown in \internal{CS}{somewhere}. The input is the
definition of one or more \emph{theories}. A \emph{theory} is a generalized
logical signature with associated axioms, similar to a Coq module signature.
The components of a theory may specify that a certain set must exist, that an
element of a set must exist, that a proposition or first-order predicate must
exist, a model of some other theory must exist, or specify an axiom that must
hold.

\internal{CS}{Actually, I should probably just follow the CASE approach of introducing the syntax in-line as each construct is mentioned.  This could save significant space for the short paper.}

As shown in in Figure~\ref{fig:typicalinput}, propositions and predicates
appearing in theories may use full first-order constructive logic, not just the
negative fragment. The grammar for logical inputs is shown in
Figure~\ref{fig:input}. Most of this should be familiar, including the use of
lambda abstraction are to define predicates.

The language of sets is rich, going well beyond the type systems of typical
programming languages. In addition to any base sets postulated in a theory, one
can construct dependent cartesian products and dependent function spaces. We
also supports disjoint unions (with labeled tags), quotient spaces (a set
modulo an equivalence relation), subsets (elements of a set satisfying a
predicate). RZ even permits explicit references to sets of realizers.

The term language primarily follows the set structure by providing
introduction and elimination constructs. For product sets we have tuples and
projections ($\iproj{\ie}{1}$, $\iproj{\ie}{2}$, \ldots), and for function spaces we have lambda abstractions and
application. One can inject a term into a tagged dependent union, or do case analyses
on the members of a union. We can produce the equivalence class of a given
term, or pick a representative of a given equivalence class (as long as what we
do with it does not depend on the particular representative). Very similarly,
we can produce the set of realizers for a term, or choose a representative from
a given set of realizers (as long as what we do with it does not depend on the
particular representative). We can inject a term into a subset (if it satisfies
the appropriate predicate), or project an element of a subset into the
enclosing set. Finally, the term language allows local definitions of term
variables, and definite descriptions (as long as there is a unique element
satisfying the predicate in question).

From the parenthesized qualifiers in the previous paragraph, it is clear that checking the
well-formedness of terms is not decidable. RZ checks what it can, but does not
attempt any serious theorem proving.  Uncheckable constraints remain as
explicitly-marked \emph{obligations} in the final output, and must be verified
by other means; the output can then be trusted to the extent that the obligations are satisfied.






\begin{figure}
	\[
	\begin{array}{rll}
		\ip,\ipp ::= 
		    & \iP\\
		  | & \itrue \ | \ \ifalse\ |\ \inot{\ip}\ |\ \iands{\ip}{\ip}\ |\ 
		       \iors{\ip}{\ip}\ | \ \iimply{\ip}{\ip}\ |\ \iiff{\ip}{\ip}\\
		  | & \ilambda{\ix}{\is}{\ip} \ |\ \iapps{\ip}{\ie_1}{\ie_n}\\
		  | & \iequal{\ie}{\ie}\\
		  | & \iforall{\ix}{\is}{\ip} \ |\ 
		      \iexists{\ix}{\is}{\ip} \ |\
		      \iunique{\ix}{\is}{\ip}\\[10pt]
		
		\is ::= 
		    & \iS\\
		  | & ()\ |\ \iprod{\ix}{\is}{\is}\\
		  | & \ilambda{\ix}{\is}{\is}\ |\ 
		      \iapps{\is}{\ie_1}{\ie_n}\\
		  | & \isums{\is}{\is}\\
		  | & \iquot{\is}{\ipp}\\
		  | & \isubset{\ix}{\is}{\ipp}\\
		  | & \irz{\is}\\[10pt] 
		
	
		\ie ::=
		    & \ix\\
		  | & \iapps{\ie}{\ie_1}{\ie_n}\\
		  | & \ilambda{\ix}{\is}{\ie}\\
		  | & \ituples{\ie}{\ie}\\
		  | & \iproj{\ie}{n}\\
		  | & \iconstrain{\ie}{\is}\\
		  | & \ithe{\ix}{\is}{\ip}\\
		  | & \ilet{\ix}{\ie}{\ie}\\
		  | & \ieclass{\ie}{\ipp}\\
		  | & \ileteclass{\ix}{\ipp}{\ie}{\ie}\\
		  | & \irz{\ie}\\
		  | & \iletrz{\ix}{\ie}{\ie}\\
	\end{array}
	\]
	\label{fig:input}
	\caption{Input Syntax}
\end{figure}



Explain the input language and its semantics.



\internal{AB}{The plan here seems to be as follows. Before this
  section we will say that assertions are written in the negative
  fragment, and that in the input language we use constructive logic,
  which turns out to be more convenient. So just go ahead and
  introduce constructive logic without worrying what its
  interpretation might be. Then in the translation secton we'll
  explain the realizability interpretation of constructive logic.
  Actually, the logic is first-order intuitionistic logic. The fact
  the the logic is intuitionistic is not a choice but reality: the
  interpretation does not validate the Law of Excluded Middle. This
  should be commented on.}


%%% Local Variables: 
%%% mode: latex
%%% TeX-master: "cie"
%%% End: 
%!TEX root = cie


\section{Translation}
\label{sec:translation}

Having seen the input and output languages for RZ, we now explain how
the translation from one to the other works. The mathematical basis
for the translation procedure is the \emph{realizability
  interpretation} of constructive type theory and logic in the
category of modest sets, which was described in
Section~\ref{sec:modest-sets-pers}. A theory is translated to a
specification, where the theory elements are translated as follows.

A set declaration $\iParameter{s}{\iSet}$ is translated to
%
\begin{align*}
  & \otyspec{s}. \\
  & \opropspec{({\approx_{s}})}{\oarrow{s}{\oarrow{s}{\oProp}}}.\\
  & \oassertion{\mathrm{per}_{s}}{
    \begin{aligned}[t]
      & \oforall{x\,y}{s}{(\oimply{\oper{s}{x}{y}}{\oper{s}{y}{x}})}
        \land {} \\
      & \oforall{x\,y\,z}{s}{(
        \oimply{
          \oper{s}{x}{y} \land \oper{s}{y}{z}
          }{\oper{s}{x}{z}}
        )}.
    \end{aligned}
  }
\end{align*}
%
This says that the programmer should define a type~$s$, and a binary
predicate~$\approx_s$ on $\values{s}$ which is symmetric and
transitive.\footnote{The predicate is \emph{not} an ocaml value of
  type $s \to s \to \mathtt{bool}$, but an abstract relation on the
  set $\values{s} \times \values{s}$. Only in special cases can we
  implement the per as a decidable test.} When a dependent set is
declared in the input, e.g., $\iParameter{t}{s \to \iSet}$, the
translation follows the interpretation of dependent sets as uniform
families, cf.\ Section~\ref{sec:uniform-families}. The underlying
output type is still non-dependent, but the per
$\approx_{\oapp{t}{{-}}}$ is parametrized by~$s$, so it is declared to
have type $\oarrow{s}{\oarrow{t}{\oarrow{t}{\oProp}}}$.

An element declaration $\iParameter{x}{s}$ is translated to
%
\begin{align*}
  & \ovalspec{x}{s} \\
  & \oassertion{\mathrm{support}_x}{\ototal{x}{s}}
\end{align*}
%
which requires the definition of a value~$x$ of type~$s$ which is in
the support of~$s$. When $s$ is not a basic set, RZ computes the
interpretation of the underlying type and support.

Constructions of sets in the input language are translated to
corresponding constructions of modest sets. In
Section~\ref{sec:modest-sets-pers} we saw how products, exponentials
and their dependent versions are formed. We briefly review the
remaining constructions. A disjoint union
$t = \isum{\il_1}{\is_1}{\il_2}{\is_2}$ is translated to the variant type
$t = \osumtyx{\il_1}{\is_1}{\il_2}{\is_2}$ with the per
%
\begin{align*}
  \oper{t}{\oinj{\il_1}{u}}{\oinj{\il_1}{v}} &\iff \oper{s_1}{u}{v},
  \\
  \oper{t}{\oinj{\il_2}{u}}{\oinj{\il_2}{v}} &\iff \oper{s_2}{u}{v}.
\end{align*}
%

quotient

subsets

rz

Propositions and axioms.

%%% Local Variables: 
%%% mode: latex
%%% TeX-master: "cie"
%%% End: 

\section{Implementation}
\label{sec:implementation}

The RZ implementation consists of several sequential passes.

After the initial parsing, a \emph{type reconstruction} phase checks
that the input is well-typed (and checks for well-formedness to the
extent that it is easily decidable), and if successful produces an
annotated result with all variables explicitly tagged with types. The
type checking phase uses a system of dependent types, with limited
subtyping (implicit coercions) for sum types and subset types. The
details are fairly standard, so are omitted here. One non-obvious
consequence of the realizability translation, however, is that the
subset types $\isubset{\ix}{\iS}{\iand{\ipp_1(\ix)}{\ipp_2(\ix)}}$ and
$\isubset{\ix}{\iS}{\iand{\ipp_2(\ix)}{\ipp_1(\ix)}}$ are not
equal, but only isomorphic in general. An
explicit coercion is required to go from one type to the other,
because subset values are pairs containing realizers for
$\iand{\ipp_1(\ix)}{\ipp_2(\ix)}$ and
$\iand{\ipp_2(\ix)}{\ipp_1(\ix)}$, and these realizers have
potentially different types $|\ipp_1(\ix)|\mathtt{*}|\ipp_2(\ix)|$ and
$|\ipp_2(\ix)|\mathtt{*}|\ipp_1(\ix)|$ respectively.

Next the realizability translation is performed as described in
Section~\ref{sec:translation}, producing interface code. The
flexibility of the full input language (e.g., $n$-ary sum types and
dependent product types) makes the translation code fairly involved,
and so it is performed in a ``naive'' fashion whenever possible. The
immediate result of the translation is not easily readable.
 
Thus, three more passes simplify the output before it is displayed to
the user. A \emph{thinning} pass removes all references to trivial
realizers produced by stable formulas. For example, direct translation
of the $\mathtt{free}$ axiom in the output for Kuratowski-finite sets
yields a value specification for $\mathtt{free}$ of type
%
\begin{equation*}
  (\mathtt{A.a} \to \mathtt{S.s}) \to 
  (\mathtt{fin} \to \mathtt{S.s}) * (\ounit * (\mathtt{A.a} \to
  \ounit) *
  (\mathtt{fin} \to \mathtt{fin} \to \ounit))
\end{equation*}
%
where $\ounit$ is the unit (terminal) type classifying the trivial
realizer. Thinning replaces this by the isomorphic type
%
\begin{equation*}
  (\mathtt{A.a} \to \mathtt{S.s}) \to \mathtt{fin} \to \mathtt{S.s}
\end{equation*}
%
and appropriately modifies references to $\mathtt{free}$ in the assertions to account for this change in type.

Next, an \emph{optimization} pass applies an ad-hoc collection of
basic logical and term simplifications in order to make the output more readable. 
Logical simplifications include applications of truth table rules
($\iand{\itrue}{\ip}$ becomes $\ip$), detection of syntactically
identical premises and conclusions
($\iimply{\ip_1}{\iand{\ip_1}{\ip_2}}$ becomes
$\iimply{\ip_1}{\ip_2}$), and optimization of other common patterns we have
seen arise
($\iforall{\ix}{\is}{\iimply{(\iequal{\ix}{\ie})}{\ipp(\ix)}}$ becomes
$\ipp(\ie)$). We do not attempt real theorem proving 
so some redundancy may remain, but in practice the optimization pass
can help significantly.

Finally, the user can specify whether two optional steps occur.
RZ can optionally performs a \emph{phase-splitting} pass~\cite{harper+:popl90}. 
This is an ML-specific transformation that replace certain
uses of parameterized modules (a heavyweight language construct) by
parameterized types and polymorphic values. The idea is that although
functors map modules containing types and terms to other modules containing types
and terms, constraints on the programming language ensure that output types
depend only on input types (and not input terms).  Thus, we can split each
functor into a mapping from input types to output types, and then a separate
(polymorphic) term mapping input types and terms to an output term.

For example (ignoring
assertions for simplicity) the entire module
\begin{source}
module Free : functor (S : Semilattice) ->
                    sig
                      val free : (A.a -> S.s) -> fin -> S.s
                    end	
\end{source}   
appearing in the output of the Kuratowski example can be replaced by the single polymorphic function
\begin{source}
val free : 's -> ('s -> 's -> 's) -> (A.a -> 's) -> fin -> 's	
\end{source}
which replaces the module parameter \texttt{S} by two extra term arguments term (corresponding to the module components \texttt{S.zero} and \texttt{S.join}) 
and a type argument \texttt{'s} for the type of lattice elements (corresponding to the module input \texttt{S.s}).

The other optional transformation is a \emph{hoisting} pass which lifts obligations in the output out to top-level positions.  This can make it easier to see exactly what one is obliged to provide.  When identical obligations appear in separate subterms of a term, hoisting can lift and merge these obligations, reducing redundancy.  However, moving obligations far from where they are used can make it harder to see why the obligation is required at all (and hence how one might satisfy the obligation), and so hoisting is turned off by default.

For example, in the following input (extracted from a larger description of an ordered field)
\begin{Verbatim}
Parameter s : Set.
Parameter zero : s.
Parameter inverse : {x : s | not (x = zero)} -> s.

Parameter lt : s -> s -> Stable.
Definition positive (x:s) := lt zero x.
Axiom lt_irr: forall x:s, not (lt x x).

Axiom order_inv: forall x:s, positive x -> positive (inverse x).
\end{Verbatim}
the axiom translates to the assertion:
\begin{Verbatim}
	(**  Assertion order_inv = 
          forall (x:||s||),  positive x ->
            positive (inv (assure (not (x =real= zero)) in x))
   *).
\end{Verbatim}
Here the system has noticed that for \Verb|inverse x| to make sense, we must know that
\Verb|x| is non-zero.   This requires non-trivial theorem proving and hence remains as 
an obligation for the user.  

We must prove \Verb|not (x =s= zero)| not for all \Verb|x|, but only under
the premises in force where the obligation occurs.  This is slightly clearer when hoisting moves the
obligation to the top level, after which it could be verified in the same way as all other assertions:
\begin{Verbatim}
   (**  Assertion order_inv = 
          assure (forall (x:||real||),  positive x -> not (x =real= zero))
            in forall (x:||real||),  positive x -> positive (inv x)
   *)	
\end{Verbatim}


%%% Local Variables: 
%%% mode: latex
%%% TeX-master: "cie"
%%% End: 

\section{Examples}
\label{sec:examples}

In this section we look at several examples which demonstrate various
points of RZ. Unfortunately, serious examples from computable
mathematics take too much space\footnote{The most basic structure
  in analysis (the real numbers) alone 
  requires several operations and a dozen or more axioms.} and will have to
be presented separately. The main theme is that constructively
reasonable axioms yield computationally reasonable operations.

\subsection{Decidable sets}
\label{sec:decidable-sets}

A set $S$ is said to be decidable when, for all $x, y \in S$, $x = y$
or $\lnot (x = y)$. In classical mathematics all sets are decidable
because decidability of equality is just an instance of the law of
excluded middle. However, RZ computes from
%
\sourcefile{decidable1.thy}
%
that the realizer for the axiom $\mathtt{eq}$ is specified by
%
\begin{source}
val eq : s -> s -> [`or0 | `or1]
(**  Assertion eq =
       forall (x:||s||, y:||s||),
         (match eq x y with
            `or0 => x =s= y
          | `or1 => not (x =s= y)
          )
*)
\end{source}
%
We read this as follows: $\mathtt{eq}$ is a function which takes
arguments~$x$ and~$y$ of type~$s$ and returns $\mathtt{`or0}$ or
$\mathtt{`or1}$. If it returns $\mathtt{`or0}$, then $x \per_s y$, and
if it returns $\mathtt{`or1}$, then $\lnot (x \per_s y)$. In other
words $\mathtt{eq}$ is a decision procedure which tells when
values~$x$ and~$y$ represent the same element of the modest set.

\subsection{Examples with obligations}
\label{sec:exampl-with-oblig}

In this section we show several small examples in which RZ outputs
obligations.

Consider how we might define division of real numbers. Given the set
of real numbers~$\mathtt{real}$ and a constant $\mathtt{zero}$
denoting~$0$, we might write
%
\sourcefile{real.thy}

\internal{Andrej}{Unfinished section.}


\subsection{Finite sets}
\label{sec:finite-sets}

There are many characterizations of finite sets, but the one that
works best constructively is due to Kuratowski, who identified the
finite subsets of~$A$ as the least family~$K(A)$ of subsets of~$A$
that contains the empty set and is closed under unions with
singletons. This characterization relies on powersets, which are not
available in RZ. But the gist of it, namely that $K(A)$ is an
\emph{inital} structure a suitable sort, can be expressed as follows.

Recall that a \emph{$\vee$-semilattice} is a set~$S$ with a
constant~$0 \in S$ and an associative, commutative, and idempotent
operation ``join'' $\vee$ on~$S$ such that $0$ is the neutral element
for~$\vee$, see Figure~\ref{fig:semilattice} for RZ axiomatization of
semilattices.
%
\begin{figure}
  \centering
\begin{source}
Definition Semilattice :=
thy
  Parameter s : Set.
  Parameter zero : s.
  Parameter join : s -> s -> s.
  Implicit Type x y z : s.
  Axiom commutative: forall x y,   join x y = join y x.
  Axiom associative: forall x y z, join (join x y) z = join x (join y z).
  Axiom idempotent:  forall x,     join x x = x.
  Axiom neutral:     forall x,     join x zero = x.
end.
\end{source}
  \caption{The theory of a semilattice}
  \label{fig:semilattice}
\end{figure}
%
The Kuratowski finite sets~$K(A)$ are the \emph{free} semilattice
generated by a set~$A$, where $\vee$ is union and $0$ is the empty
set. This is formalized in RZ as shown in Figure~\ref{fig:kuratowski}.
%
\begin{figure}
\centering
\begin{source}
Definition K (A : thy 
                Parameter a : Set.
              end) :=
thy
  include Semilattice.
  Parameter singleton : A.a -> s.
  Definition fin := s.
  Definition emptyset := zero.
  Definition union := join.

  Axiom free :
    forall S : Semilattice, forall f : A.a -> S.s,
    exists1 g : fin -> S.s, 
      g emptyset = S.zero /\
        (forall x : A.a, f x = g (singleton x))/\
        (forall u v : fin, g (union u v) = S.join (g u) (g v)).
end.
\end{source}
  \caption{Kuratowski finite sets}
  \label{fig:kuratowski}
\end{figure}
%
The theory $K$ is parametrized by a model~$A$ which contains a
set~$a$. In the first line we include the theory of semilattices. Then
we postulate an operation $\mathtt{singleton}$ which injects the
generators into the semilattice. The three definitions are just a
convenience, so that we can refer to the parts of $K(A)$ by their
natural names, e.g., $\mathtt{emptyset}$ instead of $\mathtt{zero}$.
The axiom $\mathtt{free}$ expresses the fact that $K(A)$ is the free
semilattice on~$A.a$: for every semilattice $S$ and a map $f : A.a \to
S.s$ from the generators to the underlying set of~$S$, there exists a
unique semilattice homomorphism $g : \mathtt{fin} \to S.s$ such that
$f(x) = g(\set{x})$.

The output for $\mathtt{Semilattice}$ and~$\mathtt{K}$ specifies
values of suitable types for each declared constant and operation. All
axioms but the last one are equations and have straightforward
translations in terms of underlying pers. The output for the axiom
$\mathtt{free}$ is shown in Figure~\ref{fig:free}.
%
\begin{figure}
  \centering
\begin{source}
module Free : functor (S : Semilattice) ->
sig
val free : (A.a -> S.s) -> fin -> S.s
(**  Assertion free = 
forall (f:||A.a -> S.s||), 
  let g = free f in g : ||fin -> S.s|| /\ 
  g emptyset =S.s= S.zero /\ 
  (forall (x:||A.a||),  f x =S.s= g (singleton x)) /\ 
  (forall (u:||fin||, v:||fin||), g (union u v) =S.s= S.join (g u) (g v)) /\ 
  (forall h:fin -> S.s,  h : ||fin -> S.s|| /\ 
     h emptyset =S.s= S.zero /\ 
     (forall (x:||A.a||), f x =S.s= h (singleton x)) /\ 
     (forall (u:||fin||, v:||fin||), 
        h (union u v) =S.s= S.join (h u) (h v)) ->
     forall x:fin, y:fin,  x =fin= y -> g x =S.s= h y)
*)
end
\end{source}
  \caption{Output of axiom $\texttt{free}$.}
  \label{fig:free}
\end{figure}
%
Because the axiom quantifies over all models~$S$ of the theory
$\mathtt{Semilattice}$ its translation is a functor~$\mathtt{Free}$
which accepts an implementation of a semilattice~$S$ and yields a
realizer $\mathtt{free}$ validating the axiom. The computational
meaning of $\mathtt{free}$ is a folding operation on finite sets: take
a map $f : A.a \to S.s$ and a finite set~$u = \set{x_1, \ldots, x_n}$,
and return $f(x_1) \vee \cdots \vee f(x_n)$, where $\vee$ is the join
operation on the semilattice~$S$.

The OCaml standard library contains a module $\mathtt{Set}$
implementing finite sets, which however is \emph{not} an
implementation of Kuratowski finite sets presented here. Rather,
$\mathtt{Set}$ implements something close to Kuratowski finite sets
over a set~$A$ equipped with a decidable linear order.


\subsection{Inductive types}
\label{sec:inductive-types}

To demonstrate the use of dependent types we show how RZ handles
general inductive types, also known as
\emph{W-types}~\cite{w-type-reference}. Recall that a W-type is a set
of well-founded trees, where the branching types of trees are
described by a family of sets $B = \set{T(x)}_{x \in S}$. Each node in
a tree has a \emph{branching type}~$x \in S$, which determines that
the successors of the node are labeled by the elements of~$T(x)$. For
example, to get non-empty binary trees whose leaves are labeled by
natural numbers, define
%
\begin{align*}
  S &= \set{\mathtt{cons}} \cup \set{\mathtt{leaf}(n) \such n \in \NN}
  \\
  T(\mathtt{cons}) &= \set{\mathtt{left}, \mathtt{right}}
  \\
  T(\mathtt{leaf}(n)) &= \emptyset.
\end{align*}
%
Then a node of type $\mathtt{cons}$ has two successors, indexed by
constants $\mathtt{left}$ and $\mathtt{right}$, while a node of type
$\mathtt{leaf}(n)$ does not have any successors.

Figure~\ref{fig:wtype} shows an RZ axiomatization of W-types.
%
\begin{figure}
  \centering
  \sourcefile{wtype.thy}
  \caption{General inductive types}
  \label{fig:wtype}
\end{figure}
%
The theory $\mathtt{Branching}$ describes that a branching type
consits of a set~$s$ and a set~$t$ depending on~$s$. The theory~$W$ is
parametrized by a branching type~$B$. It specifies a set~$w$ of
well-founded trees and a tree-forming operation $\mathtt{tree}$ with a
dependent type $\prod_{x \in B.s} (B.t(x) \to w) \to w$. Given a
branching type~$x$ and a map $f : B.t(x) \to w$, $\mathtt{tree}\,x\,f$
is the tree whose root has branching type~$x$ and whose successor
labeled by $\ell \in B.t(x)$ is the tree~$f(\ell)$. The inductive
nature of~$w$ is expressed with the axiom $\mathtt{induction}$, which
states that for every property $M.p$, if $M.p$ is an inductive
property then every tree satsifies it. A property is said to be
\emph{inductive} if a tree $\mathtt{tree}\,x\,f$ satisfies it whenever
all its successors satisfy it.

In the translation, see Appendix~\ref{sec:outp-induct-types},
dependencies at the level of types and terms disappear. A branching
type is determined by a pair of non-dependent types~$s$ and~$t$ but
the per $\per_t$ depends on~$\values{s}$. The theory~$W$ turns into a
signature for a functor receiving a branching type~$B$ and returning a
type~$w$, and an operation $\mathtt{tree}$ of type $B.s \to (B.t \to
w) \to w$. In this example we use phase-splitting (see
Section~\ref{sec:implementation}) to translate axiom
$\mathtt{induction}$ into a specification of a polymorphic function
(compare with output for axiom \texttt{free} in the previous example)
%
\begin{equation*}
  \mathtt{induction}:
  (B.s \to (B.t \to w) \to (B.t \to \poly{ty\_p}) \to \poly{ty\_p}) \to w \to \poly{ty\_p},
\end{equation*}
%
which is a form of recursion on well-founded trees. Instead of trying
to explain what $\mathtt{induction}$ is supposed to do, we show in
Figure~\ref{fig:wtype-implementation} a surprisingly simple, complete
hand-written implementation of W-types. The reader may entertain
himself by figuring out how $\mathtt{induction}$ works.
%
\begin{figure}
  \centering
  \sourcefile{wtype.ml}
  \caption{An implementation of general inductive types.}
  \label{fig:wtype-implementation}
\end{figure}


\subsection{Axiom of choice}
\label{sec:axiom-choice}

In this example we show how RZ can help explain why a generally
accepted axiom is not constructively valid. Consider the Axiom of
Choice:
%
\sourcefile{choice.thy}
%
The relevant part of the output is
%
\begin{source}
val ac : (a -> b * ty_r) -> (a -> b) * (a -> ty_r)
(**  Assertion ac =
  forall f:a -> b * ty_r,
    (forall (x:||a||),  let (p,q) = f x in p : ||b|| /\ r x p q) ->
    let (g,h) = ac f in g : ||a -> b|| /\
    (forall (x:||a||),  r x (g x) (h x))
*)
\end{source}
%
This requires a function $\mathtt{ac}$ which accepts a function $f$
and computes a pair of functions $(g,h)$. The input function~$f$ takes
an $\ototal{x}{a}$ and returns a pair $(p,q)$ such that $q$ realizes
the fact that $r\,x\,p$ holds. The output functions $g$ and $h$ taking
$\ototal{x}{a}$ as input must be such that $h\,x$ realizes
$r\,x\,(g\,x)$. Crucially, the requirement $\ototal{g}{\oarrow{a}{b}}$
says tht $g$ must be extensional, i.e., map equivalent realizers to
equivalent realizers. We could define~$h$ as the first component
of~$f$, but we cannot hope to implement~$g$ in general because the
second component of~$f$ is not assumed to be extensional.

The \emph{Intensional} Axiom of Choice allows the choice function to
depend on the realizers:
%
\sourcefile{ichoice.thy} Now the output is
%
\begin{source}
val iac : (a -> b * ty_r) -> (a -> b) * (a -> ty_r)
(**  Assertion iac =
  forall f:a -> b * ty_r,
    (forall (x:||a||),  let (p,q) = f x in p : ||b|| /\ r x p q) ->
    let (g,h) = iac f in (forall x:a,  x : ||a|| -> g x : ||b||) /\
    (forall (x:||a||),  r x (g x) (h x))
*)
\end{source}
%
which is exactly the same as before, \emph{except} that the
troublesome requirement $\ototal{g}{\oarrow{a}{b}}$ turned into
$\oforall{x}{a}{(\oimply{\ototal{x}{a}}{\ototal{g\,x}{b}})}$, which
is weaker. We can impement $\mathtt{iac}$ as
%
\begin{source}
let iac f = (fun x -> fst (f x)), (fun x -> snd (f x))
\end{source}
%

The Intensional Axiom of Choice is in fact just an instance of the
usual Axiom of Choice applied to~$\irz{A}$ and~$B$. Combined with the
fact that~$\irz{A}$ covers~$A$, this establishes the validity of
\emph{Presentation Axiom}~\cite{barwise75:_admis_sets_struc}, which
states that every set is an image of one satisfying the axiom of
choice.

\subsection{Modulus of Continuity}
\label{sec:we-show-modulus-of-continuity-example}

As a last example we show how certain constructive principles require
the use of computational effects. To keep the example short, we
presume that we are already given the set of natural
numbers~$\mathtt{nat}$ with the usual structure.

A \emph{type 2 function} is a map $f : (\mathtt{nat} \to \mathtt{nat})
\to \mathtt{nat}$. It is said to be continuous if the output of $f(a)$
depends only on an initial segment of the sequence~$a$. We can express
this axiom in RZ as follows:
%
\begin{source}
Axiom continuity:
forall f : (nat -> nat) -> nat, forall a : nat -> nat,
  exists k, forall b : nat -> nat,
    (forall m, m <= k -> a m = b m) -> f a = f b.
\end{source}
%
The axiom says that for any $f$ and $a$ there exists $k \in
\mathtt{nat}$ such that $f(b) = f(a)$ as soon as the sequences~$a$
and~$b$ agree on the first $k$ terms. The axiom is translated to the
specification
%
\begin{source}
val continuity : ((nat -> nat) -> nat) -> (nat -> nat) -> nat
(**  Assertion continuity =
forall (f:||(nat -> nat) -> nat||, a:||nat -> nat||),
  let p = continuity f a in p : ||nat|| /\
  (forall (b:||nat -> nat||),
     (forall (m:||nat||),  m <= p -> a m =nat= b m) -> f a =nat= f b)
*)
\end{source}
%
which says that $\mathtt{continuity}\,f\,a$ is a number~$p$ such that
$f(a) = f(b)$ whenever $a$ and $b$ agree on the first~$p$ terms. In
other words, $\mathtt{continuity}$ is a \emph{modulus of continuity}
functional. It cannot be implemented in a purely functional
language~\cite{modulus-violates-ac2}, but with the use of store we can
implement it as
%
\begin{source}
let continuity f a =
  let k = ref 0 in
  let a' n = (k := max !k n; a n) in
    f a' ; !k
\end{source}
%
To compute a modulus for~$f$ at~$a$, the program creates a
function~$a'$ which is just like~$a$ except that it stores in~$p$ the
largest argument at which it has been called. Then $f\,a'$ is
computed, its value it discarded, and the value of~$p$ is returned.
The program works because~$f$ is assumed to be extensional and must
therefore not distinguish between extensionally equal sequences~$a$
and~$a'$.



%%% Local Variables: 
%%% mode: latex
%%% TeX-master: "cie"
%%% End: 

\section{Related Work}
\label{sec:related-work}

\subsection{Coq}
\label{sec:comparison-with-coq}

Coq is a very flexible system, with complete support for
theorem-proving and creating trusted code. One common pattern of use
for Coq is to write code in Coq's functional language (values whose
types are \texttt{Set} in Coq), to state and prove theorems stating
that the code behaves correctly (where the theorems are Coq values
whose types are \texttt{Prop} in Coq), and then have Coq produce
guaranteed correct code in ML. In such cases, RZ is complementary to
Coq. RZ can clarify the constructive content of mathematical
structures and hence suggest an appropriate division between Coq's
\texttt{Set} and \texttt{Prop}, i.e., which values and which theorems
should appear in the input to Coq. (It should be easy to have RZ
produce output in Coq syntax, and we hope to do this eventually.)

In general, RZ is a smaller and more lightweight system and thus more
flexible where it applies. It is not always practical or necessary to
do theorem proving in order to provide an implementation; interfaces
generated by RZ can be implemented in any manner from theorem proving
to directly writing code. And, RZ provides a way to talk with
programmers about implications of constructive mathematics for their work without bringing in
full theorem proving.


\subsection{Other tools}

Komagata and Schmidt~\cite{komagata+:tr95} describe a system that uses
a similar realizability translation to ours. Like Coq, the system
translates formal proofs to programs.
%
An interesting technical difference is that the algorithm they use,
attributed to John Hatcliff, does thinning as it goes along, rather
than making this a separate pass. For example, the translation of the
conjunction-introduction rule has four cases, depending on whether the
left and/or right propositions being proved are [almost?] negative, in
which case the trivial contribution can be immediately discarded.

\subsection{Other Models of Computability}
\label{sec:models-of-computability}

Many formulations of computable mathematics are based on realizability
models~\cite{Bauer:00}, even though they were not initially developed,
nor are they usually presented within the framework of realizability:
Recursive Mathematics~\cite{ershov98:_handb_recur_mathem} is based on
the original realizability by Turing machines~\cite{KleeneSC:intint};
Type Two Effectivity~\cite{Wei00} on function
realizability~\cite{KleeneSC:fouim} and relative function
realizability~\cite{BirkedalL:devttc}, while topological and domain
representations~\cite{Bla97a,Bauer:Birkedal:Scott:98} are based on
realizability over the graph model
$\mathcal{P}\omega$~\cite{ScottD:dattl}. A common feature of these is
that they use models of computation which are suitable for the
theoretical studies of computability, rather than for practical
programming. 

Other approaches are based on programming languages augmented with
datatypes for real numbers~\cite{Escardo:97,marcial-romero04:_seman}
and topological algebras~\cite{TZ98}, or machine models augmented with
(suitably chosen subsets of) real numbers such as Real
RAM~\cite{borodin75}, the Blum-Smale-Shub
model~\cite{blum98:_compl_real_comput}, and the Exact Geometric
Computation model~\cite{yap06:_theor_real_comput_egc}. The motivation
behind these ranges from purely theoretical concerns about
computability and complexity to practical issues in the design of
programming languages and algorithms in computational geometry. RZ
attempts to improve on practicality by interfacing with an actual
real-world programming language, and by providing an input language
which is rich enough to allow descriptions of involved mathematical
structures that go well beyond the real numbers.

Finally, we hope that RZ and, hopefully, its forthcoming applications,
give plenty of evidence for the \emph{practical} value of Constructive
Mathematics~\cite{Bishop:Bridges:85}.



%%% Local Variables: 
%%% mode: latex
%%% TeX-master: "cie"
%%% End: 


%% Bibliography
% AB: It's weird that references are ordered by first appearance, but if
% that's what they want, that's what they get.
\bibliographystyle{splncs}
{
\bibliography{rzbib}
}

\iflong
\appendix
\section{An Interface for the Theory of General Inductive Types}
\label{sec:outp-induct-types}

To give at least one complete example, we include here an
unabridged output for the theory of inductive types shown in
Figure~\ref{fig:wtype}.

\sourcefilex{wtype.mli}

%%% Local Variables: 
%%% mode: latex
%%% TeX-master: "paper.tex"
%%% End: 

\fi % iflong

%%% Local Variables: 
%%% mode: latex
%%% TeX-master: "cie-long.tex"
%%% End: 
