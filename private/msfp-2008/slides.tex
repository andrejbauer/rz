 \documentclass[t]{beamer}

\usepackage{palatino}
\usepackage{amsfonts}
\usepackage{fancyvrb}
%\usepackage[all]{xypic}

% To show source
\DefineVerbatimEnvironment{source}{Verbatim}%
{commandchars=\\\{\}}

\DefineVerbatimEnvironment{sourcex}{Verbatim}%
{fontsize=\footnotesize,commandchars=\\\{\}}


\newcommand{\NN}{\mathbb{N}}
\newcommand{\QQ}{\mathbb{Q}}
\newcommand{\CC}{\mathbb{C}}
\newcommand{\rz}{\Vdash}
\newcommand{\nat}{\mathtt{nat}}

\title{Mathematically Structured but\\
not Necessarily Functional Programming}
\author{Andrej Bauer}
\institute{Department of Mathematics and Physics\\
  University of Ljubljana, Slovenia}
\date{Mathematically Structured Functional Programming\\Reykjavik, July 2008}

%\beamerdefaultoverlayspecification{<+->}
\mode<presentation>
% \usetheme{Goettingen}
\usecolortheme{rose}
\usefonttheme{serif}
\setbeamertemplate{navigation symbols}{}

\begin{document}

\begin{frame}
  \titlepage
\end{frame}

\begin{frame}
  \frametitle{Ways of Mathematically Structured Programming}

  \begin{itemize}
  \item Use math to develop new programming constructs (monads).
  \item Use math to reason and construct programs (Coq).
  \item Programming by proving theorems (propositions as types).
  \item Proving theorems by programing (types as propositions).
  \end{itemize}
\end{frame}

\begin{frame}
  \frametitle{Outline}

  \begin{itemize}
  \item Programming = Proving (propositions as types)
  \item Programming = Proving (realizability)
  \item RZ -- specifications via realizability
  \item Examples of non-functional realizers in constructive mathematics
  \end{itemize}
  
\end{frame}


\begin{frame}
  \frametitle{Programming by proving}

  \begin{itemize}
  \item The Curry-Howard correspondence:
    %
    \begin{gather*}
      \mathrm{Type} = \mathrm{Prop} = \mathrm{Set} \\
      \mathrm{program} = \mathrm{proof} = \mathrm{element}
    \end{gather*}
  \item Programming by proving theorems:
    %
    \begin{center}
      \emph{``Constructive proofs of \\
        mathematically meaningful theorems \\
        give useful programs.''}
  \end{center}
  \end{itemize}
\end{frame}

\begin{frame}
  \frametitle{Example: Fundamental Theorem of Algebra}

  \begin{itemize}
  \item ``Every non-constant polynomial has a complex root.''
  \item First-order logic:
    \begin{equation*}
      \forall p \in \QQ[x] .\,
      0 < \mathrm{deg}(p) \implies \exists z \in \CC .\, p(z) = 0.
    \end{equation*}
  \item Type theory:
    \begin{equation*}
      \textstyle
      \prod_{p : \mathtt{poly}} \mathtt{less}(0,
      \mathtt{deg}(p)) \to
      \sum_{z : \mathtt{complex}} \mathtt{eq}(p(z),0).
    \end{equation*}
  \item Must also define $\mathtt{poly}$, $\mathtt{less}$, $\mathtt{complex}$, and $\mathtt{eq}$.
  \item Can we get rid of $\mathtt{less}$ and $\mathtt{eq}$?
  \item Can we get rid of dependent types and have just
    %
    \begin{equation*}
      \mathtt{poly} \to \mathtt{complex} \; ?
    \end{equation*}
  \end{itemize}

\end{frame}

\begin{frame}
  \frametitle{Programming by proving a la Coq}

  \begin{itemize}
  \item Distinguish between computational and non-computational types:
    %
    \begin{align*}
      \mathtt{Set} &: \text{the sort of computational types} \\
      \mathtt{Prop} &: \text{the sort of non-computational types}
    \end{align*}
  \item We also need \emph{setoids}, which are (computational) types
    with (non-computational) equivalence relations.
  \item In the previous example:
    %
    \begin{itemize}
    \item Non-computational: $\mathtt{less}$, $\mathtt{eq}$.
    \item Setoids: $\mathtt{poly}$, $\mathtt{complex}$.
    \end{itemize}
  \item Coq's extraction mechanism gives an Ocaml or Haskell program of type
    $\mathtt{poly} \to \mathtt{complex}$.
  \end{itemize}
\end{frame}


\begin{frame}
  \frametitle{Does it actually work?}

  \begin{itemize}
  \item Programmers want to write programs, not proofs.
  \item And often it really is easier to just write a program.
  \item The most efficient proof may not correspond to the most
    efficient program.
  \item When we use complex tactics, we may lose control of what the
    extracted program does.
  \item Proofs give purely functional code. What if we want to use
    computational effects (store, exceptions, non-termination)?
  \end{itemize}

\end{frame}

\begin{frame}
  \frametitle{What really happens}

  \begin{itemize}
  \item Write programs directly, not as proofs.
  \item Then prove that the programs are correct.
  \item Coq's \textsc{program} extension does this.
  \item By adapting the type theory and the extraction mechanism, we
    can even handle non-functional programs.
  \end{itemize}

  \pause

  The connection to constructive math is almost lost.

\end{frame}

\begin{frame}
  \frametitle{Programming by proving (a la realizability)}

  \begin{itemize}
  \item Pick a reasonable programming language.
  \item $\mathrm{Proofs} \subsetneq \mathrm{Programs}$.
  \item Programs realize propositions.
  \item To each proposition $\phi$ we assign a (simple) type of
    realizers~$|\phi|$.
  \item We we define a \emph{realizability predicate} on values of $|\phi|$:
    %
    \begin{equation*}
      p \rz \phi \qquad \text{``$p$ realizers $\phi$.''}
    \end{equation*}
    %
    This is necessary because not every value in $|\phi|$ is a valid
    realizer.
  \end{itemize}
\end{frame}

\begin{frame}
  \frametitle{Types of realizer}

  \begin{align*}
    |\top| &= \mathtt{unit} \\
    |\bot| &= \mathtt{unit} \\
    |e_1 =_A e_2| &= \mathtt{unit} \\
    |\phi_1 \land \phi_2| &= |\phi_1| \times |\phi_2| \\
    |\phi_1 \lor \phi_2| &= |\phi_1| + |\phi_2| \\
    |\phi_1 \implies \phi_2| &= |\phi_1| \to |\phi_2| \\
    |\forall x \in A .\, \phi| &= |A| \to |\phi| \\
    |\exists x \in A .\, \phi| &= |A| \times |\phi|
  \end{align*}

  Propositions built only from $\top$, $\bot$, $=$, $\land$, $\to$
  have trivial realizers.

\end{frame}

\begin{frame}
  \frametitle{Realizability predicate}

  \begin{xalignat*}{3}
    () &\rz \top & & & & \\
    () &\rz e_1 =_A e_2 &\text{iff}& &
      &\text{$t_1 \simeq_A t_2$} \\
    (p_1,p_2) &\rz \phi_1 \land \phi_2 &\text{iff}& &
      &\text{$p_1 \rz \phi_1$ and $p_2 \rz \phi_2$} \\
    \mathtt{inl}(p) &\rz \phi_1 \lor \phi_2 &\text{iff}& &
      &\text{$p \rz \phi_1$} \\
    \mathtt{inr}(p) &\rz \phi_1 \lor \phi_2 &\text{iff}& &
      &\text{$p \rz \phi_2$} \\
    p &\rz \phi_1 \implies \phi_2 &\text{iff}& &
      &\text{if $q \rz \phi_1$ then $p\,q {\downarrow}$ and $p\,q \rz \phi_2$} \\
    (p,q) &\rz \exists x \in A .\, \phi(x) &\text{iff}& &
      &\text{for some $u$, $q \rz_A u$ and $p \rz \phi(u)$} \\
    p &\rz \forall x \in A .\, \phi(x) &\text{iff}& &
      &\text{if $q \rz_A u$ then $p\,q {\downarrow}$ and $p\,q \rz \phi(u)$}
  \end{xalignat*}
\end{frame}

\begin{frame}
  \frametitle{Setoids in realizability}

  \begin{itemize}
  \item In realizability setoids are types equipped with
    \emph{partial} equivalence relations (symmetric, transitive).
  \item This is necessary because not every value realizes an element.
  \item Even when the programming language is simply typed, we can
    interpret dependent setoid types.
  \end{itemize}

\end{frame}

\begin{frame}
  \frametitle{RZ --- specifications via realizability}

  \begin{itemize}
  \item A tool written by Chris Stone and me.
  \item It uses realizability to translate mathematical theories to
    program specifications.
  \item Input: mathematical theories
    %
    \begin{itemize}
    \item first-order logic
    \item rich set constructions, including dependent types
    \item support for parameterized theories, e.g., the theory of a
      vector space parameterized by a field.
    \end{itemize}
  \item Output: program specifications
    %
    \begin{itemize}
    \item Ocaml signatures
    \item Assertions about programs
    \end{itemize}
  \item Automatically eliminates non-computational realizers.
  \end{itemize}
\end{frame}

\begin{frame}
  \frametitle{Test case: Era}

  \begin{itemize}
  \item A package for exact real numbers.
  \item Written by Iztok Kavkler and me.
  \item What we did:
    \begin{itemize}
    \item wrote down theories of $\omega$-cpos,
      the interval domain and real numbers,
    \item translated them to specifications with RZ,
    \item implemented the specification efficiently.
    \end{itemize}
  \item Conclusion: it works, but we have no tool to prove that our
    programs satisfy the assertions.
  \item Plan: extend RZ so that it translates to Coq using the
    \textsc{program} extension.
  \end{itemize}
\end{frame}

\begin{frame}
  \frametitle{Non-functional realizers}

  \begin{itemize}
  \item There are constructive reasoning principles which cannot be
    proved in pure intuitionistic logic.
  \item They cannot be realized in pure type theory or pure Haskell.
  \item They are realized by non-functionals programs.
  \item Such principles express the mathematical meaning of
    non-functional programs.
  \end{itemize}

\end{frame}

\begin{frame}[fragile]
  \frametitle{Markov Principle}

  \begin{itemize}
  \item ``A sequence of $0$'s and $1$'s whose terms are not all $0$
    contains a $1$.''
  \item ``A program which does not run forever terminates.''
  \item Provable in classical logic.
  \item Cannot be proved in intuitionistic logic.
  \item $\forall a : \{0,1\}^\NN .\,
    (\lnot \forall n : \NN .\, a(n) = 0) \implies
    \exists n : \NN .\, a(n) = 1.$
  \item RZ tells us that the realizer has type
    %
    \begin{equation*}
      (\mathtt{nat} \to \mathtt{bool}) \to \mathtt{nat}.
    \end{equation*}
  \item Realized by unbounded search:
      %
    \begin{source}
let mp a =
  let n = ref 0 in
    while not (a !n) do n := !n + 1 done ;
    !n      
    \end{source}
  \end{itemize}
\end{frame}

\begin{frame}
  \frametitle{Brouwer's Continuity Principle}

  \begin{itemize}
  \item ``Every map is continuous.''
  \item ``Every map $f : \NN^\NN \to \NN$ is continuous.''
  \item 
    In other words, $f(a)$ depends only on a finite prefix of
    $a(0), a(1), a(2), \ldots$.
  \item Incompatible with classical logic.
  \item Cannot be proved in intuitionistic logic.
  \item As a formula:
    % 
    \begin{multline*}
      \forall f \in \NN^{\NN^\NN} .\,
      \forall a \in  \NN^\NN.\,
      \exists n \in \NN .\,
      \forall b \in \NN^\NN .\,\\
      ((\forall k \leq n .\, a(k) = b(k)) \implies f(a) = f(b)).
    \end{multline*}
  \item Realizers of type
    %
    \begin{equation*}
      ((\nat \to \nat) \to \nat) \to (\nat \to \nat) \to \nat
    \end{equation*}
  \end{itemize}
\end{frame}

\begin{frame}[fragile]
  \frametitle{Continuity principle with store}

  \begin{itemize}
  \item How can we discover how many terms of $a(0)$, $a(1)$, \dots
    are used by $\mathtt{f}$?
  \item Feed $\mathtt{f}$ a sequence which is just like $a$, except
    that it also stores the largest argument at which $\mathtt{f}$
    evaluated it.
  \item The code:
\begin{source}
let cont f a =
  let k = ref 0 in
  let b n = (k := max !k n; a n) in
    f b ; !k
\end{source}
  \end{itemize}

\end{frame}

\begin{frame}[fragile]
  \frametitle{Continuity principle with exceptions}


  \begin{itemize}
  \item Similar idea: throw an exception if $\mathtt{f}$ looks past a
    threshold, and keep increasing the threshold until no exception is
    raise.
  \item The code
\begin{source}
exception Abort
let cont f a =
  let rec search k =
    try
      let b n = 
        if n < k then a n else raise Abort
      in
        f b ; k
    with Abort -> search (k+1)
  in
    search 0
\end{source}
  \end{itemize}
\end{frame}

\begin{frame}[fragile]
  \frametitle{Can we prove these realizers work?}

  \begin{itemize}
  \item Store: presumably yes, using separation logic.
  \item But with global store it does \emph{not} work:

\begin{source}
let k = ref 0
let cont f a =
  let b n = (k := max !k n; a n) in
    f b ; !k
\end{source}

\item 
This version is foiled by

\begin{source}
let f a =
  let m = a 42 in k := 0 ; m
\end{source}
\item Note: Haskell's \texttt{State} monad is global store.
\end{itemize}
\end{frame}

\begin{frame}[fragile]
  \frametitle{Realizer with exceptions does not work!}

  \begin{itemize}
  \item 
    The realizer using exceptions does \emph{not} work.
  \item Foiled by

\begin{source}
let f a =
  try a 42 with Abort -> 23
\end{source}

\item 
Even if \texttt{Abort} is declared locally, we can still catch all
exceptions in ML:
  
\begin{source}
let f a =
  try a 42 with _ -> 23
\end{source}
\item Haskell also has global exceptions.

  \end{itemize}

\end{frame}

\begin{frame}
  \frametitle{Conclusion}

  \begin{itemize}
  \item Realizability is a useful alternative to propositions as
    types.
  \item We \emph{can} keep the connection between constructive math
    and programming tight, without sacrificing either mathematical
    elegance or efficiency of programs.
  \item Constructive reasoning principles are a mathematical
    abstraction of non-functional programming features.
  \item We need to study non-functional features more carefully.
  \end{itemize}

\end{frame}


\end{document}
