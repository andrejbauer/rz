\documentclass[t]{beamer}

\usepackage{palatino}
\usepackage{amsfonts}
\usepackage{fancyvrb}
%\usepackage[all]{xypic}

% To show source
\DefineVerbatimEnvironment{source}{Verbatim}%
{commandchars=\\\{\}}

\DefineVerbatimEnvironment{sourcex}{Verbatim}%
{fontsize=\footnotesize,commandchars=\\\{\}}


\newcommand{\NN}{\mathbb{N}}
\newcommand{\rz}{\Vdash}

\title{Mathematically Structured but\\
not Necessarily Functional Programming}
\author{Andrej Bauer}
\institute{Department of Mathematics and Physics\\
  University of Ljubljana, Slovenia}
\date{Mathematically Structured Functional Programming\\Reykjavik, July 2008}

%\beamerdefaultoverlayspecification{<+->}
\mode<presentation>
% \usetheme{Goettingen}
\usecolortheme{rose}
\usefonttheme{serif}
\setbeamertemplate{navigation symbols}{}

\begin{document}

\begin{frame}
  \titlepage
\end{frame}

\begin{frame}
  \frametitle{One-page summary of the talk}

  \begin{itemize}
  \item Program development and extract
  \item RZ
  \item Impure realizers
  \end{itemize}
\end{frame}

\begin{frame}
  \frametitle{A successful framework for program development}

  \begin{itemize}
  \item Mathematical foundation: type theory
  \item Interactive proof assistant such as Coq
  \item Extraction of programs from formal proofs, or
  \item Programs written directly and proved correct
  \end{itemize}
\end{frame}

\begin{frame}
  \frametitle{How it is done in Coq}

  \begin{itemize}
  \item Propositions as types
  \item Computational part -- the sort \texttt{Set}
  \item Non-computational part -- the sort \texttt{Prop}
  \item Furthermore, use \emph{setoids} instead of types
  \end{itemize}

\end{frame}

\begin{frame}
  \frametitle{What about impure programs?}

  YNot, Nanevski.
\end{frame}


\begin{frame}
  \frametitle{Realizability as an alternative}

  \begin{itemize}
  \item We embrace the fact that $\text{proofs} \subsetneq \text{programs}$.
  \item A proposition is:
    %
    \begin{itemize}
    \item \emph{provable} if witnessed by a proof
    \item \emph{realizable} if witnessed by a program
    \end{itemize}
  \item $\text{Provable} \implies \text{realizable}$
  \end{itemize}

\end{frame}

\begin{frame}[fragile]
  \frametitle{Not all programs are proofs}

  \begin{itemize}
  \item Consider
    %
    \begin{equation*}
      \forall m : \NN .\, \exists n : \NN .\, m < n
    \end{equation*}
    %
  \item The proof ``take $n$ to be $m+1$'' yields the realizer
    %
\begin{source}
  let r = fun m -> m + 1      
\end{source}
    %
  \item The following realizer does not correspond to a proof:
    %
\begin{source}
  let s =
    let k = ref 0 in
      fun m -> (k := !k + 1; m + k)
\end{source}
    %
    \end{itemize}
\end{frame}

\begin{frame}
  \frametitle{Sets and setoids}

  \begin{itemize}
  \item Propositions as types:
    %
    \begin{itemize}
    \item types
    \item setoids
    \end{itemize}
  \item Realizability:
    %
    \begin{itemize}
    \item sets
    \end{itemize}
  \end{itemize}

\end{frame}

\begin{frame}
  \frametitle{Pers}

  Comparison with setoids.

\end{frame}

\begin{frame}
  \frametitle{Propositions as types}

  \begin{itemize}
  \item A proposition $\phi$ is interpreted as a type $|\phi|$ in
    \emph{dependent} type theory:
    %
    \begin{align*}
      |\top| &= \mathtt{unit} \\
      |\phi \land \psi| &= |\phi| \times |\psi| \\
      |\phi \lor \psi| &= |\phi| + |\psi| \\
      |\forall x : A .\, \phi| &= \textstyle\prod_{x : |A|} |\phi| \\
      |\exists x : A .\, \phi| &= \textstyle\sum_{x : |A|} |\phi|
    \end{align*}
  \item All terms of type $|\phi|$ realize $\phi$:
    %
    \begin{equation*}
      \text{``$r$ realizes $\phi$''} \iff r : |\phi|
    \end{equation*}
  \end{itemize}
\end{frame}

\begin{frame}
  \frametitle{Realizability}

  \begin{itemize}
  \item A proposition $\phi$ is interpreted as a type $|\phi|$ in a
    programming language:
    %
    \begin{align*}
      |\top| &= \mathtt{unit} \\
      |\phi \land \psi| &= |\phi| \times |\psi| \\
      |\phi \lor \psi| &= |\phi| + |\psi| \\
      |\forall x : A .\, \phi| &= |A| \to |\phi| \\
      |\exists x : A .\, \phi| &= |A| \times |\phi|
    \end{align*}
  \item \emph{Not} all terms of $|\phi|$ realize $\phi$:
    %
    \begin{equation*}
      \text{``$r$ realizes $\phi$''} \iff \text{$r : |\phi|$ and $r \Vdash \phi$}
    \end{equation*}
  \end{itemize}
\end{frame}

\begin{frame}
  \frametitle{RZ}

  What it is.
\end{frame}

\begin{frame}
  \frametitle{Example use of RZ}

  Era.
\end{frame}

\begin{frame}
  \frametitle{Translating into Coq}

  Automatic separation of computational and non-computational part.
\end{frame}

\begin{frame}
  \frametitle{Non-functional realizers}

  Realizers which use effects.

  Usually they imply extra reasoning principles.
\end{frame}

\begin{frame}
  \frametitle{Unbounded search and Markov Principle}

  \begin{itemize}
  \item Markov principle: a sequence of $0$'s and $1$'s whose terms
    are not all $0$ contains a $1$.
  \item Provable in classical logic.
  \item Not provable in intuitionistc logic.
  \item As a formula:
    %
    \begin{equation*}
      \forall a : \{0,1\}^\NN .\,
      (\lnot \forall n : \NN .\, a(n) = 0) \implies
      \exists n : \NN .\, a(n) = 1.
    \end{equation*}
  \end{itemize}
\end{frame}

\begin{frame}
  \frametitle{Compactness of Cantor space}

  Escardo's programs.

  Decompile prevents it.
\end{frame}

\begin{frame}
  \frametitle{Continuity principle}

  \begin{itemize}
  \item 
    Every map $f : \NN^\NN \to \NN$ is continuous at every point $a \in
    \NN^\NN$.
  \item 
    In other words, $f(a)$ depends only on a finite prefix of
    $a(0), a(1), a(2), \ldots$.
  \item Incompatible with classical logic.
  \item Cannot be proved in intuitionistic logic.
  \item As a formula:
    % 
    \begin{multline*}
      \forall f \in \NN^{\NN^\NN} .\,
      \forall a \in  \NN^\NN.\,
      \exists n \in \NN .\,
      \forall b \in \NN^\NN .\,\\
      ((\forall k \leq n .\, a(k) = b(k)) \implies f(a) = f(b)).
    \end{multline*}
  \end{itemize}

\end{frame}

\begin{frame}
  \frametitle{With effects}

  Store.

  Exceptions.

  Can we do it in Haskell?
\end{frame}

\begin{frame}
  \frametitle{When is a realizer ``good''?}

  Longley's notion of functional realizer.

  Why it is not suitable.

  Can we do better?
\end{frame}


\end{document}
