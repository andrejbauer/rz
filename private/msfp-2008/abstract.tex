\documentclass{entcs} \usepackage{prentcsmacro}
\usepackage{amsmath}
\sloppy
% The following is enclosed to allow easy detection of differences in
% ascii coding.
% Upper-case    A B C D E F G H I J K L M N O P Q R S T U V W X Y Z
% Lower-case    a b c d e f g h i j k l m n o p q r s t u v w x y z
% Digits        0 1 2 3 4 5 6 7 8 9
% Exclamation   !           Double quote "          Hash (number) #
% Dollar        $           Percent      %          Ampersand     &
% Acute accent  '           Left paren   (          Right paren   )
% Asterisk      *           Plus         +          Comma         ,
% Minus         -           Point        .          Solidus       /
% Colon         :           Semicolon    ;          Less than     <
% Equals        =3D           Greater than >          Question mark ?
% At            @           Left bracket [          Backslash     \
% Right bracket ]           Circumflex   ^          Underscore    _
% Grave accent  `           Left brace   {          Vertical bar  |
% Right brace   }           Tilde        ~

% A couple of exemplary definitions:

\newcommand{\NN}{{\mathbb N}}
\newcommand{\RR}{{\mathbb R}}
\newcommand{\nat}{\mathtt{nat}}

\def\lastname{Bauer}
\begin{document}
\begin{frontmatter}
  \title{Mathematically Structured but\\
    not Necessarily Functional Programming\\(Extended Abstract)}
  \author{Andrej Bauer\thanksref{myemail}}
  \address{Faculty of Mathematics and Physics\\University of Ljubljana\\
    Ljubljana, Slovenia} \thanks[myemail]{Email:
    \href{mailto:Andrej.Bauer@andrej.com} {\texttt{\normalshape
        Andrej.Bauer@andrej.com}}}
\begin{abstract} 
  Realizability is an interpretation of intuitionistic logic which
  subsumes the Curry-Howard interpretation of propositions as types,
  because it allows the realizers to use computational effects such as
  non-termination, store and exceptions. Therefore, we can use
  realizability as a framework for program development and extraction
  which allows any style of programming, not just the purely
  functional one that is supported by the Curry-Howard correspondence.
  In joint work with Christopher A.~Stone we developed RZ, a tool
  which uses realizability to translate specifications written in
  constructive logic into interface code annotated with logical
  assertions. RZ does not extract code from proofs, but allows any
  implementation method, from handwritten code to code extracted from
  proofs by other tools. In our experience, RZ is useful for
  specification of non-trivial theories. While the use of
  computational effects does improve efficiency it also makes it
  difficult to reason about programs and prove their correctness. We
  demonstrate this fact by considering non-purely functional realizers
  for a Brouwerian continuity principle.
\end{abstract}
\begin{keyword}
  Realizability, constructive mathematics, specifications, functional
  programming, computational effects.
\end{keyword}
\end{frontmatter}

Realizability, a notion introduced by Kleene~\cite{kleene}, is a
mathematical formalization of the Brouwer-Heyting-Kolmogorov
interpretation of intuitionistic logic. Under the realizability
interpretation, validity of a proposition is witnessed by a program,
or a \emph{realizer}, which can be thought of as the computational
content of the proposition. For example, a $\forall x .\, \exists y
.\, \phi(x,y)$ statement is valid if there is a (not necessarily
extensional) program computing from $x$ a $y$ and a realizer for
$\phi(x,y)$. Realizability subsumes the Curry-Howard correspondence
which interprets proposition as types and proofs as programs, because
a realizer need not correspond to a proof---it may be non-terminating,
or use computational effects such as store and exceptions.

We can use realizability as a mathematical framework for program
extraction and development, just as the Curry-Howard correspondence is
used for the same purpose in systems such as Coq~\cite{coq} and
Agda~\cite{agda}. Since realizers need not correspond to proofs, they
are generally obtained as a combination of handwritten code and code
extracted from constructive proofs. This allows us to realize
statements which are not provable in intuitionistic logic. Even when a
purely functional realizer could be extracted from a proof, we might
prefer an impure handwritten one because it is more efficient, or
because it is easier to write the code than the proof. In fact, an
important advantage of realizability is the fact that it allows
programmers to implement specifications in any way they see
fit.\footnote{Whether the programmers also go to the trouble of
  proving correctness of handwritten code is another matter.}

In joint work with Chris Stone we developed \emph{RZ}~\cite{RZ}, a
tool which employs the realizability interpretation to translate
specifications in constructive logic into annotated interface code in
Objective Caml~\cite{ocaml}. The system supports a rich input language
allowing descriptions of complex mathematical structures. Currently,
RZ does not extract code from proofs, but allows any implementation
method, from handwritten code to code extracted from proofs by other
tools. Recently, we have also been experimenting with translation into
Coq using Matthieu Sozeau's \textsc{program}
extension~\cite{program}. In a related project with Iztok
Kavkler~\cite{era} we demonstrated that RZ can be used to develop
non-trivial specifications. We wrote an RZ axiomatization of real
numbers and the interval domain. We then implemented by hand the
specification produced by RZ, which gave us a remarkably efficient
library \emph{Era} for computing with exact real numbers. Most of Era
is written in a purely functional style, and we could have easily
extracted the realizers from constructive proofs. However, the ability
to use non-functional features such as caching and in-place updates,
improves performance and eases implementation.

We demonstrate the use of non-purely functional realizers by looking
at the statement
%
\begin{equation*}
  \forall f \in \NN^{\NN^\NN} .\,
  \forall a \in  \NN^\NN.\,
  \exists n \in \NN .\,
  \forall b \in \NN^\NN .\,
  ((\forall k \leq n .\, a(k) = b(k)) \implies f(a) = f(b)).
\end{equation*}
%
This is a continuity principle saying that every function from Baire
space $\NN^\NN$ to $\NN$ is continuous, where $\NN$ is equipped with
the discrete and $\NN^\NN$ with the product topology. It has important
consequences for the theory of metric spaces in Brouwerian
intuitionistic mathematics. We cannot hope to prove the statement in
pure intuitionistic logic because such a proof would also be valid
classically, but the statement is incompatible with classical
logic.\footnote{Consider $f$ defined by $f(a) = 1$ if $\exists n \in
  \NN .\, a(n) = 1$, and $f(a) = 0$ otherwise. An intuitionistic proof
  applied to $f$ would essentially give a Halting Oracle.} By feeding
the statement to RZ we find out that it is realized by a realizer
%
\begin{equation*}
  \mathtt{r} :
  ((\nat \to \nat) \to \nat) \to
  (\nat \to \nat) \to \nat
\end{equation*}
%
that takes as input realizers $\mathtt{f} : (\nat \to \nat) \to \nat$
and $\mathtt{a} : \nat \to \nat$, which realize the corresponding
functions~$f$ and~$a$, and returns a realizer $\mathtt{n} =
\mathtt{r}\, \mathtt{f} \, \mathtt{a}$ of type $\nat$ such that
$\mathtt{f}$ depends on at most the first $\mathtt{n}$ terms of the
sequence $\mathtt{a}(0), \mathtt{a}(1), \mathtt{a}(2), \ldots$ Such a
realizer is known as \emph{modulus of continuity}.  We cannot expect
to construct~$\mathtt{r}$ in a purely functional
language,\footnote{This is so because there are domain-theoretic
  models which interpret purely functional code and invalidate the
  continuity principle.}  although there are a number of ways of
getting it with the aid of computational effects. For example, by
using store we can implement $\mathtt{r}$ as (written in Objective
Caml notation)
%
\begin{verbatim}
let r f a =
  let k = ref 0 in
  let b n = (k := max !k n; a n) in
    f b ; !k
\end{verbatim}
%
Given $\mathtt{f}$ and $\mathtt{a}$, $\mathtt{r}$ evaluates
$\mathtt{f}$ at the argument $\mathtt{b}$ which behaves just
like~$\mathtt{a}$, except that it stores in location $\mathtt{k}$ the
largest argument at which it was evaluated. When $\mathtt{f}$ returns
a value, $\mathtt{r}$ discards it and returns the number stored
in~$\mathtt{k}$.

We could use exceptions instead of store:
%
\begin{verbatim}
exception Abort
let r f a =
  let rec search k =
    try
      let b n = (if n < k then a n else raise Abort) in
        f b ; k
    with Abort -> search (k+1)
  in
    search 0
\end{verbatim}
%
Now the argument $\mathtt{b}$ acts like $\mathtt{a}$ for arguments
below the threshold value $\mathtt{k}$, and raises an exception
otherwise. The program $\mathtt{r}$ keeps increasing $\mathtt{k}$ as
long as exceptions are being raised. Eventually it finds a threshold
which does not raise an exception and returns it. We leave it to the
readers to construct $\mathtt{r}$ which uses continuations.

It is not easy to prove that the realizer~$\mathtt{r}$ really does
what it is supposed to. In fact, the first version, using store, only
works because $\mathtt{k}$ is local and thus inaccessible
to~$\mathtt{f}$. Had we used a globally accessible location
$\mathtt{k}$, the following $\mathtt{f}$ would foil $\mathtt{r}$ by
resetting $\mathtt{k}$ to~$0$:
%
\begin{verbatim}
let f a =
  let m = a 42 in k := 0 ; m
\end{verbatim}
%
It is somewhat unclear whether $\mathtt{f}$ realizes an element of
$\NN^{\NN^\NN}$. We remark that Haskell \texttt{State}
monad~\cite{haskell} implements global store. It is an interesting
question how to get~$\mathtt{r}$ in Haskell using only monads that are
definable in pure Haskell.\footnote{We are asking for a monad
  $\mathtt{T}$ such that $\mathtt{r}$ can be written in the Kleisli
  category for~$\mathtt{T}$, i.e., a function accepting type $\alpha$
  and returning type $\beta$ has type $\alpha \to \mathtt{T}\,\beta$.}

The situation is even worse with the version of~$\mathtt{r}$ that uses
exceptions. It simply does not work because $\mathtt{f}$ could
intercept the exception:
%
\begin{verbatim}
let f a =
  try a 42 with Abort -> 23
\end{verbatim}
%
Note that $\mathtt{f}$ is allowed to do this because it only has to
work properly on arguments which do not raise exceptions. To mend the
situation, we would need a local exception \texttt{Abort} which cannot
be intercepted by~$\mathtt{f}$. While exceptions in Objective Caml can
be declared locally, they can also be intercepted globally. One is
tempted to either ask the designers of the language for truly local
exceptions, or to formulate an additional invariant that~$\mathtt{f}$
should satisfy, such as ``does not intercept exception
$\mathtt{Abort}$''. Neither alternative is appealing.

These considerations reflect the well known wisdom that it not easy to
reason about computational effects. However, we feel that
realizability still provides a convenient mathematical framework for
development and extraction of programs and data structures, especially
since it does not impose a particular programming style.

\bibliographystyle{entcs}
\bibliography{abstract}

\end{document}
