\chapter{Theories and Specifications}
\label{cha:theor-spec}

In this chapter we review the notions of a mathematical theory on one
side, and a program specification on the other.

%----------------------------------------------------------------------
%----------------------------------------------------------------------
\section{Theories}
\label{sec:theories}

Speaking informally, a (mathematical) theory is any mathematical topic
that we might want to study. For example, if we are interested in the
properties of a general group, our theory, call it $\thy{Group}$ would
consist of an unspecified group~$G$, equipped with unit and the usual
operations. All we know about~$G$ is that it satisfies the group
axioms. We might also wish to think about a particular group, say the
symmetric group $S_6$. This is again a theory $\thy{S6}$, but whatever
we prove in it applies only to one specific group. A third possibility
is to study the theory $\thy{Groups}$ of \emph{all} groups, which
would also speak about \emph{homomorphisms} between groups and
constructions of new groups from old ones.

\vspace{2cm}

In RZ the theory $\thy{Group}$ would be written as follows:
%
\sourcefile{group.thy}

%----------------------------------------------------------------------
\subsection{Simple Theories}
\label{sec:simple-theories}

Theories consist of sets, constants, relations and axioms.

Models.

The first-order language of theories.

Future extensions.

%----------------------------------------------------------------------
\subsection{Parametrized Theories}
\label{sec:param-theor}

Simple examples.

Advanced examples.

%----------------------------------------------------------------------
%----------------------------------------------------------------------
\section{Specifications}
\label{sec:specifications}

%----------------------------------------------------------------------
\subsection{ML Structures and Signatures}
\label{sec:ml-signatures}


%----------------------------------------------------------------------
\subsection{Parametrized Signatures (Functors)}
\label{sec:param-sign}


%----------------------------------------------------------------------
\subsection{Assertions}
\label{sec:assertions}




%%% Local Variables: 
%%% mode: latex
%%% TeX-master: "userman"
%%% End: 
