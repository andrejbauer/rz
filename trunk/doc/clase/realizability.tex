
\section{Realizability}
\label{sec:realizability}

We briefly motivate the main idea of (typed) realizability. When we
represent a set of mathematical objects~$S$ in a programming
language~$\PL$ there are two natural steps to take: first choose an
\emph{underlying type~$\ut{S}$} of representing values, and second
specify how the values of type~$\ut{S}$ represent, or \emph{realize},
elements of the set~$S$. For example, consider how we might represent
the set~$D$ of simple finite directed graphs (whose vertices are
labeled by integers). As the underlying datatype we might choose
$\ut{D} = \mathtt{int} \; \mathtt{list} *\, (\mathtt{int} *
\mathtt{int}) \mathtt{list}$, and represent a graph~$G \in D$ as a
pair of lists $(v,e)$ where $v = [x_1; \ldots; x_n]$ is the list of
vertices and $e = [e_1; \ldots; e_m]$ is the list of edges. Formally,
we write
%
\begin{equation*}
  (v, e) \rz_D G
\end{equation*}
%
and read it as ``$(v,e)$ realizes~$G \in D$''. Observe that each graph
is realized by at least one pair of lists, and that no pair of lists
represents more than one graph. (As commonly occurs, most graphs are
represented by many different pairs of lists.) This leads us to the definition given
below. We shall abuse notation slightly and write $t \in \ut{S}$ to
mean that $t$ is a closed expression of type $\ut{S}$.


\begin{figure*}
\[
  \hspace{-0.5truecm}
  \parbox[t]{0.28\textwidth}{
\footnotesize
    \begin{align*}
      \ut{\top} &= \mathtt{unit} \\
      \ut{\bot} &= \mathtt{unit} \\
      \ut{x =_S y} &= \mathtt{unit} \\
      \ut{\phi \land \psi} &= \ut{\phi} * \ut{\psi} \\
      \ut{\phi \implies \psi} &= \ut{\phi} \to \ut{\psi} \\
      \ut{\phi \lor \psi} &= \ut{\phi} + \ut{\psi} \\
      &\\
      \ut{\forall x \in A .\, \phi(x)} &= \ut{A} \to \ut{\phi} \\
      \ut{\exists x \in A .\, \phi(x)} &= \ut{A} \times \ut{\phi}
    \end{align*}
  }
\ \vrule\ 
  \parbox[t]{0.5\textwidth}{
\footnotesize
    \begin{align*}
      () \rz \top &
      \\
      &
      \\
      () \rz x =_S y
        &\quad\text{iff}\quad 
      x \per{S} y
      \\
      (t_1,t_2) \rz \phi \land \psi
        &\quad\text{iff}\quad
        \text{$t_1 \rz \phi$ and $t_2 \rz \psi$}
      \\
      t \rz \phi \implies \psi
        &\quad\text{iff}\quad
        \text{for all $u \in \ut{\phi}$, if $u \rz \phi$ then $t\,u
          \rz \psi$}
      \\
      \inl{t} \rz \phi \lor \psi
        &\quad\text{iff}\quad
        \text{$t \rz \phi$}
      \\
      \inr{t} \rz \phi \lor \psi
        &\quad\text{iff}\quad
        \text{$t \rz \psi$}
      \\
      t \rz \forall x \in A . \phi(x)
        &\quad\text{iff}\quad
        \text{for all $u \in \ut{A}$, if $u \rz_A x$ then $t\,u \rz \phi(x)$}
      \\
      (t_1, t_2) \rz \exists x \in A . \phi(x)
        &\quad\text{iff}\quad
        \text{$t_1 \rz_A x$ and $t_2 \rz \phi(x)$}
    \end{align*}
  }
\]
  \caption{Realizability interpretation of logic (outline)}
  \label{fig:rz-logic}
\end{figure*}

\begin{definition}
  A \emph{modest set}\footnote{Modest sets were so named by Dana
    Scott. They are ``modest'' because their size cannot exceed the
    number of expressions of the underlying datatype.} is a triple
  $(S, \ut{S}, {\rz_S})$ where $S$ is a set, $\ut{S}$ is a type and
  $\rz_S$ is a relation between expressions of type~$\ut{S}$ and
  elements of~$S$, satisfying:
  % 
  \begin{enumerate}
  \item For every $x \in S$ there is $t \in \ut{S}$ such that $t \rz_S
    x$.
  \item If $t \rz_S x$ and $t \rz_S y$ then $x = y$.
  \end{enumerate}
  %
  A \emph{realized function} $f : (S, \ut{S}, {\rz_S}) \to (T, \ut{T},
  {\rz_T})$ between modest sets is a function $f : S \to T$ for which
  there exists $u \in \ut{S} \to \ut{T}$ such that
  %
  \begin{equation*}
    t \rz_S x \implies u\,t \rz_T f(x) \;.
  \end{equation*}
  %
  We say that $u$ \emph{realizes}~$f$.
\end{definition}

The realizer~$u$ of a realized function~$f$ is more commonly known as
an ``implementation of~$f$'' or an ``algorithm for computing~$f$''.

Modest sets and realized functions form a category of \emph{modest
  sets~$\Mod{\PL}$}. In realizability theory this is a well known
category with good properties. It is regular and locally bi-cartesian
closed, which allows us to interpret first-order logic and a rich type
theory. Here we only outline the main ideas behind the realizability
interpretation of logic. See e.g.~\cite{Bauer:00} for details.

In the realizability interpretation of logic, each formula~$\phi$ is
assigned a set of \emph{realizers} which can be thought of as
computations that witness the validity of~$\phi$. The situation is
somewhat similar (but not equivalent) to the propositions-as-types
translation of logic into type theory, where the proofs of a
proposition correspond to terms of the corresponding type. More
precisely, to each formula~$\phi$ we assign an underlying type
$\ut{\phi}$ of realizers. However, unlike in the propositions-as-types
translation, not all terms of type $\ut{\phi}$ are necessarily valid
realizers for~$\phi$. We write $t \rz \phi$ when $t \in \ut{\phi}$ is
a realizer for~$\phi$. The underlying types and the
realizability relation~$\rz$ are defined inductively on the structure
of~$\phi$; an outline is shown in Figure~\ref{fig:rz-logic}. We say that a
formula~$\phi$ is \emph{valid} in~$\Mod{\PL}$ if it has at least one
realizer.

We shall not dwell any further on the technicalities involving the
category of modest sets, but rather proceed to a concrete description
of our realizability translation. There is one technical point,
though, which we first take care of. A modest set is a triple $(S,
\ut{S}, {\rz_S})$ in which~$S$ is an arbitrary set. For an automated
system it would be convenient if it did not have to refer to arbitrary
sets but rather just to ingredients that are already present in the
programming language, such as types and sets of expressions. Up to
equivalence of categories, modest sets can be constructed as triples
$(\ut{S}, \tot{S}, {\per{S}})$ where $\ut{S}$ is a type, $\tot{S}$ is
a subset of expressions of type~$\ut{S}$, called the \emph{total
  values},\footnote{We do \emph{not} require that a total value must
  be a terminating expression.} and $\per{S}$ is an equivalence
relation on~$\tot{S}$. The relationship between this representation of
a modest set and the original one is as follows:
%
\begin{itemize}
\item $\tot{S}$ is the set of those $t \in \ut{S}$ that
  realize something, i.e., there is $x \in S$ such that $t \rz_S x$.
  These correspond to implementations that satisfy
  the representation invariant, e.g., graphs where the list of edges
  mentions only integers in the list of nodes, a subset of
  all values of type $\mathtt{int} \; \mathtt{list} * (\mathtt{int} *
\mathtt{int}) \; \mathtt{list}$.
\item $t \per{S} u$ if $t$ and $u$ realize the same element, i.e.,
  there is $x \in S$ such that $t \rz_S x$ and $u \rz_S x$.
  This relation equates alternate concrete representations of the same
  abstract value, e.g., equating two concrete graph representations differing
  only in the order of the nodes or the order of the edges.
\end{itemize}
%
The alternative view of a modest set $(\ut{S}, \tot{S}, {\per{S}})$
only refers to objects and concepts from the programming language. It
is better suited for our purposes.

Note that the equivalence relation on~$\tot{S}$ is also a
\emph{partial} equivalence relation on~$\ut{S}$, which shows that
modest sets are in fact equivalent to PER models.


%%% Local Variables: 
%%% mode: latex
%%% TeX-master: "case"
%%% End: 
