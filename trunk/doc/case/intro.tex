\section{Introduction}
\label{sec:introduction}

We present a system, called~\emph{RZ}, for automatic generation of
program specifications from mathematical theories. We translate
mathematical theories to specifications by computing their
realizability interpretations in an ML-like language augmented with
assertions. While the system is best suited for descriptions of those
programming tasks that can be easily described in mathematical
language (e.g., algorithms on finitely presented groups, exact real
arithmetic, algorithms on graphs, etc.), it also elucidates the
relationship between data structures and constructive mathematics.

As the realizability interpretation validates the laws of
\emph{intuitionistic} logic, our input theories are intuitionistic or
constructive. Thus, RZ is a tool that extracts the computational
meaning of a constructive theory and expresses it as a programming
specification.

We emphasize that RZ does \emph{not} extract programs from proofs---in
fact, there is no way to write a proof in our system. We work ``one
level above'', so to speak, and just determine what the programs are
supposed to do, i.e., we provide specifications for them. We leave it
to the programmer, or to another tool, to construct such programs as
he or she best knows. This leaves the programmer completely free to
write programs that are efficient and may not correspond naturally to
any constructive proofs.

We hope for RZ 


We assume throughout that we have chosen a fixed programming
language~$\PL$. Any ML-like language will
do~\cite{milner+:definition}. Our implementation of RZ uses Objective
Caml~\cite{ocaml} but could easily be adopted to other variants of ML.
The essential features we require of~$\PL$ are product, function, and
sum types, as well as support for module interfaces.

Other points:

Allow programmer to write the most efficient program. Contrast with
extraction of programs from proofs, where presently one cannot expect
to get useful programs (also, who writes really long formal proofs?)

Bring constructive reasoning closer to programmers.

Overview.

%%% Local Variables: 
%%% mode: latex
%%% TeX-master: "case"
%%% End: 
