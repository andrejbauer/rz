\section{Examples}
\label{sec:examples}

In this section we show typical examples of RZ at work.

\paragraph{Decidable set.}
\label{sec:decidable-set}

As the first example, we formulate the theory of a decidable set.
Recall that in constructive mathematics a set~$S$ is said to be
\emph{decidable} if $x = y$ or $x \neq y$ for all $x, y \in S$.
The input to RZ is
%
{\small
\VerbatimInput{decidable.thy}
}
%
and the output is
%
{\small
\VerbatimInput{decidable.mli}
}
%
The output signature asks for a function \Verb|decidable| which
accepts two realizers~$x$ and~$y$ and returns one of two tokens
\Verb[`or0] and \Verb[`or1], depending on whether~$x$ and~$y$ realize
the same element. This is nothing but a computable decision procedure
for equality on \Verb|s|, with respect to \Verb|=s=| of course, as
should be expected.

We remark that nothing requires the partial equivalence relation
\Verb|=s=| to be computable, so not every modest set is decidable. In
fact, there are many natural examples of non-computable partial
equivalence relations, such as \emph{extensional equality} of function
from natural numbers to natural numbers. (If we could computably
decide whether two functions always give equal results on equal
arguments, we could write a Halting Oracle.)

\paragraph{The Free Join-Semilattice}
\label{sec:free-join-semil}



\paragraph{Axiom of Choice}
\label{sec:axiom-choice}

Axiom of choice is not realizable. This example is needed to point out
that we are not doing propositions-as-types. But intensional axiom of
choice is realized.

Also demonstrates how to express an axiom schema.

mention ZF.



%%% Local Variables: 
%%% mode: latex
%%% TeX-master: "case"
%%% End: 
