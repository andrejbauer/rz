
\section{Implementation}
\label{sec:implementation}

After parsing, our implementation does type checking.  The type
checker does rather limited type reconstruction; the types of all
bound variables must either be given at the binding site, given
through an \Verb|implicit| declaration, or be obvious from local
context.  A very small amount of implicit subtyping is done (primarily
between sums).  

The system will also try to insert explicit coercions into or out of
subset types where they were omitted by the theory author.  This can
significantly improve readability.  \comment{give example!}

After typechecking the theories are represented in abstract syntax and
translated into ML signatures as described below.  

Propositions without constructive content have trivial realizers, and
so a final ``unit elimination'' pass both removes these and does 
peephole simplification of the resulting signature.   Without the optimizer,
the axioms of the theory \Verb|SQGROUP| would produce
\begin{Verbatim}
val unit : s -> top * top
(**  Assertion unit (x:s) =  
      x * e =s= x and e * x =s= x
*)
   
val assoc : s -> s -> s -> top
(**  Assertion assoc (x:s,y:s,z:s) =
      x * (y * z) =s= (x * y) * z
*)
   
val sqrt : s -> s * top
(**  Assertion sqrt (x:s) =  
      pi0 (sqrt x) : ||s|| and 
      pi0 (sqrt x) * pi0 (sqrt x) =s= x
*)
\end{Verbatim}
requiring the presence of trivial realizers for the first two axioms,
and a more complicated (but not more useful) realizer for the final
axiom.


Here we provide more details about the translation.


The translation of axioms parameterized by models into polymorphism
(instead of functors) is essentially the phase-splitting translation
of Harper, Mitchell, and Moggi \cite{harper+:popl90}.

%%% Local Variables: 
%%% mode: latex
%%% TeX-master: "case"
%%% End: 
