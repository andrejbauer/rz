
\section{Implementation}
\label{sec:implementation}

\subsection{Pre-translation}

After parsing, our implementation does type checking.  The type
checker does simple type reconstruction.  Instead of doing full
unification, we require that the types of all bound variables must
either be given at the binding site, given through an \Verb|implicit|
declaration, or be obvious from their definition.  

Unlike ML datatypes, we do not require disjoint union type
to be declared before they are used, or to have different unions
involve different tags.  Therefore, a very small amount of implicit 
subtyping is done between sums.  Otherwise, if we had a
specifiation of queues of integers that included
\begin{Verbatim}
  set queue
  const emptyQueue
  const enqueue : int*iqueue -> iqueue
  const dequeue : 
     iqueue -> `None + `Some:int*iqueue
\end{Verbatim}
then the axiom
\begin{Verbatim}
  dequeue emptyQueue = `None
\end{Verbatim}
would fail to typecheck (the most-precise set for the left-hand-side
is a two-element disjoint unition, while the most-precise set
for the right is a one-element disjoint union.)

The type checker will also try to convert between $\Set$ and $\LBRACE
x \ \COLON \Setexp\ \BAR \Proposition \RBRACE$ as necessary in order
to type check.  Thus, one can write
\begin{Verbatim}
  set real
  set nz_real = {x:real | not (x=zero)}
  const one : real
  const reciprocal: nz_real -> nz_real
  const ( * ) : real * real -> real

  axiom field (x : real) =
    not (x=zero) => x * (inv x) = one
\end{Verbatim}
instead of needing to write
\begin{Verbatim}
  axiom field (x : real) =
    not (x=zero) => 
      x * ((inv (x :> r)) :< r') = one
\end{Verbatim}
In this case, since \Verb|:>| and \Verb|:<| has computational content
(going into a subset involves pairing the item with the realizer of
the proposition; going out of a subset is then a first projection),
the typechecker rewrites the former version of \Verb|field|
into the latter before passing it on to the translation phase.

\subsection{Realizability Translation}
Here we provide more details about the translation.

\subsection{Optimization}
Propositions without constructive content have trivial realizers, and
so a final ``unit elimination'' pass both removes these and does 
peephole simplification of the resulting signature.   Without an optimizer,
the axioms of the theory \Verb|SQGROUP| would produce
\begin{Verbatim}
val unit : s -> top * top
(** Assertion unit (x:s) =  
     x * e =s= x and e * x =s= x
*)

val sqrt : s -> s * top
(** Assertion sqrt (x:s) =  
     pi0(sqrt x) : ||s|| and 
     pi0(sqrt x) * pi0(sqrt x) =s= x
*)
\end{Verbatim}
where \Verb|top| is the type of trivial realizers; we use 
\Verb|top| instead of \Verb|unit| to emphasize that these trivial realizers are terminating and hence safe to eliminate; this would not necessarily
be true for terms of type \Verb|unit|.  The optimizer can
easily tell from the types that the realizers for the \Verb|unit| and
\Verb|assoc| axioms are trivial and can be discarded, and that
although \Verb|sqrt| cannot be discarded entirely, part of its return
value is unnecessary.  Assertions that reference discarded or optimized
constants are automatically rewritten to preserve well-typedness, and we obtain the translation of \Verb|SQGROUP| shown previously, which contains no
occurrences of \Verb|top|.


%%% Local Variables: 
%%% mode: latex
%%% TeX-master: "case"
%%% End: 
