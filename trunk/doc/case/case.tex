\documentclass{article}

\usepackage{fullpage}
\usepackage{times}
\usepackage{amsmath}
\usepackage{amssymb}
\usepackage{theorem}
\usepackage{fancyvrb,xspace}

%% MACROS

\newcommand{\ut}[1]{|#1|}
\newcommand{\tot}[1]{\|#1\|}
\newcommand{\per}[1]{\approx_{#1}}
\newcommand{\PL}{\mathcal{P}}
\newcommand{\RZ}{RZ\xspace}
\newcommand{\Mod}[1]{\mathsf{Mod}(#1)}
\newcommand{\rz}{\Vdash}
\newcommand{\note}[1]{\texttt{[#1]}}
\newcommand{\fst}[1]{\texttt{fst}\,#1}
\newcommand{\snd}[1]{\texttt{snd}\,#1}
\newcommand{\inl}[1]{\texttt{inl}\,#1}
\newcommand{\inr}[1]{\texttt{inr}\,#1}

\newcommand{\comment}[1]{\textbf{[#1]}}

%% THEOREM-LIKE ENVIRONMENTS

{
  \theorembodyfont{\itshape}

%  \newtheorem{theorem}{Theorem}[chapter]
  \newtheorem{theorem}{Theorem}[section]
  \newtheorem{Axiom}{Axiom}[section]
  \newtheorem{lemma}[theorem]{Lemma}
  \newtheorem{proposition}[theorem]{Proposition}
  \newtheorem{corollary}[theorem]{Corollary}
  \newtheorem{conjecture}[theorem]{Conjecture}
%  \newtheorem{exercise}[theorem]{Exercise}
}

{
  \theorembodyfont{\rmfamily}
  \newtheorem{definition}[theorem]{Definition}
  \newtheorem{example}[theorem]{Example}
  \newtheorem{remark}[theorem]{Remark}
  \newtheorem{question}[theorem]{Question}
  \newtheorem{problem}{Problem}
}

\newcommand{\keywd}[1]{\mbox{\texttt{#1}}\xspace}
\newcommand{\ALL}{\keywd{all}}
\newcommand{\AND}{\keywd{and}}
\newcommand{\AXIOM}{\keywd{axiom}}
\newcommand{\BOOL}{\keywd{bool}}
\newcommand{\CONST}{\keywd{const}}
\newcommand{\COROLLARY}{\keywd{corollary}}
\newcommand{\EQUIV}{\keywd{equiv}}
\newcommand{\EQUIVALENCE}{\keywd{equivalence}}
\newcommand{\EQUIVPROP}{\keywd{Equiv}}
\newcommand{\END}{\keywd{end}}
\newcommand{\EXISTS}{\keywd{exists}}
\newcommand{\EXISTSONE}{\keywd{exists1}}
\newcommand{\FALSE}{\keywd{false}}
\newcommand{\FORALL}{\keywd{forall}}
\newcommand{\IMPLICIT}{\keywd{implicit}}
\newcommand{\IFF}{\keywd{iff}}
\newcommand{\IMPLY}{\keywd{implies}}
\newcommand{\IN}{\keywd{in}}
\newcommand{\LAMBDA}{\keywd{lam}}
\newcommand{\LEMMA}{\keywd{lemma}}
\newcommand{\LET}{\keywd{let}}
\newcommand{\MATCH}{\keywd{match}}
\newcommand{\MODEL}{\keywd{model}}
\newcommand{\NOT}{\keywd{not}}
\newcommand{\OR}{\keywd{or}}
\newcommand{\PROP}{\keywd{Prop}}
\newcommand{\PROPOSITION}{\keywd{proposition}}
\newcommand{\PREDICATE}{\keywd{predicate}}
\newcommand{\RELATION}{\keywd{relation}}
%\newcommand{\RZ}{\keywd{rz}}
\newcommand{\SET}{\keywd{set}}
\newcommand{\SOME}{\keywd{some}}
\newcommand{\STABLE}{\keywd{stable}}
\newcommand{\STABLEPROP}{\keywd{Stable}}
\newcommand{\STRUCTURE}{\keywd{structure}}
\newcommand{\THE}{\keywd{the}}
\newcommand{\THEOREM}{\keywd{theorem}}
\newcommand{\THEORY}{\keywd{theory}}
\newcommand{\THY}{\keywd{thy}}
\newcommand{\TRUE}{\keywd{true}}
\newcommand{\UNIQUE}{\keywd{unique}}
\newcommand{\UNIT}{\mbox{\Verb| unit |}}
\newcommand{\WITH}{\keywd{with}}

\newcommand{\metav}[1]{\mbox{\textit{#1}}\xspace}

\newcommand{\Case}{\metav{case}}
\newcommand{\Ident}{x}
\newcommand{\Identifier}{\Ident}
\newcommand{\Label}{\metav{\Verb|`|label}}
\newcommand{\MIdentifier}{\metav{Modelname}}
\newcommand{\TIdentifier}{\metav{Theoryname}}
\newcommand{\Setexp}{\metav{set}}
\newcommand{\Set}{\metav{set}}
\newcommand{\Specification}{\metav{specification}}
\newcommand{\Proposition}{\metav{proposition}}
\newcommand{\Prop}{\metav{prop}}
\newcommand{\Param}{\metav{param}}
\newcommand{\MParam}{\metav{modelparam}}
\newcommand{\Term}{\metav{term}}
\newcommand{\Theoryexp}{\metav{theory}}

\newcommand{\AAND}{\mbox{\Verb| \&\& |}}
\newcommand{\ARROW}{\mbox{\Verb| -> |}}
\newcommand{\BAR}{\mbox{\Verb+ | +}}
\newcommand{\COLON}{\mbox{\Verb| : |}}
\newcommand{\COMMA}{\mbox{\Verb| , |}}
\newcommand{\EQUALS}{\mbox{\Verb| = |}}
\newcommand{\HASH}{\mbox{\Verb| \# |}}
\newcommand{\IIFF}{\mbox{\Verb| <=> |}}
\newcommand{\IIMPLY}{\mbox{\Verb| => |}}
\newcommand{\LBRACE}{\mbox{\Verb|\{ |}}
\newcommand{\LBRACK}{\mbox{\Verb|[ |}}
\newcommand{\LCOMMENT}{\mbox{\Verb|(* |}}
\newcommand{\LPAREN}{\mbox{\Verb|( |}}
\newcommand{\OOR}{\mbox{\Verb+ || +}}
\newcommand{\ONE}{\mbox{\Verb|1|}}
\newcommand{\PERCENT}{\mbox{\Verb+ \% +}}
\newcommand{\PLUS}{\mbox{\Verb| + |}}
\newcommand{\RBRACE}{\mbox{\Verb| \}|}}
\newcommand{\RBRACK}{\mbox{\Verb| ]|}}
\newcommand{\RCOMMENT}{\mbox{\Verb| *)|}}
\newcommand{\RPAREN}{\mbox{\Verb| )|}}
\newcommand{\PERIOD}{\mbox{\Verb| . |}}
\newcommand{\SUBIN}{\mbox{\Verb| :> |}}
\newcommand{\SUBOUT}{\mbox{\Verb| :< |}}
\newcommand{\TIMES}{\mbox{\Verb| * |}}
\newcommand{\TO}{\mbox{\Verb| -> |}}
\newcommand{\ZERO}{\mbox{\Verb|0|}}

%%%%%%%%%%%%%%%%%%%%%%%%%%%%%%%%%%%%%%%%%%%%%%%%%%
%%%%%%%%%%%%%%%%%%%%%%%%%%%%%%%%%%%%%%%%%%%%%%%%%%
\begin{document}

\twocolumn

\title{Specifications via Realizability}

\author{
  Andrej Bauer\\
  {\small Department of Mathematics and Physics}\\
  {\small University of Ljubljana, Slovenia}\\
  {\small \texttt{Andrej.Bauer@andrej.com}}
  \and
  Christopher Stone\\
  {\small Computer Science Department}\\
  {\small Harvey Mudd College, USA}\\
  {\small \texttt{stone@cs.hmc.edu}}
}

\maketitle

\begin{abstract}
We present a system, called~\RZ, for automatic generation of
program specifications from mathematical theories. We translate
mathematical theories to specifications by computing their
realizability interpretations in an ML-like language augmented with
assertions. While the system is best suited for descriptions of those
data structures that can be easily described in mathematical
language (e.g., finitely presented groups, real
arithmetic, graphs, etc.), it also elucidates the
relationship between data structures and constructive mathematics.
\end{abstract}

\section{Introduction}
\label{sec:introduction}

We present a system, called~\emph{RZ}, for automatic generation of
program specifications from mathematical theories. We translate
mathematical theories to specifications by computing their
realizability interpretations in an ML-like language augmented with
assertions. While the system is best suited for descriptions of those
programming tasks that can be easily described in mathematical
language (e.g., algorithms on finitely presented groups, exact real
arithmetic, algorithms on graphs, etc.), it also elucidates the
relationship between data structures and constructive mathematics.

As the realizability interpretation validates the laws of
\emph{intuitionistic} logic, our input theories are intuitionistic or
constructive. Thus, RZ is a tool that extracts the computational
meaning of a constructive theory and expresses it as a programming
specification.

We emphasize that RZ does \emph{not} extract programs from proofs---in
fact, there is no way to write a proof in our system. We work ``one
level above'', so to speak, and just determine what the programs are
supposed to do, i.e., we provide specifications for them. We leave it
to the programmer, or to another tool, to construct the programs as he
or she best knows. This leaves the programmer completely free to write
the most efficient programs that do not necessarily correspond to any
proofs.

The original aim of RZ was to help with development of data structures
for computable mathematics. If one sets out to actually compute
realizability interpretations of theories of constructive mathematics,
one quickly wishes for an automated way of doing it. With a tool like
RZ it is much easier to experiment and try out variations of a theory
until a suitable specification is obtained. We have also discovered
that RZ can be used to explain and teach constructive mathematics to
programmers, who are typically trained in classical mathematics,
because it translates constructive statements to easily understood
requirements about programs (expressed in classical logic).

We assume throughout that we have chosen a fixed programming
language~$\PL$. Any ML-like language will
do~\cite{milner+:definition}. Our implementation of RZ uses Objective
Caml~\cite{ocaml} but could easily be adopted to other variants of ML.
The essential features we require of~$\PL$ are product, function, and
sum types, as well as support for module interfaces.

The paper is organized as follows. Section~\ref{sec:realizability}
contains a brief overview of realizability. In
Section~\ref{sec:theories-signatures} we describe theories and
signatures, which are the inputs and the outputs of RZ, respectively. In
Section~\ref{sec:implementation} we discuss various point of
implementation. In Section~\ref{sec:examples} we show typical examples
and conclude with Section~\ref{sec:conclusion}.


%%% Local Variables: 
%%% mode: latex
%%% TeX-master: "case"
%%% End: 



\section{Realizability}
\label{sec:realizability}

We briefly motivate the main idea of (typed) realizability. When we
represent a set of mathematical objects~$S$ in a programming
language~$\PL$ there are two natural steps to take: first choose an
\emph{underlying type~$\ut{S}$} of representing values, and second
specify how the values of type~$\ut{S}$ represent, or \emph{realize},
elements of the set~$S$. For example, consider how we might represent
the set~$D$ of simple finite directed graphs (whose vertices are
labeled by integers). As the underlying datatype we might choose
$\ut{D} = \mathtt{int} \; \mathtt{list} *\, (\mathtt{int} *
\mathtt{int}) \mathtt{list}$, and represent a graph~$G \in D$ as a
pair of lists $(v,e)$ where $v = [x_1; \ldots; x_n]$ is the list of
vertices and $e = [e_1; \ldots; e_m]$ is the list of edges. Formally,
we write
%
\begin{equation*}
  (v, e) \rz_D G
\end{equation*}
%
and read it as ``$(v,e)$ realizes~$G \in D$''. Observe that each graph
is realized by at least one pair of lists, and that no pair of lists
represents more than one graph. (As commonly occurs, most graphs are
represented by many different pairs of lists.) This leads us to the definition given
below. We shall abuse notation slightly and write $t \in \ut{S}$ to
mean that $t$ is a closed expression of type $\ut{S}$.


\begin{figure*}
\[
  \hspace{-0.5truecm}
  \parbox[t]{0.28\textwidth}{
\footnotesize
    \begin{align*}
      \ut{\top} &= \mathtt{unit} \\
      \ut{\bot} &= \mathtt{unit} \\
      \ut{x =_S y} &= \mathtt{unit} \\
      \ut{\phi \land \psi} &= \ut{\phi} * \ut{\psi} \\
      \ut{\phi \implies \psi} &= \ut{\phi} \to \ut{\psi} \\
      \ut{\phi \lor \psi} &= \ut{\phi} + \ut{\psi} \\
      &\\
      \ut{\forall x \in A .\, \phi(x)} &= \ut{A} \to \ut{\phi} \\
      \ut{\exists x \in A .\, \phi(x)} &= \ut{A} \times \ut{\phi}
    \end{align*}
  }
\ \vrule\ 
  \parbox[t]{0.5\textwidth}{
\footnotesize
    \begin{align*}
      () \rz \top &
      \\
      &
      \\
      () \rz x =_S y
        &\quad\text{iff}\quad 
      x \per{S} y
      \\
      (t_1,t_2) \rz \phi \land \psi
        &\quad\text{iff}\quad
        \text{$t_1 \rz \phi$ and $t_2 \rz \psi$}
      \\
      t \rz \phi \implies \psi
        &\quad\text{iff}\quad
        \text{for all $u \in \ut{\phi}$, if $u \rz \phi$ then $t\,u
          \rz \psi$}
      \\
      \inl{t} \rz \phi \lor \psi
        &\quad\text{iff}\quad
        \text{$t \rz \phi$}
      \\
      \inr{t} \rz \phi \lor \psi
        &\quad\text{iff}\quad
        \text{$t \rz \psi$}
      \\
      t \rz \forall x \in A . \phi(x)
        &\quad\text{iff}\quad
        \text{for all $u \in \ut{A}$, if $u \rz_A x$ then $t\,u \rz \phi(x)$}
      \\
      (t_1, t_2) \rz \exists x \in A . \phi(x)
        &\quad\text{iff}\quad
        \text{$t_1 \rz_A x$ and $t_2 \rz \phi(x)$}
    \end{align*}
  }
\]
  \caption{Realizability interpretation of logic (outline)}
  \label{fig:rz-logic}
\end{figure*}

\begin{definition}
  A \emph{modest set}\footnote{Modest sets were so named by Dana
    Scott. They are ``modest'' because their size cannot exceed the
    number of expressions of the underlying datatype.} is a triple
  $(S, \ut{S}, {\rz_S})$ where $S$ is a set, $\ut{S}$ is a type and
  $\rz_S$ is a relation between expressions of type~$\ut{S}$ and
  elements of~$S$, satisfying:
  % 
  \begin{enumerate}
  \item For every $x \in S$ there is $t \in \ut{S}$ such that $t \rz_S
    x$.
  \item If $t \rz_S x$ and $t \rz_S y$ then $x = y$.
  \end{enumerate}
  %
  A \emph{realized function} $f : (S, \ut{S}, {\rz_S}) \to (T, \ut{T},
  {\rz_T})$ between modest sets is a function $f : S \to T$ for which
  there exists $u \in \ut{S} \to \ut{T}$ such that
  %
  \begin{equation*}
    t \rz_S x \implies u\,t \rz_T f(x) \;.
  \end{equation*}
  %
  We say that $u$ \emph{realizes}~$f$.
\end{definition}

The realizer~$u$ of a realized function~$f$ is more commonly known as
an ``implementation of~$f$'' or an ``algorithm for computing~$f$''.

Modest sets and realized functions form a category of \emph{modest
  sets~$\Mod{\PL}$}. In realizability theory this is a well known
category with good properties. It is regular and locally bi-cartesian
closed, which allows us to interpret first-order logic and a rich type
theory. Here we only outline the main ideas behind the realizability
interpretation of logic. See e.g.~\cite{Bauer:00} for details.

In the realizability interpretation of logic, each formula~$\phi$ is
assigned a set of \emph{realizers} which can be thought of as
computations that witness the validity of~$\phi$. The situation is
somewhat similar (but not equivalent) to the propositions-as-types
translation of logic into type theory, where the proofs of a
proposition correspond to terms of the corresponding type. More
precisely, to each formula~$\phi$ we assign an underlying type
$\ut{\phi}$ of realizers. However, unlike in the propositions-as-types
translation, not all terms of type $\ut{\phi}$ are necessarily valid
realizers for~$\phi$. We write $t \rz \phi$ when $t \in \ut{\phi}$ is
a realizer for~$\phi$. The underlying types and the
realizability relation~$\rz$ are defined inductively on the structure
of~$\phi$; an outline is shown in Figure~\ref{fig:rz-logic}. We say that a
formula~$\phi$ is \emph{valid} in~$\Mod{\PL}$ if it has at least one
realizer.

We shall not dwell any further on the technicalities involving the
category of modest sets, but rather proceed to a concrete description
of our realizability translation. There is one technical point,
though, which we first take care of. A modest set is a triple $(S,
\ut{S}, {\rz_S})$ in which~$S$ is an arbitrary set. For an automated
system it would be convenient if it did not have to refer to arbitrary
sets but rather just to ingredients that are already present in the
programming language, such as types and sets of expressions. Up to
equivalence of categories, modest sets can be constructed as triples
$(\ut{S}, \tot{S}, {\per{S}})$ where $\ut{S}$ is a type, $\tot{S}$ is
a subset of expressions of type~$\ut{S}$, called the \emph{total
  values},\footnote{We do \emph{not} require that a total value must
  be a terminating expression.} and $\per{S}$ is an equivalence
relation on~$\tot{S}$. The relationship between this representation of
a modest set and the original one is as follows:
%
\begin{itemize}
\item $\tot{S}$ is the set of those $t \in \ut{S}$ that
  realize something, i.e., there is $x \in S$ such that $t \rz_S x$.
  These correspond to implementations that satisfy
  the representation invariant, e.g., graphs where the list of edges
  mentions only integers in the list of nodes, a subset of
  all values of type $\mathtt{int} \; \mathtt{list} * (\mathtt{int} *
\mathtt{int}) \; \mathtt{list}$.
\item $t \per{S} u$ if $t$ and $u$ realize the same element, i.e.,
  there is $x \in S$ such that $t \rz_S x$ and $u \rz_S x$.
  This relation equates alternate concrete representations of the same
  abstract value, e.g., equating two concrete graph representations differing
  only in the order of the nodes or the order of the edges.
\end{itemize}
%
The alternative view of a modest set $(\ut{S}, \tot{S}, {\per{S}})$
only refers to objects and concepts from the programming language. It
is better suited for our purposes.

Note that the equivalence relation on~$\tot{S}$ is also a
\emph{partial} equivalence relation on~$\ut{S}$, which shows that
modest sets are in fact equivalent to PER models.


%%% Local Variables: 
%%% mode: latex
%%% TeX-master: "case"
%%% End: 



\section{Theories and Signatures}
\label{sec:theories-signatures}

\begin{figure*}[t]
\[
\begin{array}[t]{l}
\mbox{\textbf{Theory Elements}}\\
\SET\ s\ [\EQUALS\ \mathit{set}\ ]\\
\CONST\ c\ [\COLON \mathit{set}\ ]\ [\EQUALS\ \mathit{term}\ ]\\
{}[\STABLE{}]\ \RELATION\ r\ [\COLON \mathit{set}\ ]\ [\EQUALS \mathit{prop}\ ]\\
\EQUIVALENCE \COLON \mathit{set}\\
\MODEL\ M\COLON\mathit{theory}\\
\AXIOM\ a\ [\ M\COLON\mathit{theory}\ ]^{*}\ [x\COLON\mathit{set}\ ]^*\EQUALS\mathit{prop}\\
\\
\mbox{\textbf{Propositions}}\\
\TRUE\\
\FALSE\\
\NOT\ \Prop\\
\Prop \AAND \Prop\\
\Prop \OOR \Prop\\
\Prop \IIMPLY \Prop\\
\Prop \IIFF \Prop\\
r [\ \Term\ ]^*\\
\Term\EQUALS\Term\\ %[\IN\ \Set]\\
\ALL\ [x\COLON \Set] \PERIOD \Prop\\
\SOME\ [x\COLON \Set] \PERIOD \Prop\\
\UNIQUE\ [x\COLON \Set] \PERIOD \Prop\\
\end{array}
\qquad
\begin{array}[t]{l}
\mbox{\textbf{Sets}}\\
\ZERO\\
\ONE\\
\BOOL\\
s\\
\mathit{Model}\PERIOD \mathit{name}\\
\Set \TIMES \cdots \TIMES \Set\\
\Setexp \ARROW \Setexp\\
\Label\ [\COLON \Setexp\ ]\ \PLUS \cdots \PLUS \Label\ [\COLON \Setexp\ ]\\
\LBRACE \Ident [\ \COLON \Setexp\ ]\ \BAR \Proposition \RBRACE\\
\Setexp \PERCENT \metav{relation}\\
%\RZ\ \Setexp\\
\\
\mbox{\textbf{Terms}}\\
x\\
\LPAREN \Term\COMMA \cdots \COMMA \Term \RPAREN\\
\Term\PERIOD \metav{n}\\
\Label\ [\ \Term\ ]\\
\MATCH\ \Term\ \WITH\ \mbox{\textit{pattern-matches}}\\
\LAMBDA\ x\COLON\Set\ \PERIOD\ \Term\\
\Term\ \Term\\
\Term \PERCENT \metav{relation}\\
\LET\ x \PERCENT \metav{relation}\ \IN\ \Term\EQUALS\Term\\
\Term \SUBIN \Set\\
\Term \SUBOUT \Set\\
\THE\ x\ [\COLON\Set\ ] \PERIOD \Prop\\
\LET\ x\ [\COLON\Set\ ] \EQUALS\Term\ \IN\ \Term\\
\end{array}
\]  
\caption{Input Language Summary}
\label{fig:input}  
\end{figure*}

In this section we describe first-order theories and signatures.
Our system translates the former into the later.

\subsection{Theories}
\label{sec:theories}

A \emph{theory} is a description of a mathematical structure, such as
a group, a vector space, a directed graph, etc. A theory consists of
%
\begin{itemize}
\item a list of \emph{basic sets},
\item a list of \emph{basic constants} belonging to specified sets,
\item a list of \emph{basic relations} on specified sets,
\item a list of axioms.
\end{itemize}
%
To take a simple example, consider the theory of a semigroup in which
every element has a (possibly non-unique) square root; recall that a
semigroup is a set with an associative binary operation and a neutral
element.\footnote{An example of a semigroup with square roots is the
  complex numbers with multiplication as the binary operation.} In our
system it could be written as follows:
%
\VerbatimInput{semigroup.thy}
%
The theory is enclosed by \Verb|thy|\ldots\Verb|end|. This theory
defines one basic set \Verb|s|, and two basic constants: an element
\Verb|e| of \Verb|s| and a (curried) binary infix operator \Verb|*| on
the set \Verb|s|. The \Verb|implicit| operator is not part of the
theory proper, but signals to the type checker that bound
variables named \Verb|x| or \Verb|y| or \Verb|z| should be assumed to
range over \Verb|s| unless otherwise specified. Finally,
we have three axioms. Axiom arguments, e.g., \Verb|x|, \Verb|y|, and
\Verb|z| in the associativity axiom, name the free variables occuring
in the axiom. It is not too big a mistake to think of them as being
universally quantified.

It is important to note that theories do not include proofs, but
rather just the statements of the axioms (and theorems) specified to
hold. Thus although axioms can be defined, one cannot actually refer
to them within the theory.

There are several features of theories that our system supports other
than those shown in this example above; the input language is
summarized in Figure~\ref{fig:input}, where brackets imply optional
elements.


Theories may declare or define relations. They may be \Verb|stable|,
i.e., their computational interpretation is trivial (see
Section~\ref{sec:implementation} for further discussion of this
point). Axioms can universally quantify over all models of a theory.
This is useful for describing universaility properties, such as
initiality of an algebra or finality of a coalgebra.
  
The propositions are the familiar ones from first-order logic;
$\UNIQUE$ is unique existence ($\exists!$). In addition to the basic
empty ($\ZERO$) and unit ($\ONE$) sets, one can form cartesian
products, function spaces, tagged disjoint unions, subsets, and
quotients by stable equivalence relations. The corresponding
introduction and elimination forms appear in the language of terms.
For example, $\Term \PERCENT \metav{relation}$ is the equivalence
class under $\metav{relation}$ containing $\Term$, while $\LET\ x
\PERCENT \metav{relation} \mbox{\Verb| = |} \Term_1\ \IN\ \Term_2$
binds $x$ to a representative of the equivalence class $\Term_1$ to be
used in $\Term_2$. The expression $\Term \SUBIN \Set$ injects $\Term$
into a given subset (recording a proof obligation of the term actually
being a member of the subset), while $\Term \SUBOUT \Set$ projects
$\Term$ from a subset out into its superset $\Set$. The value of the
description operator $\THE\ x \,.\, \Prop$ is the unique $x$
satisfying $\Prop$; using it incurs the obligation of proving that
there is exactly one such~$x$.


\subsection{Signatures}
\label{sec:signatures}

On the logical side, we have models described by theories.  Thus on
the programming side we should have implementations being described by
specifications.  Our tool thus translates theories into
\emph{signatures}, which are ML's module interfaces.

Signatures allow us to require the existence of certain types, as well
as values of given type.  This allows decidable typechecking, but we
need more expressiveness in order to faithfully translate the content
of a theory.  We therefore generate signatures augmented by assertion
comments, which specify constraints on the values and functions an
implementation beyond their type.  It is the responsibility of the
programmer to check that the implementation satisfies these
assertions, as \RZ does not attempt to do any theorem proving.

Assertions are written in ordinary classical first-order logic. Since
programmers typically are not trained in constructive logic, this may
make it easier to verify the assertions.

The output for the theory \Verb|SQGROUP| above is then:
{\small \VerbatimInput{semigroup.mli}}

At the ML level we have required a type \Verb|s|, and three values
\Verb|e|, \Verb|*|, and \Verb|sqrt|, of types \Verb|s|,
\Verb|s->s->s|, and \Verb|s->s|, respectively. The third value was
generated from the square root axiom, which has a non-trivial
computational content, cf.\ Subsection~\ref{subsec:real-transl}.

Comments contain other requirements, not expressible in ML, that
further contstrain the allowed implementations. The assertion
\Verb|PER(=s=)| abbreviates the requirement that \Verb|=s=| be a
partial equivalence relation on~\Verb|s|; its domain \Verb+||s||+ is
the subset of terms of type \Verb|s| that realize semigroup
elements, and the relation \Verb|=s=| identifies (possibly different)
terms realizing the same abstract semigroup element. These data
together determine a modest set. The assertion following the
declarations of \Verb|e| asserts that \Verb|e| realizes a valid
semigroup element, and the one following \Verb|*| asserts that
\Verb|*| must not be affected by the choice of realizers. Both
\Verb|e| and \Verb|*| must of course still satisfy the \Verb|unit| and
\Verb|assoc| axioms. Finally, the new function \Verb|sqrt| derived
from the logic must compute square roots. Since the theory requires
existence but not uniqueness of square roots, there is no requirement
that \Verb|sqrt| be invariant with respect to the partial equivalence
relation on \Verb|s|; different realizers of the same semigroup
element are allowed to produce (realizers of) different square roots.


\subsection{Parameterized Theories}
\label{sec:param-theor-funct}

A theory may be parameterized by one or more models of other theories.
For example, a theory \Verb|Real| of the reals may be parameterized in
terms of a model \Verb|N| of the naturals.  A theory of free groups may be
parameterized in terms of the generating set.

Parameterized theories serve two purposes.  First a model of a
parameterized theory would be a generic implementation that, given any
implementation of the parameters, returns an implementation of the
resulting theory.  At the level of ML, this would be a function from
modules to modules, which is called a \emph{functor}, and so a
parameterized theory can be translated into the signature of a
functor.

Alternatively, once we have described a parameterized theory
\Verb|Real|, we may wish to use it to describe a single specific
implementation of real numbers based on a specific model \Verb|N1|
(implementation) of the natural numbers; this can be described as
an implementation satisfying the theory \Verb|Real(N1)|.

The dual nature of parameterized theories as being both a description
of a parameterized model (a $\Pi$ type) and something which can be
applied to a model to produce a specialized theory (a $\lambda$) is
very reminiscent the type inclusion of Automath~\cite{automath}.  ML
does not permit applications of functor signatures, however, so we
beta-reduce all theory applications before generating signatures;
\Verb|Real(N1)| would produce a signature for a real-number
implementation that refers directly to \Verb|N1| rather than to a
generic parameter \Verb|N|.


%%% Local Variables: 
%%% mode: latex
%%% TeX-master: "case"
%%% End: 


\section{Implementation}
\label{sec:implementation}

The RZ implementation consists of several sequential passes.

After the initial parsing, a \emph{type reconstruction} phase checks
that the input is well-typed (and checks for well-formedness to the
extent that it is easily decidable), and if successful produces an
annotated result with all variables explicitly tagged with types. The
type checking phase uses a system of dependent types, with limited
subtyping (implicit coercions) for sum types and subset types. The
details are fairly standard, so are omitted here. One non-obvious
consequence of the realizability translation, however, is that the
subset types $\isubset{\ix}{\iS}{\iand{\ipp_1(\ix)}{\ipp_2(\ix)}}$ and
$\isubset{\ix}{\iS}{\iand{\ipp_2(\ix)}{\ipp_1(\ix)}}$ are not
equal, but only isomorphic in general. An
explicit coercion is required to go from one type to the other,
because subset values are pairs containing realizers for
$\iand{\ipp_1(\ix)}{\ipp_2(\ix)}$ and
$\iand{\ipp_2(\ix)}{\ipp_1(\ix)}$, and these realizers have
potentially different types $|\ipp_1(\ix)|\mathtt{*}|\ipp_2(\ix)|$ and
$|\ipp_2(\ix)|\mathtt{*}|\ipp_1(\ix)|$ respectively.

Next the realizability translation is performed as described in
Section~\ref{sec:translation}, producing interface code. The
flexibility of the full input language (e.g., $n$-ary sum types and
dependent product types) makes the translation code fairly involved,
and so it is performed in a ``naive'' fashion whenever possible. The
immediate result of the translation is not easily readable.
 
Thus, three more passes simplify the output before it is displayed to
the user. A \emph{thinning} pass removes all references to trivial
realizers produced by stable formulas. For example, direct translation
of the $\mathtt{free}$ axiom in the output for Kuratowski-finite sets
yields a value specification for $\mathtt{free}$ of type
%
\begin{equation*}
  (\mathtt{A.a} \to \mathtt{S.s}) \to 
  (\mathtt{fin} \to \mathtt{S.s}) * (\ounit * (\mathtt{A.a} \to
  \ounit) *
  (\mathtt{fin} \to \mathtt{fin} \to \ounit))
\end{equation*}
%
where $\ounit$ is the unit (terminal) type classifying the trivial
realizer. Thinning replaces this by the isomorphic type
%
\begin{equation*}
  (\mathtt{A.a} \to \mathtt{S.s}) \to \mathtt{fin} \to \mathtt{S.s}
\end{equation*}
%
and appropriately modifies references to $\mathtt{free}$ in the assertions to account for this change in type.

Next, an \emph{optimization} pass applies an ad-hoc collection of
basic logical and term simplifications in order to make the output more readable. 
Logical simplifications include applications of truth table rules
($\iand{\itrue}{\ip}$ becomes $\ip$), detection of syntactically
identical premises and conclusions
($\iimply{\ip_1}{\iand{\ip_1}{\ip_2}}$ becomes
$\iimply{\ip_1}{\ip_2}$), and optimization of other common patterns we have
seen arise
($\iforall{\ix}{\is}{\iimply{(\iequal{\ix}{\ie})}{\ipp(\ix)}}$ becomes
$\ipp(\ie)$). We do not attempt real theorem proving 
so some redundancy may remain, but in practice the optimization pass
can help significantly.

Finally, the user can specify whether two optional steps occur.
RZ can optionally performs a \emph{phase-splitting} pass~\cite{harper+:popl90}. 
This is an ML-specific transformation that replace certain
uses of parameterized modules (a heavyweight language construct) by
parameterized types and polymorphic values. The idea is that although
functors map modules containing types and terms to other modules containing types
and terms, constraints on the programming language ensure that output types
depend only on input types (and not input terms).  Thus, we can split each
functor into a mapping from input types to output types, and then a separate
(polymorphic) term mapping input types and terms to an output term.

For example (ignoring
assertions for simplicity) the entire module
\begin{source}
module Free : functor (S : Semilattice) ->
                    sig
                      val free : (A.a -> S.s) -> fin -> S.s
                    end	
\end{source}   
appearing in the output of the Kuratowski example can be replaced by the single polymorphic function
\begin{source}
val free : 's -> ('s -> 's -> 's) -> (A.a -> 's) -> fin -> 's	
\end{source}
which replaces the module parameter \texttt{S} by two extra term arguments term (corresponding to the module components \texttt{S.zero} and \texttt{S.join}) 
and a type argument \texttt{'s} for the type of lattice elements (corresponding to the module input \texttt{S.s}).

The other optional transformation is a \emph{hoisting} pass which lifts obligations in the output out to top-level positions.  This can make it easier to see exactly what one is obliged to provide.  When identical obligations appear in separate subterms of a term, hoisting can lift and merge these obligations, reducing redundancy.  However, moving obligations far from where they are used can make it harder to see why the obligation is required at all (and hence how one might satisfy the obligation), and so hoisting is turned off by default.

For example, in the following input (extracted from a larger description of an ordered field)
\begin{Verbatim}
Parameter s : Set.
Parameter zero : s.
Parameter inverse : {x : s | not (x = zero)} -> s.

Parameter lt : s -> s -> Stable.
Definition positive (x:s) := lt zero x.
Axiom lt_irr: forall x:s, not (lt x x).

Axiom order_inv: forall x:s, positive x -> positive (inverse x).
\end{Verbatim}
the axiom translates to the assertion:
\begin{Verbatim}
	(**  Assertion order_inv = 
          forall (x:||s||),  positive x ->
            positive (inv (assure (not (x =real= zero)) in x))
   *).
\end{Verbatim}
Here the system has noticed that for \Verb|inverse x| to make sense, we must know that
\Verb|x| is non-zero.   This requires non-trivial theorem proving and hence remains as 
an obligation for the user.  

We must prove \Verb|not (x =s= zero)| not for all \Verb|x|, but only under
the premises in force where the obligation occurs.  This is slightly clearer when hoisting moves the
obligation to the top level, after which it could be verified in the same way as all other assertions:
\begin{Verbatim}
   (**  Assertion order_inv = 
          assure (forall (x:||real||),  positive x -> not (x =real= zero))
            in forall (x:||real||),  positive x -> positive (inv x)
   *)	
\end{Verbatim}


%%% Local Variables: 
%%% mode: latex
%%% TeX-master: "cie"
%%% End: 


\section{Examples}
\label{sec:examples}

In this section we look at several examples which demonstrate various
points of RZ. Unfortunately, serious examples from computable
mathematics take too much space\footnote{The most basic structure
  in analysis (the real numbers) alone 
  requires several operations and a dozen or more axioms.} and will have to
be presented separately. The main theme is that constructively
reasonable axioms yield computationally reasonable operations.

\subsection{Decidable sets}
\label{sec:decidable-sets}

A set $S$ is said to be decidable when, for all $x, y \in S$, $x = y$
or $\lnot (x = y)$. In classical mathematics all sets are decidable
because decidability of equality is just an instance of the law of
excluded middle. However, RZ computes from
%
\sourcefile{decidable1.thy}
%
that the realizer for the axiom $\mathtt{eq}$ is specified by
%
\begin{source}
val eq : s -> s -> [`or0 | `or1]
(**  Assertion eq =
       forall (x:||s||, y:||s||),
         (match eq x y with
            `or0 => x =s= y
          | `or1 => not (x =s= y)
          )
*)
\end{source}
%
We read this as follows: $\mathtt{eq}$ is a function which takes
arguments~$x$ and~$y$ of type~$s$ and returns $\mathtt{`or0}$ or
$\mathtt{`or1}$. If it returns $\mathtt{`or0}$, then $x \per_s y$, and
if it returns $\mathtt{`or1}$, then $\lnot (x \per_s y)$. In other
words $\mathtt{eq}$ is a decision procedure which tells when
values~$x$ and~$y$ represent the same element of the modest set.

\subsection{Examples with obligations}
\label{sec:exampl-with-oblig}

In this section we show several small examples in which RZ outputs
obligations.

Consider how we might define division of real numbers. Given the set
of real numbers~$\mathtt{real}$ and a constant $\mathtt{zero}$
denoting~$0$, we might write
%
\sourcefile{real.thy}

\internal{Andrej}{Unfinished section.}


\subsection{Finite sets}
\label{sec:finite-sets}

There are many characterizations of finite sets, but the one that
works best constructively is due to Kuratowski, who identified the
finite subsets of~$A$ as the least family~$K(A)$ of subsets of~$A$
that contains the empty set and is closed under unions with
singletons. This characterization relies on powersets, which are not
available in RZ. But the gist of it, namely that $K(A)$ is an
\emph{inital} structure a suitable sort, can be expressed as follows.

Recall that a \emph{$\vee$-semilattice} is a set~$S$ with a
constant~$0 \in S$ and an associative, commutative, and idempotent
operation ``join'' $\vee$ on~$S$ such that $0$ is the neutral element
for~$\vee$, see Figure~\ref{fig:semilattice} for RZ axiomatization of
semilattices.
%
\begin{figure}
  \centering
\begin{source}
Definition Semilattice :=
thy
  Parameter s : Set.
  Parameter zero : s.
  Parameter join : s -> s -> s.
  Implicit Type x y z : s.
  Axiom commutative: forall x y,   join x y = join y x.
  Axiom associative: forall x y z, join (join x y) z = join x (join y z).
  Axiom idempotent:  forall x,     join x x = x.
  Axiom neutral:     forall x,     join x zero = x.
end.
\end{source}
  \caption{The theory of a semilattice}
  \label{fig:semilattice}
\end{figure}
%
The Kuratowski finite sets~$K(A)$ are the \emph{free} semilattice
generated by a set~$A$, where $\vee$ is union and $0$ is the empty
set. This is formalized in RZ as shown in Figure~\ref{fig:kuratowski}.
%
\begin{figure}
\centering
\begin{source}
Definition K (A : thy 
                Parameter a : Set.
              end) :=
thy
  include Semilattice.
  Parameter singleton : A.a -> s.
  Definition fin := s.
  Definition emptyset := zero.
  Definition union := join.

  Axiom free :
    forall S : Semilattice, forall f : A.a -> S.s,
    exists1 g : fin -> S.s, 
      g emptyset = S.zero /\
        (forall x : A.a, f x = g (singleton x))/\
        (forall u v : fin, g (union u v) = S.join (g u) (g v)).
end.
\end{source}
  \caption{Kuratowski finite sets}
  \label{fig:kuratowski}
\end{figure}
%
The theory $K$ is parametrized by a model~$A$ which contains a
set~$a$. In the first line we include the theory of semilattices. Then
we postulate an operation $\mathtt{singleton}$ which injects the
generators into the semilattice. The three definitions are just a
convenience, so that we can refer to the parts of $K(A)$ by their
natural names, e.g., $\mathtt{emptyset}$ instead of $\mathtt{zero}$.
The axiom $\mathtt{free}$ expresses the fact that $K(A)$ is the free
semilattice on~$A.a$: for every semilattice $S$ and a map $f : A.a \to
S.s$ from the generators to the underlying set of~$S$, there exists a
unique semilattice homomorphism $g : \mathtt{fin} \to S.s$ such that
$f(x) = g(\set{x})$.

The output for $\mathtt{Semilattice}$ and~$\mathtt{K}$ specifies
values of suitable types for each declared constant and operation. All
axioms but the last one are equations and have straightforward
translations in terms of underlying pers. The output for the axiom
$\mathtt{free}$ is shown in Figure~\ref{fig:free}.
%
\begin{figure}
  \centering
\begin{source}
module Free : functor (S : Semilattice) ->
sig
val free : (A.a -> S.s) -> fin -> S.s
(**  Assertion free = 
forall (f:||A.a -> S.s||), 
  let g = free f in g : ||fin -> S.s|| /\ 
  g emptyset =S.s= S.zero /\ 
  (forall (x:||A.a||),  f x =S.s= g (singleton x)) /\ 
  (forall (u:||fin||, v:||fin||), g (union u v) =S.s= S.join (g u) (g v)) /\ 
  (forall h:fin -> S.s,  h : ||fin -> S.s|| /\ 
     h emptyset =S.s= S.zero /\ 
     (forall (x:||A.a||), f x =S.s= h (singleton x)) /\ 
     (forall (u:||fin||, v:||fin||), 
        h (union u v) =S.s= S.join (h u) (h v)) ->
     forall x:fin, y:fin,  x =fin= y -> g x =S.s= h y)
*)
end
\end{source}
  \caption{Output of axiom $\texttt{free}$.}
  \label{fig:free}
\end{figure}
%
Because the axiom quantifies over all models~$S$ of the theory
$\mathtt{Semilattice}$ its translation is a functor~$\mathtt{Free}$
which accepts an implementation of a semilattice~$S$ and yields a
realizer $\mathtt{free}$ validating the axiom. The computational
meaning of $\mathtt{free}$ is a folding operation on finite sets: take
a map $f : A.a \to S.s$ and a finite set~$u = \set{x_1, \ldots, x_n}$,
and return $f(x_1) \vee \cdots \vee f(x_n)$, where $\vee$ is the join
operation on the semilattice~$S$.

The OCaml standard library contains a module $\mathtt{Set}$
implementing finite sets, which however is \emph{not} an
implementation of Kuratowski finite sets presented here. Rather,
$\mathtt{Set}$ implements something close to Kuratowski finite sets
over a set~$A$ equipped with a decidable linear order.


\subsection{Inductive types}
\label{sec:inductive-types}

To demonstrate the use of dependent types we show how RZ handles
general inductive types, also known as
\emph{W-types}~\cite{w-type-reference}. Recall that a W-type is a set
of well-founded trees, where the branching types of trees are
described by a family of sets $B = \set{T(x)}_{x \in S}$. Each node in
a tree has a \emph{branching type}~$x \in S$, which determines that
the successors of the node are labeled by the elements of~$T(x)$. For
example, to get non-empty binary trees whose leaves are labeled by
natural numbers, define
%
\begin{align*}
  S &= \set{\mathtt{cons}} \cup \set{\mathtt{leaf}(n) \such n \in \NN}
  \\
  T(\mathtt{cons}) &= \set{\mathtt{left}, \mathtt{right}}
  \\
  T(\mathtt{leaf}(n)) &= \emptyset.
\end{align*}
%
Then a node of type $\mathtt{cons}$ has two successors, indexed by
constants $\mathtt{left}$ and $\mathtt{right}$, while a node of type
$\mathtt{leaf}(n)$ does not have any successors.

Figure~\ref{fig:wtype} shows an RZ axiomatization of W-types.
%
\begin{figure}
  \centering
  \sourcefile{wtype.thy}
  \caption{General inductive types}
  \label{fig:wtype}
\end{figure}
%
The theory $\mathtt{Branching}$ describes that a branching type
consits of a set~$s$ and a set~$t$ depending on~$s$. The theory~$W$ is
parametrized by a branching type~$B$. It specifies a set~$w$ of
well-founded trees and a tree-forming operation $\mathtt{tree}$ with a
dependent type $\prod_{x \in B.s} (B.t(x) \to w) \to w$. Given a
branching type~$x$ and a map $f : B.t(x) \to w$, $\mathtt{tree}\,x\,f$
is the tree whose root has branching type~$x$ and whose successor
labeled by $\ell \in B.t(x)$ is the tree~$f(\ell)$. The inductive
nature of~$w$ is expressed with the axiom $\mathtt{induction}$, which
states that for every property $M.p$, if $M.p$ is an inductive
property then every tree satsifies it. A property is said to be
\emph{inductive} if a tree $\mathtt{tree}\,x\,f$ satisfies it whenever
all its successors satisfy it.

In the translation, see Appendix~\ref{sec:outp-induct-types},
dependencies at the level of types and terms disappear. A branching
type is determined by a pair of non-dependent types~$s$ and~$t$ but
the per $\per_t$ depends on~$\values{s}$. The theory~$W$ turns into a
signature for a functor receiving a branching type~$B$ and returning a
type~$w$, and an operation $\mathtt{tree}$ of type $B.s \to (B.t \to
w) \to w$. In this example we use phase-splitting (see
Section~\ref{sec:implementation}) to translate axiom
$\mathtt{induction}$ into a specification of a polymorphic function
(compare with output for axiom \texttt{free} in the previous example)
%
\begin{equation*}
  \mathtt{induction}:
  (B.s \to (B.t \to w) \to (B.t \to \poly{ty\_p}) \to \poly{ty\_p}) \to w \to \poly{ty\_p},
\end{equation*}
%
which is a form of recursion on well-founded trees. Instead of trying
to explain what $\mathtt{induction}$ is supposed to do, we show in
Figure~\ref{fig:wtype-implementation} a surprisingly simple, complete
hand-written implementation of W-types. The reader may entertain
himself by figuring out how $\mathtt{induction}$ works.
%
\begin{figure}
  \centering
  \sourcefile{wtype.ml}
  \caption{An implementation of general inductive types.}
  \label{fig:wtype-implementation}
\end{figure}


\subsection{Axiom of choice}
\label{sec:axiom-choice}

In this example we show how RZ can help explain why a generally
accepted axiom is not constructively valid. Consider the Axiom of
Choice:
%
\sourcefile{choice.thy}
%
The relevant part of the output is
%
\begin{source}
val ac : (a -> b * ty_r) -> (a -> b) * (a -> ty_r)
(**  Assertion ac =
  forall f:a -> b * ty_r,
    (forall (x:||a||),  let (p,q) = f x in p : ||b|| /\ r x p q) ->
    let (g,h) = ac f in g : ||a -> b|| /\
    (forall (x:||a||),  r x (g x) (h x))
*)
\end{source}
%
This requires a function $\mathtt{ac}$ which accepts a function $f$
and computes a pair of functions $(g,h)$. The input function~$f$ takes
an $\ototal{x}{a}$ and returns a pair $(p,q)$ such that $q$ realizes
the fact that $r\,x\,p$ holds. The output functions $g$ and $h$ taking
$\ototal{x}{a}$ as input must be such that $h\,x$ realizes
$r\,x\,(g\,x)$. Crucially, the requirement $\ototal{g}{\oarrow{a}{b}}$
says tht $g$ must be extensional, i.e., map equivalent realizers to
equivalent realizers. We could define~$h$ as the first component
of~$f$, but we cannot hope to implement~$g$ in general because the
second component of~$f$ is not assumed to be extensional.

The \emph{Intensional} Axiom of Choice allows the choice function to
depend on the realizers:
%
\sourcefile{ichoice.thy} Now the output is
%
\begin{source}
val iac : (a -> b * ty_r) -> (a -> b) * (a -> ty_r)
(**  Assertion iac =
  forall f:a -> b * ty_r,
    (forall (x:||a||),  let (p,q) = f x in p : ||b|| /\ r x p q) ->
    let (g,h) = iac f in (forall x:a,  x : ||a|| -> g x : ||b||) /\
    (forall (x:||a||),  r x (g x) (h x))
*)
\end{source}
%
which is exactly the same as before, \emph{except} that the
troublesome requirement $\ototal{g}{\oarrow{a}{b}}$ turned into
$\oforall{x}{a}{(\oimply{\ototal{x}{a}}{\ototal{g\,x}{b}})}$, which
is weaker. We can impement $\mathtt{iac}$ as
%
\begin{source}
let iac f = (fun x -> fst (f x)), (fun x -> snd (f x))
\end{source}
%

The Intensional Axiom of Choice is in fact just an instance of the
usual Axiom of Choice applied to~$\irz{A}$ and~$B$. Combined with the
fact that~$\irz{A}$ covers~$A$, this establishes the validity of
\emph{Presentation Axiom}~\cite{barwise75:_admis_sets_struc}, which
states that every set is an image of one satisfying the axiom of
choice.

\subsection{Modulus of Continuity}
\label{sec:we-show-modulus-of-continuity-example}

As a last example we show how certain constructive principles require
the use of computational effects. To keep the example short, we
presume that we are already given the set of natural
numbers~$\mathtt{nat}$ with the usual structure.

A \emph{type 2 function} is a map $f : (\mathtt{nat} \to \mathtt{nat})
\to \mathtt{nat}$. It is said to be continuous if the output of $f(a)$
depends only on an initial segment of the sequence~$a$. We can express
this axiom in RZ as follows:
%
\begin{source}
Axiom continuity:
forall f : (nat -> nat) -> nat, forall a : nat -> nat,
  exists k, forall b : nat -> nat,
    (forall m, m <= k -> a m = b m) -> f a = f b.
\end{source}
%
The axiom says that for any $f$ and $a$ there exists $k \in
\mathtt{nat}$ such that $f(b) = f(a)$ as soon as the sequences~$a$
and~$b$ agree on the first $k$ terms. The axiom is translated to the
specification
%
\begin{source}
val continuity : ((nat -> nat) -> nat) -> (nat -> nat) -> nat
(**  Assertion continuity =
forall (f:||(nat -> nat) -> nat||, a:||nat -> nat||),
  let p = continuity f a in p : ||nat|| /\
  (forall (b:||nat -> nat||),
     (forall (m:||nat||),  m <= p -> a m =nat= b m) -> f a =nat= f b)
*)
\end{source}
%
which says that $\mathtt{continuity}\,f\,a$ is a number~$p$ such that
$f(a) = f(b)$ whenever $a$ and $b$ agree on the first~$p$ terms. In
other words, $\mathtt{continuity}$ is a \emph{modulus of continuity}
functional. It cannot be implemented in a purely functional
language~\cite{modulus-violates-ac2}, but with the use of store we can
implement it as
%
\begin{source}
let continuity f a =
  let k = ref 0 in
  let a' n = (k := max !k n; a n) in
    f a' ; !k
\end{source}
%
To compute a modulus for~$f$ at~$a$, the program creates a
function~$a'$ which is just like~$a$ except that it stores in~$p$ the
largest argument at which it has been called. Then $f\,a'$ is
computed, its value it discarded, and the value of~$p$ is returned.
The program works because~$f$ is assumed to be extensional and must
therefore not distinguish between extensionally equal sequences~$a$
and~$a'$.



%%% Local Variables: 
%%% mode: latex
%%% TeX-master: "cie"
%%% End: 


\section{Conclusions and Future Work}
\label{sec:conclusion}

By translating only at the level of specifications, we provide a
useful middle ground between ad-hoc implementations and machine-checked
implementations --- trading guaranteed correctness for 
Further, even if it turns out not to be useful on the 
addition, \emph{RZ} allows the constructive
content of mathematics to be made con

A prototype implementation of \emph{RZ} has been implemented that can
handle a number of small examples, including those shown here.



We have presented the current prototype implementation of 
\emph{RZ}, for automatic generation of
program specifications from mathematical theories. We translate
mathematical theories to specifications by computing their
realizability interpretations in an ML-like language augmented with
assertions. While the system is best suited for descriptions of those
programming tasks that can be easily described in mathematical
language (e.g., algorithms on finitely presented groups, exact real
arithmetic, algorithms on graphs, etc.), it also elucidates the
relationship between data structures and constructive mathematics.
 

We could do polymorphism
We could do dependent types: translates to non-dependent types, but generates many proof obligations.
Need to try it on larger examples.
doc/case/
\comment{The following either goes into future work, or else into a
  discussion of the free join semilattice (where polymorphism would
  yield a fold-like operator).}  The translation of axioms
parameterized by models into polymorphism (instead of functors) is
essentially the phase-splitting translation of Harper, Mitchell, and
Moggi \cite{harper+:popl90}.


%%% Local Variables: 
%%% mode: latex
%%% TeX-master: "case"
%%% End: 


%%%%%%%%%%%%%%%%%%%%%%%%%%%%%%%%%%%%%%%%%%%%%%%%%%
% Bibliography

\bibliographystyle{plain}

\bibliography{case}


\end{document}

%%% Local Variables: 
%%% mode: latex
%%% TeX-master: t
%%% End: 
