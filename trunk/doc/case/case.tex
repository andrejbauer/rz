\documentclass{article}

\usepackage{fullpage}

\begin{document}

\twocolumn

\title{Specifications via Realizability}

\author{
  Andrej Bauer\\
  {\small Department of Mathematics and Physics}\\
  {\small University of Ljubljana, Slovenia}\\
  {\small \texttt{Andrej.Bauer@andrej.com}}
  \and
  Christopher Stone\\
  {\small Computer Science Department}\\
  {\small Harvey Mudd College, USA}\\
  {\small \texttt{stone@cs.hmc.edu}}
}

\maketitle

\begin{abstract}
  Foo bar.
\end{abstract}

\section{Introduction}
\label{sec:introduction}

In this paper we present a system, called \emph{RZ}, for automatic
generation of program specifications from mathematical theories. We
translate mathematical theories to specifications by computing their
realizability interpretations in an ML-like language augmented with
assertions.

While the system is best suited for those programming tasks that can
be easily described in mathematical language (e.g., algorithms on
finitely presented groups, exact real arithmetic, algorithms on
graphs, etc.), it also 



Use realizability as a link between constructive mathematics and
programming.

Allow programmer to write the most efficient program. Contrast with
extraction of programs from proofs.

Bring constructive reasoning closer to programmers.

Overview.

\section{Realizability}
\label{sec:realizability}

Original number realizaiblity by Kleene.

Typed realizability (using ML). Modest sets.

Point how this is what programmers naturally do informally.

\section{Theories and Signatures}
\label{sec:theories-signatures}

Theories: sets, constants, axioms.

What are signatures (ML signatures + assertions).

Parametrized theories and functor signatures.


\section{The Realizability Translation}
\label{sec:real-transl}


\section{Examples}
\label{sec:examples}

Decidable sets.

Group and groups.

Finite sets.

Axiom of choice is not realizable. This is not propositions as types.
But intensional axiom of choice is realized.

Real numbers and exact arithmetic.


\section{Conclusion}
\label{sec:conclusion}



\end{document}
